\documentclass[10pt]{beamer}

\setbeamertemplate{footline}[page number]

\usepackage{color}
\usepackage{listings}
\usepackage{tikz-cd}

\definecolor{keywordcolor}{rgb}{0.7, 0.1, 0.1}
\definecolor{tacticcolor}{rgb}{0.0, 0.1, 0.6}
\definecolor{commentcolor}{rgb}{0.4, 0.4, 0.4}
\definecolor{symbolcolor}{rgb}{0.0, 0.1, 0.6}
\definecolor{sortcolor}{rgb}{0.1, 0.5, 0.1}
\definecolor{attributecolor}{rgb}{0.7, 0.1, 0.1}
\def\lstlanguagefiles{lstlean.tex}
\lstset{language=lean}

\begin{document}

\begin{frame}

\begin{center}

{\small 2022 Xena Project Undergraduate Workshop}

\vspace{0.5cm}

{\footnotesize Thursday, 29 September 2022}

\vspace{1cm}

\textbf{\large Elliptic curves and Mordell's theorem}

\vspace{1cm}

David Ang

\vspace{0.5cm}

{\scriptsize London School of Geometry and Number Theory}

\end{center}

\end{frame}

\begin{frame}[t]{Integer solutions}

Consider Mordell's equation
$$ y^2 = x^3 + k, \qquad k \in \mathbb{Z}. $$
What are the integer solutions?

\begin{minipage}{0.6\textwidth}
$$
\begin{array}{c|c}
k & \#\{(x, y) \in \mathbb{Z}^2 : y^2 = x^3 + k\} \\
\hline
-24 & 0 \\
-6 & 0 \\
-5 & 0 \\
-1 & 1 \\
7 & 0 \\
11 & 0 \\
16 & 2
\end{array}
$$
\end{minipage}
\begin{minipage}{0.3\textwidth}
\vspace{1cm}
\begin{itemize}
\item $ k = 7 $: none \\ (mod $ 4 $ and $ 8 $)
\item $ k = 16 $: $ (0, \pm 4) $ \\ (use UF of $ \mathbb{Z} $)
\item $ k = -1 $: $ (1, 0) $ \\ (use UF of $ \mathbb{Z}[i] $)
\end{itemize}
\end{minipage}

\vspace{0.5cm}

\emph{Siegel's theorem} says that there are only finitely many integer solutions.

\end{frame}

\begin{frame}[t]{Rational solutions}

Consider Mordell's equation
$$ y^2 = x^3 + k, \qquad k \in \mathbb{Z}. $$
What about the rational solutions?
$$
\begin{array}{c|c|c}
k & \#\{(x, y) \in \mathbb{Z}^2 : y^2 = x^3 + k\} & \#\{(x, y) \in \mathbb{Q}^2 : y^2 = x^3 + k\} \\
\hline
-24 & 0 & 0 \\
-6 & 0 & 0 \\
-5 & 0 & 0 \\
-1 & 1 & 1 \\
7 & 0 & 0 \\
11 & 0 & \infty \\
16 & 2 & 2
\end{array}
$$
$ k = 11 $:
$$ (-\tfrac{7}{4}, \pm\tfrac{19}{8}), (\tfrac{41825}{5776}, \pm\tfrac{8676719}{438976}), (\tfrac{6179109049}{10788145956}, \pm\tfrac{3747956961949325}{1120521567865896}), \dots $$
\emph{Mordell's theorem} says that the rational solutions are finitely generated.

\end{frame}

\begin{frame}[t]{Elliptic curves}

If $ k \ne 0 $, Mordell's equation defines an \emph{elliptic curve}.

\begin{center}
\includegraphics[width=0.4\textwidth]{ellipticcurves.png}
\end{center}

More generally, an \textbf{elliptic curve} over a field $ F $ is a pair $ (E, \mathcal{O}) $ of
\begin{itemize}
\item a smooth projective curve $ E $ of genus one defined over $ F $, and
\item a distinguished point $ \mathcal{O} $ on $ E $ defined over $ F $.
\end{itemize}

\end{frame}

\begin{frame}[fragile, t]{Weierstrass equations}

By the \emph{Riemann-Roch theorem}, any elliptic curve over a field $ F $ is the projective closure of a plane cubic equation of the form
$$ y^2 + a_1xy + a_3y = x^3 + a_2x^2 + a_4x + a_6, \qquad a_i \in F, $$
where $ \Delta \ne 0 $, \footnote{\tiny $ \Delta := -(a_1^2 + 4a_2)^2(a_1^2a_6 + 4a_2a_6 - a_1a_3a_4 + a_2a_3^2 - a_4^2) - 8(2a_4 + a_1a_3)^3 - 27(a_3^2 + 4a_6)^2 + 9(a_1^2 + 4a_2)(2a_4 + a_1a_3)(a_3^2 + 4a_6) $} and the distinguished point is the unique point at infinity.

\vspace{0.5cm}

With this definition, an elliptic curve over $ F $ is precisely the data of the five coefficients $ a_1, a_2, a_3, a_4, a_6 \in F $ and a proof that $ \Delta \ne 0 $.

\begin{lstlisting}[basicstyle=\scriptsize, frame=single]
def Δ_aux {R : Type} [comm_ring R] (a₁ a₂ a₃ a₄ a₆ : R) : R :=
  let
    b₂ := a₁^2 + 4*a₂,
    b₄ := 2*a₄ + a₁*a₃,
    b₆ := a₃^2 + 4*a₆,
    b₈ := a₁^2*a₆ + 4*a₂*a₆ - a₁*a₃*a₄ + a₂*a₃^2 - a₄^2
  in
    -b₂^2*b₈ - 8*b₄^3 - 27*b₆^2 + 9*b₂*b₄*b₆

structure EllipticCurve (R : Type) [comm_ring R] :=
  (a₁ a₂ a₃ a₄ a₆ : R) (Δ : units R) (Δ_eq : ↑Δ = Δ_aux a₁ a₂ a₃ a₄ a₆)
\end{lstlisting}

\end{frame}

\begin{frame}[fragile, t]{$ K $-rational points}

With this definition, a point on an elliptic curve $ E $ over $ F $ is either
\begin{itemize}
\item the unique point at infinity, or
\item the data of its coordinates $ x, y \in F $ and a proof that $ (x, y) \in E $.
\end{itemize}
However, it will be important to also consider points defined over a field extension $ K $ of $ F $, the \textbf{$ K $-rational points} $ E(K) $ of $ E $.

\vspace{0.5cm}

Thus, a $ K $-rational point on $ E $ is either
\begin{itemize}
\item the unique point at infinity, or
\item the data of its coordinates $ x, y \in K $ and a proof that $ (x, y) \in E(K) $.
\end{itemize}

\begin{lstlisting}[basicstyle=\scriptsize, frame=single]
variables {F : Type} [field F] (E : EllipticCurve F) (K : Type) [field K] [algebra F K]

inductive point
  | zero
  | some (x y : K) (w : y^2 + E.a₁*x*y + E.a₃*y = x^3 + E.a₂*x^2 + E.a₄*x + E.a₆)

notation E(K) := point E K
\end{lstlisting}

\end{frame}

\begin{frame}[t]{Group law}

More importantly, $ E(K) $ can be endowed with a group structure.

\vspace{0.5cm}

Group operations are characterised by $$ P + Q + R = 0 \qquad \iff \qquad P, Q, R \ \text{are collinear}. $$

\begin{center}
\includegraphics[width=0.7\textwidth]{grouplaw.png}
\end{center}

Note that if $ a_1 = a_3 = 0 $, then $ E $ is symmetric about the $ x $-axis, so $ (x, y) $ lies in the \textbf{$ 2 $-torsion subgroup} $ E[2] := \ker (E \xrightarrow{\cdot 2} E) $ precisely if $ y = 0 $.

\end{frame}

\begin{frame}[fragile, t]{Identity and negation}

More importantly, $ E(K) $ can be endowed with a group structure.

\vspace{0.5cm}

Identity is trivial.

\begin{lstlisting}[basicstyle=\scriptsize, frame=single]
instance : has_zero E(K) := ⟨zero⟩
\end{lstlisting}

Negation is easy.

\begin{lstlisting}[basicstyle=\scriptsize, frame=single]
def neg : E(K) → E(K)
  | zero := zero
  | (some x y w) := some x (-y - E.a₁*x - E.a₃)
    begin
      rw [← w],
      ring
    end

instance : has_neg E(K) := ⟨neg⟩
\end{lstlisting}

\end{frame}

\begin{frame}[fragile, t]{Addition}

More importantly, $ E(K) $ can be endowed with a group structure.

\vspace{0.5cm}

Addition is complicated.

\begin{lstlisting}[basicstyle=\scriptsize, frame=single]
def add : E(K) → E(K) → E(K)
  | zero P := P
  | P zero := P
  | (some x₁ y₁ w₁) (some x₂ y₂ w₂) :=
    if x_ne : x₁ ≠ x₂ then
      let
        L := (y₁ - y₂) / (x₁ - x₂),
        x₃ := L^2 + E.a₁*L - E.a₂ - x₁ - x₂,
        y₃ := -L*x₃ - E.a₁*x₃ - y₁ + L*x₁ - E.a₃
      in
        some x₃ y₃ ... -- 100 lines
    else if y_ne : y₁ + y₂ + E.a₁*x₂ + E.a₃ ≠ 0 then
      let
        L := (3*x₁^2 + 2*E.a₂*x₁ + E.a₄ - E.a₁*y₁) / (2*y₁ + E.a₁*x₁ + E.a₃),
        x₃ := L^2 + E.a₁*L - E.a₂ - 2*x₁,
        y₃ := -L*x₃ - E.a₁*x₃ - y₁ + L*x₁ - E.a₃
      in
        some x₃ y₃ ... -- 100 lines
    else
      zero

instance : has_add E(K) := ⟨add⟩
\end{lstlisting}

\end{frame}

\begin{frame}[fragile, t]{Group axioms}

More importantly, $ E(K) $ can be endowed with a group structure.

\vspace{0.5cm}

The remaining group axioms are doable except associativity.

\begin{lstlisting}[basicstyle=\scriptsize, frame=single]
lemma zero_add (P : E(K)) : 0 + P = P := ... -- trivial

lemma add_zero (P : E(K)) : P + 0 = P := ... -- trivial

lemma add_left_neg (P : E(K)) : -P + P = 0 := ... -- trivial

lemma add_comm (P Q : E(K)) : P + Q = Q + P := ... -- 100 lines

lemma add_assoc (P Q R : E(K)) : (P + Q) + R = P + (Q + R) := ... -- ?? lines
\end{lstlisting}

Associativity is known to be mathematically difficult with several proofs.
\begin{itemize}
\item Just bash out the algebra!
\item Via the \emph{uniformisation theorem} in complex analysis.
\item Via the \emph{Cayley-Bacharach theorem} in projective geometry.
\item Via identification with the \emph{degree zero Picard group}.
\end{itemize}
All methods require significant further work.

\end{frame}

\begin{frame}[fragile, t]{Functoriality and Galois module structure}

Modulo associativity, what basic properties can be stated or proven?

\vspace{0.5cm}

Functoriality from field extensions to abelian groups.

\begin{lstlisting}[basicstyle=\scriptsize, frame=single]
def point_hom (φ : K →ₐ[F] L) : E(K) → E(L)
  | zero := zero
  | (some x y w) := some (φ x) (φ y) $ by { ... }

lemma point_hom.id (P : E(K)) : point_hom (K→[F]K) P = P

lemma point_hom.comp (P : E(K)) :
  point_hom (L→[F]M) (point_hom (K→[F]L) P) = point_hom ((L→[F]M).comp (K→[F]L)) P
\end{lstlisting}

Structure of invariants under a Galois action.

\begin{lstlisting}[basicstyle=\scriptsize, frame=single]
def point_gal (σ : L ≃ₐ[K] L) : E(L) → E(L)
  | zero := zero
  | (some x y w) := some (σ • x) (σ • y) $ by { ... }

variables [finite_dimensional K L] [is_galois K L]

lemma point_gal.fixed :
  mul_action.fixed_points (L ≃ₐ[K] L) E(L) = (point_hom (K→[F]L)).range
\end{lstlisting}

\end{frame}

\begin{frame}[fragile, t]{Isomorphism of elliptic curves}

Modulo associativity, what basic properties can be stated or proven?

\vspace{0.5cm}

Isomorphism given by an admissible change of variables.

\begin{lstlisting}[basicstyle=\scriptsize, frame=single]
variables (u : units F) (r s t : F)

def cov : EllipticCurve F :=
{ a₁ := u.inv*(E.a₁ + 2*s),
  a₂ := u.inv^2*(E.a₂ - s*E.a₁ + 3*r - s^2),
  a₃ := u.inv^3*(E.a₃ + r*E.a₁ + 2*t),
  a₄ := u.inv^4*(E.a₄ - s*E.a₃ + 2*r*E.a₂ - (t + r*s)*E.a₁ + 3*r^2 - 2*s*t),
  a₆ := u.inv^6*(E.a₆ + r*E.a₄ + r^2*E.a₂ + r^3 - t*E.a₃ - t^2 - r*t*E.a₁),
  disc := ⟨u.inv^12*E.disc.val, u.val^12*E.disc.inv, by { ... }, by { ... }⟩,
  disc_eq := by { simp only, rw [disc_eq, disc_aux, disc_aux], ring } }

def cov.to_fun : (E.cov u r s t)(K) → E(K)
  | zero := zero
  | (some x y w) := some (u.val^2*x + r) (u.val^3*y + u.val^2*s*x + t) $ by { ... }

def cov.inv_fun : E(K) → (E.cov u r s t)(K)
  | zero := zero
  | (some x y w) := some (u.inv^2*(x - r)) (u.inv^3*(y - s*x + r*s - t)) $ by { ... }

def cov.equiv_add : (E.cov u r s t)(K) ≃+ E(K) :=
  ⟨cov.to_fun u r s t, cov.inv_fun u r s t, by { ... }, by { ... }, by { ... }⟩
\end{lstlisting}

\end{frame}

\begin{frame}[fragile, t]{$ 2 $-division polynomial and $ 2 $-torsion subgroup}

Modulo associativity, what basic properties can be stated or proven?

\vspace{0.5cm}

Polynomial determining points in the $ 2 $-torsion subgroup.

\begin{lstlisting}[basicstyle=\scriptsize, frame=single]
def ψ₂_x : cubic K := ⟨4, E.a₁^2 + 4*E.a₂, 4*E.a₄ + 2*E.a₁*E.a₃, E.a₃^2 + 4*E.a₆⟩

lemma ψ₂_x.disc_eq_disc : (ψ₂_x E K).disc = 16*E.disc
\end{lstlisting}

Structure and cardinality of the $ 2 $-torsion subgroup.

\begin{lstlisting}[basicstyle=\scriptsize, frame=single]
notation E(K)[n] := ((•) n : E(K) →+ E(K)).ker

lemma E₂.x {x y w} : some x y w ∈ E(K)[2] ↔ x ∈ (ψ₂_x E K).roots

theorem E₂.card_le_four : fintype.card E(K)[2] ≤ 4

variables [algebra ((ψ₂_x E F).splitting_field) K]

theorem E₂.card_eq_four : fintype.card E(K)[2] = 4

lemma E₂.gal_fixed (σ : L ≃ₐ[K] L) (P : E(L)[2]) : σ • P = P
\end{lstlisting}

\end{frame}

\begin{frame}[fragile, t]{Mordell's theorem}

Modulo associativity, what basic properties can be stated or proven?

\vspace{0.5cm}

\begin{theorem}[Mordell]
$ E(\mathbb{Q}) $ is finitely generated.
\end{theorem}

\begin{lstlisting}[basicstyle=\scriptsize, frame=single]
instance : add_group.fg E(ℚ)
\end{lstlisting}

As a consequence of the structure theorem, $ E(\mathbb{Q}) $ can be written as the product of a finite group and a finite number of copies of $ \mathbb{Z} $.

\vspace{0.5cm}

\begin{proof}[Proof of Mordell's theorem]
Three steps.
\begin{itemize}
\item \textbf{Weak Mordell}: $ E(\mathbb{Q}) / 2E(\mathbb{Q}) $ is finite.
\item \textbf{Heights}: $ E(\mathbb{Q}) $ can be endowed with a ``height function".
\item \textbf{Descent}: An abelian group $ A $ endowed with a ``height function", such that $ A / 2A $ is finite, is necessarily finitely generated.
\end{itemize}
\end{proof}

\end{frame}

\begin{frame}[t]{Weak Mordell}

\begin{proof}[Proof that $ E(\mathbb{Q}) / 2E(\mathbb{Q}) $ is finite.]
\begin{itemize}
\item Reduce to $ a_1 = a_3 = 0 $, so that $ y^2 = x^3 + a_2x^2 + a_4x + a_6 $.
\item Reduce to $ E[2] \subset K $, so that $ y^2 = (x - e_1)(x - e_2)(x - e_3) $.
\item Define a homomorphism
$$
\begin{array}{rrrcc}
\delta & : & E(K) & \longrightarrow & K^\times / (K^\times)^2 \times K^\times / (K^\times)^2 \\
& & \mathcal{O} & \longmapsto & (1, 1) \\
& & (e_1, 0) & \longmapsto & ((e_1 - e_2)(e_1 - e_3), e_1 - e_2) \\
& & (e_2, 0) & \longmapsto & (e_2 - e_1, (e_2 - e_1)(e_2 - e_3)) \\
& & (x, y) & \longmapsto & (x - e_1, x - e_2)
\end{array}.
$$
\item Prove $ \ker(\delta) = 2E(K) $ by an explicit computation.
\item Prove $ \mathrm{im}(\delta) \subseteq K(S, 2) $ by a simple $ p $-adic analysis.
\item Prove $ K(S, 2) $ is finite by classical algebraic number theory.
\end{itemize}
\end{proof}

\end{frame}

\begin{frame}[t]{Selmer groups}

Here $ K(S, 2) $ is a \textbf{Selmer group}, more generally given by
$$ K(S, n) := \{x(K^\times)^n \in K^\times / (K^\times)^n : \forall p \notin S, \ \mathrm{ord}_p(x) \equiv 0 \mod n\}, $$
where $ S $ is a finite set of primes of $ K $.

\vspace{0.5cm}

The finiteness of $ K(S, n) $ reduces to the finiteness of $ K(\emptyset, n) $, which boils down to two fundamental results in classical algebraic number theory.
\begin{itemize}
\item The \emph{class group} $ \mathrm{Cl}_K $ is finite.
\item The \emph{unit group} $ \mathcal{O}_K^\times $ is finitely generated.
\end{itemize}

\vspace{0.5cm}

Then $ K(\emptyset, n) $ can be nested in a short exact sequence
$$ 0 \to \mathcal{O}_K^\times / (\mathcal{O}_K^\times)^n \to K(\emptyset, n) \to \mathrm{Cl}_K[n] \to 0, $$
whose flanking groups are both finite, so $ K(\emptyset, n) $ is also finite.

\end{frame}

\begin{frame}[t]{Heights and descent}

\begin{proof}[Proof that $ E(\mathbb{Q}) / 2E(\mathbb{Q}) $ finite implies $ E(\mathbb{Q}) $ finitely generated.]
There is a function $ h : E(\mathbb{Q}) \to \mathbb{R} $ with the following three properties.
\begin{itemize}
\item For all $ Q \in E(\mathbb{Q}) $, there exists $ C_1 \in \mathbb{R} $ such that for all $ P \in E(\mathbb{Q}) $,
$$ h(P + Q) \le 2h(P) + C_1. $$
\item There exists $ C_2 \in \mathbb{R} $ such that for all $ P \in E(\mathbb{Q}) $,
$$ 4h(P) \le h(2P) + C_2. $$
\item For all $ C_3 \in \mathbb{R} $, the set
$$ \{P \in E(\mathbb{Q}) : h(P) \le C_3\} $$
is finite.
\end{itemize}
To prove that an abelian group $ A $ endowed with such a function, such that $ A / 2A $ is finite, is finitely generated is an exercise in algebra.
\end{proof}

\end{frame}

\begin{frame}{Future}

Potential future projects:
\begin{itemize}
\item Generalise $ 2 $-division polynomials into $ n $-division polynomials to determine the structure of $ n $-torsion subgroups in general.
\item Explore the theory over finite fields and prove the Hasse-Weil bound.
\item Verify the correctness of Schoof's and Lenstra's algorithms.
\item Explore the theory over local fields via defining formal groups.
\item Define the classical Selmer group and the Tate-Shafarevich group with Galois cohomology of elliptic curves.
\item Define an elliptic curve as a projective scheme and reprove all results using this definition and some form of the Riemann-Roch theorem.
\item Explore the theory over global function fields.
\item Explore the complex theory to prove the uniformisation theorem and state some version of the modularity theorem.
\end{itemize}
Thank you!

\end{frame}

\end{document}