\documentclass{article}

\usepackage{amssymb}
\usepackage{amsthm}
\usepackage{commath}
\usepackage[margin=1in]{geometry}
\usepackage{enumitem}
\usepackage{tikz-cd}

\newtheorem*{example}{Example}
\newtheorem*{proposition}{Proposition}
\newtheorem*{theorem}{Theorem}

\renewcommand{\AA}{\mathbb{A}}
\newcommand{\br}{\del}
\newcommand{\Br}{\mathrm{Br}}
\newcommand{\CC}{\mathbb{C}}
\newcommand{\ch}{\mathrm{char}}
\newcommand{\E}{\mathrm{E}}
\newcommand{\et}{\mathrm{\acute{e}t}}
\newcommand{\FF}{\mathbb{F}}
\newcommand{\G}{\mathrm{G}}
\newcommand{\Gal}{\mathrm{Gal}}
\newcommand{\Gm}{\mathbb{G}_\mathrm{m}}
\renewcommand{\H}{\mathrm{H}}
\newcommand{\HH}{\mathbb{H}}
\newcommand{\Hom}{\mathrm{Hom}}
\newcommand{\inv}{\mathrm{inv}}
\newcommand{\Mat}{\mathrm{Mat}}
\newcommand{\Nm}{\mathrm{Nm}}
\newcommand{\NN}{\mathbb{N}}
\newcommand{\OOO}{\mathcal{O}}
\newcommand{\Pic}{\mathrm{Pic}}
\newcommand{\PP}{\mathbb{P}}
\newcommand{\Q}{\mathrm{Q}}
\newcommand{\QQ}{\mathbb{Q}}
\newcommand{\RR}{\mathbb{R}}
\newcommand{\res}{\mathrm{res}}
\newcommand{\s}{\mathrm{s}}
\newcommand{\Spec}{\mathrm{Spec}}
\newcommand{\V}{\mathrm{V}}
\newcommand{\ZZ}{\mathbb{Z}}

\title{Examples of Brauer groups}
\author{David Ang}
\date{Wednesday, 30 November 2022}

\begin{document}

\maketitle

\section{Brauer groups of fields}

Recall that the Brauer group $ \Br\br{K} $ of a field $ K $ can be defined in two ways.
\begin{itemize}
\item The classical Brauer group is the set of equivalence classes of central simple algebras over $ K $.
\item The cohomological Brauer group is the second Galois cohomology group $ \H^2\br{K, \Gm} $.
\end{itemize}
By Galois descent, these central simple algebras are classified by certain cocycles with projective linear coefficients, so the two Brauer groups are isomorphic, by the long exact sequence in Galois cohomology and Speiser's theorem. Both interpretations turn out to be useful for computing different examples.

\subsection{The classical Brauer group}

There are many fields with trivial Brauer groups, which are amenable to prove with the classical definition.

\begin{proposition}
Let $ K $ be a finite field. Then $ \Br\br{K} = 0 $.
\end{proposition}

\begin{proof}
It suffices to show that a central division algebra over $ K $ is $ K $ itself. A central division algebra over $ K $ is finite, but a finite division algebra is a field and the only field with centre $ K $ is $ K $.
\end{proof}

This also follows from the non-trivial fact that finite fields have cohomological dimension one.

\begin{proposition}
Let $ K $ be an algebraically closed field. Then $ \Br\br{K} = 0 $.
\end{proposition}

\begin{proof}
It suffices to show that a central division algebra over $ K $ is $ K $ itself. The algebra generated by $ K $ and any element of a central simple algebra is a finite-dimensional vector space over $ K $ and an integral domain, so it is a finite field extension of $ K $, but $ K $ does not have any finite field extensions.
\end{proof}

This follows from the fact that $ \Gal\br{K^\s / K} $ is trivial, and the same statement also holds for separably closed fields, \footnote{CTS21 Theorem 1.2.6} as well as fields of transcendence degree one over an algebraically closed field.

\begin{theorem}[Tsen]
Let $ X $ be a curve over an algebraically closed field $ K $. Then $ \Br\br{K\br{X}} = 0 $.
\end{theorem}

\begin{proof}
Poo17 Theorem 1.5.33.
\end{proof}

In fact, the statement also holds for a wider class of fields called \textbf{quasi-algebraically closed fields}. \footnote{CTS21 Theorem 1.2.10} This is a field $ K $ where any homogeneous form of degree $ d $ in greater than $ d $ variables with coefficients in $ K $ has a non-trivial zero in $ K $. For instance, finite fields are quasi-algebraically closed by the Chevalley-Warning theorem, \footnote{CTS21 Theorem 1.2.11} as well as actual algebraically closed fields by definition, but other examples include function fields of curves over an algebraically closed field \footnote{CTS21 Theorem 1.2.12} and maximal unramified extensions of non-archimedean local fields, \footnote{CTS21 Corollary 1.2.14} while local and global fields are not quasi-algebraically closed in general.

\pagebreak

\subsection{The cohomological Brauer group}

The fields with non-trivial Brauer groups are more easily computed with cohomological techniques.

\begin{proposition}
$ \Br\br{\RR} \cong \tfrac{1}{2}\ZZ / \ZZ $.
\end{proposition}

\begin{proof}
This is the first cohomology of the cyclic group $ G = \Gal\br{\CC / \RR} $ with coefficients in $ \CC^\times $, so
$$ \Br\br{\RR} = \H^2\br{G, \CC^\times} = \br{\CC^\times}^G / \Nm_{\CC / \RR}\br{\CC^\times} = \RR^\times / \RR^+ \cong \cbr{\pm}. $$
\end{proof}

Thus the only non-trivial real central simple algebra is Hamilton's quaternions $ \HH $, \footnote{CTS21 Theorem 1.2.3} which is also Frobenius's theorem. On the other hand, Hasse gave a description for the Brauer groups of the other local fields.

\begin{theorem}[Hasse]
Let $ K $ be a non-archimedean local field. Then there is an isomorphism
$$ \inv_K : \Br\br{K} \xrightarrow{\sim} \QQ / \ZZ. $$
\end{theorem}

\begin{proof}
CTS21 Corollary 3.6.3.
\end{proof}

The construction of this local invariant map is essential in the cohomological formulation of local class field theory, while global class field theory is essentially a computation of the Brauer group of a global field.

\begin{theorem}[Albert-Brauer-Hasse-Noether]
Let $ K $ be a global field. Then there is a short exact sequence
$$ 0 \to \Br\br{K} \to \bigoplus_v \Br\br{K_v} \xrightarrow{\sum_v \inv_{K_v}} \QQ / \ZZ \to 0, $$
where the direct sum ranges over all places of $ K $.
\end{theorem}

\begin{proof}
CTS21 Theorem 12.1.8.
\end{proof}

Here the map $ \Br\br{K} \to \Br\br{K_v} $ is induced covariant functorially from the natural inclusion $ K \hookrightarrow K_v $, so the main content is its injectivity. This reduces to the vanishing of the first Galois cohomology of the id\'ele class group associated to an algebraic number field, and to the vanishing of the first Galois cohomology of the Jacobian of the curve over a finite field associated to a global function field.

\section{Brauer groups of schemes}

Recall that the Brauer group of a field generalises to that of a scheme $ X $ in two ways.
\begin{itemize}
\item The Brauer-Azumaya group is the set of equivalence classes of Azumaya algebras over $ X $.
\item The Brauer-Grothendieck group is the second \'etale cohomology group $ \H_\et^2\br{X, \Gm} $.
\end{itemize}
The Galois descent argument generalises to an argument in fpqc descent for \v Cech cohomology, but in general the Brauer-Azumaya group merely injects into the Brauer-Grothendieck group. \footnote{CTS21 Theorem 3.3.1(iii)}
\begin{itemize}
\item If $ X $ has finitely many connected components, then the Brauer-Azumaya group is torsion. \footnote{Poo17 Theorem 6.6.17(ii)}
\item If $ X $ is quasi-compact and separated with an ample invertible sheaf, so has finitely many connected components, then Gabber proved that this injection has image precisely the torsion subgroup. \footnote{CTS21 Theorem 3.3.2}
\item If $ X $ is integral and geometrically locally factorial, then the Brauer-Grothendieck group is torsion. \footnote{CTS21 Theorem 3.5.2}
\end{itemize}
For instance, $ X $ can be quasi-projective over an affine scheme. The former interpretation is great for constructing explicit elements in the Brauer group via quaternion algebras, while the latter interpretation allows for the machinery of spectral sequences to compute the structure of the Brauer group abstractly, but both involve reducing the computation to the Brauer group $ \Br\br{K} = \Br\br{\Spec\br{K}} $ of some field $ K $.

\pagebreak

\subsection{The Brauer-Azumaya group}

If $ X $ is integral, its generic point is the spectrum of its function field $ K\br{X} $, so the natural inclusion $ \Spec\br{K\br{X}} \hookrightarrow X $ induces a contravariant functorial map $ \Br\br{X} \to \Br\br{K\br{X}} $. If $ X $ is also geometrically locally factorial, then this map is also injective, \footnote{CTS21 Theorem 3.5.4} and its image can be characterised by certain residue homomorphisms under further smoothness conditions, as a consequence of Gabber's absolute purity theorem.

\begin{theorem}[Purity]
Let $ X $ be a regular proper variety over a field $ K $ of $ \ch\br{K} = 0 $. Then there is an exact sequence
$$ 0 \to \Br\br{X} \to \Br\br{K\br{X}} \xrightarrow{\bigoplus_D \res_D} \bigoplus_D \H^1\br{\kappa\br{D}, \QQ / \ZZ}, $$
where the direct sum ranges over all prime divisors $ D $ of $ X $, and $ \kappa\br{D} $ is the residue field at $ D $.
\end{theorem}

\begin{proof}
CTS21 Corollary 3.7.3.
\end{proof}

For the purposes of an example, it suffices to define the residue homomorphisms $ \res_D : \Br\br{K\br{X}} \to \H^1\br{\kappa\br{D}, \QQ / \ZZ} $ at $ D $ for degree two central simple algebras called \textbf{quaternion algebras}. More explicitly, over a field $ K $ of $ \ch\br{K} \ne 2 $, this is a four-dimensional associative $ K $-algebra $ \Q_K\br{a, b} $ with basis $ 1 $, $ i $, $ j $, and $ ij $ such that $ i^2 = a $, $ j^2 = b $, and $ ij = -ij $ for some $ a, b \in K^\times $. The typical example is the usual quaternions $ \HH $, which would be $ \Q_\RR\br{-1, -1} $ under this notation. Before defining its residues, it is worth remarking that any $ \Q_K\br{a, b} $ is associated to a unique projective plane conic $ ax^2 + by^2 = z^2 $ over $ K $. Its class in $ \Br\br{K} $ is trivial precisely when this conic has a $ K $-rational point, and hence the triviality of the Hilbert symbol $ \br{a, b}_K $, in which case $ \Q_K\br{a, b} $ is said to be \textbf{split}. \footnote{CTS21 Proposition 1.1.7} Now since the class of $ \Q_K\br{a, b} $ is $ 2 $-torsion in $ \Br\br{K} $, its residue at $ D $ lies in $ \H^1\br{\kappa\br{D}, \tfrac{1}{2}\ZZ / \ZZ} \cong \kappa\br{D}^\times / \kappa\br{D}^{\times2} $, by Hilbert's theorem 90. In fact, \footnote{CTS21 Formula 1.16}
$$ \res_D\br{\Q_K\br{a, b}} = \sbr{\br{-1}^{\nu_D\br{a}\nu_D\br{b}}\dfrac{b^{\nu_D\br{a}}}{a^{\nu_D\br{b}}}}, $$
where $ \nu_D : K\br{X}^\times \to \ZZ $ is a valuation associated to $ D $. If the residue of $ \Q_K\br{a, b} $ is trivial at all prime divisors of $ X $, it is said to be \textbf{unramified}, and hence arises from some Azumaya algebra in $ \Br\br{X} $ by purity.

\begin{example}[Reichardt-Lind]
Let $ X $ be the projective closure of the smooth affine curve
$$ ay^2 = x^4 + b, \qquad a, b \in K^\times, $$
where $ K $ is a field of $ \ch\br{K} \ne 2 $. Then $ \Q_K\br{y, b} $ is unramified.
\end{example}

\begin{proof}
Let $ \nu_D : K\br{X}^\times \to \ZZ $ be a valuation associated to a prime divisor $ D $ of $ X $, so that
$$ \res_D\br{\Q_K\br{y, b}} = \sbr{\br{-1}^{\nu_D\br{y}\nu_D\br{b}}\dfrac{b^{\nu_D\br{y}}}{y^{\nu_D\br{b}}}} = \sbr{b^{\nu_D\br{y}}}. $$
If $ b $ is a quadratic residue in $ \kappa\br{D}^\times $, then $ \res_D\br{\Q_K\br{y, b}} $ is trivial. Otherwise observe that
$$ 2\nu_D\br{y} = \nu_D\br{y^2} = \nu_D\br{ay^2} = \nu_D\br{x^4 - b}. $$
If $ \nu_D\br{x} < 0 $, then $ \nu_D\br{x^4 - b} = \nu_D\br{x^4} = 4\nu_D\br{x} $, so $ \nu_D\br{y} = 2\nu_D\br{x} $ is even, and hence $ \res_D\br{\Q_K\br{y, b}} $ is trivial. Otherwise $ \nu_D\br{x} \ge 0 $, so $ \nu_D\br{x^4 - b} \ge 0 $. If $ \nu_D\br{x^4 - b} > 0 $, then $ x^4 = b $ in $ \kappa\br{D}^\times $, which is a contradiction to $ b $ being a quadratic non-residue in $ \kappa\br{D}^\times $. Otherwise $ \nu_D\br{x^4 - b} = 0 $, so $ \nu_D\br{y} = 0 $, and hence $ \res_D\br{\Q_K\br{y, b}} $ is also trivial. Thus $ \res_D\br{\Q_K\br{y, b}} $ is always trivial.
\end{proof}

In fact, $ \Q_\QQ\br{y, -17} $ is a Brauer-Manin obstruction to the Hasse principle for the curve $ 2y^2 = x^4 - 17 $, which in turn represents a non-trivial element of the Tate-Shafarevich group of the elliptic curve $ y^2 = x^3 + 17x $.

\pagebreak

\subsection{The Brauer-Grothendieck group}

The purity theorem is typically proven via the cohomological machinery known as \textbf{spectral sequences}. Informally, this is a sequence of bigraded groups $ \cbr{E_r^{p, q}}_{p, q \in \NN} $ indexed by $ r \in \NN $ satisfying certain homological compatibility conditions, with a notion of convergence to a group $ G^n $ indexed by $ n \in \NN $ denoted $ E_r^{p, q} \implies G^n $. For the purposes of an example, it suffices to know that $ E_2^{p, q} \implies G^{p + q} $ implies the exactness of the sequence
$$ 0 \to E_2^{1, 0} \to G^1 \to E_2^{0, 1} \to E_2^{2, 0} \to \ker\br{G^2 \to E_2^{0, 2}} \to E_2^{1, 1} \to E_2^{3, 0}. $$
This fact is then applied to a special case of the Grothendieck spectral sequence for \'etale sites.

\begin{theorem}[Leray]
Let $ X $ be a variety over a field $ K $. Then there is a spectral sequence
$$ \H^p\br{K, \H_\et^q\br{X^\s, \Gm}} \implies \H_\et^{p + q}\br{X, \Gm}, $$
where $ X^\s $ is the base change of $ X $ to a separable closure of $ K $.
\end{theorem}

\begin{proof}
CTS21 Formula 4.7.
\end{proof}

Now by construction, $ \H_\et^0\br{X, \Gm} $ is simply the group of invertible global sections $ \OOO\br{X}^\times $ of $ X $, while $ \H_\et^1\br{X, \Gm} $ can be shown to be isomorphic to the Picard group $ \Pic\br{X} $ of equivalence classes of invertible sheaves on $ X $. \footnote{Poo17 Proposition 6.6.1} Thus the Leray spectral sequence induces an exact sequence ending with
$$ \dots \to \H^2\br{K, \OOO\br{X^\s}^\times} \to \ker\br{\Br\br{X} \xrightarrow{\phi} \Br\br{X^\s}} \to \H^1\br{K, \Pic\br{X^\s}} \to \H^3\br{K, \OOO\br{X^\s}^\times}. $$
Here the natural map $ \phi : \Br\br{X} \to \Br\br{X^\s} $ is induced contravariant functorially from the natural projection $ X^\s \to X $, and its kernel is called the \textbf{algebraic Brauer group}, often denoted by $ \Br_1\br{X} $.

\begin{example}
Let $ X $ be the affine line $ \AA_K^1 $ or the projective line $ \PP_K^1 $ over a perfect field $ K $. Then
$$ \Br\br{X} \cong \Br\br{K}. $$
\end{example}

\begin{proof}
The regular functions on $ \PP_{K^\s}^1 $ are the constants, while the regular functions on $ \AA_{K^\s}^1 $ are the polynomials in one variable, but in both cases the invertible ones are both the non-zero constants, so
$$ \H^2\br{K, \OOO\br{X^\s}^\times} = \H^2\br{K, \Gm} = \Br\br{K}. $$
Since $ X^\s $ is integral and geometrically locally factorial, $ \Br\br{X^\s} $ injects into $ \Br\br{K^\s\br{X}} $, which is trivial by Tsen's theorem, since $ K^\s $ is algebraically closed by perfectness, so $ \Br_1\br{X} = \Br\br{X} $ and the sequence
$$ \Br\br{K} \xrightarrow{\phi} \Br\br{X} \to \H^1\br{K, \Pic\br{X^\s}} $$
is exact. Now the existence of a $ K $-point in $ X $ induces a map $ \Spec\br{K} \to X $ and hence a contravariant functorial map $ \Br\br{X} \to \Br\br{K} $, which is a retraction to $ \phi $ by construction, so $ \phi $ is injective. Finally, there are no invertible sheaves on $ \AA_{K^\s}^1 $ since its regular functions form a principal ideal domain, while the invertible sheaves on $ \PP_{K^\s}^1 $ are indexed by $ \ZZ $ via the degree function, but in both cases
$$ \H^1\br{K, \Pic\br{X^\s}} = \Hom\br{\Gal\br{K^\s / K}, \Pic\br{X^\s}} = 0, $$
so $ \phi $ is also surjective.
\end{proof}

In fact, by refining the same argument with a generalisation of Tsen's theorem, it can be shown that $ \Br\br{\PP_K^n} \cong \Br\br{K} $ for any field $ K $, \footnote{CTS21 Theorem 5.1.3} but $ \Br\br{\AA_K^n} \cong \Br\br{K} $ only holds for fields $ K $ of $ \ch\br{K} = 0 $. \footnote{CTS21 Theorem 5.1.1}

\section*{References}

\begin{itemize}[leftmargin=0.5in]
\item[CTS21.] J-L Colliot-Th\'el\`ene and A Skorobogatov (2021) \emph{The Brauer-Grothendieck group}
\item[Poo17.] B Poonen (2017) \emph{Rational points on varieties}
\end{itemize}

\end{document}