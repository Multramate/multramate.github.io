\documentclass[10pt]{beamer}

\setbeamertemplate{footline}[page number]

\usepackage{color}
\usepackage{listings}
\usepackage{tikz-cd}

\definecolor{keywordcolor}{rgb}{0.7, 0.1, 0.1}
\definecolor{tacticcolor}{rgb}{0.0, 0.1, 0.6}
\definecolor{commentcolor}{rgb}{0.4, 0.4, 0.4}
\definecolor{symbolcolor}{rgb}{0.0, 0.1, 0.6}
\definecolor{sortcolor}{rgb}{0.1, 0.5, 0.1}
\definecolor{attributecolor}{rgb}{0.7, 0.1, 0.1}
\def\lstlanguagefiles{lstlean.tex}
\lstset{language=lean}

\begin{document}

\begin{frame}

\begin{center}

{\scriptsize London School of Geometry and Number Theory}

\vspace{0.5cm}

{\small London Learning Lean}

\vspace{1cm}

\textbf{\large Elliptic curves and the Mordell-Weil theorem}

\vspace{1cm}

David Ang

\vspace{0.5cm}

{\footnotesize Thursday, 26 May 2022}

\end{center}

\end{frame}

\begin{frame}{Overview}

\begin{itemize}
\item Introduction
\item Abstract definition
\item Concrete definition
\item Implementation
\item Associativity
\item The Mordell-Weil theorem
\item Selmer groups
\item Future
\end{itemize}

\end{frame}

\begin{frame}[t]{Introduction --- informally}

What are elliptic curves?

\vspace{0.5cm}

\only<2->{
\begin{itemize}
\item A curve --- solutions to $ y^2 = x^3 + Ax + B $ for fixed $ A $ and $ B $.
\end{itemize}
}

\only<3->{
\begin{center}
\includegraphics[width=0.4\textwidth]{ellipticcurves.png}
\end{center}
}

\only<4>{
\begin{itemize}
\item A group --- notion of addition of points!
\end{itemize}
}

\end{frame}

\begin{frame}[t]{Introduction --- applications}

Why do we care?

\vspace{0.5cm}

\only<2->{Make or break cryptography.
\begin{itemize}
\item<3-> Lenstra's integer factorisation algorithm (RSA).
\item<4-> Discrete logarithm problem --- solve $ nQ = P $ given $ P $ and $ Q $ (DH).
\item<5-> Post-quantum cryptography (SIDH).
\end{itemize}
}

\vspace{0.5cm}

\only<6->{Number theory and algebraic geometry.
\begin{itemize}
\item<7-> The simplest non-trivial objects in algebraic geometry.
\begin{itemize}
\item Abelian variety of dimension one, projective curve of genus one, etc...
\end{itemize}
\item<8-> Rational elliptic curve associated to $ a^p + b^p = c^p $ is not modular.
\begin{itemize}
\item But modularity theorem --- rational elliptic curves are modular!
\end{itemize}
\item<9> Distribution of ranks of rational elliptic curves.
\begin{itemize}
\item The BSD conjecture --- analytic rank equals algebraic rank?
\end{itemize}
\end{itemize}
}

\end{frame}

\begin{frame}[t]{Abstract definition --- globally}

An \textbf{elliptic curve $ E $ over a scheme $ S $} is a diagram
$$
\begin{tikzcd}[row sep=small]
E \arrow{d}[swap]{f} \\
S \only<2->{\arrow[bend right=90]{u}[swap]{0}}
\end{tikzcd}
$$
\only<3->{with a few technical conditions. \footnote{$ f $ is smooth, proper, and all its geometric fibres are integral curves of genus one}}

\vspace{0.5cm}

\only<4->{For a scheme $ T $ over $ S $,}
\only<5->{define the \textbf{set of $ T $-points} of $ E $ by $$ E(T) := \mathrm{Hom}_S(T, E), $$}
\only<6->{which is naturally identified with a \textbf{Picard group} $ \mathrm{Pic}_{E / S}^0(T) $ of $ E $.}

\vspace{0.5cm}

\only<7->{This defines a contravariant functor $ \textbf{Sch}_S \to \textbf{Ab} $ given by $ T \mapsto E(T) $.}

\vspace{0.5cm}

\only<8>{Good for algebraic geometry, but not very friendly...}

\end{frame}

\begin{frame}[t]{Abstract definition --- locally}

Let $ S = \mathrm{Spec} \ F $ and $ T = \mathrm{Spec} \ K $ for a field extension $ K / F $. \footnote{or even a ring extension $ K / F $ whose class group has no $ 12 $-torsion}

\vspace{0.5cm}

\only<2->{An \textbf{elliptic curve $ E $ over a field $ F $} is a tuple $ (E, 0) $.
\begin{itemize}
\item<3-> $ E $ is a nice \only<3->{\footnote{smooth, proper, and geometrically integral}} genus one curve over $ F $.
\item<4-> $ 0 $ is an $ F $-point.
\end{itemize}
}

\vspace{0.5cm}

\only<5->{The Picard group is $$ \mathrm{Pic}_{E / F}^0(K) = \dfrac{\{\text{degree zero divisors of} \ E \ \text{over} \ K\}}{\{\text{principal divisors of} \ E \ \text{over} \ K\}}. $$}

\only<6->{This defines a covariant functor $ \textbf{Alg}_F \to \textbf{Ab} $ given by $ K \mapsto E(K) $.}

\vspace{0.5cm}

\only<7>{Group law is free, but still need equations...}

\end{frame}

\begin{frame}[t]{Concrete definition -- Weierstrass equations}

The Riemann-Roch theorem gives \textbf{Weierstrass equations}.

\vspace{0.5cm}

\only<2->{
\begin{corollary}[of Riemann-Roch]
An elliptic curve $ E $ over a field $ F $ is a projective plane curve $$ Y^2Z + a_1XYZ + a_3YZ^2 = X^3 + a_2X^2Z + a_4XZ^2 + a_6Z^3, \qquad a_i \in F, $$
\only<3->{with $ \Delta \ne 0 $. \footnote{\tiny $ \Delta := -(a_1^2 + 4a_2)^2(a_1^2a_6 + 4a_2a_6 - a_1a_3a_4 + a_2a_3^2 - a_4^2) - 8(2a_4 + a_1a_3)^3 - 27(a_3^2 + 4a_6)^2 + 9(a_1^2 + 4a_2)(2a_4 + a_1a_3)(a_3^2 + 4a_6) $}}
\end{corollary}
}

\vspace{0.5cm}

\only<4->{If $ \mathrm{char} \ F \ne 2, 3 $, can reduce this to $$ Y^2Z = X^3 + AXZ^2 + BZ^3, \qquad A, B \in F, $$}
\only<5->{with $ \Delta := 4A^3 + 27B^2 \ne 0 $.}

\vspace{0.5cm}

\only<6>{Note the unique \textbf{point at infinity} when $ Z = 0 $! Call this point $ 0 $.}

\end{frame}

\begin{frame}[t]{Concrete definition --- group law}

The \textbf{group law} from $ E(K) \cong \mathrm{Pic}_{E / F}^0(K) $ is reduced to drawing lines.

\vspace{0.5cm}

\only<2->{Operations are characterised by $$ P + Q + R = 0 \qquad \iff \qquad P, Q, R \ \text{are collinear}. $$}

\only<3->{
\begin{center}
\includegraphics[width=0.7\textwidth]{grouplaw.png}
\end{center}
}

\only<4->{Note that $ (x, y) \in E[2] := \ker (E \xrightarrow{\cdot 2} E) $ if and only if $ y = 0 $. \footnote{Assume $ a_1 = a_3 = 0 $.}}

\vspace{0.5cm}

\only<5>{Many cases... but all completely explicit!}

\end{frame}

\begin{frame}[fragile, t]{Implementation --- the curve}

Three definitions of elliptic curves:
\begin{enumerate}
\item Abstract definition over a scheme
\item Abstract definition over a field
\item Concrete definition over a field
\end{enumerate}
Generality: $ 1. \supset 2. \overset{\text{RR}}{=} 3. $
\only<2->{
\begin{itemize}
\item<2-> $ 1. \ \& \ 2. $ require much algebraic geometry (properness, genus, ...).
\item<3-> $ 2. = 3. $ also requires algebraic geometry (divisors, differentials, ...).
\item<4-> $ 3. $ requires just five coefficients (and $ \Delta \ne 0 $)!
\end{itemize}
}

\begin{onlyenv}<5->
\begin{lstlisting}[basicstyle=\scriptsize, frame=single]
def disc_aux {R : Type} [comm_ring R] (a₁ a₂ a₃ a₄ a₆ : R) : R :=
  -(a₁^2 + 4*a₂)^2*(a₁^2*a₆ + 4*a₂*a₆ - a₁*a₃*a₄ + a₂*a₃^2 - a₄^2)
  - 8*(2*a₄ + a₁*a₃)^3 - 27*(a₃^2 + 4*a₆)^2
  + 9*(a₁^2 + 4*a₂)*(2*a₄ + a₁*a₃)*(a₃^2 + 4*a₆)

structure EllipticCurve (R : Type) [comm_ring R] :=
  (a₁ a₂ a₃ a₄ a₆ : R) (disc : units R) (disc_eq : disc.val = disc_aux a₁ a₂ a₃ a₄ a₆)
\end{lstlisting}
\end{onlyenv}

\only<6>{This is the \emph{curve} $ E $ --- what about the \emph{group} $ E(K) $?}

\end{frame}

\begin{frame}[fragile, t]{Implementation --- the group}

\begin{lstlisting}[basicstyle=\scriptsize, frame=single]
variables {F : Type} [field F] (E : EllipticCurve F) (K : Type) [field K] [algebra F K]

inductive point
  | zero
  | some (x y : K) (w : y^2 + E.a₁*x*y + E.a₃*y = x^3 + E.a₂*x^2 + E.a₄*x + E.a₆)

notation E(K) := point E K
\end{lstlisting}

\only<2-3>{
\begin{itemize}
\item Identity is trivial!
\end{itemize}
}

\begin{onlyenv}<2-3>
\begin{lstlisting}[basicstyle=\scriptsize, frame=single]
instance : has_zero E(K) := ⟨zero⟩
\end{lstlisting}
\end{onlyenv}

\only<3>{
\begin{itemize}
\item Negation is easy.
\end{itemize}
}

\begin{onlyenv}<3>
\begin{lstlisting}[basicstyle=\scriptsize, frame=single]
def neg : E(K) → E(K)
  | zero := zero
  | (some x y w) := some x (-y - E.a₁*x - E.a₃) $
    begin
      rw [← w],
      ring
    end

instance : has_neg E(K) := ⟨neg⟩
\end{lstlisting}
\end{onlyenv}

\only<4>{
\begin{itemize}
\item Addition is complicated...
\end{itemize}
}

\begin{onlyenv}<4>
\begin{lstlisting}[basicstyle=\scriptsize, frame=single]
def add : E(K) → E(K) → E(K)
  | zero P := P
  | P zero := P
  | (some x₁ y₁ w₁) (some x₂ y₂ w₂) :=
    if x_ne : x₁ ≠ x₂ then                                     -- add distinct points
      let L := (y₁ - y₂) / (x₁ - x₂),
          x₃ := L^2 + E.a₁*L - E.a₂ - x₁ - x₂,
          y₃ := -L*x₃ - E.a₁*x₃ - y₁ + L*x₁ - E.a₃
      in some x₃ y₃ $ by { ... }
    else if y_ne : y₁ + y₂ + E.a₁*x₂ + E.a₃ ≠ 0 then -- double a point
      ...
    else                                                             -- draw vertical line
      zero

instance : has_add E(K) := ⟨add⟩
\end{lstlisting}
\end{onlyenv}

\only<5->{
\begin{itemize}
\item Commutativity is... doable.
\end{itemize}
}

\begin{onlyenv}<5->
\begin{lstlisting}[basicstyle=\scriptsize, frame=single]
lemma add_comm (P Q : E(K)) : P + Q = Q + P :=
  begin
    rcases ⟨P, Q⟩ with ⟨_ | _, _ | _⟩,
    ... -- six cases
  end
\end{lstlisting}
\end{onlyenv}

\only<6>{
\begin{itemize}
\item Associativity is... impossible?
\end{itemize}
}

\begin{onlyenv}<6>
\begin{lstlisting}[basicstyle=\scriptsize, frame=single]
lemma add_assoc (P Q R : E(K)) : (P + Q) + R = P + (Q + R) :=
  begin
    rcases ⟨P, Q, R⟩ with ⟨_ | _, _ | _, _ | _⟩,
    ... -- ??? cases
  end
\end{lstlisting}
\end{onlyenv}

\end{frame}

\begin{frame}[t]{Associativity --- explaining the problem}

Known to be difficult with several proofs:
\only<2->{
\begin{itemize}
\item<2-> Just do it!
\begin{itemize}
\item Probably(?) times out with 130,000(!) coefficients.
\end{itemize}
\item<3-> Uniformisation.
\begin{itemize}
\item Requires theory of elliptic functions.
\end{itemize}
\item<4-> Cayley-Bacharach.
\begin{itemize}
\item Requires intersection multiplicity and B\'ezout's theorem.
\end{itemize}
\item<5-> $ E(K) \cong \mathrm{Pic}_{E / F}^0(K) $.
\begin{itemize}
\item Requires divisors, differentials, and the Riemann-Roch theorem.
\end{itemize}
\end{itemize}
}

\vspace{0.5cm}

\only<6->{Current status:
\begin{itemize}
\item<6-> Left as a \texttt{sorry}.
\item<7-> Ongoing attempt (by Marc Masdeu) to bash it out.
\item<8-> Proof in Coq (by Evmorfia-Iro Bartzia and Pierre-Yves Strub \only<8->{\footnote{A Formal Library for Elliptic Curves in the Coq Proof Assistant (2015)}}) \\
that $ E(K) \cong \mathrm{Pic}_{E / F}^0(K) $ but only for $ \mathrm{char} \ F \ne 2, 3 $.
\end{itemize}
}

\end{frame}

\begin{frame}[fragile, t]{Associativity --- ignoring the problem}

Modulo associativity, what has been done?

\only<2-3>{
\begin{itemize}
\item Functoriality $ \textbf{Alg}_F \to \textbf{Ab} $.
\end{itemize}
}

\begin{onlyenv}<2-3>
\begin{lstlisting}[basicstyle=\scriptsize, frame=single]
def point_hom (φ : K →ₐ[F] L) : E(K) → E(L)
  | zero := zero
  | (some x y w) := some (φ x) (φ y) $ by { ... }

lemma point_hom.id (P : E(K)) : point_hom (K→[F]K) P = P

lemma point_hom.comp (P : E(K)) :
  point_hom (L→[F]M) (point_hom (K→[F]L) P) = point_hom ((L→[F]M).comp (K→[F]L)) P
\end{lstlisting}
\end{onlyenv}

\only<3>{
\begin{itemize}
\item Galois module structure $ \mathrm{Gal}(L/K) \curvearrowright E(L) $.
\end{itemize}
}

\begin{onlyenv}<3>
\begin{lstlisting}[basicstyle=\scriptsize, frame=single]
def point_gal (σ : L ≃ₐ[K] L) : E(L) → E(L)
  | zero := zero
  | (some x y w) := some (σ • x) (σ • y) $ by { ... }

variables [finite_dimensional K L] [is_galois K L]

lemma point_gal.fixed :
  mul_action.fixed_points (L ≃ₐ[K] L) E(L) = (point_hom (K→[F]L)).range
\end{lstlisting}
\end{onlyenv}

\only<4>{
\begin{itemize}
\item Isomorphisms $ (x, y) \mapsto (u^2x + r, u^3y + u^2sx + t) $.
\end{itemize}
}

\begin{onlyenv}<4>
\begin{lstlisting}[basicstyle=\scriptsize, frame=single]
variables (u : units F) (r s t : F)

def cov : EllipticCurve F :=
{ a₁ := u.inv*(E.a₁ + 2*s),
  a₂ := u.inv^2*(E.a₂ - s*E.a₁ + 3*r - s^2),
  a₃ := u.inv^3*(E.a₃ + r*E.a₁ + 2*t),
  a₄ := u.inv^4*(E.a₄ - s*E.a₃ + 2*r*E.a₂ - (t + r*s)*E.a₁ + 3*r^2 - 2*s*t),
  a₆ := u.inv^6*(E.a₆ + r*E.a₄ + r^2*E.a₂ + r^3 - t*E.a₃ - t^2 - r*t*E.a₁),
  disc := ⟨u.inv^12*E.disc.val, u.val^12*E.disc.inv, by { ... }, by { ... }⟩,
  disc_eq := by { simp only, rw [disc_eq, disc_aux, disc_aux], ring } }

def cov.to_fun : (E.cov u r s t)(K) → E(K)
  | zero := zero
  | (some x y w) := some (u.val^2*x + r) (u.val^3*y + u.val^2*s*x + t) $ by { ... }

def cov.inv_fun : E(K) → (E.cov u r s t)(K)
  | zero := zero
  | (some x y w) := some (u.inv^2*(x - r)) (u.inv^3*(y - s*x + r*s - t)) $ by { ... }

def cov.equiv_add : (E.cov u r s t)(K) ≃+ E(K) :=
  ⟨cov.to_fun u r s t, cov.inv_fun u r s t, by { ... }, by { ... }, by { ... }⟩
\end{lstlisting}
\end{onlyenv}

\only<5-6>{
\begin{itemize}
\item $ 2 $-division polynomial $ \psi_2(x) $.
\end{itemize}
}

\begin{onlyenv}<5-6>
\begin{lstlisting}[basicstyle=\scriptsize, frame=single]
def ψ₂_x : cubic K := ⟨4, E.a₁^2 + 4*E.a₂, 4*E.a₄ + 2*E.a₁*E.a₃, E.a₃^2 + 4*E.a₆⟩

lemma ψ₂_x.disc_eq_disc : (ψ₂_x E K).disc = 16*E.disc
\end{lstlisting}
\end{onlyenv}

\only<6>{
\begin{itemize}
\item Structure of $ E(K)[2] $.
\end{itemize}
}

\begin{onlyenv}<6>
\begin{lstlisting}[basicstyle=\scriptsize, frame=single]
notation E(K)[n] := ((•) n : E(K) →+ E(K)).ker

lemma E₂.x {x y w} : some x y w ∈ E(K)[2] ↔ x ∈ (ψ₂_x E K).roots

theorem E₂.card_le_four : fintype.card E(K)[2] ≤ 4

variables [algebra ((ψ₂_x E F).splitting_field) K]

theorem E₂.card_eq_four : fintype.card E(K)[2] = 4

lemma E₂.gal_fixed (σ : L ≃ₐ[K] L) (P : E(L)[2]) : σ • P = P
\end{lstlisting}
\end{onlyenv}

\end{frame}

\begin{frame}[t]{The Mordell-Weil theorem --- statement and proof}

\begin{theorem}[Mordell-Weil]
Let $ K $ be a number field. Then $ E(K) $ is finitely generated.
\end{theorem}

\only<2->{By the structure theorem (Pierre-Alexandre Bazin),
$$ E(K) \cong T \oplus \mathbb{Z}^r. $$
\vspace{-0.5cm}
\begin{itemize}
\item<3-> $ T $ is a finite \textbf{torsion subgroup}.
\item<4-> $ r \in \mathbb{N} $ is the \textbf{algebraic rank}.
\end{itemize}
}

\only<5->{
\begin{proof}
\renewcommand{\qedsymbol}{}
Three steps.
\begin{itemize}
\item<6-> \textbf{Weak Mordell-Weil}: $ E(K) / 2E(K) $ is finite.
\item<7-> \textbf{Heights}: $ E(K) $ can be endowed with a ``height function".
\item<8-> \textbf{Descent}: An abelian group $ A $ endowed with a ``height function", such that $ A / 2A $ is finite, is necessarily finitely generated. $ \square $
\end{itemize}
\end{proof}
}

\only<9>{\vspace{-0.5cm} The descent step is done (Jujian Zhang).}

\end{frame}

\begin{frame}[fragile, t]{The Mordell-Weil theorem --- weak Mordell-Weil}

Prove that $ E(K) / 2E(K) $ is finite with \textbf{complete $ 2 $-descent}.

\only<2-5>{\vspace{-0.5cm}}

\only<2->{$$ E(K) = \{(x, y) \mid y^2 \only<2-5>{+ a_1xy + a_3y} = \only<2-10>{x^3 + a_2x^2 + a_4x + a_6} \only<11->{(x - e_1)(x - e_2)(x - e_3)}\} \cup \{0\} $$}

\only<3->{
\begin{itemize}
\item Reduce to $ a_1 = a_3 = 0 $.

\only<4-6>{\vspace{0.5cm} Completing the square is an isomorphism
$$
\begin{array}{rcl}
E(K) & \longrightarrow & E'(K) \\
(x, y) & \longmapsto & (x, y - \tfrac{1}{2}a_1x -\tfrac{1}{2}a_3)
\end{array}.
$$
\visible<5-6>{Thus $$ E(K) / 2E(K) \ \text{finite} \qquad \iff \qquad E'(K) / 2E'(K) \ \text{finite}. $$} \vspace{-0.5cm}}
\end{itemize}
}

\begin{onlyenv}<4-6>
\begin{lstlisting}[basicstyle=\scriptsize, frame=single]
def covₘ.equiv_add : (E.cov _ _ _ _)(K) ≃+ E(K) := cov.equiv_add 1 0 (-E.a₁/2) (-E.a₃/2)
\end{lstlisting}
\end{onlyenv}

\only<7->{
\begin{itemize}
\item Reduce to $ E[2] \subset E(K) $.

\only<8-11>{\vspace{0.5cm} Let $ L = K(E[2]) $. Suffices to show $$ E(L) / 2E(L) \ \text{finite} \qquad \implies \qquad E(K) / 2E(K) \ \text{finite}. $$}
\only<9-11>{Suffices to show finiteness of $$ \Phi := \ker (E(K) / 2E(K) \to E(L) / 2E(L)). $$}
\only<10-11>{Define an injection $$ \kappa : \Phi \hookrightarrow \mathrm{Hom}(\mathrm{Gal}(L / K), E(L)[2]). $$}
\end{itemize}
}

\begin{onlyenv}<12>
\begin{lstlisting}[basicstyle=\scriptsize, frame=single]
variables [finite_dimensional K L] [is_galois K L] (n : ℕ)

lemma range_le_comap_range : n⬝E(K) ≤ add_subgroup.comap (point_hom _) n⬝E(L)

def Φ : add_subgroup E(K)/n :=
  (quotient_add_group.map _ _ _ $ range_le_comap_range n).ker

lemma Φ_mem_range (P : Φ n E L) : point_hom _ P.val.out' ∈ n⬝E(L)

def κ : Φ n E L → L ≃ₐ[K] L → E(L)[n] :=
  λ P σ, ⟨σ • (Φ_mem_range n P).some - (Φ_mem_range n P).some, by { ... }⟩

lemma κ.injective : function.injective $ κ n

def coker_2_of_fg_extension.fintype : fintype E(L)/2 → fintype E(K)/2
\end{lstlisting}
\end{onlyenv}

\only<13->{
\begin{itemize}
\item Define a \textbf{complete $ 2 $-descent} homomorphism $$ \delta \ : \ E(K) \quad \longrightarrow \quad K^\times / (K^\times)^2 \times K^\times / (K^\times)^2. \qquad $$

\only<14-17>{Map: \vspace{-0.4cm}
$$
\begin{array}{rcccccc}
\visible<14-17>{0 & \longmapsto & ( & 1 & , & 1 & ) \\}
\visible<15-17>{(x, y) & \longmapsto & ( & x - e_1 & , & x - e_2 & ) \\}
\visible<16-17>{(e_1, 0) & \longmapsto & ( & \dfrac{e_1 - e_3}{e_1 - e_2} & , & e_1 - e_2 & ) \\}
\visible<17>{(e_2, 0) & \longmapsto & ( & e_2 - e_1 & , & \dfrac{e_2 - e_3}{e_2 - e_1} & )}
\end{array}
$$
}
\end{itemize}
}

\begin{onlyenv}<18>
\begin{lstlisting}[basicstyle=\scriptsize, frame=single]
variables (ha₁ : E.a₁ = 0) (ha₃ : E.a₃ = 0) (h3 : (ψ₂_x E K).roots = {e₁, e₂, e₃})

def δ : E(K) → (units K) / (units K)^2 × (units K) / (units K)^2
  | zero := 1
  | (some x y w) :=
    if he₁ : x = e₁ then
      (units.mk0 ((e₁ - e₃) / (e₁ - e₂)) $ by { ... }, units.mk0 (e₁ - e₂) $ by { ... })
    else if he₂ : x = e₂ then
      (units.mk0 (e₂ - e₁) $ by { ... }, units.mk0 ((e₂ - e₃) / (e₂ - e₁)) $ by { ... })
    else
      (units.mk0 (x - e₁) $ by { ... }, units.mk0 (x - e₂) $ by { ... })
\end{lstlisting}
\end{onlyenv}

\only<19->{
\begin{itemize}
\item Prove $ \ker \delta = 2E(K) $.

\only<20>{\vspace{0.5cm} Here $ \supseteq $ is obvious, while $ \subseteq $ is long but constructive.}
\end{itemize}
}

\begin{onlyenv}<20>
\begin{lstlisting}[basicstyle=\scriptsize, frame=single]
lemma δ.ker : (δ ha₁ ha₃ h3).ker = 2⬝E(K) :=
  begin
    ... -- completely constructive proof
  end
\end{lstlisting}
\end{onlyenv}

\only<21->{
\begin{itemize}
\item Prove $ \mathrm{im} \ \delta \le K(S, 2) \times K(S, 2) $ for some $ K(S, 2) \le K^\times / (K^\times)^2 $.

\only<22>{\vspace{0.5cm} Here $ S $ is a finite set of ``ramified" places of $ K $.}
\end{itemize}
}

\begin{onlyenv}<22>
\begin{lstlisting}[basicstyle=\scriptsize, frame=single]
lemma δ.range_le : (δ ha₁ ha₃ h3).range ≤ K(S, 2) × K(S, 2) := sorry -- ramification theory?
\end{lstlisting}
\end{onlyenv}

\end{frame}

\begin{frame}[fragile, t]{Interlude --- Selmer groups}

Let $ S $ be a finite set of places of $ K $.
\only<2->{The \textbf{$ n $-Selmer group} of $ K $ is $$ K(S, n) := \{x(K^\times)^n \in K^\times / (K^\times)^n \mid \forall p \notin S, \ \mathrm{ord}_p(x) \equiv 0 \mod n\}. $$}
\only<3->{Claim that $ K(S, n) $ is finite.}

\only<4->{
\begin{itemize}
\item Reduce to $ K(\emptyset, n) $.

\only<5-6>{\vspace{0.5cm} There is a homomorphism
$$
\begin{array}{rcl}
K(S, n) & \longrightarrow & (\mathbb{Z} / n\mathbb{Z})^{|S|} \\
x(K^\times)^n & \longmapsto & (\mathrm{ord}_p(x))_{p \in S}
\end{array},
$$
with kernel $ K(\emptyset, n) $.}
\only<6>{Thus $$ K(S, n) \ \text{finite} \qquad \iff \qquad K(\emptyset, n) \ \text{finite}. $$}
\end{itemize}
}

\begin{onlyenv}<7>
\begin{lstlisting}[basicstyle=\scriptsize, frame=single]
def selmer : subgroup $ (units K) / (units K)^n :=
  { carrier := {x | ∀ p ∉ S, val_of_ne_zero_mod p x = 1},
    one_mem' := by { ... },
    mul_mem' := by { ... },
    inv_mem' := by { ... } }

notation K(S, n) := selmer K S n

def selmer.val : K(S, n) →* S → multiplicative (zmod n) :=
  { to_fun := λ x p, val_of_ne_zero_mod p x,
    map_one' := by { ... },
    map_mul' := by { ... } }

lemma selmer.val_ker : selmer.val.ker = K(∅, n).subgroup_of K(S, n)
\end{lstlisting}
\end{onlyenv}

\only<8->{
\begin{itemize}
\item Define an exact sequence $$ 0 \to \mathcal{O}_K^\times / (\mathcal{O}_K^\times)^n \xrightarrow{f} K(\emptyset, n) \xrightarrow{g} \mathrm{Cl}_K. $$
\end{itemize}
}

\begin{onlyenv}<9>
\begin{lstlisting}[basicstyle=\scriptsize, frame=single]
def f : units (O K) →* K(∅, n) :=
  { to_fun := λ x, ⟨quotient_group.mk $ ne_zero_of_unit x, λ p _, val_of_unit_mod p x⟩,
    map_one' := rfl,
    map_mul' := λ ⟨⟨_, _⟩, ⟨_, _⟩, _, _⟩ ⟨⟨_, _⟩, ⟨_, _⟩, _, _⟩, rfl } -- lol

lemma f_ker : f.ker = (units (O K))^n

def g : K(∅, n) →* class_group (O K) K := ... -- hmm

lemma g_ker : g.ker = f.range
\end{lstlisting}
\end{onlyenv}

\only<10->{
\begin{itemize}
\item Prove $ \mathrm{Cl}_K $ is finite. \only<11->{Done (Baanen, Dahmen, Narayanan, Nuccio).}
\end{itemize}
}

\only<12->{
\begin{itemize}
\item Prove $ \mathcal{O}_K^\times / (\mathcal{O}_K^\times)^n $ is finite. \only<13->{Suffices to show $ \mathcal{O}_K^\times $ is finitely generated.} \only<14->{Consequence of \textbf{Dirichlet's unit theorem (help wanted!)}.}
\end{itemize}
}

\only<15>{\vspace{0.5cm} Note the classical $ n $-Selmer group of $ E $ is $$ \mathrm{Sel}(K, E[n]) \le K(S, n) \times K(S, n). $$}

\end{frame}

\begin{frame}[t]{The Mordell-Weil theorem --- heights}

Prove that $ E(K) $ can be endowed with a ``height function".

\vspace{0.5cm}

\only<2->{A \textbf{height function} $ h : E(K) \to \mathbb{R} $ satisfies the following.
\begin{itemize}
\item<3-> For all $ Q \in E(K) $, there exists $ C_1 \in \mathbb{R} $ such that for all $ P \in E(K) $, $$ h(P + Q) \le 2h(P) + C_1. $$
\item<4-> There exists $ C_2 \in \mathbb{R} $ such that for all $ P \in E(K) $, $$ 4h(P) \le h(2P) + C_2. $$
\item<5-> For all $ C_3 \in \mathbb{R} $, the set $$ \{P \in E(K) \mid h(P) \le C_3\} $$ is finite.
\end{itemize}
}

\vspace{0.5cm}

\only<6>{Ongoing for $ K = \mathbb{Q} $. Probably not ready for general $ K $?}

\end{frame}

\begin{frame}{Future}

Potential future projects:
\begin{itemize}
\item $ n $-division polynomials and structure of $ E(K)[n] $
\item formal groups and local theory
\item ramification theory $ \implies $ full Mordell-Weil theorem
\item Galois cohomology $ \implies $ Selmer and Tate-Shafarevich groups
\item modular functions $ \implies $ complex theory
\item algebraic geometry $ \implies $ associativity, finally
\end{itemize}
Thank you!

\end{frame}

\end{document}