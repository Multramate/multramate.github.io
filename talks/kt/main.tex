\documentclass[10pt]{beamer}

\setbeamertemplate{footline}[page number]

\DeclareFontFamily{U}{wncyr}{}
\DeclareFontShape{U}{wncyr}{m}{n}{<->wncyr10}{}
\DeclareSymbolFont{cyr}{U}{wncyr}{m}{n}
\DeclareMathSymbol{\Sha}{\mathord}{cyr}{"58}

\usepackage{tikz-cd}

\begin{document}

\begin{frame}

\begin{center}

{\scriptsize University College London}

\vspace{0.5cm}

{\small Study group on the conjecture of Birch and Swinnerton-Dyer}

\vspace{1cm}

\textbf{\large Kolyvagin's theorem}

\vspace{1cm}

David Ang

\vspace{0.5cm}

{\footnotesize Wednesday, 13 March 2024}

\end{center}

\end{frame}

\begin{frame}[t]{Some recapitulation}

Let $ E $ be a rational elliptic curve of conductor $ N $, and let $ K = \mathbb{Q}(\sqrt{-D}) $ be an imaginary quadratic field satisfying the \textbf{Heegner hypothesis}
$$ \ell \mid N \qquad \implies \qquad \ell \ \text{is split in} \ K. $$
For any $ n $ coprime to $ N $, define a \textbf{Heegner point of conductor $ n $}
$$ P_n := \Phi_N(\mathbb{C} / \mathcal{O}_n, \mathbb{C} / \mathcal{N}_n) \in E(H_n). $$
For any $ \ell $ coprime to $ nN $ that is inert in $ K $, there are norm compatibilities
$$ \mathrm{Tr}_{H_n}^{H_{n\ell}} P_{n\ell} = a_\ell P_n. $$
These form a \textbf{Heegner system for $ (E, K) $}.

\vspace{0.5cm} Furthermore, define the \textbf{basic Heegner point}
$$ P_K := \mathrm{Tr}_K^{H_1}(P_1) \in E(K). $$

\end{frame}

\begin{frame}[t]{Application to BSD}

We will do the following next week:

\begin{theorem}[Gross--Zagier '86]
There is an explicit constant $ \alpha \ne 0 $ such that $ L'(E / K, 1) = \alpha \cdot \widehat{h}(P_K) $.
\end{theorem}

\vspace{0.5cm} We will do the following this week:

\begin{theorem}[Kolyvagin '90]
If $ \widehat{h}(P_K) \ne 0 $, then $ \mathrm{rk}_\mathbb{Z} E(K) = 1 $ and $ \#\Sha(E / K) < \infty $.
\end{theorem}

In particular, $ E(K)_{/ \mathrm{tors}} = \mathbb{Z} \cdot \tfrac{1}{n}P_K $.

\vspace{0.5cm} This almost proves the following:

\begin{corollary}[of Gross--Zagier '86, Kolyvagin '90, etc]
If $ \mathrm{ord}_{s = 1} L(E, s) \le 1 $, then $ \mathrm{rk}_\mathbb{Z} E(\mathbb{Q}) = \mathrm{ord}_{s = 1} L(E, s) $ and $ \#\Sha(E) < \infty $.
\end{corollary}

The missing ingredient is the existence of $ K $.

\end{frame}

\begin{frame}[t]{Existence of Heegner fields}

Let $ -\epsilon $ be the sign in the functional equation
$$ \Lambda(E, s) = -\epsilon \cdot \Lambda(E, 2 - s). $$

\begin{theorem}[Waldspurger '85, Murty--Murty '97]
If $ \epsilon = + $, there are many imaginary quadratic fields $ K = \mathbb{Q}(\sqrt{-D}) $ satisfying the Heegner hypothesis such that $ \mathrm{ord}_{s = 1} L(E_D, s) = 0 $.
\end{theorem}

In particular,
$$ \mathrm{ord}_{s = 1} L(E, s) = \mathrm{ord}_{s = 1} L(E / K, s). $$

\begin{theorem}[Bump--Friedberg--Hoffstein '90, Murty--Murty '91]
If $ \epsilon = - $, there are many imaginary quadratic fields $ K = \mathbb{Q}(\sqrt{-D}) $ satisfying the Heegner hypothesis such that $ \mathrm{ord}_{s = 1} L(E_D, s) = 1 $.
\end{theorem}

In particular,
$$ \mathrm{ord}_{s = 1} L(E, s) = \mathrm{ord}_{s = 1} L(E / K, s) - 1. $$

\end{frame}

\begin{frame}[t]{Complex conjugation on Heegner points}

\begin{lemma}[$ \tau $]
Complex conjugation $ \tau $ maps $ P_n \in E(H_n)_{/ \mathrm{tors}} $ to
$$ \tau(P_n) = \epsilon \cdot \sigma(P_n), $$
for some $ \sigma \in \mathrm{Gal}(H_n / K) $.
\end{lemma}

\begin{proof}
Note that $ \epsilon $ is precisely the eigenvalue of the Fricke involution $ w_N $ on the eigenform $ f_E $ associated to $ E $. On the other hand,
$$ w_N(\mathbb{C} / \mathcal{O}_n, \mathbb{C} / \mathcal{N}_n) = (\mathbb{C} / \mathcal{N}_n^{-1}, \mathbb{C} / \overline{\mathcal{N}_n}), $$
which differs from $ \tau(\mathbb{C} / \mathcal{O}_n, \mathbb{C} / \mathcal{N}_n) $ by some $ \sigma \in \mathrm{Gal}(H_n / K) \cong \mathrm{Cl}(\mathcal{O}_n) $. Now apply $ \Phi_N $ and the Manin--Drinfeld theorem.
\end{proof}

\vspace{0.5cm} In particular, $ P_K \in E(\mathbb{Q})_{/ \mathrm{tors}} $ precisely if $ \epsilon = + $.

\end{frame}

\begin{frame}[t]{Proof of Gross--Zagier--Kolyvagin}

\begin{proof}[Proof of BSD for $ \mathrm{ord}_{s = 1} L(E, s) \le 1 $]
The functional equation says that $ L(E, 1) = -\epsilon \cdot L(E, 1) $ and $ L'(E, 1) = \epsilon \cdot L'(E, 1) $. Since $ \mathrm{ord}_{s = 1} L(E, s) \le 1 $,
$$ \mathrm{ord}_{s = 1} L(E, s) = \begin{cases} 1 & \epsilon = + \\ 0 & \epsilon = - \end{cases}. $$
Choose any imaginary quadratic field $ K $ satisfying the Heegner hypothesis such that $ \mathrm{ord}_{s = 1} L(E / K, s) = 1 $, which exists by W/MM and BFH/MM. By Gross--Zagier and Kolyvagin, $ E(K)_{/ \mathrm{tors}} = \mathbb{Z} \cdot \tfrac{1}{n}P_K $. By Lemma $ (\tau) $,
$$ \mathrm{rk}_\mathbb{Z} E(\mathbb{Q}) = \begin{cases} 1 & \epsilon = + \\ 0 & \epsilon = - \end{cases}. $$
Finally, $ \#\Sha(E) < \infty $ follows from $ \#\Sha(E / K) < \infty $ by Kolyvagin.
\end{proof}

\end{frame}

\begin{frame}[t]{A weaker version of Kolyvagin}

\begin{theorem}[Kolyvagin '90]
If $ \widehat{h}(P_K) \ne 0 $, then $ \mathrm{rk}_\mathbb{Z} E(K) = 1 $ and $ \#\Sha(E / K) < \infty $.
\end{theorem}

For any prime $ \ell $,
$$ 0 \to E(K) / \ell E(K) \xrightarrow{\delta} \mathrm{Sel}_\ell(E / K) \to \Sha(E / K)[\ell] \to 0. $$
Choose any prime $ \ell \nmid 6ND $ such that $ \overline{\rho_{E, \ell}} $ is surjective and $ P_K \notin \ell E(K) $. Then $ E(K)[\ell] = 0 $, so $ \mathrm{rk}_{\mathbb{Z}} E(K) = \dim_{\mathbb{F}_\ell} E(K) / \ell E(K) $.

\begin{theorem}[weak Kolyvagin '90]
$ \mathrm{Sel}_\ell(E / K) = \mathbb{F}_\ell \cdot \delta(P_K) $, so $ \mathrm{rk}_\mathbb{Z} E(K) \le 1 $ and $ \#\Sha(E / K)[\ell] < \infty $.
\end{theorem}

\vspace{0.5cm} When $ E $ has no complex multiplication, this excludes finitely many primes by Serre's theorem, so this proves that $ \widehat{h}(P_K) \ne 0 $ implies $ \mathrm{rk}_\mathbb{Z} E(K) = 1 $. Kolyvagin proves $ \#\Sha(E / K) < \infty $ by refining the argument for these primes and bounding the $ \ell $-primary components using Iwasawa theory.

\end{frame}

\begin{frame}[t]{Selmer structures}

Let $ M $ be a discrete finite irreducible self-dual $ \mathbb{F}_\ell[G_K] $-module.

\vspace{0.5cm} The inflation-restriction exact sequence says
$$ 0 \to H^1(G_p^{\mathrm{ur}}, M^{I_p}) \to H^1(K_p, M) \to H^1(I_p, M)^{G_p^{\mathrm{ur}}} \to 0. $$
For $ M = E[\ell] $ and good $ p \nmid \ell $, this can be identified with
$$ 0 \to E(K_p) / \ell E(K_p) \to H^1(K_p, E[\ell]) \to H^1(K_p, E)[\ell] \to 0. $$
More generally, a \textbf{Selmer structure} for $ (K, M) $ is an assignment
$$ p \longmapsto H_f^1(K_p, M) \subseteq H^1(K_p, M), $$
such that $ H_f^1(K_p, M) = H^1(G_p^{\mathrm{ur}}, M^{I_p}) $ for all but finitely many places $ p $ of $ K $. Its associated \textbf{singular quotient} $ H_s^1(K_p, M) $ sits in
$$ 0 \to H_f^1(K_p, M) \to H^1(K_p, M) \xrightarrow{(\cdot)^s} H_s^1(K_p, M) \to 0. $$

\end{frame}

\begin{frame}[t]{Selmer groups}

The \textbf{Selmer group} $ \mathrm{Sel} := \mathrm{Sel}(K, M) $ sits in
$$ 0 \to \mathrm{Sel}(K, M) \to H^1(K, M) \xrightarrow{\prod_p (\cdot)_p^s} \prod_p H_s^1(K_p, M). $$
For $ M = E[\ell] $ and $ H_f^1(K_p, M) = E(K_p) / \ell E(K_p) $, this is just $ \mathrm{Sel}_\ell(E / K) $.

\vspace{0.5cm} Let $ S $ be a finite set of places of $ K $.
\begin{itemize}
\item The \textbf{relaxed Selmer group} $ \mathrm{Sel}^S := \mathrm{Sel}^S(K, M) $ sits in
$$ 0 \to \mathrm{Sel}(K, M) \to \mathrm{Sel}^S(K, M) \xrightarrow{\bigoplus_{p \in S} (\cdot)_p^s} \bigoplus_{p \in S} H_s^1(K_p, M). $$
\item The \textbf{restricted Selmer group} $ \mathrm{Sel}_S := \mathrm{Sel}_S(K, M) $ sits in
$$ 0 \to \mathrm{Sel}_S(K, M) \to \mathrm{Sel}(K, M) \xrightarrow{\bigoplus_{p \in S} (\cdot)_p} \bigoplus_{p \in S} H_f^1(K_p, M). $$
\end{itemize}

\end{frame}

\begin{frame}[t]{Duality of Selmer groups}

\begin{corollary}[of Tate duality]
\vspace{-0.5cm}
$$
\begin{tikzcd}[ampersand replacement=\&, column sep=small, row sep=0.1in]
0 \arrow{r} \& \mathrm{Sel} \arrow{r} \& \mathrm{Sel}^S \arrow{r} \& \displaystyle\bigoplus_{p \in S} H_s^1(K_p, M) \arrow[dash]{d}{\mid} \& \& \& \\
\& \& \& \displaystyle\bigoplus_{p \in S} H_f^1(K_p, M)^\vee \arrow{r} \& \mathrm{Sel}^\vee \arrow{r} \& \mathrm{Sel}_S^\vee \arrow{r} \& 0.
\end{tikzcd}
$$
\vspace{-0.5cm}
\end{corollary}

\begin{proof}
Local Tate duality gives a perfect pairing $ H_s^1(K_p, M) \times H_f^1(K_p, M) \to \mathbb{F}_\ell $. The Poitou-Tate exact sequence gives exactness at
$$ \mathrm{Sel}^S \to \bigoplus_{p \in S} H^1(K_p, M) \to \mathrm{Sel}^{S\vee}. $$
Now apply the snake lemma and diagram chase.
\end{proof}

\end{frame}

\begin{frame}[t]{Complex conjugation on Selmer groups}

To compute $ \mathrm{Sel} $, it suffices to consider the last three terms
$$ 0 \to \mathrm{coker}\left(\mathrm{Sel}^S \to \displaystyle\bigoplus_{p \in S} H_s^1(K_p, M)\right) \to \mathrm{Sel}^\vee \to \mathrm{Sel}_S^\vee \to 0, $$
for some appropriate finite set of places $ S $ of $ K $.

\vspace{0.5cm} If $ \tau \in G_\mathbb{Q} $ is an involution with non-zero eigenspaces $ M^+ $ and $ M^- $, then
$$ 0 \to \mathrm{coker}\left(\mathrm{Sel}^{S_1+} \to \displaystyle\bigoplus_{p \in S_1} H_s^1(K_p, M)^+\right) \to \mathrm{Sel}^{+\vee} \to \mathrm{Sel}_{S_1}^{+\vee} \to 0, $$
$$ 0 \to \mathrm{coker}\left(\mathrm{Sel}^{S_2-} \to \displaystyle\bigoplus_{p \in S_2} H_s^1(K_p, M)^-\right) \to \mathrm{Sel}^{-\vee} \to \mathrm{Sel}_{S_2}^{-\vee} \to 0, $$
for some appropriate finite sets of places $ S_1 $ and $ S_2 $ of $ K $.

\end{frame}

\begin{frame}[t]{Computing Selmer groups}

Now consider $ M = E[\ell] $.

\begin{corollary}[of Chebotarev density]
There is a finite set $ S $ of primes of $ \mathbb{Q} $ inert in $ K $ such that
$$ \mathrm{coker}\Biggl(\underbrace{\mathrm{Sel}^{S-\epsilon}}_{\bigoplus_{p \in S} \mathbb{F}_\ell \cdot c(p)_p^s} \to \bigoplus_{p \in S} \underbrace{H^1(K_p, E)[\ell]^{-\epsilon}}_{\mathbb{F}_\ell \cdot c(p)_p^s}\Biggr) \to \mathrm{Sel}^{-\epsilon\vee} \to \underbrace{\mathrm{Sel}_S^{-\epsilon\vee}}_0. $$
For any $ p \in S $, there is a finite set $ S_p $ of primes of $ \mathbb{Q} $ inert in $ K $ such that
$$ \mathrm{coker}\Biggl(\underbrace{\mathrm{Sel}^{S_p\epsilon}}_{\bigoplus_{q \in S_p} \mathbb{F}_\ell \cdot c(pq)_q^s} \to \bigoplus_{q \in S_p} \underbrace{H^1(K_p, E)[\ell]^\epsilon}_{\mathbb{F}_\ell \cdot c(pq)_q^s}\Biggr) \to \mathrm{Sel}^{\epsilon\vee} \to \underbrace{\mathrm{Sel}_{S_p}^{\epsilon\vee}}_{\mathbb{F}_\ell \cdot \delta(P_K)} \to 0. $$
\vspace{-0.5cm}
\end{corollary}

\begin{proof}
Chebotarev density and a lot of Galois cohomology.
\end{proof}

\end{frame}

\begin{frame}[t]{Derivative operators}

The classes $ c(n) \in H^1(K, E[\ell]) $ are derived from $ P_n \in E(H_n) $.

\vspace{0.5cm} It suffices to let $ n $ be a product of primes $ p \nmid ND\ell $ inert in $ K $, so
$$ \mathrm{Gal}(H_n / H_1) \cong \prod_{p \mid n} \mathrm{Gal}(H_p / H_1) \cong \prod_{p \mid n} \mathbb{Z} / (p + 1)\mathbb{Z} \cdot \sigma_p. $$
Define the \textbf{derivative operator} $ D_n \in \mathbb{Z}[\mathrm{Gal}(H_n / H_1)] $ by
$$ D_n := \prod_{p \mid n} D_p, $$
where $ D_p $ is any solution to $ (\sigma_p - 1)D_p = p + 1 - \mathrm{Tr}_{H_1}^{H_p} $, and define
$$ \mathcal{P}_n := \sum_{\tau \in T_n} \tau(D_nP_n), $$
where $ T_n $ is a set of coset representatives for $ \mathrm{Gal}(H_n / H_1) $ in $ \mathrm{Gal}(H_n / K) $.

\end{frame}

\begin{frame}[t]{Derived classes}

\begin{lemma}
The class of $ \mathcal{P}_n $ in $ E(H_n) / \ell E(H_n) $ is invariant under the action of $ G_n := \mathrm{Gal}(H_n / K) $ and lies in the $ \epsilon_n := \epsilon \cdot (-1)^{\sigma_0(n)} $ eigenspace.
\end{lemma}

\begin{proof}
Norm compatibilities and Lemma $ (\tau) $.
\end{proof}

\vspace{0.5cm} Define the \textbf{derived class} $ c(n) \in H^1(K, E[\ell])^{\epsilon_n} $ by $ \mathrm{res}_n(c(n)) = \delta_n(\mathcal{P}_n) $ in
$$
\begin{tikzcd}[ampersand replacement=\&, column sep=small, row sep=small]
\& H^1(G_n, E(H_n)[\ell])^{\epsilon_n} = 0 \arrow{d}{\mathrm{inf}_n} \& \\
H_f^1(K, E[\ell])^{\epsilon_n} \arrow{r}{\delta} \arrow{d} \& H^1(K, E[\ell])^{\epsilon_n} \arrow{r} \arrow{d}{\mathrm{res}_n} \& H_s^1(K, E[\ell])^{\epsilon_n} \arrow{d} \\
H_f^1(H_n, E[\ell])^{G_n\epsilon_n} \arrow{r}[swap]{\delta_n} \& H^1(H_n, E[\ell])^{G_n\epsilon_n} \arrow{r} \arrow{d}{\mathrm{tra}_n} \& H_s^1(H_n, E[\ell])^{G_n\epsilon_n} \\
\& H^2(G_n, E(H_n)[\ell])^{\epsilon_n} = 0 \&
\end{tikzcd}.
$$

\end{frame}

\begin{frame}[t]{Ramification of derived classes}

\begin{lemma}
\begin{enumerate}
\item If $ p \nmid n $, then $ c(n)_p^s = 0 $, so $ c(n) \in \mathrm{Sel}^{\{p \mid n\}\epsilon_n} $.
\item If $ p \mid n $, then $ c(n)_p^s = 0 $ if and only if $ \mathcal{P}_{n / p} \in \ell E(K_p) $.
\end{enumerate}
\end{lemma}

\begin{proof}[Proof of 1 for good $ p \nmid \ell $]
Note that $ H_s^1(I_p, E[\ell]) = \mathrm{Hom}(I_p, E[\ell])^{G_p^{\mathrm{ur}}} $. Since $ (H_n)_p / K $ is unramified at $ p $, the inertia subgroups of $ K_p $ and $ (H_n)_p $ are both $ I_p $, so
$$
\begin{tikzcd}[ampersand replacement=\&, row sep=0.1in]
H_f^1(K_p, E[\ell]) \arrow{r} \arrow{d} \& H^1(K_p, E[\ell]) \arrow{r}{(\cdot)^s} \arrow{d}{\mathrm{res}_n} \& \mathrm{Hom}(I_p, E[\ell]) \arrow[dash]{d}{\mid} \\
H_f^1((H_n)_p, E[\ell]) \arrow{r}[swap]{\delta_n} \& H^1((H_n)_p, E[\ell]) \arrow{r}[swap]{(\cdot)^s} \& \mathrm{Hom}(I_p, E[\ell])
\end{tikzcd}.
$$
Thus $ c(n)_p^s = (\mathrm{res}_n(c(n)_p))^s = 0 $ by exactness.
\end{proof}

\vspace{0.5cm} Note that 2 is precisely the reason for the assumption $ P_K \notin \ell E(K) $.

\end{frame}

\begin{frame}[t]{References}

Different accounts of Kolyvagin's paper on Euler systems:
\begin{itemize}
\item K Rubin (1989) The work of Kolyvagin on the arithmetic of elliptic curves
\item B Gross (1991) Kolyvagin's work on modular elliptic curves
\item T Weston (2001) The Euler system of Heegner points
\item H Darmon (2004) Rational points on modular elliptic curves
\end{itemize}
Relevant papers on non-vanishing of L-functions:
\begin{itemize}
\item J-L Waldspurger (1985) Sur les valeurs de certaines fonctions L automorphes en leur centre de symetrie
\item D Bump, S Friedberg, and J Hoffstein (1990) Nonvanishing theorems for L-functions of modular forms
and their derivatives
\item M R Murty and V K Murty (1991) Mean values of derivatives of modular L-series
\item M R Murty and V K Murty (1997) Non-vanishing of L-functions and applications
\end{itemize}

\end{frame}

\end{document}