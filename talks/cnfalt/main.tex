\documentclass{article}

\usepackage{amssymb}
\usepackage{amsthm}
\usepackage{bbm}
\usepackage{commath}
\usepackage{enumitem}
\usepackage[margin=1in]{geometry}
\usepackage{mathtools}
\usepackage{tikz-cd}

\newtheorem*{theorem}{Theorem}

\newcommand{\1}{\mathbbm{1}}
\renewcommand{\AA}{\mathbb{A}}
\newcommand{\br}{\del}
\renewcommand{\c}{\mathrm{c}}
\newcommand{\CC}{\mathbb{C}}
\newcommand{\CCC}{\mathcal{C}}
\newcommand{\Cl}{\mathrm{Cl}}
\newcommand{\D}{\mathrm{D}}
\newcommand{\DDD}{\mathcal{D}}
\newcommand{\f}{\mathrm{f}}
\newcommand{\h}{\mathrm{h}}
\newcommand{\I}{\mathrm{I}}
\newcommand{\Nm}{\mathrm{Nm}}
\newcommand{\OOO}{\mathcal{O}}
\newcommand{\ppp}{\mathfrak{p}}
\newcommand{\q}{\mathrm{q}}
\renewcommand{\r}{\mathrm{r}}
\newcommand{\R}{\mathrm{R}}
\newcommand{\Res}{\mathrm{Res}}
\newcommand{\RR}{\mathbb{R}}
\newcommand{\st}{\ \middle| \ }
\newcommand{\U}{\mathrm{U}}
\newcommand{\V}{\mathrm{V}}
\newcommand{\w}{\mathrm{w}}
\newcommand{\Z}{\mathrm{Z}}
\newcommand{\ZZ}{\mathbb{Z}}

\newcommand{\abr}[1]{\left\langle #1 \right\rangle}

\newcommand{\function}[5]{
  \begin{array}{ccrcl}
    \displaystyle #1 & : & \displaystyle #2 & \longrightarrow & \displaystyle #3 \\
                     &   & \displaystyle #4 & \longmapsto     & \displaystyle #5
  \end{array}
}

\title{Class number formula, \`a la Tate}
\author{David Ang}
\date{Tuesday, 17 January 2023}

\begin{document}

\maketitle

\begin{abstract}
The analytic class number formula, as its name might suggest, relates the class number of a number field, an algebraic invariant, to the residue of its zeta-function, an analytic quantity. In his thesis, Tate reinterpreted such zeta-functions as global zeta-integrals over the locally compact topological group of id\`eles, where the class number formula becomes a relatively straightforward computation in Fourier analysis. This talk outlines an ad\`elic proof of the class number formula as an application of Tate's thesis.
\end{abstract}

\section{The statement}

Let $ K $ be a number field with $ r $ real and $ c $ complex embeddings. The class number formula relates its \textbf{class number} $ \h_K $, the size of its ideal class group $ \Cl_K $, to the residue at $ s = 1 $ of its \textbf{Dedekind $ \zeta $-function}
$$ \zeta_K\br{s} \coloneqq \sum_{0 \ne I \trianglelefteq \OOO_K} \dfrac{1}{\Nm\br{I}^s}, $$
which has a simple pole at $ s = 1 $. A common formulation is as follows.

\begin{theorem}[Class number formula]
$$ \Res_{s = 1} \zeta_K\br{s} = \dfrac{2^{r}\br{2\pi}^{c}\h_K\R_K}{\w_K\sqrt{\abs{\D_K}}}. $$
\end{theorem}

Here, if $ \V_K \coloneqq \V_K^\r \cup \V_K^\c \cup \V_K^\f $ is the set of real, complex, and finite places of $ K $ respectively, then
\begin{itemize}
\item $ \w_K $ is the size of the group of roots of unity $ \mu_K $,
\item $ \D_K $ is the \textbf{absolute discriminant}, which is the product of $ \q_v^{v\br{\DDD_v}} $ over all $ v \in \V_K^\f $, where $ \q_v $ is the size of the residue field at $ v \in \V_K^\f $ and $ \DDD_v $ is the absolute different at $ v \in \V_K^\f $, and
\item $ \R_K $ is the \textbf{regulator}, which is the determinant of any $ \br{\br{r + c - 1} \times \br{r + c - 1}} $-minor of the matrix
$$ \br{\log \abs{\epsilon_i}_v}_{1 \le i \le r + c - 1, \ v \in \V_K^\r \cup \V_K^\c}, $$
where the $ \epsilon_i $ are the non-torsion generators in $ \OOO_K^\times \cong \mu_K \times \abr{\epsilon_1, \dots, \epsilon_{r + c - 1}} $ by Dirichlet's unit theorem.
\end{itemize}
It is worth noting the analogy with the invariants in the strong Birch and Swinnerton-Dyer conjecture.

Classically, its proof falls into the realm of the geometry of numbers, and the residue is computed by counting lattice points in the ring of integers $ \OOO_K $ within a region obtained by embedding $ K \hookrightarrow \RR^{r + c} $, not unlike the proof of Dirichlet's unit theorem. However, Tate's Fourier analytic proof for the meromorphic continuation and functional equation of the Dedekind $ \zeta $-function, and more generally those of Hecke $ L $-functions in his thesis, paved way for an elegant ad\`elic proof of the class number formula, by reducing the residue computation to determining the measure of the norm one id\`ele class group.

After recalling a few basic constructions of ad\`eles and id\`eles, a brief overview of Tate's thesis will be provided, including the necessary integral calculations to finally compute this residue.

\pagebreak

\section{Ad\`eles and id\`eles}

For a place $ v \in \V_K $, let $ \OOO_v $ be the valuation ring of the completion $ K_v $. Recall that the ring of \textbf{ad\`eles}
$$ \AA_K \coloneqq \cbr{\br{x_v}_{v \in \V_K} \in \prod_{v \in \V_K} K_v \st x_v \in \OOO_v \ \text{for almost all} \ v \in \V_K}, $$
and its unit group of \textbf{id\`eles} $ \AA_K^\times $, can both can be endowed with the restricted product topology, which make them locally compact groups by Tychonoff's theorem. An open basis of $ \AA_K^\times $ is given by the open subsets of
$$ \AA_S^\times \coloneqq \prod_{v \in \V_K^\r \cup \V_K^\c \cup S} K_v^\times \times \prod_{\V_K^\f \setminus S} \OOO_v^\times $$
under the usual product topology, for some finite subset of places $ S \subset \V_K^\f $, so that for instance,
$$ \AA_\emptyset^\times = K_\infty^\times \times \widehat{\OOO_K}^\times \coloneqq \prod_{v \in \V_K^\r \cup \V_K^\c} K_v^\times \times \prod_{\V_K^\f} \OOO_v^\times. $$
Note that $ \AA_K^\times \hookrightarrow \AA_K $ is finer than the subspace topology to allow inversion to be continuous.

There are natural discrete diagonal embeddings $ K \hookrightarrow \AA_K $ and $ K^\times \hookrightarrow \AA_K^\times $, but only the former is cocompact. The latter quotient is the \textbf{id\`ele class group} $ \CCC_K \coloneqq \AA_K^\times / K^\times $, and has a compact subgroup, the \textbf{norm one id\`ele class group} $ \CCC_K^1 $, defined as the kernel of the \textbf{id\`ele norm} homomorphism
$$ \function{\abs{\cdot}_{\AA_K}}{\CCC_K}{\RR^+}{\br{x_v}_{v \in \V_K}}{\prod_{v \in \V_K} \abs{x_v}_v}, $$
which is well-defined by the product formula and continuous under the real topology on the positive reals $ \RR^+ $. Now the id\`ele norm is clearly surjective since $ K_\infty^\times / K^\times \hookrightarrow \CCC_K $, and has a set-theoretic section given by
$$ \function{\iota}{\RR^+}{K_\infty^\times / K^\times}{x}{\br{x^{\tfrac{1}{r + 2c}}}_{v \in \V_K^\r \cup \V_K^\c}}. $$
Thus there is a split short exact sequence $ 1 \to \CCC_K^1 \to \CCC_K \xleftarrow{\iota} \RR^+ \to 1 $.

For a place $ v \in \V_K $, let $ \ppp_v $ be the corresponding prime ideal. There is a surjective \textbf{content} homomorphism
$$ \function{\kappa}{\CCC_K}{\Cl_K}{\br{x_v}_{v \in \V_K}}{\prod_{v \in \V_K^\f} \ppp_v^{v\br{x_v}}}, $$
which is well-defined since the content of a id\`ele arising from some $ x \in K^\times $ is precisely the principal ideal $ \abr{x} $, and is continuous under the discrete topology on $ \Cl_K $ since its kernel is the open subgroup $ \AA_\emptyset^\times / \OOO_K^\times $. Thus there is a short exact sequence $ 1 \to \AA_\emptyset^\times / \OOO_K^\times \to \CCC_K \xrightarrow{\kappa} \Cl_K \to 1 $. Since $ \kappa \circ \iota : \RR^+ \to \Cl_K $ is the zero map, the content map also factors through $ \kappa : \CCC_K^1 \to \Cl_K $, which is a continuous surjection from a compact group to a discrete group, so the finiteness of $ \Cl_K $ follows. Note that an argument akin to Minkowski's theorem in the geometry of numbers is still present, albeit hidden in the proof that $ \CCC_K^1 $ is compact. With slightly more effort, Dirichlet's unit theorem may also be proven by considering the surjective \textbf{id\`ele logarithm} homomorphism
$$ \function{\lambda}{\AA_\emptyset^\times}{\RR^{r + c}}{\br{x_v}_{v \in \V_K}}{\br{\log \abs{x_v}_v}_{v \in \V_K^\r \cup \V_K^\c}}, $$
or more precisely its continuous surjective quotient $ \AA_\emptyset^\times / \OOO_K^\times \to \RR^{r + c} / \lambda\br{\OOO_K^\times} $ from the closed, and hence compact, subgroup $ \AA_\emptyset^\times / \OOO_K^\times < \CCC_K^1 $. The id\`ele logarithm also clearly factors through $ \lambda : K_\infty^\times \to \RR^{r + c} $, whose kernel is $ \cbr{\pm}^r \times \U\br{1}^c $, and thus there is a short exact sequence $ 1 \to \cbr{\pm}^r \times \U\br{1}^c \to K_\infty^\times \xrightarrow{\lambda} \RR^{r + c} \to 1 $.

\pagebreak

\section{Tate's thesis}

The main input behind Tate's thesis is the fact that a locally compact group $ G $, such as $ \AA_K $ and $ \AA_K^\times $ or their local counterparts, can be endowed with a certain translation-invariant measure that is unique up to scaling by a non-zero constant, called the \textbf{Haar measure} $ \mu = \int \dif x $, which allows for a notion of integration over $ G $. The ad\`elic Haar measure happens to be the product measure of the Haar measures at each completion, and these local Haar measures can be described and normalised explicitly for an easier computation later.
\begin{itemize}
\item If $ v \in \V_K^\r $, the additive Haar measure is normalised to be the usual Lebesgue measure, so that $ \dif_v x = \dif x $, while the multiplicative Haar measure is normalised such that $ \dif_v^\times x = \dif_v x / \abs{x}_v $.
\item If $ v \in \V_K^\c $, the additive Haar measure is normalised to be twice the usual Lebesgue measure, so that $ \dif_v \br{x + iy} = 2\dif x\dif y $, while the multiplicative Haar measure is normalised such that $ \dif_v^\times z = \dif_v z / \abs{z}_v $.
\item If $ v \in \V_K^\f $, the additive Haar measure is normalised such that $ \mu_v\br{\OOO_v} = \q_v^{-v\br{\DDD_v} / 2} $, so that
$$ \mu_v\br{\pi_v^n\OOO_v} = \q_v^{-n} \cdot \mu_v\br{\OOO_v} = \q_v^{-n - \tfrac{v\br{\DDD_v}}{2}}, $$
for any uniformiser $ \pi_v \in \OOO_v $, while the multiplicative Haar measure is normalised such that
$$ \dif_v^\times x = \dfrac{\q_v^{\tfrac{v\br{\DDD_v}}{2}}}{1 - \q_v^{-1}} \cdot \dfrac{\dif_v x}{\abs{x}_v}, $$
so that
$$ \mu_v^\times\br{\OOO_v^\times} = \dfrac{\q_v^{\tfrac{v\br{\DDD_v}}{2}}}{1 - \q_v^{-1}} \cdot \br{\mu_v\br{\OOO_v} - \mu_v\br{\pi_v\OOO_v}} = 1. $$
Thus the normalised ad\`elic Haar measure on all of $ \widehat{\OOO_K}^\times $ is also $ 1 $.
\end{itemize}

The \textbf{global $ \zeta $-integral} $ \zeta\br{f, s} $ may then be defined as the product of \textbf{local $ \zeta $-integrals}
$$ \zeta_v\br{f_v, s} \coloneqq \int_{K_v^\times} f_v\br{x}\abs{x}_v^s\dif_v^\times x, $$
for a class of well-behaved functions $ f_v : K_v^\times \to \CC $. These functions are equipped with \textbf{Fourier transforms}
$$ \widehat{f_v}\br{y} \coloneqq \int_{K_v} e^{2\pi i\chi_v\br{xy}}f_v\br{x}\dif_v x, $$
for some additive function $ \chi_v : K_v \to \CC $ depending only on the choice of normalising factor in $ \mu_v $, such that the \textbf{Fourier inversion formula} $ \widehat{\widehat{f_v}} = f_v $ holds. For the sake of brevity, the exact choices for these additive characters will be omitted, but the resulting Fourier transforms for each function will be provided directly.
\begin{itemize}
\item If $ v \in \V_K^\r $, let $ f_v\br{x} \coloneqq e^{-\pi x^2} $. Then $ \widehat{f_v} = f_v $, and by substituting $ y \coloneqq \pi x^2 $,
$$ \zeta_v\br{f_v, s} = \int_0^\infty e^{-\pi x^2}x^s\dfrac{2\dif x}{x} = \int_0^\infty e^{-y}\br{\dfrac{y}{\pi}}^{\tfrac{s}{2}}\dfrac{\dif y}{y} = \pi^{-\tfrac{s}{2}}\Gamma\br{\dfrac{s}{2}} \eqqcolon \Gamma_\RR\br{s}. $$
\item If $ v \in \V_K^\c $, let $ f_v\br{z} \coloneqq \tfrac{1}{\pi}e^{-2\pi z\overline{z}} $. Then $ \widehat{f_v} = f_v $, and by substituting $ z \coloneqq re^{i\theta} $,
$$ \zeta_v\br{f_v, s} = \int_0^{2\pi} \int_0^\infty \dfrac{1}{\pi}e^{-2\pi r^2}r^{2s}\dfrac{2r\dif r\dif \theta}{r^2} = 4\int_0^\infty e^{-2\pi r^2}r^{2s}\dfrac{\dif r}{r} = 2\br{2\pi}^{-s}\Gamma\br{s} \eqqcolon \Gamma_\CC\br{s}. $$
\item If $ v \in \V_K^\f $, let $ f_v \coloneqq \1_{\OOO_v} $. Then $ \widehat{f_v} = \q_v^{-v\br{\DDD_v} / 2}\1_{\DDD_v^{-1}} $, and
$$ \zeta_v\br{f_v, s} = \dfrac{\q_v^{\tfrac{v\br{\DDD_v}}{2}}}{1 - \q_v^{-1}} \cdot \int_{\OOO_v} \abs{x}_v^s\dfrac{\dif_v x}{\abs{x}_v} = \sum_{n = 0}^\infty \dfrac{\q_v^{\tfrac{v\br{\DDD_v}}{2} + n\br{1 - s}}}{1 - \q_v^{-1}} \cdot \br{\mu_v\br{\pi_v^n\OOO_v} - \mu_v\br{\pi_v^{n + 1}\OOO_v}} = \sum_{n = 0}^\infty \q_v^{-ns}. $$
Thus the product of $ \zeta_v\br{f_v, s} $ for all $ v \in \V_K^\f $ is simply $ \zeta_K\br{s} $.
\end{itemize}

\pagebreak

In this sense, $ \zeta\br{f, s} $ supplements $ \zeta_K\br{s} $ with local Euler factors corresponding to the archimedean places, which also explains the functional equation $ \Z_K\br{s} = \Z_K\br{1 - s} $ of the \textbf{completed Dedekind $ \zeta $-function}
$$ \Z_K\br{s} \coloneqq \abs{\D_K}^{\tfrac{s}{2}} \cdot \Gamma_\RR\br{s} \cdot \Gamma_\CC\br{s} \cdot \zeta_K\br{s}, $$
whose fudge factors were poorly understood prior to Tate's thesis. He further proved the analytic continuation of $ \zeta\br{f, s} $ to $ \CC $ apart from simple poles at $ s = 0 $ and $ s = 1 $, with residues $ -\mu\br{\CCC_K^1}f\br{0} $ and $ \mu\br{\CCC_K^1}\widehat{f}\br{0} $ respectively, but the proof of this will be omitted for the sake of brevity. It is worth noting that the residue contributions at $ s = 0 $ and $ s = 1 $ arise precisely from introducing a $ 0 $-term to a certain integral over $ K^\times $ to utilise an ad\`elic \textbf{Poisson summation formula} $ \sum_{x \in K} f\br{x} = \sum_{x \in K} \widehat{f}\br{x} $.

\section{The proof}

The class number formula can now be proven as follows.

\begin{proof}[Proof of Theorem]
Consider the global Schwartz-Bruhat function $ f \coloneqq \prod_{v \in \V_K} f_v $, where
$$ f_v\br{x} =
\begin{cases}
e^{-\pi x^2} & v \in \V_K^\r \\
\dfrac{1}{\pi}e^{-2\pi x\overline{x}} & v \in \V_K^\c \\
\1_{\OOO_v}\br{x} & v \in \V_K^\f
\end{cases}.
$$
Then its global $ \zeta $-integral $ \zeta\br{f, s} $ is related to the Dedekind $ \zeta $-function $ \zeta_K\br{s} $ by
$$ \zeta_K\br{s} = \Gamma_\RR\br{s}^{-r} \cdot \Gamma_\CC\br{s}^{-c} \cdot \zeta\br{f, s}, $$
so their residues at $ s = 1 $ are
$$ \Res_{s = 1} \zeta_K\br{s} = \Gamma_\RR\br{1}^{-r} \cdot \Gamma_\CC\br{1}^{-c} \cdot \Res_{s = 1} \zeta\br{f, s} = 1^{-r} \cdot \br{\dfrac{1}{\pi}}^{-c} \cdot \mu\br{\CCC_K^1}\widehat{f}\br{0}, $$
where $ \mu $ is the normalised ad\`elic Haar measure. On one hand, the norm one id\`ele class group has measure
\begin{align*}
\mu\br{\CCC_K^1}
& = \mu\br{\CCC_K / \abr{\iota\br{e}}} & \text{by} \ 1 \to \CCC_K^1 \to \CCC_K \xleftarrow{\iota} \RR^+ \to 1 \\
& = \h_K \cdot \mu\br{\AA_\emptyset^\times / \OOO_K^\times\abr{\iota\br{e}}} & \text{by} \ 1 \to \AA_\emptyset^\times / \OOO_K^\times \to \CCC_K \xrightarrow{\kappa} \Cl_K \to 1 \\
& = \dfrac{\h_K}{\w_K} \cdot \mu\br{\AA_\emptyset^\times / \abr{\epsilon_1, \dots, \epsilon_{r + c - 1}, \iota\br{e}}} & \text{by} \ \OOO_K^\times = \mu_K \times \abr{\epsilon_1, \dots, \epsilon_{r + c - 1}} \\
& = \dfrac{\h_K}{\w_K} \cdot \mu\br{K_\infty^\times / \abr{\epsilon_1, \dots, \epsilon_{r + c - 1}, \iota\br{e}}} & \text{by} \ \AA_\emptyset^\times = K_\infty^\times \times \widehat{\OOO_K}^\times \\
& = \dfrac{2^{r}\br{2\pi}^{c}\h_K\R_K}{\w_K} & \text{by} \ 1 \to \cbr{\pm}^{r} \times \U\br{1}^{c} \to K_\infty^\times \xrightarrow{\lambda} \RR^{r + c} \to 1.
\end{align*}
On the other hand, $ f $ has Fourier transform $ \widehat{f} \coloneqq \prod_{v \in \V_K} \widehat{f_v} $, where
$$ \widehat{f_v}\br{x} =
\begin{cases}
e^{-\pi x^2} & v \in \V_K^\r \\
\dfrac{1}{\pi}e^{-2\pi x\overline{x}} & v \in \V_K^\c \\
\q_v^{-\tfrac{v\br{\DDD_v}}{2}}\1_{\DDD_v^{-1}}\br{x} & v \in \V_K^\f
\end{cases},
$$
so
$$ \widehat{f}\br{0} = \prod_{v \in \V_K^\r} e^{-\pi0^2} \cdot \prod_{v \in \V_K^\c} \dfrac{1}{\pi}e^{-2\pi0\overline{0}} \cdot \prod_{v \in \V_K^\f} \q_v^{-\tfrac{v\br{\DDD_v}}{2}} = 1^{r} \cdot \br{\dfrac{1}{\pi}}^{c} \cdot \dfrac{1}{\sqrt{\abs{\D_K}}}. $$
Combining both expressions yields the class number formula.
\end{proof}

\end{document}