\documentclass[10pt]{beamer}

\setbeamertemplate{footline}[page number]

\usepackage{tikz}

\title{Application of additive combinatorics}
\subtitle{Hilbert's tenth problem over rings of integers of number fields \footnote{Koymans and Pagano (2025) Hilbert's tenth problem via additive combinatorics}}
\author{David Kurniadi Angdinata}
\institute{London School of Geometry and Number Theory}
\date{Wednesday, 18 June 2025}

\begin{document}

\frame{\titlepage}

\begin{frame}{Notation}

\begin{itemize}
\item[$ K $] a number field of degree $ n $, such that $ r := \#V_K^\mathbb{R} \ge 32 $
\item[$ E $] an elliptic curve $ y^2 = (x - a_1)(x - a_2)(x - a_3) $ over $ K $ of root number $ 1 $, such that $ -1, a_1 - a_2, a_1 - a_3, a_2 - a_3 \in K^\times $ are linearly independent as elements of $ K^\times / (K^\times)^2 $
\item[$ T $] a finite set of places of $ K $ that includes the $ 2 $-adic primes, the $ 3 $-adic primes, the primes of bad reduction for $ E $, and the archimedean places, such that $ [K(T) : K(V_K^\mathbb{R})] \ge 2^r $
\item[$ \tau_\ell $] six places in $ V_K^\mathbb{R} $, such that
\begin{align*}
\tau_1(a_3) > \tau_1(a_1) > \tau_1(a_2), & \qquad \tau_2(a_3) > \tau_2(a_2) > \tau_2(a_1), \\
\tau_3(a_1) > \tau_3(a_3) > \tau_3(a_2), & \qquad \tau_4(a_2) > \tau_4(a_3) > \tau_4(a_1), \\
\tau_5(a_3) > \tau_5(a_1) > \tau_5(a_2), & \qquad \tau_6(a_3) > \tau_6(a_2) > \tau_6(a_1)
\end{align*}
\end{itemize}

\end{frame}

\begin{frame}[t]{A suitable twist}

Recall that $ t \in K^\times / (K^\times)^2 $ is a \textbf{suitable twist} if it satisfies the following.
\begin{itemize}
\item[P1] The quadratic character $ \psi_t $ is trivial at the places in $ T $.
\item[P2] There is some $ \kappa \in K^\times / (K^\times)^2 $ whose quadratic character $ \psi_\kappa $ is ramified at some primes $ \mathfrak{p}_1, \dots, \mathfrak{p}_s $ of $ K $ and satisfies
$$ \psi_t = \psi_\kappa + \psi_{q_1} + \psi_{q_2} + \psi_{q_3} + \psi_{q_4}, $$
for some primes $ q_1, q_2, q_3, q_4 $ of $ K $ not in $ T' := T \cup \{\mathfrak{p}_1, \dots, \mathfrak{p}_s\} $.
\item[P3] There is a basis $ (z_1, z_2), \dots, (z_{11}, z_{12}) $ of $ \operatorname{Sel}_{\mathcal{L}_{s, t}}(K, E[2]) $ such that
$$ \prod_{v \in T'} (z_i, q_j)_v =
\begin{cases}
-1 & \text{if} \ (i, j) = (1, 1), (5, 2), (9, 3), \\
& \qquad \qquad (4, 1), (8, 2), (12, 3), \\
1 & \text{otherwise}.
\end{cases}
$$
\item[P4] The rank of $ E^{-t}(K) $ is positive.
\end{itemize}
Given a suitable twist $ t $, the ranks of $ E^t(K) $ and $ E^t(K(i)) $ can be shown to be equal and positive, which proves that $ \mathcal{O}_K $ is Diophantine over $ \mathcal{O}_{K(i)} $.

\end{frame}

\begin{frame}[t]{An auxiliary twist}

It turns out that constructing a suitable twist reduces to constructing an \textbf{auxiliary twist} $ \kappa \in K^\times / (K^\times)^2 $ satisfying the following.
\begin{itemize}
\item[K1] The quadratic character $ \psi_\kappa $ is
\begin{itemize}
\item a unit at the $ 2 $-adic primes of $ K $,
\item unramified at the odd primes in $ T $, and
\item trivial at $ \tau_1, \tau_2, \tau_3 $ and non-trivial at $ \tau_4, \tau_5, \tau_6 $.
\end{itemize}
\item[K2] There is a basis $ (z_1, z_2), \dots, (z_{11}, z_{12}) $ of $ \operatorname{Sel}_{\mathcal{L}_{s, \pi}}(K, E[2]) $ such that
$$ \operatorname{sgn}(\tau_\ell(z_i)) =
\begin{cases}
- & \text{if} \ (i, \ell) = (1, 1), (5, 3), (9, 5), \\
& \qquad \qquad (4, 2), (8, 4), (12, 6), \\
+ & \text{otherwise},
\end{cases}
$$
for some tuple $ \pi = (\pi_1, \dots, \pi_s) $ of primes of $ K $.
\end{itemize}
Given an auxiliary twist $ \kappa $, a generalisation of the Green--Tao theorem by Kai says that a generic family of polynomials $ L_j(X, Y) \in \mathcal{O}_K[X, Y] $ admits simultaneously prime values $ q_j $ that satisfy certain congruence conditions. Then $ \kappa\prod_j q_j $ will turn out to be a suitable twist.

\end{frame}

\begin{frame}[t]{The four bivariate polynomials}

Let $ m \in \mathcal{O}_K $ be a generator of the ideal
$$ 8\prod_{\mathfrak{p} \in T \ \text{odd}} \mathfrak{p}^{\#\operatorname{Cl}(K)}. $$
Assume that $ \kappa \in \mathcal{O}_K $ is coprime to $ m $, which is possible by K1 and strong approximation. Let $ \lambda \in \mathcal{O}_K $ be an inverse of $ \kappa $ modulo $ m $ coprime to $ \kappa $.

\vspace{0.5cm} Let $ c(X) := m^2\kappa X + 1 $ and $ d(Y) := m^2\kappa(m^2\kappa Y + \lambda) $, and define
$$ L_1(X, Y) := c(X) + a_1d(Y), \qquad L_2(X, Y) := c(X) + a_2d(Y), $$
$$ L_3(X, Y) := c(X) + a_3d(Y), \qquad L_4(X, Y) := d(Y) / m^2\kappa. $$

\vspace{0.5cm} Kai's theorem will give infinitely many quadruples $ (q_1, q_2, q_3, q_4) $ of primes of $ K $ such that $ L_j(x, y) = q_j $ for each $ j = 1, 2, 3, 4 $ for some $ x, y \in \mathcal{O}_K $, so that $ t := \kappa q_1q_2q_3q_4 $ clearly satisfies P2.

\end{frame}

\begin{frame}[t]{Point of infinite order}

For $ t $ to satisfy P4, observe that $ E^{-t} $ is given by
$$ -(c + a_1d)(c + a_2d)(c + a_3d)\dfrac{dy^2}{m^2} = (x - a_1)(x - a_2)(x - a_3), $$
for some $ c, d \in \mathcal{O}_K $, which always has a rational point
$$ P_t := \left(-\dfrac{c}{d}, \dfrac{m}{d^2}\right). $$
It then suffices to show that $ P_t $ is almost always non-torsion.

\begin{lemma}[3.2]
For all but finitely many $ d \in K^\times / (K^\times)^2 $,
$$ E^d(K)_{\text{tors}} = \{\mathcal{O}, (a_1, 0), (a_2, 0), (a_3, 0)\}. $$
\end{lemma}

\vspace{-0.5cm}

\begin{proof}
If $ E^d(K)[p] $ is non-trivial for some prime $ p > 2 $, then $ \overline{\rho}_{E, p} $ factors through the quadratic character $ \psi_d $, but $ \overline{\rho}_{E, p} $ is almost always irreducible.
\end{proof}

\end{frame}

\begin{frame}[t]{Signs for the polynomials}

For $ t $ to satisfy P1 and P3, $ q_j $ need to satisfy additional conditions at the real places $ \sigma \in T $, obtained from enforcing the signs
$$ \operatorname{sgn}(\sigma(L_j(X, Y))) =
\begin{cases}
- & \text{if} \ j = 1 \ \text{and} \ \sigma = \tau_1, \tau_2, \\
& \quad \! j = 2 \ \text{and} \ \sigma = \tau_1, \tau_2, \tau_3, \tau_4, \\
& \quad \! j = 3 \ \text{and} \ \sigma = \tau_3, \tau_5, \tau_6, \\
\operatorname{sgn}(\sigma(\kappa)) & \text{if} \ j = 4 \ \text{and} \ \sigma \ne \tau_1, \dots, \tau_6, \\
+ & \text{otherwise}.
\end{cases}
$$
Along with K1, these conditions force $ t $ to be trivial at $ \sigma $.
\begin{itemize}
\item If $ \ell = 1, 2, 3 $, then $ \tau_\ell(q_1q_2q_3) > 0 $, $ \tau_\ell(q_4) > 0 $, and $ \tau_\ell(\kappa) > 0 $.
\item If $ \ell = 4, 5, 6 $, then $ \tau_\ell(q_1q_2q_3) < 0 $, $ \tau_\ell(q_4) > 0 $, and $ \tau_\ell(\kappa) < 0 $.
\item Otherwise, $ \sigma(q_1q_2q_3) > 0 $ and $ \sigma(q_4\kappa) > 0 $.
\end{itemize}
Furthermore, $ q_1 \equiv q_2 \equiv q_3 \equiv 1 \mod m\kappa $ and $ q_4\kappa \equiv \lambda\kappa \equiv 1 \mod m $, so that $ t $ is trivial at the primes in $ T $, and hence $ t $ satisfies P1.

\end{frame}

\begin{frame}[t]{Computation of Hilbert symbols}

Finally, since $ \sigma(q_1), \sigma(q_2), \sigma(q_3) > 0 $ at the real places $ \sigma \ne \tau_1, \dots, \tau_6 $ and since $ q_1 \equiv q_2 \equiv q_3 \equiv 1 \mod m\kappa $,
$$ (z_i, q_j)_v = 1, \qquad i = 1, \dots, 12, \qquad j = 1, 2, 3, $$
for the places $ v \in T' \setminus \{\tau_1, \dots, \tau_6\} $, so that
$$ \prod_{v \in T'} (z_i, q_j)_v = \prod_{\ell = 1}^6 (z_i, q_j)_{\tau_\ell}, \qquad i = 1, \dots, 12, \qquad j = 1, 2, 3. $$
Now $ (z_i, q_j)_{\tau_\ell} = -1 $ precisely if $ \tau_\ell(z_i), \tau_\ell(q_j) < 0 $, which occur when
\begin{align*}
(i, j, \ell)
& = (1, 1, 1), (1, 2, 1), (4, 1, 2), (4, 2, 2), (5, 2, 3), \\
& \qquad (5, 3, 3), (8, 2, 4), (9, 3, 5), (12, 3, 6).
\end{align*}
These are precisely the Hilbert symbol conditions enforced in P3, noting that it does not enforce conditions for $ (i, j) = (1, 2), (4, 2), (5, 3) $.

\end{frame}

\begin{frame}[t]{Admissibility of the polynomials}

Observe that $ L_j(X, Y) $ form an admissible family, in the sense that they satisfy the analogue of Bunyakovsky's property in Dickson's conjecture.

\begin{lemma}[5.5]
For any prime $ \mathfrak{p} $ of $ K $, there are $ x, y \in \mathcal{O}_K $ such that
$$ \mathfrak{p} \nmid L_1(x, y)L_2(x, y)L_3(x, y)L_4(x, y). $$
\end{lemma}

\vspace{-0.5cm}

\begin{proof}
If $ \mathfrak{p} \mid m\kappa $, then $ L_1(0, 0) = L_2(0, 0) = L_3(0, 0) = 1 $, and $ L_4(0, 0) = \lambda $ is coprime to $ m\kappa $. Otherwise $ \mathfrak{p} \nmid m\kappa $, then there is some $ y \in \mathcal{O}_K $ such that $ \mathfrak{p} \nmid m^2\kappa y + \lambda $, so that $ \mathfrak{p} \nmid L_4(x, y) $ for any $ x \in \mathcal{O}_K $. On the other hand, $ \#(\mathcal{O}_K / \mathfrak{p}) \ge 5 $ since $ \mathfrak{p} \nmid 6 $, so that there is some $ x \in \mathcal{O}_K $ such that
$$ x \not\equiv \dfrac{-a_1d(y) - 1}{m^2\kappa}, \dfrac{-a_2d(y) - 1}{m^2\kappa}, \dfrac{-a_3d(y) - 1}{m^2\kappa} \mod \mathfrak{p}, $$
and hence $ \mathfrak{p} \nmid L_1(x, y)L_2(x, y)L_3(x, y) $.
\end{proof}

\end{frame}

\begin{frame}[t]{Statement of Kai's theorem}

A version of Kai's theorem can be stated as follows.

\begin{theorem}[A.8]
Let $ \phi_1, \dots, \phi_k : \mathbb{Z}^d \to \mathcal{O}_K $ be affine linear forms for some $ d \in \mathbb{N}_{> 1} $ such that the restriction of $ \phi_j $ to the kernel of $ \phi_{j'} $ has finite cokernel whenever $ j \ne j' $, and let $ \Omega \subseteq \mathbb{R}^d $ be a convex region such that the volume of $ \Omega_N := \Omega \cap [-N, N]^d $ is asymptotically $ N^d $. Then
$$ \sum_{\vec{x} \in \Omega_N \cap \mathbb{Z}^d} \prod_{j = 1}^k \Lambda_K(\phi_j(\vec{x})) \sim \dfrac{N^d}{\operatorname{res}_{s = 1} \zeta_K(s)^k} \cdot \prod_p \beta_p, $$
where $ \Lambda_K := \Lambda \circ \operatorname{Nm}_{K / \mathbb{Q}} $ is the von Mangoldt function for $ K $ and
$$ \beta_p := \left(\dfrac{p^n}{\#(\mathcal{O}_K / \langle p\rangle)^\times}\right)^k \cdot \dfrac{\#\{\vec{x} \in \mathbb{F}_p^d : \mathfrak{p} \nmid \prod_{j = 1}^k \phi_j(\vec{x}) \ \text{for all} \ \mathfrak{p} \mid p\}}{p^d}. $$
\end{theorem}

Note that the full version considers affine linear forms $ \mathbb{Z}^d \to I $ with a uniformity condition over all fractional ideals $ I $ of $ K $.

\end{frame}

\begin{frame}[t]{Assumptions in Kai's theorem}

For each $ j = 1, 2, 3, 4 $, the $ 1 $-homogeneous part of $ L_j(X, Y) $ defines an affine linear form $ \phi_j : \mathbb{Z}^{2n} \to \mathcal{O}_K $ by fixing a basis of $ \mathcal{O}_K $ over $ \mathbb{Z} $. If $ j \ne j' $, then $ \phi_j(\vec{x}, \vec{y}) = 0 $ implies that $ \phi_{j'}(\vec{x}, \vec{y}) \not\equiv 0 $ since $ a_j \ne a_{j'} $, or in other words that the restriction of $ \phi_j $ to the kernel of $ \phi_{j'} $ has finite cokernel.

\vspace{0.5cm} The signs enforced on $ \sigma(L_j(X, Y)) $ define a convex region $ \Omega \subseteq \mathbb{R}^{2n} $.

\begin{lemma}[5.6, 5.7]
The volume of $ \Omega_N $ is asymptotically $ N^{2n} $.
\end{lemma}

\begin{proof}[Sketch of proof]
For a surjective linear operator $ T : \mathbb{R}^{2n} \to \mathbb{R}^{2r} $, the volume of
$$ T^{-1}\left(\prod_{\ell = 1}^{2r} (x_\ell, \infty)\right) \cap [-N, N]^{2n}, \qquad (x_1, \dots, x_{2r}) \in \mathbb{R}^{2r} $$
is asymptotically $ N^{2n} $. While $ \Omega $ is defined by $ 4r $ embeddings, the signs enforced on $ \sigma(L_j(X, Y)) $ and $ \sigma(a_j) $ reduce this to $ 2r $ embeddings.
\end{proof}

\end{frame}

\begin{frame}[t]{Intuition for Kai's theorem}

Assume now that the coefficients of $ \phi_j : \mathbb{Z}^d \to \mathcal{O}_K $ are fixed. If $ \phi_j(\vec{x}) $ is prime for some $ \vec{x} \in \Omega_N \cap \mathbb{Z}^d $, then $ \Lambda_K(\phi_j(\vec{x})) \sim \log dN $, since composite prime powers are asymptotically negligible compared to primes. Thus
$$ \sum_{\vec{x} \in \Omega_N \cap \mathbb{Z}^d} \prod_{j = 1}^k \Lambda_K(\phi_j(\vec{x})) \sim \#S_N \cdot \log^k dN, $$
where $ S_N $ is the set of $ \vec{x} \in \Omega_N \cap \mathbb{Z}^d $ such that $ \phi_1(\vec{x}), \dots, \phi_r(\vec{x}) $ are simultaneously prime. Kai's theorem then says $ \#S_N > 0 $ whenever $ \prod_p \beta_p > 0 $, which is equivalent to the admissibility of $ \phi_j $.

\vspace{0.5cm} Note that when $ K = \mathbb{Q} $, this says that
$$ \#S_N \sim \dfrac{N^d}{\log^k dN} \cdot \prod_p \left(\dfrac{p}{p - 1}\right)^k \cdot \dfrac{\#\{\vec{x} \in \mathbb{F}_p^d : p \nmid \prod_{j = 1}^k \phi_j(\vec{x})\}}{p^d}, $$
which is simply a multivariate version of the Hardy--Littlewood conjecture.

\end{frame}

\end{document}