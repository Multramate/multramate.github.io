\documentclass[10pt]{beamer}

\newtheorem{conjecture}{Conjecture}
\newtheorem{exercise}{Exercise}

\setbeamertemplate{footline}[page number]

\title{Elliptic divisibility sequences in Lean}
\subtitle{British Mathematical Colloquium}
\author{David Kurniadi Angdinata (joint with Junyan Xu)}
\institute{London School of Geometry and Number Theory}
\date{Wednesday, 25 June 2025}

\begin{document}

\frame{\titlepage}

\begin{frame}[t]{Formalising mathematics}

The process of formalising mathematics is interesting for many reasons.

\vspace{0.5cm} One important reason is to ensure that a mathematical argument is sound and complete, as the standard literature may sometimes be hazy.

\vspace{0.5cm} Throughout my PhD, I have been formalising the algebraic foundations of elliptic curves in the \textbf{Lean 4 theorem prover} as a side project.

\vspace{0.5cm} In the process, we accidentally discovered a novel purely algebraic proof of the group law on the points of an elliptic curve over a field.

\vspace{0.5cm} Two years ago, I was stuck trying to formalise a result on division points simply because the standard literature turned out to be incomplete.

\vspace{0.5cm} Since this is a joint session, \emph{there will be no Lean in this talk}!

\end{frame}

\begin{frame}[t]{Elliptic divisibility sequences}

Let $ W := \{W_n\}_{n \in \mathbb{Z}} $ be a sequence of elements in a commutative ring $ R $.

\vspace{0.5cm} Then $ W $ is
\begin{itemize}
\item a \textbf{divisibility sequence} if for all $ m, n \in \mathbb{Z} $,
$$ m \mid n \qquad \implies \qquad W_m \mid W_n, $$
\item an \textbf{elliptic sequence} if for all $ p, q, r \in \mathbb{Z} $,
$$ W_{p + q}W_{p - q}W_r^2 + W_{q + r}W_{q - r}W_p^2 + W_{r + p}W_{r - p}W_q^2 = 0, $$
\item an \textbf{elliptic divisibility sequence (EDS)} if it is simply a divisibility sequence that is elliptic.
\end{itemize}

\vspace{0.5cm} EDSs were first introduced by Morgan Ward (1948), where he studied their arithmetic properties in relation to elliptic curves.

\end{frame}

\begin{frame}[t]{Examples in nature}

There are many examples of EDSs in nature with integer values.

\begin{examples}
\begin{itemize}
\item The constant sequence $ W_n = 0 $ for all $ n \in \mathbb{Z} $ is an EDS.
\item The identity sequence $ W_n = n $ for all $ n \in \mathbb{Z} $ is an EDS.
\item If $ W $ is an EDS, then $ \{cW_n\}_{n \in \mathbb{Z}} $ is an EDS for any $ c \in \mathbb{Z} $.
\item The subsequence of even terms of the Fibonacci sequence is an EDS:
$$ 1, 3, 8, 21, 55, 144, 377, 987, 2584, 6765, 17711, 46368, 121393, \dots. $$
\item Certain subsequences of Lucas sequences $ L := \{L_n\}_{n \in \mathbb{Z}} $ given by $ L_1 = 1 $, $ L_2 = \ell $, and $ L_{n + 2} = \ell \cdot L_{n + 1} - L_n $ for all $ n > 2 $ are EDSs.
\item Certain generalised Somos-4 sequences $ a := \{a_n\}_{n \in \mathbb{Z}} $ given by $ a_na_{n - 4} = a_{n - 1}a_{n - 3} + a_{n - 2}^2 $ are EDSs. For instance, the generalised Somos-4 sequence with $ (a_1, a_2, a_3, a_4) = (1, 1, -1, 2) $ is an EDS:
$$ 0, 1, 1, -1, 2, 3, 1, -11, -16, 35, -129, -299, -386, 3977, 8063, \dots. $$
\end{itemize}
\end{examples}

\end{frame}

\begin{frame}[t]{Division polynomials}

Perhaps the most important example of an EDS is the sequence arising from division points on an elliptic curve $ E $ over a field $ F $.

\begin{exercise}[The Arithmetic of Elliptic Curves, 3.7(d)]
Prove that for any point $ (x : y : 1) $ on $ E $ we have for all $ n \in \mathbb{Z} $,
$$ [n](x : y : 1) = (\phi_{E, n}(x, y)\psi_{E, n}(x, y) : \omega_{E, n}(x, y) : \psi_{E, n}(x, y)^3). $$
\end{exercise}

Here, $ \phi_{E, n}, \omega_{E, n} \in F[X, Y] $ are defined in terms of \textbf{division polynomials} $ \psi_{E, n} \in F[X, Y] $. Then the sequence $ \psi_E := \{\psi_{E, n}\}_{n \in \mathbb{Z}} $ is an EDS.

\vspace{0.5cm} This is \emph{one path} to formalising the isomorphism of Galois representations
$$ T_pE_{\overline{F}} \cong \begin{cases} \mathbb{Z}_p^2 & \operatorname{char}(F) \ne p \\ 0 \ \text{or} \ \mathbb{Z}_p & \operatorname{char}(F) = p \end{cases}, $$
which is useful for Buzzard's formalisation of Fermat's last theorem.

\end{frame}

\begin{frame}[t]{Special cases}

EDSs can be generated easily by inspecting special cases of $ (p, q, r) \in \mathbb{Z}^3 $.

\vspace{0.5cm} Let $ p, q \in \mathbb{Z} $ be arbitrary, and let $ r = 0 $. Then
$$ W_{p + q}W_{p - q}W_0^2 + W_qW_qW_p^2 + W_pW_{-p}W_q^2 = 0. $$
Since $ W_0 \nmid W_n $ for any $ n \in \mathbb{Z} $, it is \emph{sensible} to set $ W_0 = 0 $. This forces $ W_pW_{-p}W_q^2 = -W_p^2W_q^2 $, so it is \emph{sensible} to set $ W_{-p} = -W_p $.

\vspace{0.5cm} If two of $ p, q, r > 0 $ are the same, say $ q = r $, then
$$ W_{p + q}W_{p - q}W_q^2 + W_{2q}W_0W_p^2 + W_{q + p}W_{q - p}W_q^2 = 0. $$
This is trivial, so assume that $ p > q > r > 1 $.
\begin{itemize}
\item If $ (p, q, r) = (3, 2, 1) $, then $ W_5W_1^3 + W_3^3W_1 - W_4W_2^3 = 0 $.
\item If $ (p, q, r) = (4, 2, 1) $, then $ W_6W_2W_1^2 + W_3W_1W_4^2 - W_5W_3W_2^2 = 0 $.
\end{itemize}
It turns out that all non-trivial relations can be generated this way.

\end{frame}

\begin{frame}[t]{Recursive cases}

If $ (p, q, r) = (n + 1, n, 1) $ for some $ n > 0 $, then
$$ W_{2n + 1}W_1^3 + W_{n + 1}^3W_{n - 1} - W_{n + 2}W_n^3 = 0. $$
If $ W_1 $ is not a zero divisor, then this gives a non-trivial relation
$$ W_{2n + 1} = \dfrac{W_{n + 2}W_n^3 - W_{n + 1}^3W_{n - 1}}{W_1^3} \qquad \text{for all} \ n > 1. $$
If $ (p, q, r) = (n + 1, n - 1, 1) $ for some $ n > 0 $, then
$$ W_{2n}W_2W_1^2 + W_nW_{n - 2}W_{n + 1}^2 - W_{n + 2}W_nW_{n - 1}^2 = 0. $$
If $ W_1 $ and $ W_2 $ are not zero divisors, then this gives a non-trivial relation
$$ W_{2n} = \dfrac{W_{n + 2}W_nW_{n - 1}^2 - W_nW_{n - 2}W_{n + 1}^2}{W_2W_1^2} \qquad \text{for all} \ n > 2. $$
Thus a sensible EDS is \emph{completely determined} by its first four values.

\end{frame}

\begin{frame}[t]{The canonical EDS}

Let $ a, b, c, d \in R $ such that $ a $ and $ b $ are not zero divisors. The \textbf{canonical EDS defined by $ (a, b, c, d) $} is the sequence $ C := \{C_n\}_{n \in \mathbb{Z}} $ given by
\begin{align*}
C_0 & := 0, \\
C_1 & := a, \\
C_2 & := ab, \\
C_3 & := ac, \\
C_4 & := abd, \\
C_{-n} & := -C_n & \text{for all} \ n < 0, \\
C_{2n + 1} & := \dfrac{C_{n + 2}C_n^3 - C_{n + 1}^3C_{n - 1}}{C_1^3} & \text{for all} \ n > 1, \\
C_{2n} & := \dfrac{C_{n + 2}C_nC_{n - 1}^2 - C_nC_{n - 2}C_{n + 1}^2}{C_2C_1^2} & \text{for all} \ n > 2.
\end{align*}
Now $ \psi_E $ is simply defined as $ C $, with parameters $ a, b, c, d $ given in terms of the coefficients of $ E $, but the fact that $ C $ is an EDS is \emph{not obvious}!

\end{frame}

\begin{frame}[t]{An infamous exercise}

\begin{exercise}[The Arithmetic of Elliptic Curves, 3.34(a)]
Prove that a sequence $ W := \{W_n\}_{n \in \mathbb{Z}} $ of elements of a field with $ W_1W_2W_3 \ne 0 $ is an EDS if and only if it satisfies the two conditions
\begin{align*}
W_{2n + 1}W_1^3 & = W_{n + 2}W_n^3 - W_{n + 1}^3W_{n - 1} & \text{for all} \ n > 1, \\
W_{2n}W_2W_1^2 & = W_{n + 2}W_nW_{n - 1}^2 - W_nW_{n - 2}W_{n + 1}^2 & \text{for all} \ n > 2.
\end{align*}
\end{exercise}

In the literature, \emph{every} complete argument I could find only proves this for $ W = \psi_E $ using complex analysis, but this is not covered until Chapter 6!

\vspace{0.5cm} An interesting conversation in Math Stack Exchange (paraphrased):

\begin{itemize}
\item Question (2013): how can this be done without elliptic functions?
\item Answer (2013): you can use the addition formulae and some algebra
\item Comment (2020): has anyone actually done the algebraic approach?
\item Reply (2020): I expect the answer is yes but I do not know who
\end{itemize}

\end{frame}

\begin{frame}[t]{Elliptic nets}

It turns out that you cannot solve Exercise 3.34(a) with direct induction: the inductive hypothesis is \emph{too weak} to establish the inductive step.

\vspace{0.5cm} Instead, it turns out that a canonical EDS $ C $ also satisfies the stronger relation of an \textbf{elliptic net}, that for all $ p, q, r, s \in \mathbb{Z} $,
\begin{align*}
\operatorname{EN}(p, q, r, s) \ : \ C_{p + q}C_{p - q}C_{r + s}C_{r - s}
\ = & \ C_{p + r}C_{p - r}C_{q + s}C_{q - s} \\
\ - & \ C_{q + r}C_{q - r}C_{p + s}C_{p - s}.
\end{align*}
Elliptic nets were first introduced and studied by Katherine Stange (2008), which generalise elliptic sequences by setting $ s = 0 $.

\vspace{0.5cm} Xu gave an elegant proof of this in Math Stack Exchange.

\begin{theorem}[Xu, 2024]
A canonical EDS is an elliptic net, and hence an elliptic sequence.
\end{theorem}

I will now briefly describe his inductive argument on four variables.

\end{frame}

\begin{frame}[t]{Xu's argument}

By $ C_{-n} = -C_n $, it suffices to prove $ \operatorname{EN}(p, q, r, s) $ by strong induction on $ p $ assuming that $ p, q, r, s > 0 $. Firstly,
$$ \operatorname{EN}(p, q, 1, 0) = \operatorname{EN}(\tfrac{p + q + 1}{2}, \tfrac{p + q - 1}{2}, \tfrac{p - q + 1}{2}, \tfrac{p - q - 1}{2}). $$

If $ p = q + 1 $, then $ \operatorname{EN}(q + 1, q, 1, 0) $ holds by definition of $ C_{2q + 1} $. Otherwise $ p > q + 1 $, then inductive hypothesis applies since $ \tfrac{p + q + 1}{2} < p $. This gives $ \operatorname{EN}(p, q, 1, 0) $ for all $ p, q > 1 $. Furthermore,

\vspace{-0.5cm}
$$
\arraycolsep=1pt
\begin{array}{rcrcrcr}
\scriptstyle \operatorname{EN}(p, q, r, 0) & = & \scriptstyle C_r^2 \cdot \operatorname{EN}(p, q, 1, 0) & - & \scriptstyle C_q^2 \cdot \operatorname{EN}(p, r, 1, 0) & + & \scriptstyle C_p^2 \cdot \operatorname{EN}(q, r, 1, 0), \\
\scriptstyle \operatorname{EN}(p, q, r, 1) & = & \scriptstyle C_{r + 1}C_{r - 1} \cdot \operatorname{EN}(p, q, 1, 0) & - & \scriptstyle C_{q + 1}C_{q - 1} \cdot \operatorname{EN}(p, r, 1, 0) & + & \scriptstyle C_{p + 1}C_{p - 1} \cdot \operatorname{EN}(q, r, 1, 0).
\end{array}
$$
\vspace{-0.25cm}

This gives $ \operatorname{EN}(p, q, r, 0) $ and $ \operatorname{EN}(p, q, r, 1) $ for all $ p, q, r > 1 $. Finally,

\vspace{-0.5cm}
$$
\arraycolsep=2pt
\begin{array}{rcrcrcr}
\scriptstyle \operatorname{EN}(p, q, r, s) & = & \scriptstyle C_q^2 \cdot \operatorname{EN}(p, r, s, 1) & + & \scriptstyle C_{q + 1}C_{q - 1} \cdot \operatorname{EN}(p, r, s, 0) & + & \scriptstyle C_{q + r}C_{q - r} \cdot \operatorname{EN}(p, s, 1, 0) \\
& - & \scriptstyle C_r^2 \cdot \operatorname{EN}(p, q, s, 1) & - & \scriptstyle C_{r + 1}C_{r - 1} \cdot \operatorname{EN}(p, q, s, 0) & - & \scriptstyle C_{q + s}C_{q - s} \cdot \operatorname{EN}(p, r, 1, 0) \\
& + & \scriptstyle C_s^2 \cdot \operatorname{EN}(p, q, r, 1) & + & \scriptstyle C_{s + 1}C_{s - 1} \cdot \operatorname{EN}(p, q, r, 0) & + & \scriptstyle C_{r + s}C_{r - s} \cdot \operatorname{EN}(p, q, 1, 0) \\
& - & \scriptstyle 2C_p^2 \cdot \operatorname{EN}(q, r, s, 1) & . & & &
\end{array}
$$
\vspace{-0.25cm}

This gives $ \operatorname{EN}(p, q, r, s) $ for all $ p, q, r, s > 1 $.

\end{frame}

\begin{frame}[t]{Final remarks}

Note that the complete argument also needs the case when $ p, q, r, s $ are all half-integers, so that $ \operatorname{EN}(\tfrac{p + q + 1}{2}, \tfrac{p + q - 1}{2}, \tfrac{p - q + 1}{2}, \tfrac{p - q - 1}{2}) $ is well-defined when $ p $ and $ q $ have the same parity, and this uses the definition of $ C_{2q} $.

\vspace{0.5cm} Xu's result also allows for the construction of an explicit \textbf{complement sequence} $ C^c := \{C_{m, n}^c\}_{m, n \in \mathbb{Z}} $ such that for all $ m, n \in \mathbb{Z} $,
$$ C_m \cdot C_{m, n}^c = C_{mn}, $$
which proves that $ C $ is a divisibility sequence, and hence an EDS.

\vspace{0.5cm} Finally, the fact that $ \omega_E $ is a sequence of \emph{polynomials} turned out to be highly non-trivial! Xu showed this by establishing the invariant
$$ \mathcal{I}_W(n) := \dfrac{W_{n - 1}^2W_{n + 2} + W_{n - 2}W_{n + 1}^2 + W_2^2W_n^3}{W_{n + 1}W_nW_{n - 1}}, \qquad n \in \mathbb{Z}, $$
for any EDS $ W $, which holds for $ C $ and in particular for $ \psi_E $.

\end{frame}

\end{document}