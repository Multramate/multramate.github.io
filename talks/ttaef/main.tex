\documentclass[10pt]{beamer}

\setbeamertemplate{footline}[page number]

\usepackage{tikz-cd}
\usepackage{transparent}

\begin{document}

\begin{frame}

\begin{center}

{\small Galois representations and root numbers}

\vspace{0.5cm}

{\scriptsize Tuesday, 22 November 2022}

\vspace{1cm}

\textbf{\large Tate's thesis \footnote{Tate (1950) \emph{Fourier analysis in number fields and Hecke's zeta-functions}} and epsilon factors}

\vspace{1cm}

David Ang

\vspace{0.5cm}

{\footnotesize University College London}

\end{center}

\end{frame}

\begin{frame}[t]{Overview}

Consider the Riemann $ \zeta $-function
$$ \zeta(s) := \sum_{n \in \mathbb{N}^+} \dfrac{1}{n^s}. $$

\only<2->{
\begin{theorem}[Riemann (1859)]
$ \zeta(s) $ has an analytic continuation to $ \mathbb{C} $ with simple poles at $ s = 0, 1 $ and satisfies a functional equation $ Z(s) = Z(1 - s) $ where
$$ Z(s) := \pi^{-\tfrac{s}{2}}\Gamma\left(\dfrac{s}{2}\right) \cdot \zeta(s). $$
\end{theorem}
}

\only<3->{\vspace{-0.5cm}
\begin{proof}[Sketch of proof]
Write $ Z(s) $ as a real integral of the theta series $ \Theta(z) := \sum_{n \in \mathbb{Z}} e^{-\pi n^2z} $. The Poisson summation formula for $ \mathbb{Z} \subset \mathbb{R} $ relates $ \Theta(z) $ and $ \Theta(1 / z) $.
\end{proof}
}

\only<4>{\vspace{0.5cm} Can you generalise this to a number field $ K $?}

\end{frame}

\begin{frame}[t]{Overview}

Can you generalise this to a number field $ K $?

\only<2->{\vspace{0.5cm} Consider the Dedekind $ \zeta $-function
$$ \zeta_K(s) := \sum_{0 \ne I \trianglelefteq \mathcal{O}_K} \dfrac{1}{\mathrm{Nm}(I)^s}. $$
}

\only<3->{
\begin{theorem}[Hecke (1917)]
$ \zeta_K(s) $ has an analytic continuation to $ \mathbb{C} $ with simple poles at $ s = 0, 1 $ and satisfies a functional equation $ Z_K(s) = Z_K(1 - s) $ where
$$ Z_K(s) := |\Delta_K|^{\tfrac{s}{2}} \cdot \left(\pi^{-\tfrac{s}{2}}\Gamma\left(\dfrac{s}{2}\right)\right)^{r_1} \cdot \left(2(2\pi)^{-s}\Gamma(s)\right)^{r_2} \cdot \zeta_K(s). $$
\end{theorem}
}

\only<4>{\vspace{-0.5cm}
\begin{proof}[Sketch of proof]
Write $ Z_K(s) $ as a real integral of a generalised theta series $ \Theta_K(s) $ and apply the Poisson summation formula for a lattice in $ \mathbb{R}^n $.
\end{proof}
}

\end{frame}

\begin{frame}[t]{Overview}

Can you generalise this to a number field $ K $?

\begin{theorem}[Hecke (1917)]
$ \zeta_K(s) $ has an analytic continuation to $ \mathbb{C} $ with simple poles at $ s = 0, 1 $ and satisfies a functional equation $ Z_K(s) = Z_K(1 - s) $ where
$$ Z_K(s) := |\Delta_K|^{\tfrac{s}{2}} \cdot \left(\pi^{-\tfrac{s}{2}}\Gamma\left(\dfrac{s}{2}\right)\right)^{r_1} \cdot \left(2(2\pi)^{-s}\Gamma(s)\right)^{r_2} \cdot \zeta_K(s). $$
\end{theorem}

Here are some basic questions.

\begin{itemize}
\item<2-> Can you explain the $ \Gamma $-factors in the functional equation?
\item<3-> Can you generalise this to $ L $-functions $ L(\chi, s) $ twisted by characters?
\end{itemize}

\only<4->{Tate (1950) answered both questions by giving a different proof of this.}

\only<5>{\vspace{0.5cm} Idea: lift $ \zeta_K(s) $ or $ L(\chi, s) $ into global $ \zeta $-integrals over the locally compact topological group of id\`eles $ \mathbb{A}_K^\times $ and apply techniques of Fourier analysis.}

\end{frame}

\begin{frame}[t]{Overview}

Idea: lift $ \zeta_K(s) $ or $ L(\chi, s) $ into global $ \zeta $-integrals over the locally compact topological group of id\`eles $ \mathbb{A}_K^\times $ and apply techniques of Fourier analysis.

\vspace{0.5cm}

Note that there is an Euler product
\only<1>{
$$ \zeta_K(s) = \prod_{v \in V_K^f} \dfrac{1}{1 - q_v^{-s}} = \prod_{v \in V_K^f} \left(\sum_{n = 0}^\infty q_v^{-ns}\right), $$
}%
\only<2->{
$$ Z_K(s) = |\Delta_K|^{\tfrac{s}{2}} \cdot \left(\pi^{-\tfrac{s}{2}}\Gamma\left(\dfrac{s}{2}\right)\right)^{r_1} \cdot \left(2(2\pi)^{-s}\Gamma(s)\right)^{r_2} \cdot \prod_{v \in V_K^f} \left(\sum_{n = 0}^\infty q_v^{-ns}\right), $$
}%
where $ V_K^f $ is the set of primes of $ K $.
\only<3->{On the other hand,
$$ \mathbb{A}_K^\times = (\mathbb{R}^\times)^{r_1} \times (\mathbb{C}^\times)^{r_2} \times \prod_{v \in V_K^f}^{\text{---}} K_v^\times. $$
}

\only<4>{Idea: the global $ \zeta $-integral over $ \mathbb{A}_K^\times $ is the product of local $ \zeta $-integrals over $ K_v^\times $, and the $ \Gamma $-factors are local $ \zeta $-integrals at the archimedean places.}

\end{frame}

\begin{frame}[t]{Local theory --- Fourier analysis}

Let $ F $ be a completion of a number field $ K_v $, so $ F / \mathbb{R} $ or $ F / \mathbb{Q}_p $.

\only<2->{\vspace{0.5cm} For $ F = \mathbb{R} $, the Fourier transform
$$ \widehat{f}(y) = \int_{-\infty}^\infty e^{-2\pi ixy}f(x) \ \mathrm{d}x $$
has three components.}
\only<3->{These are
\begin{itemize}
\item<3-> the integrable function $ f $,
\item<4-> the Lebesgue measure $ \mathrm{d}x $, and
\item<5-> the additive factor $ e^{-2\pi ixy} $.
\end{itemize}
}

\only<6>{\vspace{0.5cm} Each of these can be generalised for $ F = \mathbb{C} $ and $ F / \mathbb{Q}_p $.}

\end{frame}

\begin{frame}[t]{Local theory --- Haar measures}

How do you integrate over $ F^\times $?

\only<2->{\vspace{0.5cm} A locally compact topological group $ G $ can be endowed with a translation-invariant \textbf{Haar measure} $ \mu_G = \int \mathrm{d}_Gx $ unique up to scaling.}

\only<3->{
\begin{examples}
\begin{itemize}
\only<3-6>{\item Let $ \mathrm{d}_{\mathbb{R}}x := \mathrm{d}x $ be the Lebesgue measure}\only<4-6>{, and let
$$ \mathrm{d}_{\mathbb{R}^\times}x := \dfrac{\mathrm{d}_{\mathbb{R}}x}{|x|_{\mathbb{R}}}. $$
}
\only<5-6>{\item Let $ \mathrm{d}_{\mathbb{C}}(x + iy) := 2\mathrm{d}x\mathrm{d}y $ be twice the Lebesgue measure}\only<6>{, and let
$$ \mathrm{d}_{\mathbb{C}^\times}z := \dfrac{\mathrm{d}_{\mathbb{C}}z}{|x|_{\mathbb{C}}}. $$
}
\only<7->{\item Normalise $ \mathrm{d}_{\mathbb{Q}_p}x $ such that $ \mu_{\mathbb{Q}_p}(\mathbb{Z}_p) := 1 $}\only<8->{, so that
$$ \mu_{\mathbb{Q}_p}(a + p^n\mathbb{Z}_p) = \mu_{\mathbb{Q}_p}(p^n\mathbb{Z}_p) = p^{-n}\mu_{\mathbb{Q}_p}(\mathbb{Z}_p) = p^{-n}, $$
for all $ a \in \mathbb{Q}_p $}\only<9->{, and let
$$ \mathrm{d}_{\mathbb{Q}_p^\times}x := \dfrac{1}{1 - p^{-1}}\dfrac{\mathrm{d}_{\mathbb{Q}_p}x}{|x|_v}, $$
so that $ \mu_{\mathbb{Q}_p^\times}(\mathbb{Z}_p^\times) = 1 $.}
\only<10>{If $ G / \mathbb{Q}_p $, then $ \mu_G $ and $ \mu_{G^\times} $ should account for the valuation $ \delta_v $ of the different ideal $ \mathcal{D}_{G / \mathbb{Q}_p} \trianglelefteq \mathcal{O}_G $.}
\end{itemize}
\end{examples}
}

\end{frame}

\begin{frame}[t]{Local theory --- Schwartz-Bruhat functions}

What do you integrate over $ F^\times $?

\only<2->{\vspace{0.5cm} The \textbf{Schwartz-Bruhat} functions $ f : F \to \mathbb{C} $.}

\only<3-8>{
\begin{itemize}
\item If $ F = \mathbb{R} $, this is a function such that for all $ n \in \mathbb{N} $ and $ m \in \mathbb{N} $,
$$ \lim_{|x| \to \infty} \left(|x|^n\left|\dfrac{\mathrm{d}^mf}{\mathrm{d}x^m}\right|\right) = 0. $$
\end{itemize}
}

\only<4-5>{\vspace{-0.5cm}
\begin{example}
Let $ f(x) = f_0(x) := e^{-\pi x^2} $.
\only<5>{Then
\begin{align*}
\int_{\mathbb{R}^\times} f(x)|x|_\mathbb{R}^s \ \mathrm{d}_{\mathbb{R}^\times}x
& = 2\int_0^\infty e^{-\pi x^2}x^s \ \dfrac{\mathrm{d}x}{x} \\
& = \int_0^\infty e^{-y}\left(\dfrac{y}{\pi}\right)^{\tfrac{s}{2}} \ \dfrac{\mathrm{d}y}{y} & y = \pi x^2 \\
& = \pi^{-\tfrac{s}{2}}\Gamma\left(\dfrac{s}{2}\right)
=: \Gamma_{\mathbb{R}}(s).
\end{align*}
}
\end{example}
}

\only<6-8>{
\begin{itemize}
\item If $ F = \mathbb{C} $, this is a function such that for all $ n \in \mathbb{N} $ and $ m_1, m_2 \in \mathbb{N} $,
$$ \lim_{|x + iy| \to \infty} \left(|x + iy|_\mathbb{C}^n\left|\dfrac{\partial^{m_1 + m_2}f}{\partial x^{m_1}\partial y^{m_2}}\right|_\mathbb{C}\right) = 0. $$
\end{itemize}
}

\only<7-8>{\vspace{-0.5cm}
\begin{example}
Let $ f(z) = f_0(z) := \tfrac{1}{\pi}e^{-2\pi z\overline{z}} $.
\only<8>{Then
$$ \int_{\mathbb{C}^\times} f(z)|z|_\mathbb{C}^s \ \mathrm{d}_{\mathbb{C}^\times}z = \dots = 2(2\pi)^{-s}\Gamma(s) =: \Gamma_{\mathbb{C}}(s). $$
}
\end{example}
}

\only<9->{
\begin{itemize}
\item If $ F = K_v / \mathbb{Q}_p $, this is a linear combination of characteristic functions
$$ \mathbb{I}_{a + \pi_v^n\mathcal{O}_v}(x) =
\begin{cases}
1 & x \in a + \pi_v^n\mathcal{O}_v \\
0 & x \notin a + \pi_v^n\mathcal{O}_v
\end{cases}.
$$
\end{itemize}
}

\only<10->{\vspace{-0.5cm}
\begin{example}
Let $ f(x) = f_0(x) := \mathbb{I}_{\mathbb{Z}_p}(x) $.
\only<11->{Then
\begin{align*}
\int_{\mathbb{Q}_p^\times} f(x)|x|_p^s \ \mathrm{d}_{\mathbb{Q}_p^\times}x
& = \dfrac{1}{1 - p^{-1}}\int_{\mathbb{Z}_p} |x|_p^s \ \dfrac{\mathrm{d}_{\mathbb{Q}_p}x}{|x|_p} \\
& = \sum_{n = 0}^\infty \dfrac{p^{n - ns}}{1 - p^{-1}}\int_{p^n\mathbb{Z}_p \setminus p^{n + 1}\mathbb{Z}_p} \ \mathrm{d}_{\mathbb{Q}_p}x
= \sum_{n = 0}^\infty p^{-ns}.
\end{align*}
}
\only<12>{\vspace{-0.5cm} If $ F / \mathbb{Q}_p $, let $ f_0(x) := \mathbb{I}_{\mathcal{O}_F}(x) $ instead.}
\end{example}
}

\end{frame}

\begin{frame}[t]{Local theory --- additive characters}

Why do you need Schwartz-Bruhat functions?

\only<2->{\vspace{0.5cm} A Schwartz-Bruhat function $ f : F \to \mathbb{C} $ has a Fourier transform
$$ \widehat{f}(y) := \int_F \psi_F(xy)f(x) \ \mathrm{d}_Fx, $$
where $ \psi_F : F \to \mathbb{C} $ is an \textbf{additive character}.}

\begin{itemize}
\item<3-> If $ F = \mathbb{R} $, then $ \psi_\mathbb{R}(x) := e^{-2\pi ix} $.
\item<4-> If $ F = \mathbb{C} $, then $ \psi_\mathbb{C}(z) := e^{-2\pi i(z + \overline{z})} $.
\item<5-> If $ F = \mathbb{Q}_p $, then $ \psi_{\mathbb{Q}_p}(x) := e^{2\pi iy} $, where $ y \in \mathbb{Z}[p^{-1}] $ is such that $ x \in y + \mathbb{Z}_p $.
\only<6->{If $ F / \mathbb{Q}_p $, apply the trace $ \mathrm{Tr} : F \to \mathbb{Q}_p $ first.}
\end{itemize}

\only<7->{\vspace{0.5cm} These are defined in such a way so that the Fourier inversion formula $ \widehat{\widehat{f}}(x) = f(-x) $ holds, giving a duality between $ \psi_F $ and $ \mathrm{d}_Fx $.}
\only<8>{Indeed $ \widehat{\widehat{f_0}} = f_0 $, which is necessary in the Poisson summation formula.}

\end{frame}

\begin{frame}[t]{Local theory --- $ \zeta $-integrals}

What can you prove with this?

\only<2-5>{\vspace{0.5cm} Let $ f : F \to \mathbb{C} $ be a Schwartz-Bruhat function, and let $ \chi : F^\times \to \mathbb{C}^\times $ be a multiplicative character.}
\only<3-5>{The \textbf{local $ \zeta $-integral} is defined to be
$$ \zeta_F(f, \chi) := \int_{F^\times} f(x)\chi(x) \ \mathrm{d}_{F^\times} x, $$
which is independent of the dual pair $ (\psi_F, \mathrm{d}_Fx) $.}

\only<4-5>{
\begin{example}
$ \zeta_F(f_0, |\cdot|_F^s) $ are the $ \Gamma $-factors and local Euler factors.
\end{example}
}

\only<5->{
\begin{theorem}[Functional equation for the local $ \zeta $-integral]
There is a meromorphic function $ L_F : \mathrm{Hom}_{\mathrm{cts}}(F^\times, \mathbb{C}^\times) \to \mathbb{C}^\times $ and a holomorphic function $ \epsilon_F : \mathrm{Hom}_{\mathrm{cts}}(F^\times, \mathbb{C}^\times) \to \mathbb{C}^\times $ such that
$$ \dfrac{\zeta_F(\widehat{f}, \chi^{-1}|\cdot|_F)}{L_F(\chi^{-1}|\cdot|_F)} = \epsilon_F(\chi)\dfrac{\zeta_F(f, \chi)}{L_F(\chi)}. $$
\end{theorem}
\vspace{-0.5cm}}

\only<6->{\vspace{0.5cm} Here $ L_F(\chi) $ is the \textbf{local $ L $-factor} and $ \epsilon_F(\chi) $ is the \textbf{local $ \epsilon $-factor}, which are both independent of the choice of $ f $.}

\only<7>{\vspace{0.5cm} The \textbf{local root number} is then defined to be
$$ w_F(\chi) := \dfrac{\epsilon_F(\chi)}{|\epsilon_F(\chi)|} \in U(1). $$
}

\end{frame}

\begin{frame}[t]{Local theory --- $ \epsilon $-factors}

How do you compute $ \epsilon_F(\chi) $?

\only<2->{\vspace{0.5cm} Determine multiplicative characters $ \chi : F^\times \to \mathbb{C}^\times $ completely.}

\only<3-11>{
\begin{itemize}
\only<3-8>{\item Let $ F = \mathbb{R} $. Then
$$ \chi(x) = \eta(x)|x|_\mathbb{R}^s, \qquad \eta \in \{1, \mathrm{sgn}\}. $$
\vspace{-0.5cm}
\begin{itemize}
\item<4-8> If $ \eta = 1 $, set $ f(x) := f_0(x) = e^{-\pi x^2} $ and $ L_\mathbb{R}(\chi) := \Gamma_\mathbb{R}(s) $. \\
Then compute $ \epsilon_\mathbb{R}(\chi) = 1 $.
\item<5-8> If $ \eta = \mathrm{sgn} $, set $ f(x) := xe^{-\pi x^2} $ and $ L_\mathbb{R}(\chi) := \Gamma_\mathbb{R}(s + 1) $. \\
Then compute $ \epsilon_\mathbb{R}(\chi) = -i $.
\end{itemize}
}
\only<6-8>{\item Let $ F = \mathbb{C} $. Then
$$ \chi(z) = (z / \sqrt{z\overline{z}})^n|z|_\mathbb{C}^s, \qquad n \in \mathbb{Z}. $$
\vspace{-0.5cm}
\begin{itemize}
\item<7-8> If $ n = 0 $, set $ f(z) := f_0(z) = \tfrac{1}{\pi}e^{-2\pi z\overline{z}} $ and $ L_\mathbb{C}(\chi) := \Gamma_\mathbb{C}(s) $. \\
Then compute $ \epsilon_\mathbb{C}(\chi) = 1 $.
\item<8> In general, set $ f(z) := \tfrac{1}{\pi}z^ne^{-2\pi z\overline{z}} $ and $ L_\mathbb{C}(\chi) := \Gamma_\mathbb{C}(s + \tfrac{1}{2}|n|) $. \\
Then compute $ \epsilon_\mathbb{C}(\chi) = i^{-|n|} $.
\end{itemize}
}
\only<9-11>{\item Let $ F = K_v / \mathbb{Q}_p $. The \textbf{conductor} of $ \chi $ is the least $ n \in \mathbb{N} $ such that
$$ \chi((1 + \pi_v^n\mathcal{O}_v) \cap \mathcal{O}_v^\times) = 1. $$
If $ n = 0 $, then $ \chi $ is said to be \textbf{unramified}.
\begin{itemize}
\item<10-11> If $ n = 0 $, set $ f := \mathbb{I}_{\mathcal{O}_v} $ and $ L_{K_v}(\chi) := (1 - \chi(\pi_v)^{-1})^{-1} $. \\
Then compute
$$ \epsilon_{K_v}(\chi) = q_v^{\tfrac{\delta_v}{2}}\chi(\pi_v)^{\delta_v}. $$
\item<11> If $ n > 0 $, set $ f := \mathbb{I}_{1 + \pi_v^n\mathcal{O}_v} $ and $ L_{K_v}(\chi) := 1 $. \\
Then compute
$$ \epsilon_{K_v}(\chi) = \int_{K_v^\times} \psi_v(x)\chi(x)^{-1} \ \mathrm{d}_{K_v}x. $$
\end{itemize}
}
\end{itemize}
}

\only<12>{\renewcommand{\arraystretch}{2}
$$
\begin{array}{c|c|c|c}
F & \chi & L_F(\chi) & \epsilon_F(\chi) \\
\hline
\mathbb{R} & |x|_\mathbb{R}^s & \Gamma_\mathbb{R}(s) & 1 \\
\mathbb{R} & \mathrm{sgn}(x)|x|_\mathbb{R}^s & \Gamma_\mathbb{R}(s + 1) & -i \\
\mathbb{C} & (z / \sqrt{z\overline{z}})^n|z|_\mathbb{C}^s & \Gamma_\mathbb{C}(s + \tfrac{1}{2}|n|) & i^{-|n|} \\
K_v & \text{unramified} & (1 - \chi(\pi_v)^{-1})^{-1} & q_v^{\tfrac{\delta_v}{2}}\chi(\pi_v)^{\delta_v} \\
K_v & \text{ramified} & 1 & \int_{K_v^\times} \psi_v(x)\chi(x)^{-1} \ \mathrm{d}_{K_v}x
\end{array}
$$
}

\end{frame}

\begin{frame}[t]{Global theory --- ad\`eles and id\`eles}

Let $ V_K = V_K^f \cup V_K^\infty $ be the set of places of a number field $ K $.

\only<2-4>{\vspace{0.5cm} Consider the ad\`ele ring
$$ \mathbb{A}_K := \left\{(x_v)_{v \in V_K} \in \prod_{v \in V_K} K_v \ \middle| \ x_v \in \mathcal{O}_v \ \text{for almost all} \ v \in V_K\right\}. $$
}%
\only<3-4>{Its unit group is the id\`ele group
$$ \mathbb{A}_K^\times := \left\{(x_v)_{v \in V_K} \in \prod_{v \in V_K} K_v^\times \ \middle| \ x_v \in \mathcal{O}_v^\times \ \text{for almost all} \ v \in V_K\right\}. $$
}

\only<4>{
\begin{example}
If $ K = \mathbb{Q} $, then
$$ \mathbb{A}_\mathbb{Q} \cong \mathbb{R} \times \bigcup_{n \in \mathbb{N}^+} \dfrac{1}{n}\prod_{p < \infty} \mathbb{Z}_p. $$
\end{example}
}

\only<5->{\vspace{0.5cm} The id\`ele group is endowed with the restricted product topology such that
$$ \prod_{v \in S} U_v \times \prod_{v \in V_K \setminus S} \mathcal{O}_v^\times, $$
is an open basis for some finite $ V_K^\infty \subseteq S \subset V_K $ and some open $ U_v \subseteq K_v^\times $.}

\only<6->{\vspace{0.5cm} There is a diagonal embedding $ K^\times \hookrightarrow \mathbb{A}_K^\times $.}
\only<7->{By the product formula,
$$ |x|_{\mathbb{A}_K} := \prod_{v \in V_K} |x|_v = 1, \qquad x \in K^\times. $$
}%
\only<8>{By Tychonoff's theorem, both the id\`ele group $ \mathbb{A}_K^\times $ and the id\`ele class group $ C_K := \mathbb{A}_K^\times / K^\times $ are locally compact topological groups.}

\end{frame}

\begin{frame}[t]{Global theory --- Hecke characters}

A \textbf{Hecke character} is a character of the id\`ele class group, that is a continuous homomorphism $ C_K \to \mathbb{C}^\times $ with the discrete topology on $ \mathbb{C}^\times $.

\only<2-5>{
\begin{examples}
\begin{itemize}
\item<2-5> A Dirichlet character $ \phi : (\mathbb{Z} / n\mathbb{Z})^\times \to \mathbb{C}^\times $ induces a Hecke character
$$ C_\mathbb{Q} \cong \mathbb{R}^+ \times \prod_{p < \infty} \mathbb{Z}_p^\times \twoheadrightarrow \prod_{p \mid n} (\mathbb{Z}_p / n\mathbb{Z}_p)^\times \cong (\mathbb{Z} / n\mathbb{Z})^\times \xrightarrow{\phi} \mathbb{C}^\times $$
of finite order.
\only<3-5>{Indeed, Hecke characters of $ \mathbb{Q} $ of finite order correspond precisely to primitive Dirichlet characters of $ \mathbb{Q} $.}
\item<4-5> In fact, any Hecke character of $ \mathbb{Q} $ is of the form $ \eta|\cdot|_{\mathbb{A}_K}^s $ for some $ s \in \mathbb{C} $, where $ \eta $ is a Hecke character of finite order.
\item<5> In general, a Hecke character $ \chi : C_K \to \mathbb{C}^\times $ is uniquely determined by local multiplicative characters $ \chi|_{K_v^\times} : K_v^\times \to \mathbb{C}^\times $, which are unramified, so $ \chi|_{K_v^\times}(\mathcal{O}_v^\times) = 1 $, for almost all $ v \in V_K $.
\end{itemize}
\end{examples}
}

\only<6->{\vspace{0.5cm} A \textbf{Hecke L-function} of $ \chi $ is
$$ L(\chi) := \prod_{v \in V_K^f} L_{K_v}(\chi|_{K_v^\times}), $$
where $ L_{K_v} $ are the local $ L $-factors
$$ L_{K_v}(\chi) =
\begin{cases}
(1 - \chi(\pi_v))^{-1} & \chi \ \text{is unramified} \\
1 & \chi \ \text{is not unramified}
\end{cases}.
$$
}

\only<7->{\vspace{-0.5cm}
\begin{examples}
\begin{itemize}
\item<7-> If $ \chi = |\cdot|_{\mathbb{A}_K}^s $, then $ L(\chi) $ is the Dedekind $ \zeta $-function $ \zeta_K(s) $.
\item<8> If $ K = \mathbb{Q} $ and $ \chi $ has finite order, then $ L(\chi) $ is the Dirichlet $ L $-function of a primitive Dirichlet character.
\end{itemize}
\end{examples}
}

\end{frame}

\begin{frame}[t]{Global theory --- Fourier analysis}

The three components for the global Fourier transform are simply defined as the product of their local counterparts with the unramified condition.
\begin{itemize}
\item<2-> The global Schwartz-Bruhat functions on $ \mathbb{A}_K $ are linear combinations of products of local Schwartz-Bruhat functions $ f_v : K_v \to \mathbb{C} $ for all $ v \in V_K $, such that $ f_v = \mathbb{I}_{\mathcal{O}_v} $ for almost all $ v \in V_K $.
\item<3-> The global Haar measure on $ \mathbb{A}_K $ is such that
$$ \int_{\mathbb{A}_K} f(x) \ \mathrm{d}_{\mathbb{A}_K}x := \prod_{v \in V_K} \int_{K_v} f|_{K_v}(x) \ \mathrm{d}_{K_v}x. $$
\item<4-> The global additive character on $ \mathbb{A}_K $ is such that
$$ \psi_{\mathbb{A}_K}((x_v)_{v \in V_K}) := \prod_{v \in V_K} \psi_{K_v}(x_v). $$
\end{itemize}
\only<5>{By construction, since the Fourier inversion formula holds in all completions of $ K $, the Poisson summation formula holds in $ \mathbb{A}_K $.}

\end{frame}

\begin{frame}[t]{Global theory --- $ \zeta $-integrals}

\only<1-4>{Let $ f : \mathbb{A}_K \to \mathbb{C} $ be a Schwartz-Bruhat function, and let $ \chi : C_K \to \mathbb{C}^\times $ be a Hecke character.}
\only<2-4>{The \textbf{global $ \zeta $-integral} is defined to be
$$ \zeta(f, \chi) := \prod_{v \in V_K} \zeta_{K_v}(f|_{K_v^\times}, \chi|_{K_v^\times}), $$
which is an infinite product.}

\only<3-4>{
\begin{theorem}[Functional equation for the global $ \zeta $-integral]
$ \zeta $ has a meromorphic continuation to $ \mathbb{C} $ and satisfies a functional equation
$$ \zeta(f, \chi) = \zeta(\widehat{f}, \chi^{-1}|\cdot|_{\mathbb{A}_K}). $$
\end{theorem}
}

\only<4>{
\begin{proof}[Sketch of proof]
The Poisson summation formula $ \mathbb{A}_K $ relates $ f $ and $ \widehat{f} $.
\end{proof}
}

\only<5->{
\begin{theorem}[Tate (1950)]
$ L(\chi) $ has a meromorphic continuation to $ \mathbb{C} $ and satisfies a functional equation $ \Lambda(\chi) = \epsilon(\chi)\Lambda(\chi^{-1}|\cdot|_{\mathbb{A}_K}) $ where
$$ \Lambda(\chi) := L_\mathbb{R}(s)^{r_1} \cdot L_\mathbb{C}(s)^{r_2} \cdot L(\chi), \qquad \epsilon(\chi) := \prod_{v \in V_K} \epsilon_{K_v}(\chi). $$
\end{theorem}
}

\only<6->{Here $ \epsilon(\chi) $ is the \textbf{global $ \epsilon $-factor}, and similarly the \textbf{global root number} is defined to be $ w(\chi) := \prod_{v \in V_K} w_{K_v}(\chi) \in U(1) $.}

\only<7>{
\begin{proof}
The product of the functional equations for the local $ \zeta $-integrals is
$$ \dfrac{\zeta(\widehat{f}, \chi^{-1}|\cdot|_{\mathbb{A}_K})}{\Lambda(\chi^{-1}|\cdot|_{\mathbb{A}_K})} = \epsilon(\chi)\dfrac{\zeta(f, \chi)}{\Lambda(\chi)}. $$
Divide this by the functional equation for the global $ \zeta $-integral.
\end{proof}
}

\end{frame}

\begin{frame}

\begin{center}
Thank you!
\end{center}

\end{frame}

\end{document}