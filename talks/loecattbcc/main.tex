\documentclass{article}

\usepackage{amssymb}
\usepackage{amsthm}
\usepackage{commath}
\usepackage[margin=1in]{geometry}
\usepackage[hidelinks]{hyperref}
\usepackage{mathtools}
\usepackage{stmaryrd}
\usepackage{tikz-cd}

\DeclareFontFamily{U}{wncyr}{}
\DeclareFontShape{U}{wncyr}{m}{n}{<->wncyr10}{}
\DeclareSymbolFont{cyr}{U}{wncyr}{m}{n}
\DeclareMathSymbol{\Sha}{\mathord}{cyr}{"58}

\theoremstyle{plain}
\newtheorem*{theorem}{Theorem}

\theoremstyle{definition}
\newtheorem*{example}{Example}

\newcommand{\BSD}{\mathrm{BSD}}
\renewcommand{\d}{\mathrm{d}}
\newcommand{\FF}{\mathbb{F}}
\newcommand{\Fr}{\mathrm{Fr}}
\newcommand{\Gal}{\mathrm{Gal}}
\newcommand{\im}{\mathrm{im}}
\renewcommand{\L}{\mathrm{L}}
\newcommand{\LLL}{\mathcal{L}}
\newcommand{\Nm}{\mathrm{Nm}}
\newcommand{\ord}{\mathrm{ord}}
\newcommand{\QQ}{\mathbb{Q}}
\newcommand{\Reg}{\mathrm{Reg}}
\newcommand{\rk}{\mathrm{rk}}
\newcommand{\Tam}{\mathrm{Tam}}
\newcommand{\tor}{\mathrm{tor}}
\newcommand{\tr}{\mathrm{tr}}
\newcommand{\ZZ}{\mathbb{Z}}

\newcommand{\abr}[2][-1]{\ensuremath{\mathinner{\ifthenelse{\equal{#1}{-1}}{\left\langle#2\right\rangle}{}\ifthenelse{\equal{#1}{1}}{\bigl\langle#2\bigr\rangle}{}}}}
\newcommand{\br}{\!\del}
\let\cb\cbr\renewcommand{\cbr}{\!\cb}
\newcommand{\fbr}[2][-1]{\ensuremath{\mathinner{\ifthenelse{\equal{#1}{-1}}{\left\lfloor#2\right\rfloor}{}\ifthenelse{\equal{#1}{1}}{\bigl\lfloor#2\bigr\rfloor}{}}}}
\let\sb\sbr\renewcommand{\sbr}{\!\sb}
\newcommand{\st}{\ \middle| \ }

\newcommand{\function}[5]{
  \begin{array}{rcrcl}
    #1 & : & \displaystyle #2 & \longrightarrow & \displaystyle #3 \\
       &   & \displaystyle #4 & \longmapsto     & \displaystyle #5
  \end{array}
}

\title{L-values of elliptic curves twisted by cubic characters}
\author{David Ang}
\date{Wednesday, 24 April 2024}

\begin{document}

\maketitle

\section{Motivational background}

Let $ E $ be an elliptic curve over $ \QQ $. Associated to $ E $ is its Hasse-Weil L-function
$$ \L\br{E, s} \coloneqq \prod_p \dfrac{1}{\det\br[1]{1 - p^{-s} \cdot \Fr_p^{-1} \mid \br[1]{\rho_{E, q}^\vee}^p}}, $$
where $ \Fr_p $ is an arithmetic Frobenius at a prime $ p $, and $ \rho_{E, q} $ is the $ q $-adic representation associated to the $ q $-adic Tate module of $ E $ for any prime $ q \ne p $. The algebraic and analytic properties of these L-functions are studied extensively in the literature, and they are the subject of many problems in the arithmetic of elliptic curves. Most notably, the Birch--Swinnerton-Dyer conjecture says that the order of vanishing $ r $ of $ \L\br{E, s} $ at $ s = 1 $ is precisely the Mordell-Weil rank $ \rk\br{E} $, and its leading term is given by
$$ \lim_{s \to 1} \dfrac{\L\br{E, s}}{\br{s - 1}^r} \cdot \dfrac{1}{\Omega\br{E}} = \dfrac{\Tam\br{E} \cdot \#\Sha\br{E} \cdot \Reg\br{E}}{\#\tor\br{E}^2}, $$
where $ \Omega\br{E} $ denotes the real period, $ \Tam\br{E} $ denotes the Tamagawa number, $ \Sha\br{E} $ denotes the Tate--Shafarevich group, $ \Reg\br{E} $ denotes the elliptic regulator, and $ \tor\br{E} $ denotes the torsion subgroup. As Tate once said, this remarkable conjecture relates the behaviour of a function $ \L\br{E, s} $ at a point where it is not at present known to be defined, to the order of a group $ \Sha\br{E} $ which is not known to be finite. Since then, the modularity theorem of Taylor--Wiles shows that $ \L\br{E, s} $ admits analytic continuation to the entire complex plane, and $ \Sha\br{E} $ is now known to be finite for $ r \le 1 $ thanks to the works of Gross--Zagier and Kolyvagin. For the sake of convenience, call the left hand side the algebraic L-value of $ E $, denoting it by $ \LLL\br{E} $, and call the right hand side the Birch--Swinnerton-Dyer quotient of $ E $, denoting it by $ \BSD\br{E} $.

When $ E $ is base changed to a finite Galois extension $ K $ of $ \QQ $, analogous quantities $ \L\br[0]{E / K, s} $, $ \rk\br[0]{E / K} $, $ \Omega\br[0]{E / K} $, $ \Tam\br[0]{E / K} $, $ \Sha\br[0]{E / K} $, $ \Reg\br[0]{E / K} $, and $ \tor\br[0]{E / K} $ can be defined to formulate a generalisation of the conjecture over $ K $. However, the modularity theorem has yet to be extended to elliptic curves beyond specific number fields, so the conjectural equality remains ill-defined in general. On the other hand, Artin's formalism for L-functions says that $ \L\br[0]{E / K, s} $ decomposes into a product of twisted L-functions
$$ \L\br{E, \rho, s} \coloneqq \prod_p \dfrac{1}{\det\br[1]{1 - p^{-s} \cdot \Fr_p^{-1} \mid \br[1]{\rho_{E, q}^\vee \otimes \rho^\vee}^p}}, $$
over all irreducible Artin representations $ \rho $ that factor through $ K $, so the behaviour of $ \L\br[0]{E / K, s} $ is completely governed by $ \L\br{E, \rho, s} $. These twisted L-functions can in turn be analytically continued to the entire complex plane by expressing them as Rankin-Selberg convolutions of $ \L\br{E, s} $, so the validity of the conjecture can be asked at the level of twisted L-functions. For instance, the Deligne--Gross conjecture states that the order of vanishing of $ \L\br{E, \rho, s} $ at $ s = 1 $ is precisely the multiplicity of $ \rho $ in the Artin representation associated to $ E\br{K} $. Analogous to the classical leading term conjecture that $ \LLL\br{E} = \BSD\br{E} $, the twisted leading term conjecture would be a statement about a twisted algebraic L-value $ \LLL\br{E, \rho} $ of $ E $. For the sake of simplicity, when $ K $ is a cyclotomic extension of $ \QQ $, the corresponding twisted algebraic L-value is given by
$$ \LLL\br{E, \chi} \coloneqq \lim_{s \to 1} \dfrac{\L\br{E, \chi, s}}{\br{s - 1}^r} \cdot \dfrac{p}{\tau\br{\chi}\Omega\br{E}}, $$
where $ \tau\br{\chi} $ is the Gauss sum of the primitive Dirichlet character $ \chi $ associated to $ K $. When $ E $ is semistable $ \Gamma_0 $-optimal of conductor $ N $ and $ \chi $ has prime conductor $ p \nmid N $ and order $ q > 1 $, it is known that $ \LLL\br{E, \chi} \in \ZZ\sbr[0]{\zeta_q} $.

\pagebreak

\section{Known results}

Unfortunately, there seems to be a barrier to formulating a twisted leading term conjecture for $ \LLL\br{E, \chi} $, even assuming classical leading term conjectures over general number fields. Dokchitser--Evans--Wiersema gave many explicit pairs of examples of elliptic curves $ E_1 $ and $ E_2 $ over $ \QQ $, with $ \LLL\br{E_1, \chi} \ne \LLL\br{E_2, \chi} $ for some fixed Dirichlet character $ \chi $, but are arithmetically identical over the number field $ K $ cut out by $ \chi $.

\begin{example}[DEW21, Example 45]
Let $ E_1 $ and $ E_2 $ be the elliptic curves given by the Cremona labels 1356d1 and 1356f1 respectively, and let $ \chi $ be the cubic character of conductor $ 7 $ such that $ \chi\br{3} = \zeta_3^2 $. Then $ \BSD\br{E_i} = \BSD\br{E_i / K} = 1 $ for $ i = 1, 2 $, but $ \LLL\br{E_1, \chi} = \zeta_3^2 $ and $ \LLL\br{E_2, \chi} = -\zeta_3^2 $.
\end{example}

This phenomenon can be partially explained with the assumption of standard arithmetic conjectures. For instance, under Stevens's Manin constant conjecture and the leading term conjectures over $ \QQ $ and over $ K $, Dokchitser--Evans--Wiersema expressed the norm of $ \LLL\br{E, \chi} $ in terms of Birch--Swinnerton-Dyer quotients.

\begin{theorem}[DEW21, Theorem 38]
Let $ E $ be a semistable $ \Gamma_0 $-optimal elliptic curve over $ \QQ $ of conductor $ N $, let $ \chi $ be a primitive Dirichlet character of odd prime conductor $ p \nmid N $ and odd prime order $ q \nmid \BSD\br{E}\#E\br{\FF_p} $, and let $ \zeta \coloneqq \chi\br{N}^{\br{q - 1} / 2} $. Then $ \LLL\br{E, \chi} \cdot \zeta \in \ZZ\sbr[0]{\zeta_q}^+ $, and has norm
$$ \Nm_\QQ^{\QQ\br[0]{\zeta_q}^+}\br{\LLL\br{E, \chi} \cdot \zeta} = \sqrt{\dfrac{\BSD\br[0]{E / K}}{\BSD\br{E}}}. $$
In particular, if $ \BSD\br{E} = \BSD\br[0]{E / K} $, then there is a unit $ u \in \ZZ\sbr[0]{\zeta_q}^+ $ such that $ \LLL\br{E, \chi} = u \cdot \zeta^{-1} $.
\end{theorem}

In the relevant case of $ \BSD\br{E} = \BSD\br[0]{E / K} $, this predicts the ideal of $ \QQ\br[0]{\zeta_q}^+ $ generated by $ \LLL\br{E, \chi} $, but not the precise value of $ \LLL\br{E, \chi} $. Note that in general, the exact prime ideal factorisation of $ \LLL\br{E, \chi} $ can be recovered from the $ \Gal\br[0]{K / \QQ} $-module structure of $ \Sha\br[0]{E / K} $ under stronger Iwasawa-theoretic assumptions.

From a purely analytic perspective, a natural problem is to determine the asymptotic distribution of $ \LLL\br{E, \chi} $ as $ \chi $ varies over primitive Dirichlet characters of some fixed prime order $ q $ but arbitrarily high prime conductor $ p \nmid N $, for some fixed elliptic curve $ E $ of conductor $ N $. However, for each such $ p $, there are $ q - 1 $ primitive Dirichlet characters $ \chi $ of conductor $ p $ and order $ q $, giving rise to $ q - 1 $ conjugates of $ \LLL\br{E, \chi} $, so a uniform choice of $ \chi $ for each $ p $ has to be made for any meaningful analysis. One solution is to observe that the residue class of $ \LLL\br{E, \chi} $ modulo $ \abr{1 - \zeta_q} $ is independent of the choice of $ \chi $, so a simpler problem would be to determine the asymptotic distribution of these residue classes instead. Let $ X_{E, q}^{< n} $ be the set of equivalence classes of primitive Dirichlet characters of odd order $ q $ and odd prime conductor $ p \nmid N $ less than $ n $, where two primitive Dirichlet characters in $ X_{E, q}^{< n} $ are equivalent if they have the same conductor. Define the residual densities $ \delta_{E, q} $ of $ \LLL\br{E, \chi} $ to be the natural densities of $ \LLL\br{E, \chi} $ modulo $ \abr{1 - \zeta_q} $, namely
$$ \delta_{E, q}\br{\lambda} \coloneqq \lim_{n \to \infty} \dfrac{\#\cbr{\chi \in X_{E, q}^{< n} \st \LLL\br{E, \chi} \equiv \lambda \mod \abr{1 - \zeta_q}}}{\#X_{E, q}^{< n}}, \qquad \lambda \in \FF_q, $$
if such a limit exists. Fixing six elliptic curves $ E $ and five small orders $ q $, Kisilevsky--Nam numerically computed $ \delta_{E, q} $ by varying $ \chi $ over millions of conductors $ p $, and observed inherent biases.

\begin{example}[KN22, Section 7]
Let $ E $ be the elliptic curve given by the Cremona label 11a1. Then
$$ \delta_{E, 3}\br{0} \approx \tfrac{3}{8}, \qquad \delta_{E, 3}\br{1} \approx \tfrac{3}{8}, \qquad \delta_{E, 3}\br{2} \approx \tfrac{1}{4}. $$
\end{example}

Note that their actual computational results seemingly give
$$ \delta_{E, 3}\br{0} \approx \tfrac{9}{24}, \qquad \delta_{E, 3}\br{1} \approx \tfrac{15}{24}, \qquad \delta_{E, 3}\br{2} \approx \tfrac{1}{24}, $$
but this is simply due to a difference in normalisation. Instead of considering the residual density of $ \LLL\br{E, \chi} $, they computed that of the norms of $ \LLL^+\br{E, \chi} / \gcd_{E, q} $, where
$$ \LLL^+\br{E, \chi} \coloneqq
\begin{cases}
\LLL\br{E, \chi} & \text{if} \ \chi\br{N} = 1 \\
\LLL\br{E, \chi} \cdot \br{1 + \overline{\chi\br{N}}} & \text{if} \ \chi\br{N} \ne 1
\end{cases},
$$
and $ \gcd_{E, q} $ is the greatest common divisor of these norms as $ \chi $ varies, which is determined empirically.

\pagebreak

\section{New results}

I refined the result of Dokchitser--Evans--Wiersema by predicting the precise value of $ \LLL\br{E, \chi} $ in terms of an abstract generator of the ideal of $ \QQ\br[0]{\zeta_q}^+ $ generated by $ \LLL\br{E, \chi} $. When $ \chi $ is cubic, this can be made explicit.

\begin{theorem}[Ang24, Corollary 5.2]
Let $ E $ be a semistable $ \Gamma_0 $-optimal elliptic curve over $ \QQ $ of conductor $ N $, and let $ \chi $ be a cubic primitive Dirichlet character of odd prime conductor $ p \nmid N $ such that $ 3 \nmid \BSD\br{E}\#E\br{\FF_p} $. Then
$$ \LLL\br{E, \chi} = u \cdot \overline{\chi\br{N}}\sqrt{\dfrac{\BSD\br[0]{E / K}}{\BSD\br{E}}}, $$
for some sign $ u = \pm1 $, chosen such that
$$ u \equiv -\#E\br{\FF_p}\sqrt{\dfrac{\BSD\br{E}^3}{\BSD\br[0]{E / K}}} \mod 3. $$
\end{theorem}

This clarifies the original example given by Dokchitser--Evans--Wiersema, as well as all of their other cubic examples, in the sense that $ \LLL\br{E_1, \chi} \ne \LLL\br{E_2, \chi} $ precisely because $ \#E_1\br{\FF_p} \not\equiv \#E_2\br{\FF_p} \mod 3 $.

\begin{example}[Ang24, Example 5.3]
Let $ E_1 $ and $ E_2 $ be the elliptic curves given by the Cremona labels 1356d1 and 1356f1 respectively, and let $ \chi $ be the cubic character of conductor $ 7 $ such that $ \chi\br{3} = \zeta_3^2 $. Then $ \LLL\br{E_i, \chi} = u \cdot \zeta_3^2 $ for $ u \equiv -\#E_i\br{\FF_7} \mod 3 $ for $ i = 1, 2 $, and indeed $ \#E_1\br{\FF_7} = 11 $ and $ \#E_2\br{\FF_7} = 7 $.
\end{example}

When $ \chi $ has order $ q > 3 $, the same proof only yields a congruence on the unit $ u \in \ZZ\sbr[0]{\zeta_q}^+ $ modulo $ q $, since the group of units of $ \ZZ\sbr[0]{\zeta_q}^+ $ is infinite. This does clarify all of the quintic examples given by Dokchitser--Evans--Wiersema with $ \BSD\br{E} = \BSD\br{E / K} $, in the sense that $ \LLL\br{E_1, \chi} \ne \LLL\br{E_2, \chi} $ precisely because $ \#E_1\br{\FF_p} \not\equiv \#E_2\br{\FF_p} \mod 5 $. Unfortunately, enforcing the congruence on $ \#E\br{\FF_p} $ modulo $ q $ remains insufficient to determine the precise value of $ \LLL\br{E, \chi} $, as the following rare example shows.

\begin{example}[Ang24, Remark 5.7]
Let $ E_1 $ and $ E_2 $ be the the elliptic curves given by the Cremona labels 544b1 and 544f1 respectively, and let $ \chi $ be the quintic character of conductor $ 11 $ such that $ \chi\br{2} = \zeta_5 $. Then $ \BSD\br{E_i} = \BSD\br{E_i / K} = 1 $, but $ \LLL\br{E_1, \chi} = -\zeta_5^3 - \zeta_5 $ and $ \LLL\br{E_2, \chi} = -2\zeta_5^3 - 3\zeta_5^2 - 2\zeta_5 $.
\end{example}

I also classified the possible residual densities of $ \LLL\br{E, \chi} $ in terms of the mod-$ q^m $ representations $ \overline{\rho_{E, q^m}} $.

\begin{theorem}[Ang24, Proposition 6.1]
Let $ E $ be a semistable $ \Gamma_0 $-optimal elliptic curve over $ \QQ $ such that $ \L\br{E, 1} \ne 0 $, and let $ q $ be an odd prime. If $ \ord_q\br{\BSD\br{E}} > 0 $, then $ \delta_{E, q}\br{0} = 1 $ and $ \delta_{E, q}\br{\lambda} = 0 $ for any $ \lambda \in \FF_q^\times $. Otherwise, for any $ \lambda \in \FF_q $,
$$ \delta_{E, q}\br{\lambda} = \dfrac{\#\cbr{M \in G_{E, q^m} \st 1 + \det\br{M} - \tr\br{M} \equiv -\lambda\BSD\br{E}^{-1} \mod q^m}}{\#G_{E, q^m}}, $$
where $ m \coloneqq 1 - \ord_q\br{\BSD\br{E}} $ and $ G_{E, q^m} \coloneqq \cbr{M \in \im\overline{\rho_{E, q^m}} \st \det\br{M} \equiv 1 \mod q} $, and furthermore if $ \overline{\rho_{E, q}} $ is surjective, then for any $ \lambda \in \FF_q $,
$$ \delta_{E, q}\br{\lambda} =
\begin{cases}
\tfrac{1}{q - 1} & \text{if} \ \lambda_{E, q} = 1 \\
\tfrac{q}{q^2 - 1} & \text{if} \ \lambda_{E, q} = 0 \\
\tfrac{1}{q + 1} & \text{if} \ \lambda_{E, q} = -1
\end{cases},
\qquad \lambda_{E, q} \coloneqq \br{\dfrac{\lambda\BSD\br{E}^{-1}}{q}}\br{\dfrac{\lambda\BSD\br{E}^{-1} + 4}{q}}. $$
\end{theorem}

When $ \chi $ is cubic, this can be made very explicit.

\begin{theorem}[Ang24, Theorem 6.4]
Let $ E $ be a semistable $ \Gamma_0 $-optimal elliptic curve over $ \QQ $ such that $ \L\br{E, 1} \ne 0 $. Then there is an explicit algorithm to determine the ordered triple $ \br{\delta_{E, 3}\br{0}, \delta_{E, 3}\br{1}, \delta_{E, 3}\br{2}} $ in terms of only $ \BSD\br{E} $ and $ \im\overline{\rho_{E, 9}} $. In particular, they can only be one of
$$ \br{1, 0, 0}, \ \br{\tfrac{3}{8}, \tfrac{3}{8}, \tfrac{1}{4}}, \ \br{\tfrac{3}{8}, \tfrac{1}{4}, \tfrac{3}{8}}, \ \br{\tfrac{1}{2}, \tfrac{1}{2}, 0}, \ \br{\tfrac{1}{2}, 0, \tfrac{1}{2}}, \ \br{\tfrac{1}{8}, \tfrac{3}{4}, \tfrac{1}{8}}, $$
$$ \br{\tfrac{1}{8}, \tfrac{1}{8}, \tfrac{3}{4}}, \ \br{\tfrac{1}{4}, \tfrac{1}{2}, \tfrac{1}{4}}, \ \br{\tfrac{1}{4}, \tfrac{1}{4}, \tfrac{1}{2}}, \ \br{\tfrac{5}{9}, \tfrac{2}{9}, \tfrac{2}{9}}, \ \br{\tfrac{1}{3}, \tfrac{2}{3}, 0}, \ \br{\tfrac{1}{3}, 0, \tfrac{2}{3}}. $$
\end{theorem}

This algorithm is in the form of two tables and will be omitted for brevity, but ultimately does recover the predicted residual densities in the six examples of Kisilevsky--Nam.

\pagebreak

\section{Proof ingredients}

The proofs of all of these results crucially rely on the following fundamental congruence.

\begin{theorem}[Ang24, Corollary 3.7]
Let $ E $ be a semistable $ \Gamma_0 $-optimal elliptic curve of conductor $ N $, and let $ \chi $ be a primitive Dirichlet character of odd prime conductor $ p \nmid N $ and order $ q > 1 $. Then
$$ \LLL\br{E, \chi} \equiv -\LLL\br{E}\#E\br{\FF_p} \mod \abr{1 - \zeta_q}. $$
\end{theorem}

This is a consequence of writing $ \L\br{E, 1} $ and $ \L\br{E, \chi, 1} $ as sums of modular symbols
$$ \mu_E\br{q} \coloneqq \int_0^q 2\pi if\br{z}\d z, $$
where $ f $ is the normalised cuspidal eigenform associated to $ E $ by the modularity theorem. Specifically, the Hecke action on the space of modular symbols and a modification of Birch's formula respectively give
$$ -\L\br{E, 1}\#E\br{\FF_p} = \sum_{a = 1}^{p - 1} \mu_E\br{\tfrac{a}{p}}, \qquad \L\br{E, \chi, 1} = \dfrac{\tau\br{\chi}}{n}\sum_{a = 1}^{p - 1} \overline{\chi\br{a}}\mu_E\br{\tfrac{a}{p}}. $$
By Manin's formalism for modular symbols, it turns out that $ \mu_E\br{q} + \mu_E\br{1 - q} $ is an integer multiple of $ \Omega\br{E} $ for any $ q \in \QQ $, so the modular symbols in both expressions can be paired up and normalised accordingly to give an expression for $ -\LLL\br{E}\#E\br{\FF_p} $ in $ \ZZ $ and an expression for $ \LLL\br{E, \chi} $ in $ \ZZ\sbr{\zeta_q} $. The congruence then follows immediately by comparing both integral expressions, noting that $ \overline{\chi\br{a}} \equiv 1 \mod \abr{1 - \zeta_q} $.

This essentially proves the algebraic result, while the analytic results require more work. As the conductor $ p $ of $ \chi $ varies over odd primes congruent to $ 1 $ modulo the order $ q $ of $ \chi $, the congruence says that $ \LLL\br{E, \chi} $ varies according to $ \#E\br{\FF_p} = 1 + \det\br[1]{\rho_{E, q}\br{\Fr_p}} - \tr\br[1]{\rho_{E, q}\br{\Fr_p}} $ modulo $ q $. On the other hand, $ \rho_{E, q}\br{\Fr_p} $ varies over $ G_{E, q^\infty} \coloneqq \cbr{M \in \im\rho_{E, q} \st \det\br{M} \equiv 1 \mod q} $, but Chebotarev's density theorem says that this is asymptotically uniformly distributed. It turns out that it suffices to compute densities in the finite group $ G_{E, q^m} $ rather than the infinite group $ G_{E, q^\infty} $, and $ m $ is bounded above by the following general result.

\begin{theorem}[Ang24, Theorem 4.4]
Let $ E $ be a semistable $ \Gamma_0 $-optimal elliptic curve over $ \QQ $ such that $ \L\br{E, 1} \ne 0 $, and let $ q $ be an odd prime. Then $ \ord_q\br{\LLL\br{E}} \ge -1 $ assuming the Birch--Swinnerton-Dyer conjecture. If $ E $ has no rational $ q $-isogeny, then $ \ord_q\br{\LLL\br{E}} \ge 0 $ unconditionally.
\end{theorem}

The proof of this turned out to be quite subtle, involving many cases using a multitude of recent results. Mazur's torsion theorem first reduces this to a finite number of cases depending on $ \tor\br{E} $, and all of which can be dealt with by Lorenzini's theorem on cancellations between torsion and Tamagawa numbers [Lor11, Proposition 1.1], except for when $ q = 3 $ and $ \tor\br{E} \cong \ZZ / 3\ZZ $. The proof of this last case follows from an application of Tate's algorithm, the aforementioned integrality of $ \LLL\br{E}\#E\br{\FF_p} $, and a case-by-case analysis on the possible mod-$ 3 $ and $ 3 $-adic Galois images of $ E $ classified by Rouse--Sutherland--Zureick-Brown [RSZB22, Corollary 1.3.1 and Corollary 12.3.3]. The analytic results can then be derived by computing the densities of $ \rho_{E, 3}\br{\Fr_p} $ in all possible finite groups $ G_{E, 3} $ and $ G_{E, 9} $ given by the same classification.

Finally, note that all hypotheses that $ E $ is semistable $ \Gamma_0 $-optimal can be weakened by considering Manin constants, which is possible thanks to \v Cesnavi\v cius's theorem on Manin constants [Ces18, Theorem 1.2].

\section*{References}

\begin{itemize}
\item[Ang24] D Angdinata (2024) L-values of elliptic curves twisted by cubic characters
\item[Ces18] K \v Cesnavi\v cius (2018) The Manin constant in the semistable case
\item[DEW21] V Dokchitser, R Evans, and H Wiersema (2021) On a BSD-type formula for L-values
of Artin twists of elliptic curves
\item[KN22] H Kisilevsky and J Nam (2022) Small algebraic central values of twists of elliptic L-functions
\item[Lor11] D Lorenzini (2011) Torsion and Tamagawa numbers
\item[RSZB22] J Rouse, A Sutherland, and D Zureick-Brown (2022) $ \ell $-adic images of Galois for elliptic curves over $ \QQ $
\end{itemize}

\end{document}