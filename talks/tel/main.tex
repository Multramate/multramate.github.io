\documentclass[10pt]{beamer}

\setbeamertemplate{footline}[page number]

\usetheme{Hannover}

\newtheorem{conjecture}{Conjecture}

\DeclareFontFamily{U}{wncyr}{}
\DeclareFontShape{U}{wncyr}{m}{n}{<->wncyr10}{}
\DeclareSymbolFont{cyr}{U}{wncyr}{m}{n}
\DeclareMathSymbol{\Sha}{\mathord}{cyr}{"58}

\title{Twisted elliptic L-values}

\author[David Ang]{David Kurniadi Angdinata}

\institute[LSGNT]{London School of Geometry and Number Theory}

\date[ENTR23]{\small Early Number Theory Researchers Workshop 2023 \\ \vspace{0.5cm} \footnotesize Friday, 25 August 2023}

\begin{document}

\frame{\titlepage}

\section{Introduction}

\begin{frame}[t]{A tale of two ranks}

Let $ E $ be an elliptic curve over $ \mathbb{Q} $, and let $ K $ be a number field.

\pause

\vspace{0.5cm}

\begin{theorem}[Mordell--Weil]
The set of $ K $-points $ E(K) $ is a finitely generated abelian group.
\end{theorem}

In particular, $ E(K) \cong \mathrm{tor}_{E / K} \times \mathbb{Z}^{\mathrm{rk}_{E / K}} $, where
\begin{itemize}
\item $ \mathrm{tor}_{E / K} $ is the \emph{torsion subgroup}, and
\item $ \mathrm{rk}_{E / K} $ is the \emph{(algebraic) rank}.
\end{itemize}
While $ \mathrm{tor}_{E / K} $ is classified, $ \mathrm{rk}_{E / K} $ remains mysterious.

\pause

\vspace{0.5cm}

\begin{conjecture}[Birch--Swinnerton-Dyer, weak form]
The order of vanishing of $ L_{E / K}(s) $ at $ s = 1 $ is equal to $ \mathrm{rk}_{E / K} $.
\end{conjecture}

This is called the \emph{analytic rank}.

\end{frame}

\begin{frame}[t]{L-functions}

For any $ G_K $-representation $ \rho $, its \textbf{local Euler factor} is given by
\vspace{-0.2cm} $$ \vspace{-0.2cm} L_\mathfrak{p}(\rho, T) := \det(1 - T \cdot \phi_\mathfrak{p} \mid \rho^{I_\mathfrak{p}}), $$
where $ \phi_\mathfrak{p} \in G_K $ is a Frobenius and $ I_\mathfrak{p} \le G_K $ is the inertia subgroup.

\pause

\vspace{0.5cm} The \textbf{(Hasse-Weil) L-function of $ E / K $} is given by
\vspace{-0.2cm} $$ \vspace{-0.2cm} L_{E / K}(s) := \prod_\mathfrak{p} \dfrac{1}{L_\mathfrak{p}(\rho_{E, \ell}, \mathrm{Nm}(\mathfrak{p})^{-s})}, $$
where $ \rho_{E, \ell} $ is the rational $ \ell $-adic Tate module as a $ G_K $-representation.

\pause

\begin{example}[$ K = \mathbb{Q} $]
Let $ a_p := 1 + p - \#E(\mathbb{F}_p) $. Then
\vspace{-0.2cm} $$ \vspace{-0.2cm} L_p(\rho_{E, \ell}, p^{-s}) =
\begin{cases}
1 - a_pp^{-s} + p^{1 - 2s} & p \nmid \Delta(E) \\
1 - a_pp^{-s} & p \mid \Delta(E)
\end{cases}.
$$
\end{example}

\end{frame}

\begin{frame}[t]{The BSD conjecture}

\begin{conjecture}[Birch--Swinnerton-Dyer, strong form]
The leading term of $ L_{E / K}(s) $ at $ s = 1 $ satisfies
\vspace{-0.2cm} $$ \vspace{-0.2cm} \lim_{s \to 1} \dfrac{L_{E / K}(s)}{(s - 1)^{\mathrm{rk}_{E / K}}} \cdot \dfrac{\sqrt{|\Delta_K|}}{\Omega_{E / K}} = \dfrac{C_{E / K} \cdot R_{E / K} \cdot \#\Sha_{E / K}}{\#\mathrm{tor}_{E / K}^2}. $$
\end{conjecture}

\pause

\vspace{-0.2cm} Here,
\begin{itemize}
\item $ \Omega_{E / K}$ is the \emph{global period},
\item $ C_{E / K} $ is the \emph{Tamagawa product},
\item $ R_{E / K} $ is the \emph{regulator}, where $ R_{E / K} = 1 $ if $ \mathrm{rk}_{E / K} = 0 $, and
\item $ \Sha_{E / K} $ is the \emph{Tate-Shafarevich group}, conjecturally finite.
\end{itemize}

\pause

\vspace{0.5cm} If $ \mathrm{rk}_{E / K} = 0 $, the LHS is called the \textbf{algebraic L-value}, given by
\vspace{-0.2cm} $$ \vspace{-0.2cm} \mathcal{L}_{E / K} := L_{E / K}(1) \cdot \dfrac{\sqrt{\Delta_K}}{\Omega_{E / K}}. $$

\end{frame}

\section{Dirichlet twists}

\begin{frame}[t]{Twisted L-functions}

Let $ K = \mathbb{Q}(\zeta_m) $, and let $ \chi : (\mathbb{Z} / m\mathbb{Z})^\times \to \mathbb{C}^\times $ be a Dirichlet character.

\pause

\vspace{0.5cm} The \textbf{(Hasse-Weil) L-function of $ E $ twisted by $ \chi $} is given by
\vspace{-0.2cm} $$ \vspace{-0.2cm} L_{E, \chi}(s) := \prod_p \dfrac{1}{L_p(\rho_{E, \ell} \otimes \chi, p^{-s})}. $$

\pause

Note that
\vspace{-0.2cm} $$ \vspace{-0.2cm} L_E(s) = \sum_{n \in \mathbb{N}} \dfrac{a_n}{n^s} \quad \overset{\chi}{\longrightarrow} \quad L_{E, \chi}(s) = \sum_{n \in \mathbb{N}} \dfrac{a_n\chi(n)}{n^s}. $$

\pause

\vspace{0.5cm} By representation theory, there is a factorisation
\vspace{-0.2cm} $$ \vspace{-0.2cm} L_{E / K}(s) = \prod_\chi L_{E, \chi}(s), $$
where $ \chi : (\mathbb{Z} / m\mathbb{Z})^\times \to \mathbb{C}^\times $ runs over primitive Dirichlet characters.

\end{frame}

\begin{frame}[t]{A twisted BSD conjecture}

\begin{conjecture}[Deligne--Gross]
The order of vanishing of $ L_{E, \chi}(s) $ at $ s = 1 $ is equal to $ \langle\chi, E(K)_\mathbb{C}\rangle $.
\end{conjecture}

\pause

\vspace{0.5cm} Unfortunately, a twisted version of strong BSD seems difficult.

\pause

\begin{example}[Dokchitser--Evans--Wiersema]
Let $ E_1 $ and $ E_2 $ be elliptic curves given by Cremona labels 307a1 and 307c1, and let $ \chi : (\mathbb{Z} / 11\mathbb{Z})^\times \to \mathbb{C}^\times $ be the primitive Dirichlet character of order $ 5 $ and conductor $ 11 $ given by $ \chi(2) = \zeta_5 $. \pause Then
\vspace{-0.2cm} $$ \vspace{-0.2cm} C_{E_i / K} = R_{E_i / K} = \Sha_{E_i / K} = \mathrm{tor}_{E_i / K} = 1, $$
for $ K \subseteq \mathbb{Q}(\zeta_{11})^+ $, \pause but
\vspace{-0.2cm} $$ \vspace{-0.2cm} \mathcal{L}_{E_1, \chi} = 1, \qquad \mathcal{L}_{E_2, \chi} = \zeta_5(1 + \zeta_5^4)^2. $$
\end{example}

\end{frame}

\begin{frame}[t]{Algebraic twisted L-values}

If $ \mathrm{rk}_E = 0 $, the \textbf{algebraic twisted L-value} is given by
\vspace{-0.2cm} $$ \vspace{-0.2cm} \mathcal{L}_{E, \chi} := L_{E, \chi}(1) \cdot \dfrac{\tau(\overline{\chi})}{\Omega_E}, $$
where $ \tau(\overline{\chi}) $ is the \textbf{Gauss sum}
\vspace{-0.2cm} $$ \vspace{-0.2cm} \tau(\overline{\chi}) := \sum_{n \in (\mathbb{Z} / m\mathbb{Z})^\times} \overline{\chi}(n)\zeta_m^n. $$

\pause

In general $ \mathcal{L}_{E, \chi} \in \overline{\mathbb{Q}} $, but some integrality statements are known.

\pause

\vspace{0.5cm}

\begin{theorem}[Wiersema--Wuthrich]
If $ E $ is semistable optimal of conductor $ N_E $, and if $ \chi $ is primitive of order $ k $ and conductor coprime to $ N_E $, then $ \mathcal{L}_{E, \chi} \in \mathbb{Z}[\zeta_k] $.
\end{theorem}

There are stronger statements under the \emph{Manin constant conjecture}.

\end{frame}

\section{Real values}

\begin{frame}[t]{Real algebraic twisted L-values}

Assume that $ \mathrm{rk}_E = 0 $, and that $ \chi $ is primitive of order $ k > 2 $.

\pause

\vspace{0.5cm}

\begin{lemma}[Kisilevsky--Nam]
Let $ \omega_E $ be the ``root number'' of $ E $. Then $ \lambda_\chi \cdot \mathcal{L}_{E, \chi} \in \mathbb{Z}[\zeta_k]^+ $, where
\vspace{-0.2cm} $$ \vspace{-0.2cm} \lambda_\chi :=
\begin{cases}
1 & \omega_E \cdot \chi(-N_E) = 1 \\
\chi(m) - \overline{\chi(m)} & \omega_E \cdot \chi(-N_E) = -1 \\
1 + \omega_E \cdot \overline{\chi(-N_E)} & \omega_E \cdot \chi(-N_E) \ne \pm 1
\end{cases},
\qquad m \in \mathbb{Z}. $$
\end{lemma}

\pause

This is called the \textbf{real algebraic twisted L-value} $ \mathcal{L}_{E, \chi}^+ $.

\pause

\begin{example}[$ k = 3 $]
\vspace{-0.2cm} $$ \vspace{-0.2cm} \mathbb{Z} \ni \mathcal{L}_{E, \chi}^+ =
\begin{cases}
\Re(\mathcal{L}_{E, \chi}) & \omega_E \cdot \chi(N_E) = 1 \\
2\Re(\mathcal{L}_{E, \chi}) & \omega_E \cdot \chi(N_E) \ne 1
\end{cases}.
$$
\end{example}

\end{frame}

\begin{frame}[t]{Some observations}

Let $ \chi : (\mathbb{Z} / p\mathbb{Z})^\times \to \mathbb{C}^\times $ run over primes $ p \equiv 1 \mod 3 $.

\pause

\begin{example}[Kisilevsky--Nam]
Let $ E $ be the elliptic curve given by the Cremona label 11a1.

\pause

\vspace{0.2cm}

\begin{scriptsize}
\begin{tabular}{r|rrrrrrrrrrr}
$ p $ & 7 & 13 & 19 & 31 & 37 & 43 & 61 & 67 & 73 & 79 & 97 \\
\hline
$ \mathcal{L}_{E, \chi}^+ $ & 5 & -10 & -10 & 5 & 20 & 5 & -10 & 15 & 5 & 15 & -30 \\
\visible<4-7>{$ \overline{\mathcal{L}}_{E, \chi}^+ $ & 1 & -2 & -2 & 1 & 4 & 1 & -2 & 3 & 1 & 3 & -6} \\
\visible<5-7>{$ [\overline{\mathcal{L}}_{E, \chi}^+]_3 $ & 1 & 1 & 1 & 1 & 1 & 1 & 1 & 0 & 1 & 0 & 0} \\
\visible<6-7>{$ [\#E(\mathbb{F}_p)]_3 $ & 1 & 1 & 2 & 1 & 2 & 2 & 2 & 0 & 1 & 0 & 0} \\
\visible<7>{$ \chi(N_E) $ & $ \zeta_3 $ & $ \zeta_3 $ & 1 & $ \overline{\zeta_3} $ & 1 & 1 & 1 & $ \overline{\zeta_3} $ & $ \zeta_3 $ & $ \overline{\zeta_3} $ & $ \overline{\zeta_3} $}
\end{tabular}
\end{scriptsize}

\vspace{0.2cm}

\begin{scriptsize}
\begin{tabular}{r|rrrrrrrrrr}
$ p $ & 103 & 109 & 127 & 139 & 151 & 157 & 163 & 181 & 193 & 199 \\
\hline
$ \mathcal{L}_{E, \chi}^+ $ & 30 & 5 & 15 & 5 & 0 & 0 & 80 & 50 & -5 & -55 \\
\visible<4-7>{$ \overline{\mathcal{L}}_{E, \chi}^+ $ & 6 & 1 & 3 & 1 & 0 & 0 & 16 & 10 & -1 & -11} \\
\visible<5-7>{$ [\overline{\mathcal{L}}_{E, \chi}^+]_3 $ & 0 & 1 & 0 & 1 & 0 & 0 & 1 & 1 & 2 & 1} \\
\visible<6-7>{$ [\#E(\mathbb{F}_p)]_3 $ & 0 & 1 & 0 & 1 & 0 & 0 & 1 & 1 & 1 & 2} \\
\visible<7>{$ \chi(N_E) $ & $ \zeta_3 $ & $ \overline{\zeta_3} $ & $ \overline{\zeta_3} $ & $ \zeta_3 $ & $ \zeta_3 $ & $ \zeta_3 $ & $ \overline{\zeta_3} $ & $ \overline{\zeta_3} $ & 1 & 1}
\end{tabular}
\end{scriptsize}

\vspace{0.2cm}

\end{example}

\pause

Here, $ \overline{\mathcal{L}}_{E, \chi}^+ := \mathcal{L}_{E, \chi}^+ / \gcd_{\chi'}\{\mathcal{L}_{E, \chi'}^+\} $.

\end{frame}

\begin{frame}[t]{Some phenomena}

If $ E $ is the elliptic curve given by the Cremona label 11a1,
\vspace{-0.2cm} $$ \vspace{-0.2cm} \overline{\mathcal{L}}_{E, \chi}^+ \equiv_3
\begin{cases}
0 & \#E(\mathbb{F}_p) \equiv 0 \mod 3 \\
2 & \#E(\mathbb{F}_p) \equiv 1 \mod 3 \ \text{and} \ \chi(N_E) = 1 \\
1 & \text{otherwise}
\end{cases}.
$$

\pause

KN computed $ \overline{\mathcal{L}}_{E, \chi}^+ $ modulo $ 3 $ for 11a1, 14a1, 15a1, 17a1, 19a1, 37b1.

\pause

\begin{itemize}
\item For 14a1, $ \overline{\mathcal{L}}_{E, \chi}^+ \equiv 2 \mod 3 $ often occurs.

\pause

\item For 11a1, 15a1, 17a1, $ \overline{\mathcal{L}}_{E, \chi}^+ \equiv 2 \mod 3 $ rarely occurs.

\pause

\item For 19a1, 37b1, $ \overline{\mathcal{L}}_{E, \chi}^+ \equiv 2 \mod 3 $ never occurs.
\end{itemize}

\pause

\vspace{0.5cm}

\begin{theorem}[A.]
I can partially explain the DEW and KN phenomena.
\end{theorem}

\end{frame}

\section{Modular symbols}

\begin{frame}[t]{The modularity theorem}

Let $ E $ be a semistable optimal elliptic curve over $ \mathbb{Q} $ of conductor $ N_E $.

\pause

\vspace{0.5cm}

\begin{theorem}[Taylor--Wiles]
There is an eigenform $ f_E \in S_2(\Gamma_0(N_E)) $ with (Hecke) L-function $ L_{f_E}(s) = L_E(s) $, such that the Hecke operator $ T_p $ has eigenvalue $ a_p $.
\end{theorem}

\pause

\vspace{0.5cm} For any cusp form $ f \in S_k(\Gamma) $, its L-function is a Mellin transform
\vspace{-0.2cm} $$ \vspace{-0.2cm} L_f(s) := \dfrac{(-2\pi i)^s}{\Gamma(s)}\int_0^\infty z^{s - 1}f(z)\mathrm{d}z. $$

\pause

Set $ s = 1 $:
\vspace{-0.2cm} $$ \vspace{-0.2cm} L_f(1) = -2\pi i\int_0^\infty f(z)\mathrm{d}z =: -\langle\{0, \infty\}, f\rangle. $$
This is a \emph{period} of the \emph{modular symbol} $ \{0, \infty\} $.

\end{frame}

\begin{frame}[t]{Classical modular symbols}

Let $ \mathcal{H} $ be the extended upper half plane, and let $ \phi : \mathcal{H} \twoheadrightarrow \mathcal{H} / \Gamma =: X_\Gamma $.

\pause

\vspace{0.5cm} A \textbf{modular symbol} is a class $ \{x, y\} \in H_1(X_\Gamma, \mathbb{R}) $ for any $ x, y \in \mathcal{H} $.

\pause

\begin{itemize}
\item If $ \Gamma \cdot x = \Gamma \cdot y $, then $ \phi(x \rightsquigarrow y) \in H_1(X_\Gamma, \mathbb{Z}) $, and conversely any $ \gamma \in H_1(X_\Gamma, \mathbb{Z}) $ arises from $ x, y \in \mathcal{H} $ in the same $ \Gamma $-orbit. Define
\vspace{-0.2cm} $$ \vspace{-0.2cm} \{x, y\} := \phi(x \rightsquigarrow y). $$

\pause

\item The map $ H_1(X_\Gamma, \mathbb{Z}) \to \mathrm{Hom}_\mathbb{C}(S_2(\Gamma), \mathbb{C}) $ given by $ \gamma \mapsto \langle\gamma, \cdot\rangle $ extends to $ \psi : H_1(X_\Gamma, \mathbb{R}) \xrightarrow{\sim} \mathrm{Hom}_\mathbb{C}(S_2(\Gamma), \mathbb{C}) $. Define
\vspace{-0.2cm} $$ \vspace{-0.2cm} \{x, y\} := \psi^{-1}\langle\phi(x \rightsquigarrow y), \phi^*(\cdot)\rangle. $$
\end{itemize}

\pause

Note that
\vspace{-0.2cm} $$ \vspace{-0.2cm} \{x, x\} = 0, \quad \{x, y\} = -\{y, x\}, \quad \{x, y\} + \{y, z\} = \{x, z\}, $$
but also
\vspace{-0.2cm} $$ \vspace{-0.2cm} \langle\{x, y\}, M \cdot f\rangle = \langle\{M \cdot x, M \cdot y\}, f\rangle, \qquad M \in \Gamma. $$

\end{frame}

\begin{frame}[t]{L-values as periods}

Let $ p \nmid N_E $. The Hecke operator $ T_p $ acts on $ H_1(X_\Gamma, \mathbb{Q}) $ by
\vspace{-0.2cm} $$ \vspace{-0.2cm} T_p \cdot \{x, y\} = \{px, py\} + \sum_{n = 0}^{p - 1} \{\tfrac{x + n}{p}, \tfrac{y + n}{p}\}. $$

\pause

\begin{lemma}[Manin]
\vspace{-0.6cm} $$ \vspace{-0.6cm} -\#E(\mathbb{F}_p) \cdot \mathcal{L}_E = \dfrac{1}{\Omega_E}\sum_{n = 1}^{p - 1} \langle\{0, \tfrac{n}{p}\}, f_E\rangle. $$
\end{lemma}

\pause

\begin{proof}
Set $ \{x, y\} = \{0, \infty\} $ in the Hecke action and apply the pairing $ \langle\cdot, f_E\rangle $:
\vspace{-0.2cm} $$ \vspace{-0.2cm} \only<3>{(1 + p - a_p)}\only<4-5>{\underbrace{(1 + p - a_p)}_{\#E(\mathbb{F}_p)}} \cdot \only<3>{\langle\{0, \infty\}, f_E\rangle}\only<4-5>{\underbrace{\langle\{0, \infty\}, f_E\rangle}_{-L_E(1)}} = \sum_{n = 1}^{p - 1} \only<3>{\langle\{0, \tfrac{n}{p}\}, f_E\rangle}\only<4-5>{\underbrace{\langle\{0, \tfrac{n}{p}\}, f_E\rangle}_{???}}. $$

\pause\pause

Multiply by $ \tfrac{1}{\Omega_E} $.
\end{proof}

\end{frame}

\begin{frame}[t]{Twisted L-values as periods}

Let $ \chi : (\mathbb{Z} / p\mathbb{Z})^\times \to \mathbb{C}^\times $ be a Dirichlet character.

\begin{lemma}[Manin]
\vspace{-0.6cm} $$ \vspace{-0.6cm} \mathcal{L}_{E, \chi} = \dfrac{1}{\Omega_E}\sum_{n = 1}^{p - 1} \overline{\chi}(n)\langle\{0, \tfrac{n}{p}\}, f_E\rangle. $$
\end{lemma}

\pause

\begin{proof}
For any $ m \in (\mathbb{Z} / p\mathbb{Z})^\times $, the discrete Fourier transform of $ \chi(m) $ is
\vspace{-0.2cm} $$ \vspace{-0.2cm} \chi(m) = \dfrac{1}{\tau(\overline{\chi})}\sum_{n = 1}^{p - 1} \overline{\chi}(n)\zeta_p^{mn}. $$

\pause

Substitute into $ \sum_m a_m\chi(m)q^m $ and apply the Mellin transform:
\vspace{-0.2cm} $$ \vspace{-0.2cm} L_{E, \chi}(1) = \dfrac{1}{\tau(\overline{\chi})}\sum_{n = 1}^{p - 1} \overline{\chi}(n)\only<3>{\langle\{0, \infty\}, M \cdot f_E\rangle}\only<4-5>{\underbrace{\langle\{0, \infty\}, M \cdot f_E\rangle}_{\text{apply properties}}}, \qquad M := \begin{pmatrix} p & k \\ 0 & p \end{pmatrix}. $$

\pause\pause

Multiply by $ \tfrac{\tau(\overline{\chi})}{\Omega_E} $.
\end{proof}

\end{frame}

\begin{frame}[t]{A congruence for L-values}

Let $ E $ be a semistable optimal elliptic curve over $ \mathbb{Q} $ of conductor $ N_E $, let $ p \nmid N_E $, and let $ \chi : (\mathbb{Z} / p\mathbb{Z})^\times \to \mathbb{C}^\times $ be a Dirichlet character. \pause Then

\vspace{-0.5cm}

\begin{align*}
-\#E(\mathbb{F}_p) \cdot \mathcal{L}_E & = \dfrac{1}{\Omega_E}\sum_{n = 1}^{p - 1} \langle\{0, \tfrac{n}{p}\}, f_E\rangle, \\
\mathcal{L}_{E, \chi} & = \dfrac{1}{\Omega_E}\sum_{n = 1}^{p - 1} \overline{\chi}(n)\langle\{0, \tfrac{n}{p}\}, f_E\rangle.
\end{align*}

\pause

\begin{corollary}[A]
If $ \chi $ has prime order $ k $, then
\vspace{-0.2cm} $$ \vspace{-0.2cm} -\#E(\mathbb{F}_p) \cdot \mathcal{L}_E \equiv \mathcal{L}_{E, \chi} \mod (1 - \zeta_k). $$
\end{corollary}

\begin{proof}
By integrality, the lemmata, and $ \overline{\chi} \equiv 1 \mod (1 - \zeta_k) $.
\end{proof}

\end{frame}

\begin{frame}[t]{A congruence for cubic twists}

Let $ \chi : (\mathbb{Z} / p\mathbb{Z})^\times \to \mathbb{C}^\times $ be a cubic primitive Dirichlet character.

\pause

\begin{corollary}[B]
\vspace{-0.4cm} $$ \vspace{-0.4cm} \mathcal{L}_{E, \chi}^+ \equiv_3 \#E(\mathbb{F}_p) \cdot \mathcal{L}_E \cdot
\begin{cases}
2 & \omega_E \cdot \chi(N_E) = 1 \\
1 & \omega_E \cdot \chi(N_E) \ne 1
\end{cases}.
$$
\end{corollary}

\begin{proof}
By cases of corollary (A).
\end{proof}

\pause

\begin{corollary}[C]
\vspace{-0.4cm} $$ \vspace{-0.4cm} \overline{\mathcal{L}}_{E, \chi}^+ \equiv_3 \#E(\mathbb{F}_p) \cdot \mathcal{L}_E \cdot \gcd_{\chi'}\{\mathcal{L}_{E, \chi'}^+\} \cdot
\begin{cases}
2 & \omega_E \cdot \chi(N_E) = 1 \\
1 & \omega_E \cdot \chi(N_E) \ne 1
\end{cases}.
$$
\end{corollary}

\begin{proof}
By cases of corollary (B).
\end{proof}

\end{frame}

\section{Explanations}

\begin{frame}[t]{DEW phenomena}

Recall that if $ E_1 $ and $ E_2 $ are elliptic curves given by Cremona labels 307a1 and 307c1, and $ \chi : (\mathbb{Z} / 11\mathbb{Z})^\times \to \mathbb{C}^\times $ is the primitive Dirichlet character of order $ 5 $ and conductor $ 11 $ given by $ \chi(2) = \zeta_5 $, then
\vspace{-0.2cm} $$ \vspace{-0.2cm} C_{E_i / K} = R_{E_i / K} = \Sha_{E_i / K} = \mathrm{tor}_{E_i / K} = 1, $$
for $ K \subseteq \mathbb{Q}(\zeta_{11})^+ $, but
\vspace{-0.2cm} $$ \vspace{-0.2cm} \mathcal{L}_{E_1, \chi} = 1, \qquad \mathcal{L}_{E_2, \chi} = \zeta_5(1 + \zeta_5^4)^2. $$

\pause

Note that $ \mathcal{L}_{E_i} = 1 $, and
\vspace{-0.2cm} $$ \vspace{-0.2cm} \#E_1(\mathbb{F}_{11}) = 9, \qquad \#E_2(\mathbb{F}_{11}) = 16, $$
so corollary (A) says
\vspace{-0.2cm} $$ \vspace{-0.2cm} \mathcal{L}_{E_1, \chi} \not\equiv \mathcal{L}_{E_2, \chi} \mod (1 - \zeta_5). $$

\pause

\vspace{0.5cm} In fact, corollary (A) partially explains all examples in DEW where $ \chi $ is quintic, and fully explains all examples in DEW where $ \chi $ is cubic.

\end{frame}

\begin{frame}[t]{Insufficiency of congruence}

Unfortunately, there are elliptic curves $ E_1 $ and $ E_2 $ over $ \mathbb{Q} $, where $ \mathcal{L}_{E_1, \chi} \equiv \mathcal{L}_{E_2, \chi} \mod (1 - \zeta_5) $, but $ \mathcal{L}_{E_1, \chi} \ne \mathcal{L}_{E_2, \chi} $.

\pause

\vspace{0.5cm}

\begin{example}[A.]
Let $ E_1 $ and $ E_2 $ be elliptic curves given by Cremona labels 130b3 and 312c3, and let $ \chi : (\mathbb{Z} / 11\mathbb{Z})^\times \to \mathbb{C}^\times $ be the primitive Dirichlet character of order $ 5 $ and conductor $ 11 $ given by $ \chi(2) = \zeta_5 $. \pause Then
\vspace{-0.2cm} $$ \vspace{-0.2cm} C_{E_i / K} = 2, \qquad R_{E_i / K} = 1, \qquad \Sha_{E_i / K} \cong \mathrm{tor}_{E_i / K} \cong (\mathbb{Z} / 2\mathbb{Z})^2, $$
for $ K \subseteq \mathbb{Q}(\zeta_{11})^+ $, \pause and furthermore $ \#E_i(\mathbb{F}_{11}) = 12 $ and $ \mathcal{L}_{E_i} = \tfrac{1}{2} $, \pause but
\vspace{-0.2cm} $$ \vspace{-0.2cm} \mathcal{L}_{E_1, \chi} = -4\zeta_5^3, \qquad \mathcal{L}_{E_2, \chi} = -4\zeta_5, $$
which are not equal but congruent modulo $ (1 - \zeta_5) $.
\end{example}

\pause

\vspace{0.5cm} Heuristically, the norm of $ \overline{\mathcal{L}}_{E, \chi}^+ $ is the $ \chi $-component of $ \Sha_E $.

\end{frame}

\begin{frame}[t]{KN phenomena}

Recall that if $ E $ is the elliptic curve given by the Cremona label 11a1,
\vspace{-0.2cm} $$ \vspace{-0.2cm} \overline{\mathcal{L}}_{E, \chi}^+ \equiv_3
\begin{cases}
0 & \#E(\mathbb{F}_p) \equiv 0 \mod 3 \\
2 & \#E(\mathbb{F}_p) \equiv 1 \mod 3 \ \text{and} \ \chi(N_E) = 1 \\
1 & \text{otherwise}
\end{cases}.
$$

\pause

Note that $ \mathcal{L}_E \cdot \gcd_{\chi'}\{\mathcal{L}_{E, \chi'}^+\} = 1 $ and $ \omega_E = 1 $, so corollary (C) says
\vspace{-0.2cm} $$ \vspace{-0.2cm} \overline{\mathcal{L}}_{E, \chi}^+ \equiv_3
\begin{cases}
2\#E(\mathbb{F}_p) & \chi(N_E) = 1 \\
\#E(\mathbb{F}_p) & \chi(N_E) \ne 1
\end{cases}.
$$

\pause

This fully explains the three cases, except for
\vspace{-0.2cm} $$ \vspace{-0.2cm} \#E(\mathbb{F}_p) \equiv 2 \mod 3 \quad \implies \quad \chi(N_E) = 1. $$

\pause

\vspace{0.5cm} In fact, corollary (C) fully explains $ \overline{\mathcal{L}}_{E, \chi}^+ $ modulo $ 3 $ for any $ E $ where
\begin{itemize}
\item $ E $ does not have rational $ 3 $-isogenies, and
\item the $ 3 $-division field $ \mathbb{Q}(x(E[3])) $ of $ E $ contains $ \sqrt[3]{N_E} $.
\end{itemize}

\end{frame}

\begin{frame}[t]{The missing piece}

\begin{theorem}[A.--Dokchitser]
Assume that
\begin{itemize}
\item $ E $ does not have rational $ 3 $-isogenies, and
\item the $ 3 $-division field $ \mathbb{Q}(x(E[3])) $ of $ E $ contains $ \sqrt[3]{N_E} $.
\end{itemize}
If $ \#E(\mathbb{F}_p) \equiv 2 \mod 3 $, then $ \chi(N_E) = 1 $.
\end{theorem}

\pause

\begin{proof}
The assumptions imply that
\vspace{-0.2cm} $$ \vspace{-0.2cm} \mathrm{Gal}(\mathbb{Q}(x(E[3])) / \mathbb{Q}) \cong \mathrm{PGL}_2(\mathbb{F}_3) \cong S_4. $$

\pause

This has a quotient
\vspace{-0.2cm} $$ \vspace{-0.2cm} \mathrm{Gal}(\mathbb{Q}(\sqrt[3]{N_E}, \zeta_3) / \mathbb{Q}) \cong S_4 / K_4 \cong S_3. $$

\pause

If $ \#E(\mathbb{F}_p) \equiv 2 \mod 3 $, then $ a_p \equiv 0 \mod 3 $, so $ \phi_p^2 = 1 $ in $ S_4 $. \pause By group theory, $ \phi_p = 1 $ in $ S_3 $, but it acts as $ \chi(N_E) $ on $ \mathbb{Q}(\sqrt[3]{N_E}, \zeta_3) $.
\end{proof}

\end{frame}

\begin{frame}[t]{Other KN phenomena}

Key idea: understand how $ \phi_p $ acts on $ \mathbb{Q}(x(E[3])) $ and $ \mathbb{Q}(\sqrt[3]{N_E}, \zeta_3) $.

\pause

\vspace{0.5cm} In general, $ \phi_p \ne 1 $ in $ \mathrm{Gal}(\mathbb{Q}(x(E[3])) / \mathbb{Q}) \le \mathrm{PGL}_2(\mathbb{F}_3) $.

\pause

\begin{itemize}
\item If $ E $ has rational $ 3 $-isogenies, then $ \overline{\mathcal{L}}_{E, \chi}^+ $ modulo $ 3 $ is partially explained by how $ \phi_p $ acts on the $ 9 $-division field $ \mathbb{Q}(x(E[9])) $.

\pause

\item If $ \mathbb{Q}(x(E[3])) $ does not contain $ \sqrt[3]{N_E} $, then $ \overline{\mathcal{L}}_{E, \chi}^+ $ modulo $ 3 $ is fully explained by how $ \phi_p $ acts on $ \mathbb{Q}(x(E[3]), \sqrt[3]{N_E}) $.
\end{itemize}

\pause

The elliptic curves given by Cremona labels 11a1, 15a1, 17a1 are generic, but those given by 14a1, 19a1, 37b1 are special.

\pause

\vspace{0.5cm}

\begin{theorem}[A.]
I understand how $ \phi_p $ acts on $ \mathbb{Q}(x(E[9]), \sqrt[3]{N_E}) $.
\end{theorem}

\pause

This crucially uses the classification of $ 3 $-adic images of Galois for elliptic curves over $ \mathbb{Q} $ by Rouse--Sutherland--Zureick-Brown.

\end{frame}

\section{}

\begin{frame}{Future work}

Here are some potential extensions, listed in increasing difficulty:
\begin{itemize}
\item replace $ \mathbb{Q} $ with a global field
\item replace $ \chi $ with an Artin representation
\item replace $ E $ with the Jacobian of a higher genus curve
\item remove the $ \mathrm{rk}_E = 0 $ assumption
\end{itemize}
Thank you!

\end{frame}

\end{document}