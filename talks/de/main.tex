\documentclass[10pt]{beamer}

\setbeamertemplate{footline}[page number]

\usepackage{multirow}
\usepackage{stmaryrd}

\theoremstyle{definition}
\newtheorem{answer}{Answer}
\newtheorem{conjecture}{Conjecture}
\newtheorem{question}{Question}

\title{Diophantine equations}

\author{David Ang}

\institute{University College London}

\date{Monday, 20 May 2024}

\begin{document}

\frame{\titlepage}

\begin{frame}[t]{A friendly problem}

\begin{center}
\includegraphics[width=0.7\textwidth]{values.png}
\end{center}

\begin{minipage}{0.4\textwidth}
Let's write:
\begin{itemize}
\item $ a $ for \textbf{a}pple
\item $ b $ for \textbf{b}anana
\item $ c $ for ananas \textbf{c}omosus
\end{itemize}
\end{minipage}
\begin{minipage}{0.3\textwidth}
Real values:
\begin{itemize}
\item $ a = 2 + \sqrt{3} $
\item $ b = 1 $
\item $ c = 0 $
\end{itemize}
\end{minipage}
\begin{minipage}{0.28\textwidth}
Integer values:
\begin{itemize}
\item $ a = 11 $
\item $ b = 4 $
\item $ c = -1 $
\end{itemize}
\end{minipage}

\end{frame}

\begin{frame}[t]{A fiendish problem}

\begin{center}
\includegraphics[width=0.7\textwidth]{positive_whole_values.png}
\end{center}

Smallest positive whole values:
\begin{itemize}
\item $ a = $ {\tiny $ 154476802108746166441951315019919837485664325669565431700026634898253202035277999 $}
\item $ b = $ {\tiny $ 36875131794129999827197811565225474825492979968971970996283137471637224634055579 $}
\item $ c = $ {\tiny $ 4373612677928697257861252602371390152816537558161613618621437993378423467772036 $}
\end{itemize}

\end{frame}

\begin{frame}[t]{Diophantine equations}

A \textbf{Diophantine equation} is a polynomial equation in two or more unknown variables with \emph{integer} coefficients.

\begin{examples}
\vspace{-0.3cm}
$$ X - 2Y - 3Z = 4 \qquad 2X^2 - 3XY + 4Y^2 - 5X + 6Y - 7 = 0 $$
$$ 3X^3 + 4Y^3 + 5Z^3 = 0 \qquad X^4 + Y^4 = Z^4 \qquad Y^2 = X^5 + 1 $$
\end{examples}

\vspace{0.5cm} To \textbf{solve} a Diophantine equation means to find its \emph{integer} solutions:
\begin{itemize}
\item Is there an integer solution?
\item Can we write down an integer solution?
\item Are there infinitely many integer solutions?
\item Can we generate new integer solutions from old ones?
\item Is there a way to write down all integer solutions?
\item Can we describe the distribution of integer solutions?
\end{itemize}

\end{frame}

\begin{frame}[t]{Fermat's last theorem}

\begin{columns}[T]

\begin{column}{0.8\textwidth}
In 1637, Pierre de Fermat claimed the following theorem.

\vspace{0.5cm}

\begin{conjecture}[Fermat's last theorem]
The only integer solutions to $ X^n + Y^n = Z^n $ for some $ n > 2 $ satisfy $ XYZ = 0 $.
\end{conjecture}
\end{column}

\begin{column}{0.1\textwidth}
\hspace{-1cm}
\includegraphics[width=1.5\textwidth]{fermat.jpg}
\end{column}

\end{columns}

\vspace{0.5cm} ``I have discovered a truly marvelous proof of this, which this margin is too narrow to contain.''

\vspace{0.5cm}

\begin{columns}[T]

\begin{column}{0.8\textwidth}
In 1995, Andrew Wiles published the first complete proof, which involved \emph{very advanced} 20th century mathematics.

\vspace{0.5cm} I think Fermat was mistaken.
\end{column}

\begin{column}{0.1\textwidth}
\hspace{-1cm}
\includegraphics[width=1.5\textwidth]{wiles.jpg}
\end{column}

\end{columns}

Why are Diophantine equations so difficult?

\end{frame}

\begin{frame}[t]{Hilbert's tenth problem}

\begin{columns}[T]

\begin{column}{0.8\textwidth}
In 1900, David Hilbert published a list of 23 unsolved problems ranging over all areas of mathematics.

\vspace{0.5cm}

\begin{question}[Hilbert]
Is there an algorithm to solve \emph{any} Diophantine equation?
\end{question}
\end{column}

\begin{column}{0.1\textwidth}
\hspace{-1cm}
\includegraphics[width=1.5\textwidth]{hilbert.jpg}
\end{column}

\end{columns}

\vspace{0.5cm}

\begin{answer}[Davis, Matiyasevich, Putnam, Robinson]
No.
\end{answer}

\begin{center}
\includegraphics[width=0.2\textwidth]{davis.jpg}
\hspace{0.5cm}
\includegraphics[width=0.1\textwidth]{matiyasevich.jpg}
\hspace{0.5cm}
\includegraphics[width=0.1\textwidth]{putnam.jpg}
\hspace{0.5cm}
\includegraphics[width=0.2\textwidth]{robinson.jpg}
\end{center}

\vspace{0.5cm} We have to get creative!

\end{frame}

\begin{frame}[t]{Overview}

Diophantine equations become more difficult to solve with more variables, so we will focus on two or three variables.

\vspace{0.5cm} Diophantine equations can also be classified by their degree, and the approaches to solve them typically depend on the degree.

\vspace{0.5cm} For the rest of the talk, we will consider the following examples:
\begin{itemize}
\item Linear equations
\begin{itemize}
\item $ aX + bY = c $ for fixed $ a, b, c \in \mathbb{Z} $
\end{itemize}
\item Quadratic equations
\begin{itemize}
\item $ X^2 + aY^2 = b $ for fixed $ a, b \in \mathbb{Z} $
\end{itemize}
\item Cubic equations
\begin{itemize}
\item $ X^3 + Y^2Z = aZ^3 $ for fixed $ a \in \mathbb{Z} $
\end{itemize}
\end{itemize}
Ultimately we will develop ideas leading to Fermat's last theorem.

\end{frame}

\begin{frame}[t]{Linear equations}

Observe that an integer solution gives a solution modulo $ n $ for any $ n \in \mathbb{N} $.

\vspace{0.5cm}

\begin{question}
Is there an integer solution to $ 15X + 21Y = 35 $?
\end{question}

\begin{answer}
No, because $ 15X + 21Y \equiv 0 \mod 3 $, but $ 35 \equiv 2 \mod 3 $.
\end{answer}

\vspace{0.5cm}

\begin{theorem}[B\'ezout's identity]
There is an integer solution to $ aX + bY = c $ iff $ \gcd(a, b) \mid c $. Furthermore, there is an algorithm to determine all of its solutions.
\end{theorem}

\vspace{0.5cm} For a proof, refer to MATH0006 Algebra 2.

\end{frame}

\begin{frame}[t]{B\'ezout's identity}

\begin{question}
Can we write down an integer solution to $ 15X + 21Y = 36 $?
\end{question}

\begin{answer}
Yes, because $ 36 $ is divisible by $ \gcd(15, 21) = 3 $. By the division algorithm:
\begin{align*}
21 & = 1 \cdot 15 + 6 & \text{divide $ 21 $ by $ 15 $} \\
15 & = 2 \cdot 6 + 3 & \text{divide $ 15 $ by $ 6 $}
\end{align*}
By reversing the division algorithm:
\begin{align*}
3 & = 15 - 2 \cdot 6 & \text{substitute $ 3 $} \\
& = 15 - 2 \cdot (21 - 1 \cdot 15) & \text{substitute $ 6 $} \\
& = 3 \cdot 15 - 2 \cdot 21 & \text{rearrange}
\end{align*}
Thus $ X = \tfrac{36}{3} \cdot 3 = 36 $ and $ Y = \tfrac{36}{3} \cdot -2 = -24 $ works!
\end{answer}

\end{frame}

\begin{frame}[t]{Quadratic equations}

Can we do something similar for quadratic equations $ X^2 + Y^2 = b $?

\vspace{0.5cm}

\begin{question}
Is there an integer solution to $ X^2 + Y^2 = 7^5 $?
\end{question}

\begin{answer}
No, because $ X^2, Y^2 \equiv 0, 1 \mod 4 $, but $ 7^5 \equiv 3 \mod 4 $.
\end{answer}

\vspace{0.5cm}

\begin{theorem}[Sum of two squares theorem]
There is an integer solution to $ X^2 + Y^2 = b $ iff $ b $ is not divisible by a prime congruent to $ 3 $ modulo $ 4 $ with odd exponent.
\end{theorem}

\vspace{0.5cm} For a proof, refer to MATH0034 Number Theory.

\end{frame}

\begin{frame}[t]{Sum of two squares theorem}

\begin{question}
Can we write down an integer solution to $ X^2 + Y^2 = 5^3 $?
\end{question}

\begin{answer}
Yes, because $ 5 $ is a prime congruent to $ 1 $ modulo $ 4 $. In particular, $ 5^3 $ is not divisible by any prime congruent to $ 3 $ modulo $ 4 $ with odd exponent. In the ring of Gaussian integers $ \mathbb{Z}[i] $:
$$ 5^3 = X^2 + Y^2 = (X + iY)(X - iY) $$
By \emph{unique factorisation in $ \mathbb{Z}[i] $}, write $ X \pm iY = (W \pm iZ)^3 $. Then:
$$ 5^3 = ((W + iZ)(W - iZ))^3 = (W^2 + Z^2)^3 $$
Now $ W = 2 $ and $ Z = 1 $ is an integer solution to $ W^2 + Z^2 = 5 $. Moreover:
$$ X + iY = (W + iZ)^3 = (W^3 - 3WZ^2) + i(3W^2Z - Z^3) $$
Thus $ X = W^3 - 3WZ^2 = 2 $ and $ Y = 3W^2Z - Z^3 = 11 $ works!
\end{answer}

\end{frame}

\begin{frame}[t]{Number rings}

Can we do something similar for quadratic equations $ X^2 + aY^2 = b $?

\vspace{0.5cm}

\begin{question}
Is there an integer solution to $ X^2 + 2Y^2 = 7^2 $?
\end{question}

\begin{answer}
Consider the number ring $ R := \mathbb{Z}[\sqrt{-2}] $. Factorise:
$$ 7^2 = X^2 + 2Y^2 = (X + \sqrt{-2}Y)(X - \sqrt{-2}Y) $$
\emph{By unique factorisation in $ R $}, write $ X \pm \sqrt{-2}Y = (W \pm \sqrt{-2}Z)^2 $. Then:
$$ 7^2 = ((W + \sqrt{-2}Z)(W - \sqrt{-2}Z))^2 = (W^2 + 2Z^2)^2 $$
There are no integer solutions to $ W^2 + 2Z^2 = 7 $!
\end{answer}

\vspace{0.5cm} Note that $ W^2 + 2Z^2 > 0 $, so it is easy to rule out solutions.

\end{frame}

\begin{frame}[t]{Failure of unique factorisation}

Solving the quadratic equation $ X^2 + aY^2 = b $ seems to rely on unique factorisation in the ring $ R := \mathbb{Z}[\sqrt{-a}] $, but this might fail.

\begin{examples}
\begin{itemize}
\item In $ R = \mathbb{Z}[\sqrt{-5}] $, we have $ 6 = 2 \cdot 3 = (1 + \sqrt{-5}) \cdot (1 - \sqrt{-5}) $.
\item In $ R = \mathbb{Z}[\sqrt{10}] $, we have $ 10 = 2 \cdot 5 = \sqrt{10} \cdot \sqrt{10} $.
\end{itemize}
\end{examples}

The solution is to replace $ X + \sqrt{-a} $ with the \textbf{ideal}
$$ \langle X + \sqrt{-a}\rangle := \{(X + \sqrt{-a})r : r \in \mathbb{Z}[\sqrt{-a}]\}, $$
which has unique factorisation into \textbf{prime ideals} if $ a \not\equiv 3 \mod 4 $.

\vspace{0.5cm} The failure of unique factorisation into \emph{primes} is measured by the \textbf{ideal class group} $ \mathrm{Cl}(R) $. For some $ \mathrm{Cl}(R) $, a similar argument still works!

\vspace{0.5cm} For more details, refer to MATH0035 Algebraic Number Theory.

\end{frame}

\begin{frame}[t]{Cyclotomic rings}

\begin{columns}[T]

\begin{column}{0.8\textwidth}
In the 19th century, Ernst Kummer proved Fermat's last theorem for many exponents using this approach.

\vspace{0.5cm}

\begin{theorem}[Kummer]
If $ p $ is a regular odd prime, then the only integer solutions to $ X^p + Y^p = Z^p $ satisfy $ XYZ = 0 $.
\end{theorem}
\end{column}

\begin{column}{0.1\textwidth}
\hspace{-1cm}
\includegraphics[width=1.5\textwidth]{kummer.jpg}
\end{column}

\end{columns}

\begin{proof}
Consider the \textbf{cyclotomic ring} $ R := \mathbb{Z}[\zeta_p] $, where $ \zeta_p := e^{\frac{2\pi i}{p}} $. Then:
$$ Z^p = X^p + Y^p = (X + Y) \cdot (X + \zeta_pY) \cdot (X + \zeta_p^2Y) \cdot \dots \cdot (X + \zeta_p^{p - 1}Y) $$
A ``similar'' argument still works if $ p $ is a regular prime!
\end{proof}

\vspace{0.5cm} Say that a prime $ p $ is \textbf{regular} if it does not divide the size of $ \mathrm{Cl}(R) $, which conjecturally accounts for 61\% of all primes.

\end{frame}

\begin{frame}[t]{Rational projective plane}

Observe that $ X^n + Y^n = Z^n $ is \textbf{homogeneous} of degree $ n $.

\vspace{0.5cm} In particular, this \emph{almost} gives a correspondence:
$$
\begin{array}{rcl}
\{(X, Y, Z) \in \mathbb{Z}^3 : X^n + Y^n = Z^n\} & \leftrightsquigarrow & \{(x, y) \in \mathbb{Q}^2 : x^n + y^n = 1\} \\
(X, Y, Z) & \mapsto & (\tfrac{X}{Z}, \tfrac{Y}{Z}) \\
(xz, yw, wz) & \mapsfrom & (\tfrac{x}{w}, \tfrac{y}{z})
\end{array}
$$
This correspondence is not quite bijective:
\begin{itemize}
\item Both $ (X, Y, Z) $ and $ (\lambda X, \lambda Y, \lambda Z) $ map to $ (\tfrac{X}{Z}, \tfrac{Y}{Z}) $.
\item Where does $ (X, Y, 0) $ map to?
\end{itemize}
Both of these issues can be fixed by working in the \textbf{projective plane}.
\begin{itemize}
\item Replace the left hand side with equivalence classes up to scaling.
\item Supplement the right hand side with ``points at infinity''.
\end{itemize}

\vspace{0.5cm} For more details, refer to MATH0076 Algebraic Geometry.

\end{frame}

\begin{frame}[t]{Fermat curves}

By working in the projective plane, the integer solutions of $ X^n + Y^n = Z^n $ are \emph{essentially} the rational solutions of $ x^n + y^n = 1 $.

\vspace{0.5cm} The cubic equation $ x^3 + y^3 = 1 $ defines an object in algebraic geometry called an \textbf{elliptic curve} that lives in the projective plane.

\vspace{0.5cm}

\begin{center}
\includegraphics[width=0.4\textwidth]{ellipticcurve.png}
\end{center}

\end{frame}

\begin{frame}[t]{Elliptic curves}

The set of rational solutions of an elliptic curve forms an abelian group:
$$ P + Q + R = 0 \qquad \iff \qquad P, Q, R \ \text{are collinear} $$

\begin{center}
\includegraphics[width=0.9\textwidth]{grouplaw.png}
\end{center}

This gives a way to generate new rational solutions from old ones!

\end{frame}

\begin{frame}[t]{Cubic equations}

\begin{question}
Can we write down two rational solutions to $ x^3 - y^2 = 4 $?
\end{question}

\begin{answer}
This defines an elliptic curve, with an obvious solution $ x = 2 $ and $ y = 2 $. The tangent of $ e(x, y) = x^3 - y^2 - 4 $ at the point $ (x, y) = (2, 2) $ is:
$$ \dfrac{\partial e}{\partial x}(2) \cdot (x - 2) + \dfrac{\partial e}{\partial y}(2) \cdot (y - 2) = 0 $$
This simplifies as $ y = 3x - 4 $, which substitutes into $ e(x, y) = 0 $ to yield:
$$ 0 = x^3 - (3x - 4)^2 - 4 = (x - 2)^2(x - 5) $$
Thus $ x = 5 $ and $ y = 3(5) - 4 = 11 $ works!
\end{answer}

\vspace{0.5cm} In fact, \emph{adding} the solution $ x = 2 $ and $ y = 2 $ to itself repeatedly generates the \emph{only} infinite family of rational solutions to $ x^3 - y^2 = 4 $.

\end{frame}

\begin{frame}[t]{Mordell's theorem}

\begin{columns}[T]

\begin{column}{0.8\textwidth}
In 1922, Louis Mordell classified the abstract group structure of rational solutions of elliptic curves.

\vspace{0.5cm}

\begin{theorem}[Mordell]
The rational solutions of an elliptic curve can be generated by a finite set of initial rational solutions.
\end{theorem}
\end{column}

\begin{column}{0.1\textwidth}
\hspace{-1cm}
\includegraphics[width=1.5\textwidth]{mordell.jpg}
\end{column}

\end{columns}

\vspace{0.5cm}

Associated to an elliptic curve $ E $ is a complex-analytic function $ L_E(s) $.

\begin{conjecture}[Birch, Swinnerton-Dyer]
An elliptic curve $ E $ has infinitely many rational solutions iff $ L_E(1) = 0 $.
\end{conjecture}

\begin{center}
\includegraphics[width=0.2\textwidth]{birch.jpg}
\vspace{0.5cm}
\includegraphics[width=0.15\textwidth]{swinnertondyer.jpg}
\end{center}

For more details, refer to MATH0036 Elliptic Curves.

\end{frame}

\begin{frame}[t]{Modular forms}

Andrew Wiles proved Fermat's last theorem by contradiction.

\vspace{0.5cm} This requires classifying certain highly-symmetric functions on the upper half $ \mathcal{H} $ of the complex plane called \textbf{modular forms}.

\vspace{0.5cm}

\begin{center}
\includegraphics[width=0.4\textwidth]{modularform.jpg}
\end{center}

\end{frame}

\begin{frame}[t]{Newforms}

The modular forms of interest are the so-called \textbf{level $ N $ newforms}.

\vspace{0.5cm} These are functions $ f : \mathcal{H} \to \mathbb{C} $ satisfying the \textbf{modular condition}:
$$ f\left(\dfrac{az + b}{cz + d}\right) = (cz + d)^2 \cdot f(z) $$
for any $ a, b, c, d \in \mathbb{Z} $ such that $ ad - bc = 1 $ and $ N \mid c $.

\vspace{0.5cm}

\begin{theorem}[Valence formula]
For fixed $ N $, there are finitely many level $ N $ newforms.
\end{theorem}

\vspace{0.5cm} In fact, there are \emph{no} level $ N $ newforms for:
$$ N \in \{1, 2, 3, 4, 5, 6, 7, 8, 9, 10, 12, 13, 16, 18, 22, 25, 28, 60\} $$
For more details, refer to MATH0104 Modular Forms.

\end{frame}

\begin{frame}[t]{Modularity theorem}

Also associated to a modular form $ f $ is a complex-analytic function $ L_f(s) $.

\vspace{0.5cm}

\begin{columns}[T]

\begin{column}{0.8\textwidth}
Call an elliptic curve $ E $ \textbf{modular} if there is a level $ N $ newform $ f $ such that $ L_E(s) = L_f(s) $ for some $ N $.

\vspace{0.5cm}

\begin{theorem}[Wiles]
For squarefree $ N $, all elliptic curves are modular.
\end{theorem}
\end{column}

\begin{column}{0.1\textwidth}
\hspace{-1cm}
\includegraphics[width=1.5\textwidth]{wiles.jpg}
\end{column}

\end{columns}

\vspace{0.5cm}

\begin{theorem}[Breuil, Conrad, Diamond, Taylor]
All elliptic curves are modular.
\end{theorem}

\begin{center}
\includegraphics[width=0.1\textwidth]{breuil.jpg}
\hspace{0.5cm}
\includegraphics[width=0.2\textwidth]{conrad.jpg}
\hspace{0.5cm}
\includegraphics[width=0.2\textwidth]{diamond.jpg}
\hspace{0.5cm}
\includegraphics[width=0.2\textwidth]{taylor.jpg}
\end{center}

\end{frame}

\begin{frame}[t]{Fermat's last theorem}

Fermat's last theorem can now be deduced from the modularity theorem.

\vspace{0.5cm} Assume for a contradiction that $ X^n + Y^n = Z^n $ has an integer solution not satisfying $ XYZ = 0 $. Consider the elliptic curve $ E $ given by:
$$ y^2 = x(x - X^n)(x + Y^n) $$
This is called the \textbf{Frey curve} associated to the triple $ (X, Y, Z) $.

\vspace{0.5cm} The modularity theorem says that $ E $ corresponds to a level $ N $ newform $ f $.

\begin{columns}[T]

\begin{column}{0.8\textwidth}
\vspace{0.5cm}

\begin{theorem}[Ribet]
$ f $ can be ``level-lowered'' to a level $ 2 $ newform.
\end{theorem}

\vspace{0.5cm} There are no level $ 2 $ newforms, hence a contradiction!
\end{column}

\begin{column}{0.1\textwidth}
\begin{center}
\hspace{-1cm}
\includegraphics[width=1.5\textwidth]{ribet.jpg}
\end{center}
\end{column}

\end{columns}

\end{frame}

\begin{frame}[t]{Formalising Fermat}

\begin{columns}[T]

\begin{column}{0.8\textwidth}
My PhD supervisor Kevin Buzzard started a massive project to teach the modularity theorem to a computer.
\end{column}

\begin{column}{0.1\textwidth}
\begin{center}
\vspace{-1cm}
\hspace{-1cm}
\includegraphics[width=1.5\textwidth]{buzzard.jpg}
\end{center}
\end{column}

\end{columns}

\vspace{0.5cm} This means \emph{formally} defining all the relevant objects (elliptic curves, modular forms) and \emph{rigorously} verifying all the details of the proof.

\vspace{0.5cm}

\begin{center}
\texttt{https://imperialcollegelondon.github.io/FLT/}
\end{center}

\vspace{0.5cm} This is a \emph{huge} amount of work, and we need \emph{all} the help we can get!

\vspace{0.5cm} To get started, check out MATH0109 Theorem Proving in Lean!

\begin{center}
\includegraphics[width=0.5\textwidth]{lean.png}
\end{center}

\end{frame}

\end{document}