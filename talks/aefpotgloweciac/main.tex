\documentclass[10pt]{beamer}

\setbeamertemplate{footline}[page number]

\usetheme{Warsaw}

\usepackage{color}
\usepackage{listings}

\definecolor{keywordcolor}{rgb}{0.7, 0.1, 0.1}
\definecolor{tacticcolor}{rgb}{0.0, 0.1, 0.6}
\definecolor{commentcolor}{rgb}{0.4, 0.4, 0.4}
\definecolor{symbolcolor}{rgb}{0.0, 0.1, 0.6}
\definecolor{sortcolor}{rgb}{0.1, 0.5, 0.1}
\definecolor{attributecolor}{rgb}{0.7, 0.1, 0.1}
\def\lstlanguagefiles{lstlean.tex}
\lstset{language=lean}

\title{The Group Law on Weierstrass Elliptic Curves}

\subtitle{An Elementary Formal Proof in Any Characteristic}

\author[David Ang]{\textbf{David Kurniadi Angdinata} \inst{1} \and Junyan Xu \inst{2}}

\institute[LSGNT]{\inst{1} \textbf{London School of Geometry and Number Theory, UK} \and \inst{2} Cancer Data Science Laboratory, National Cancer Institute, Bethesda, MD, USA}

\date[ITP 2023]{\small Fourteenth International Conference on Interactive Theorem Proving \\ \vspace{0.5cm} \footnotesize Wednesday, 2 August 2023}

\begin{document}

\frame{\titlepage}

\begin{frame}[t]{Elliptic curves}

An \textbf{elliptic curve} over a field $ F $ is a pair $ (E, \mathcal{O}) $:
\begin{itemize}
\item $ E $ is a \emph{smooth projective curve} of \emph{genus one} defined over $ F $
\item $ \mathcal{O} $ is a distinguished point on $ E $ defined over $ F $
\end{itemize}

\begin{center}
\includegraphics[width=0.3\textwidth]{ellipticcurves.png}
\end{center}

\only<2>{Applications:
\begin{itemize}
\item computational mathematics
\begin{itemize}
\item primality testing, integer factorisation, public-key cryptography
\end{itemize}
\item algebraic geometry and number theory
\begin{itemize}
\item Fermat's last theorem, the Birch and Swinnerton-Dyer conjecture
\end{itemize}
\end{itemize}
}

\end{frame}

\begin{frame}[fragile, t]{Weierstrass equations}

\begin{theorem}[corollary of \emph{Riemann-Roch}]
Any elliptic curve $ E $ over $ F $ can be given by $ E(X, Y) = 0 $, where \vspace{-0.2cm} $$ \vspace{-0.2cm} E(X, Y) := Y^2 + a_1XY + a_3Y - (X^3 + a_2X^2 + a_4X + a_6), $$ for some $ a_i \in F $ such that $ \Delta(a_i) \ne 0 $, with $ \mathcal{O} $ the point at infinity.
\end{theorem}

\only<2-4>{This is the \textbf{Weierstrass model} for $ E $, but $ E $ has other models.
\begin{itemize}
\item<3-4> If $ \mathrm{char}(F) \ne 2, 3 $, then $ E $ has a \textbf{short Weierstrass model} \vspace{-0.2cm} $$ \vspace{-0.2cm} E(X, Y) := Y^2 - (X^3 + aX + b), \qquad a, b \in F, $$
where $ \Delta(a, b) = -16(4a^3 + 27b^2) $.
\item<4> If $ \mathrm{char}(F) \ne 2 $, then $ E $ has an \textbf{Edwards model} \vspace{-0.2cm} $$ \vspace{-0.2cm} E(X, Y) := X^2 + Y^2 - (1 + dX^2Y^2), \qquad d \in F \setminus \{0, 1\}, $$
with $ \mathcal{O} := (1, 0) $.
\end{itemize}
}

\only<5-6>{In the Weierstrass model, an \textbf{elliptic curve} over $ F $ is the data of:
\begin{itemize}
\item five coefficients $ a_1, a_2, a_3, a_4, a_6 \in F $, and
\item a proof that $ \Delta(a_1, a_2, a_3, a_4, a_6) \ne 0 $.
\end{itemize}
}

\begin{onlyenv}<6>
\begin{lstlisting}[backgroundcolor=\color{lime}, basicstyle=\scriptsize, frame=single]
structure weierstrass_curve (F : Type) := (a₁ a₂ a₃ a₄ a₆ : F)

def weierstrass_curve.Δ {F : Type} [comm_ring F] (W : weierstrass_curve F) : F :=
  -(E.a₁^2 + 4*E.a₂)*(E.a₁^2*E.a₆ + 4*E.a₂*E.a₆ - E.a₁*E.a₃*E.a₄ + E.a₂*E.a₃^2 - E.a₄^2)
    - 8*(2*E.a₄ + E.a₁*E.a₃)^3 - 27*(E.a₃^2 + 4*E.a₆)^2
    + 9*(E.a₁^2 + 4*E.a₂)*(2*E.a₄ + E.a₁*E.a₃)*(E.a₃^2 + 4*E.a₆)

structure elliptic_curve (F : Type) [comm_ring F] extends weierstrass_curve F :=
  (Δ' : units F) (coe_Δ' : ↑Δ' = to_weierstrass_curve.Δ)
\end{lstlisting}
\end{onlyenv}

\only<7-8>{In the Weierstrass model, a \textbf{point} on $ E $ is either:
\begin{itemize}
\item the point at infinity $ \mathcal{O} $, or
\item two affine coordinates $ x, y \in F $ and a proof that $ (x, y) \in E $.
\end{itemize}
}

\begin{onlyenv}<8>
\begin{lstlisting}[backgroundcolor=\color{lime}, basicstyle=\scriptsize, frame=single]
variables {F : Type} [field F] (E : elliptic_curve F)

def polynomial : F[X][Y] :=
  Y^2 + C (C E.a₁*X + C E.a₃)*Y - C (X^3 + C E.a₂*X^2 + C E.a₄*X + C E.a₆)

def equation (x y : F) : Prop := (E.polynomial.eval (C y)).eval x = 0

inductive point
  | zero
  | some {x y : F} (h : E.equation x y)
\end{lstlisting}
\end{onlyenv}

\end{frame}

\begin{frame}[fragile, t]{Group law}

\begin{theorem}[the group law]
The points of $ E $ form an abelian group under a geometric addition law.
\end{theorem}

\only<2-3>{Identity is given by $ \mathcal{O} $.}

\begin{onlyenv}<2-3>
\begin{lstlisting}[backgroundcolor=\color{lime}, basicstyle=\scriptsize, frame=single]
instance : has_zero E.point := ⟨zero⟩
\end{lstlisting}
\end{onlyenv}

\only<3>{Negation and addition are characterised by \vspace{-0.2cm} $$ \vspace{-0.2cm} P + Q + R = 0 \qquad \iff \qquad P, Q, R \ \text{are collinear}. $$}

\only<3>{
\begin{center}
\includegraphics[width=0.7\textwidth]{grouplaw.png}
\end{center}
}

\only<4-6>{Negation is given by $ -(x, y) := (x, \sigma(y)) $, where \vspace{-0.2cm} $$ \vspace{-0.2cm} \sigma(Y) := -Y - a_1X - a_3. $$}

\begin{onlyenv}<4-6>
\begin{lstlisting}[backgroundcolor=\color{lime}, basicstyle=\scriptsize, frame=single]
def neg_polynomial : F[X][Y] := -Y - C (C E.a₁ * X + C E.a₃)

def neg_Y (x y : F) : F := (E.neg_polynomial.eval (C y)).eval x

lemma equation_neg {x y : F} : E.equation x y → E.equation x (E.neg_Y x y) := ...

def neg : E.point → E.point
  | zero := zero
  | (some h) := some (equation_neg h)

instance : has_neg E.point := ⟨neg⟩
\end{lstlisting}
\end{onlyenv}

\only<5-6>{\underline{\textbf{Note:}} \visible<6>{in the \textbf{coordinate ring} $ F[E] := F[X, Y] / \langle E(X, Y) \rangle $,} \vspace{-0.2cm} $$ \vspace{-0.2cm} -(Y \cdot \sigma(Y)) = Y^2 + a_1XY + a_3Y \visible<6>{\equiv X^3 + a_2X^2 + a_4X + a_6.} $$}

\only<7-8>{Addition is given by $ (x_1, y_1) + (x_2, y_2) := -(x_3, y_3) $, where \vspace{-0.2cm} \begin{align*} x_3 & := \lambda^2 + a_1\lambda - a_2 - x_1 - x_2, \\ y_3 & := \lambda(x_3 - x_1) + y_1. \end{align*} \vspace{-0.6cm}}

\begin{onlyenv}<7-8>
\begin{lstlisting}[backgroundcolor=\color{lime}, basicstyle=\scriptsize, frame=single]
def add : E.point → E.point → E.point
  | zero P := P
  | P zero := P
  | (some h₁) (some h₂) := some (equation_add h₁ h₂)

instance : has_add E.point := ⟨add⟩
\end{lstlisting}
\end{onlyenv}

\only<8>{Here, \vspace{-0.2cm} $$ \vspace{-0.2cm} \lambda := \begin{cases} \dfrac{y_1 - y_2}{x_1 - x_2} & x_1 \ne x_2 \vspace{0.2cm} \\ \dfrac{3x_1^2 + 2a_2x_1 + a_4 - a_1y_1}{y_1 - \sigma(y_1)} & y_1 \ne \sigma(y_1) \\ \infty & \text{otherwise} \end{cases}. $$}

\end{frame}

\begin{frame}[fragile, t]{Attempts at proof}

\only<1-3>{One may attempt to prove the axioms directly.}

\begin{onlyenv}<1-3>
\begin{lstlisting}[backgroundcolor=\color{lime}, basicstyle=\scriptsize, frame=single]
instance : add_group E.point :=
  { zero            := zero,
    neg              := neg,
    add              := add,
    zero_add       := rfl,     -- by definition
    add_zero       := rfl,     -- by definition
    add_left_neg := ...,      -- by cases
    add_comm       := ...,      -- by cases
    add_assoc     := sorry } -- seems impossible?
\end{lstlisting}
\end{onlyenv}

\only<2-3>{Associativity is a proof that \vspace{-0.2cm} $$ \vspace{-0.2cm} (P + Q) + R = P + (Q + R), $$ where each $ + $ has five cases!}

\only<3>{\vspace{0.5cm} In the generic case, this is an equality of polynomials with 26,082 terms. \\ \vspace{0.5cm} In contrast, the \texttt{ring} tactic in Lean can handle at most 1,000 terms.}

\only<4-6>{Associativity is known to be mathematically difficult with many proofs.}

\only<5-6>{\vspace{0.5cm} Proof 1: just do it.
\begin{itemize}
\item elementary but slow
\item several known formalisations
\begin{itemize}
\item Th\'ery (Coq, 2007): short Weierstrass model $ Y^2 = X^3 + aX + b $
\item Hales, Raya (Isabelle, 2020): Edwards model $ X^2 + Y^2 = 1 + dX^2Y^2 $
\item Fox, Gordon, Hurd (HOL4, 2006): long Weierstrass model $ Y^2 + a_1XY + a_3Y = X^3 + a_2X^2 + a_4X + a_6 $ but no associativity
\end{itemize}
\end{itemize}
}

\only<6>{\vspace{0.5cm} Proof 2: ad-hoc argument with projective geometry.
\begin{itemize}
\item only works generically via \emph{Cayley-Bacharach}
\item no known formalisations
\begin{itemize}
\item our original attempt
\end{itemize}
\end{itemize}
}

\only<7-10>{One may instead identify the set of points $ E(F) $ with a known group.}

\only<8-10>{\vspace{0.5cm} Proof 3: identify with a quotient of $ \mathbb{C} $ by the \emph{fundamental lattice} $ \Lambda_E $.
\begin{itemize}
\item only works in characteristic zero via \emph{uniformisation}
\item no known formalisations
\begin{itemize}
\item needs a lot of theory
\end{itemize}
\end{itemize}
}

\only<9-10>{\vspace{0.5cm} Proof 4: identify with the \emph{degree zero Weil divisor class group} $ \mathrm{Pic}_F^0(E) $.
\begin{itemize}
\item algebro-geometric and usually uses \emph{Riemann-Roch}
\item one known formalisation
\begin{itemize}
\item Bartzia, Strub (10,000 lines of Coq, 2014): short Weierstrass model
\end{itemize}
\end{itemize}
}

\only<10>{\vspace{0.5cm} Proof 5: identify with the \emph{ideal class group} $ \mathrm{Cl}(F[E]) $.
\begin{itemize}
\item purely algebraic and uses commutative algebra
\item one known formalisation
\begin{itemize}
\item our final proof (1,000 lines of Lean, 2023): long Weierstrass model
\end{itemize}
\end{itemize}
}

\end{frame}

\begin{frame}[t]{Sketch of proof}

\begin{proof}[Proof of the group law]
\begin{enumerate}
\item Construct a function $ E(F) \to \mathrm{Cl}(F[E]) $. \only<6-8>{$ \checkmark $}
\item Prove that $ E(F) \to \mathrm{Cl}(F[E]) $ respects addition. \only<8>{$ \checkmark $}
\item Prove that $ E(F) \to \mathrm{Cl}(F[E]) $ is injective.
\end{enumerate}
\vspace{-0.5cm}
\end{proof}

\only<2-5>{Here, the \textbf{ideal class group} $ \mathrm{Cl}(R) $ of an integral domain $ R $ is the quotient group of \emph{invertible fractional ideals} by \emph{principal fractional ideals}.}

\only<3-5>{
\begin{example}
Any nonzero ideal $ I \trianglelefteq R $ such that $ I \cdot J $ is principal for some ideal $ J \trianglelefteq R $ is an invertible fractional ideal of $ R $.
\end{example}
}

\only<4-5>{Ideal class groups were formalised in Lean's mathematical library \texttt{mathlib} by Baanen, Dahmen, Narayanan, Nuccio (2021).}

\only<5>{\vspace{0.5cm} \underline{\textbf{Key:}} the coordinate ring $ F[E] $ is an integral domain.}

\only<6-8>{Consider the function \texttt{point.to\_class} given by $$ \begin{array}{rcl} E(F) & \longrightarrow & \mathrm{Cl}(F[E]) \\ \mathcal{O} & \longmapsto & [\langle 1 \rangle] \\ (x, y) & \longmapsto & [\langle X - x, Y - y \rangle] \end{array}. $$}

\only<7-8>{\underline{\textbf{Note:}} $ \langle X - x, Y - y \rangle $ is invertible, since \vspace{-0.2cm} $$ \vspace{-0.2cm} \langle X - x, Y - y \rangle \cdot \langle X - x, Y - \sigma(y) \rangle = \langle X - x \rangle. $$}

\only<8>{\vspace{0.5cm} The function \texttt{point.to\_class} respects addition, since \vspace{-0.2cm} $$ \vspace{-0.2cm} \langle X - x_1, Y - y_1 \rangle \cdot \langle X - x_2, Y - y_2 \rangle \cdot \langle X - x_3, Y - \sigma(y_3) \rangle = \langle Y - \lambda(X - x_3) - y_3 \rangle. $$}

\end{frame}

\begin{frame}[t]{Proof of injectivity}

\begin{theorem}[Xu, 2022]
The function \texttt{point.to\_class} is injective.
\end{theorem}

\only<2-10>{\underline{\textbf{Key:}} $ F[E] = F[X, Y] / \langle E(X, Y) \rangle $ is free over $ F[X] $ with basis $ \{1, Y\} $, so it has a norm $ \mathrm{Nm} : F[E] \to F[X] $ given by $ \mathrm{Nm}(f) := \det([\cdot f]) $.}

\only<3-4>{
\begin{lemma}[A]
If $ f \in F[E] $, then $ \deg(\mathrm{Nm}(f)) \ne 1 $.
\end{lemma}
}

\only<4>{
\begin{proof}[Proof of Lemma (A)]
Let $ f = p + qY $ for $ p, q \in F[X] $. Then
\vspace{-0.2cm}
\begin{align*}
\mathrm{Nm}(f)
& \equiv \det\begin{pmatrix} p & q \\ q(X^3 + a_2X^2 + a_4X + a_6) & p - q(a_1X + a_3) \end{pmatrix} \\
& = p^2 - pq(a_1X + a_3) - q^2(X^3 + a_2X^2 + a_4X + a_6).
\end{align*}
Then $ \deg(\mathrm{Nm}(f)) = \max(2\deg(p), 2\deg(q) + 3) $.
\end{proof}
}

\only<5-6>{
\begin{lemma}[B]
If $ f \in F[E] $, then $ \deg(\mathrm{Nm}(f)) = \dim_F(F[E] / \langle f \rangle) $.
\end{lemma}
}

\only<6>{
\begin{proof}[Proof of Lemma (B)]
Multiplication by $ f $ has Smith normal form
\vspace{-0.2cm}
$$ [\cdot f] \sim \begin{pmatrix} p & 0 \\ 0 & q \end{pmatrix}, \qquad p, q \in F[X]. $$
\vspace{-0.5cm}
\begin{itemize}
\item Taking determinants gives $ \mathrm{Nm}(f) = pq $.
\item Taking quotients gives $ F[E] / \langle f \rangle \cong F[X] / \langle p \rangle \oplus F[X] / \langle q \rangle $.
\end{itemize}
\vspace{-0.5cm}
\end{proof}
}

\only<7-10>{
\begin{proof}[Proof of Theorem]
Suffices to show if $ (x, y) \in E(F) $, then $ \langle X - x, Y - y \rangle $ is not principal.

\vspace{0.5cm}

\visible<8-10>{Suppose otherwise that $ \langle X - x, Y - y \rangle = \langle f \rangle $ for some $ f \in F[E] $. Then \vspace{-0.2cm} $$ \vspace{-0.5cm} F \overset{1^{\mathrm{st}} \mathrm{iso}}{\cong} F[X, Y] / \langle X - x, Y - y \rangle \overset{3^{\mathrm{rd}} \mathrm{iso}}{\cong} F[E] / \langle X - x, Y - y \rangle = F[E] / \langle f \rangle. $$}

\visible<9-10>{Taking dimensions gives \vspace{-0.2cm} $$ \vspace{-0.5cm} 1 = \dim_F(F) = \dim_F(F[E] / \langle f \rangle) \overset{(B)}{=} \deg(\mathrm{Nm}(f)) \overset{(A)}{\ne} 1. $$}

\visible<10>{Contradiction!}
\end{proof}
}

\end{frame}

\begin{frame}[t]{Concluding retrospectives}

Some thoughts:
\begin{itemize}
\item proof works for nonsingular points of Weierstrass curves
\item formalisation encouraged proof accessible to undergraduates
\item heavy use of linear algebra and ring theory in \texttt{mathlib}
\item fully integrated to \texttt{mathlib} and even ported to \texttt{mathlib4}
\end{itemize}

\pause \vspace{0.5cm}

Some projects:
\begin{itemize}
\item division polynomials, torsion subgroups, and Tate modules
\item elliptic curves over discrete valuation rings and the reduction map
\item verification of computational algorithms and cryptographic protocols
\item equivalence with scheme-theoretic definitions via Riemann-Roch
\item elliptic curves over specific fields: finite fields, local fields, number fields, global function fields, complete fields
\end{itemize}

\pause \vspace{0.5cm}

Thank you!

\end{frame}

\end{document}