\documentclass[10pt]{beamer}

\setbeamertemplate{footline}[page number]

\usetheme{berkeley}

\usepackage{multirow}

\title{Elliptic curves in mathlib}

\author{David Ang}

\institute{London School of Geometry and Number Theory}

\date{Wednesday, 26 June 2024}

\begin{document}

\frame{\titlepage}

\section{Overview}

\begin{frame}[t]{Overview: definitions}

Elliptic curves in \texttt{Mathlib/AlgebraicGeometry/EllipticCurve} are defined in terms of Weierstrass curves over a commutative ring $ R $.

\begin{definition}[\texttt{WeierstrassCurve} in \texttt{Weierstrass.lean}]
A \textbf{Weierstrass curve} $ W_R $ is a tuple $ (a_1, a_2, a_3, a_4, a_6) \in R^5 $.

An \textbf{elliptic curve} $ E_R $ is a Weierstrass curve such that $ \Delta(a_i) \in R^\times $.
\end{definition}

\pause

Their points are defined via the affine model.

\begin{definition}[\texttt{WeierstrassCurve.Affine.Point} in \texttt{Affine.lean}]
An \textbf{affine point} of $ W_R $ is a pair $ (x, y) \in R^2 $ such that $ W(x, y) = 0 $ and either $ W_X(x, y) \ne 0 $ or $ W_Y(x, y) \ne 0 $, where
$$ W := Y^2 + a_1XY + a_3Y - (X^3 + a_2X^2 + a_4X^3 + a_6). $$

The \textbf{points} $ W_R(R) $ are the affine points of $ W_R $ and a point at infinity.
\end{definition}

\end{frame}

\begin{frame}[t]{Overview: files}

Current:
\begin{itemize}
\item \texttt{Weierstrass.lean}
\item \texttt{Affine.lean}
\item \texttt{Projective.lean}
\item \texttt{Jacobian.lean}
\item \texttt{Group.lean}
\item \texttt{DivisionPolynomial/Basic.lean}
\item \texttt{DivisionPolynomial/Degree.lean}
\end{itemize}

Future:
\begin{itemize}
\item \texttt{Universal.lean}
\item \texttt{DivisionPolynomial/Group.lean}
\item \texttt{Torsion.lean}
\item \texttt{Scheme.lean} (NEW!)
\end{itemize}

\end{frame}

\section{Group law}

\begin{frame}[t]{Group law: theorem}

\begin{theorem}[in \texttt{Group.lean}]
If $ F $ is a field, then $ W_F(F) $ is an abelian group under an addition law.
\end{theorem}

\pause

\vspace{0.5cm}

Elementary proofs of associativity:
\begin{itemize}
\item polynomial manipulation via \texttt{ring}

\pause

\item geometric argument via B\'ezout's theorem
\end{itemize}

\pause

Proofs by identification with known groups:
\begin{itemize}
\item a quotient $ \mathbb{C} / \Lambda $ of the complex numbers by a lattice

\pause

\item the group of degree-zero Weil divisors $ \mathrm{Pic}^0(W_F) $

\pause

\item the ideal class group $ \mathrm{Cl}(F[W_F]) $ of the coordinate ring
\end{itemize}

\pause

Junyan gave an pure algebraic proof via norms.

\end{frame}

\begin{frame}[t]{Group law: formalisation}

Define
$$
\begin{array}{rcrcl}
\phi & : & W_F(F) & \longrightarrow & \mathrm{Cl}(F[W_F]) \\
& & 0 & \longmapsto & [\langle 1 \rangle] \\
& & (x, y) & \longmapsto & [\langle X - x, Y - y \rangle]
\end{array}.
$$

\pause

Note that $ F[W_F] $ is a free algebra over $ F[X] $ with basis $ \{1, Y\} $, so it has a norm given by $ \mathrm{Nm}(p + qY) = \det([\cdot (p + qY)]) $. \pause On one hand,
$$ \deg(\mathrm{Nm}(p + qY)) = \max(2\deg(p), 2\deg(q) + 3) \ne 1. $$

\pause

On the other hand, $ F[W_F] / \langle p + qY\rangle \cong F[X] / \langle p\rangle \oplus F[X] / \langle q\rangle $, so
$$ \deg(\mathrm{Nm}(p + qY)) = \deg(pq) = \dim(F[W_F] / \langle p + qY\rangle). $$

\pause

Thus if $ \langle X - x, Y - y \rangle = \langle p + qY\rangle $, then
$$ F[W_F] / \langle p + qY\rangle = F[X, Y] / \langle W(X, Y), X - x, Y - y\rangle \cong F, $$
which contradicts $ \dim(F[W_F] / \langle p + qY\rangle) \ne 1 $.

\end{frame}

\section{Torsion subgroup}

\begin{frame}[t]{Torsion subgroup: theorem}

\begin{theorem}[in \texttt{Torsion.lean}]
If $ F $ is a field where $ n \ne 0 $, then $ E_F(\overline{F})[n] \cong (\mathbb{Z} / n\mathbb{Z})^2 $.
\end{theorem}

\pause

\vspace{0.5cm}

Some standard proofs:
\begin{itemize}
\item identification with $ (\mathbb{C} / \Lambda)[n] $

\pause

\item induced map of isogenies on $ \mathrm{Pic}^0(E_{\overline{F}}) $

\pause

\item existence of polynomials $ \psi_n, \phi_n, \omega_n \in \overline{F}[X, Y] $ such that
$$ [n](x, y) = \left(\dfrac{\phi_n(x)}{\psi_n(x)^2}, \dfrac{\omega_n(x, y)}{\psi_n(x, y)^3}\right) $$
and a proof that $ \deg(\psi_n^2) = n^2 - 1 $
\end{itemize}

\end{frame}

\begin{frame}[t]{Torsion subgroup: formalisation}

The latter proof turned out to be incredibly tricky.

\pause

\begin{itemize}
\item The identity holds in the universal ring $ \mathbb{Z}[A_i, X, Y] / \langle W\rangle $, so needs a specialisation map or projective coordinates

\pause

\item The definition of $ \psi_n $ is strong even-odd recursive with five base cases and an awkward even case, so proofs are very lengthy

\pause

\item The definition of $ \omega_n $ is very elusive, and seemingly involves division by two in characteristic two

\pause

\item The polynomials $ \phi_n $ and $ \psi_n^2 $ are bivariate, so needs a conversion to univariate polynomials for degree computations

\pause

\item The identity cannot be proven directly via induction, and needs elliptic divisibility sequences and elliptic nets
\end{itemize}

\pause

These have been formalised in \texttt{Projective.lean}, \texttt{Jacobian.lean}, \texttt{DivisionPolynomial/*.lean}, and \texttt{Universal.lean}. These also use lemmas in \texttt{Algebra/Polynomial/Bivariate.lean} and \texttt{NumberTheory/EllipticDivisibilitySequence.lean}

\end{frame}

\section{Progress}

\begin{frame}[t]{Progress: current}

Already in master:
\begin{itemize}
\item Weierstrass curves and variable changes of standard quantities
\item elliptic curves with prescribed j-invariant
\item affine group law and functoriality of base change
\item Jacobian group law and equivalence with affine group law
\item division polynomials and degree computations
\end{itemize}

\pause

Already in branches:
\begin{itemize}
\item Galois theory on points and n-torsion points
\item projective group law and equivalence with affine group law
\item the coordinate ring and other universal constructions
\item elliptic divisibility sequences and elliptic nets
\item multiplication by n in terms of division polynomials
\item structure of the n-torsion subgroup and the Tate module
\item the affine scheme associated to an elliptic curve
\end{itemize}

\end{frame}

\begin{frame}[t]{Progress: future}

Projects without algebraic geometry:
\begin{itemize}
\item algorithms that only use the group law
\item finite fields: the Hasse--Weil bound, the Weil conjectures
\item local fields: the reduction homomorphism, Tate's algorithm, the Neron--Ogg--Shafarevich criterion, the Hasse--Weil L-function
\item number fields: Neron-Tate heights, the Mordell--Weil theorem, Tate--Shafarevich groups, the Birch--Swinnerton-Dyer conjecture
\item complete fields: complex uniformisation, p-adic uniformisation
\end{itemize}

\pause

Projects with algebraic geometry:
\begin{itemize}
\item elliptic curves over global function fields
\item the projective scheme associated to an elliptic curve
\item integral models and finite flat group schemes
\item divisors on curves and the Riemann--Roch theorem
\item modular curves and Mazur's theorem
\end{itemize}

\end{frame}

\begin{frame}

\begin{center}
\Huge THANK YOU!
\end{center}

\end{frame}

\end{document}