\documentclass[10pt]{beamer}

\setbeamertemplate{footline}[page number]

\usepackage{tikz-cd}
\usepackage{transparent}

\theoremstyle{definition}\newtheorem*{remark}{Remark}

\begin{document}

\begin{frame}

\begin{center}

{\scriptsize Abelian varieties over finite fields}

\vspace{0.5cm}

{\scriptsize Wednesday, 18 January 2023}

\vspace{1cm}

\textbf{\large Introduction to abelian varieties over finite fields}

\vspace{0.5cm}

{\large Dual abelian varieties \footnote{J S Milne (2008) Abelian Varieties}}

\vspace{1cm}

{\footnotesize David Ang}

\vspace{0.5cm}

{\footnotesize University College London}

\end{center}

\end{frame}

\begin{frame}[t]{Dual elliptic curves}

Let $ (E, O) $ be an elliptic curve over a field $ K $.
\only<2->{Recall that
$$
\begin{array}{rcrcl}
\lambda_{(O)} & : & E & \xrightarrow{\sim} & \mathrm{Cl}^0(E) \le \mathrm{Cl}(E) \\
& & P & \mapsto & (-P) - (O)
\end{array}.
$$
}%
\only<3->{Here $ \mathrm{Cl}(E) $ is the \textbf{class group} of Weil divisors $ \sum_{P \in E} n_P(P) $ modulo $ \sim $, where $ D \sim 0 $ if $ D $ is the divisor $ (f) $ of some rational function $ f \in \overline{K}(E)^\times $, and $ \mathrm{Cl}^0(E) $ is its subgroup with $ \sum_{P \in E} n_P = 0 $.}

\only<4->{\vspace{0.5cm} Idea: for any $ D \in \mathrm{Cl}^0(E) $, the Riemann-Roch space $ \mathrm{L}(D + (O)) $, where
$$ \mathrm{L}(D) := \{f \in \overline{K}(E)^\times : (f) + D \ge 0\} \cup \{0\}, $$
is one-dimensional, so $ D \sim (-P) - (O) $ for some $ P \in E $.}

\only<5>{\vspace{0.5cm} For an elliptic curve $ E $, its \emph{dual} is $ \mathrm{Cl}^0(E) $.}

\end{frame}

\begin{frame}[t]{Invertible sheaves on smooth varieties}

Let $ X / K $ be a smooth variety.
\only<2->{Then identify
$$
\begin{array}{rcl}
\mathrm{Cl}(X) & \xrightarrow{\sim} & \mathrm{Pic}(X) \\
D & \mapsto & \mathcal{L}(D)
\end{array}.
$$
}%
\only<3->{Here $ \mathrm{Pic}(X) $ is the \textbf{Picard group} of invertible sheaves $ \mathcal{L} $ modulo $ \cong $, with
$$ \mathcal{L} \cdot \mathcal{L}' := \mathcal{L} \otimes_{\mathcal{O}_X} \mathcal{L}', \qquad \mathcal{L}^{-1} := \mathcal{H}om(\mathcal{L}, \mathcal{O}_X), $$
}%
\only<4->{and $ \mathcal{L}(D) $ is the sheaf of $ \mathcal{O}_X $-modules such that for any open $ U \subseteq X $,
$$ \Gamma(U, \mathcal{L}(D)) := \{f \in K(X)^\times : (f) + D \ge 0 \ \text{on} \ U\} \cup \{0\}. $$
}%
\only<5>{If $ f : Y \to X $ is a morphism, then there is also a \textbf{pull-back}
$$ f^*\mathcal{L} := f^{-1}\mathcal{L} \otimes_{f^{-1}\mathcal{O}_Y} \mathcal{O}_X \in \mathrm{Pic}(Y). $$
}

\end{frame}

\begin{frame}[t]{Invertible sheaves on abelian varieties}

Let $ A / K $ be an abelian variety.
\only<2->{For any $ a \in A(K) $, the translation map $ \tau_a : A \to A $ induces $ \tau_a^* : \mathrm{Pic}(A) \to \mathrm{Pic}(A) $.}
\only<3->{For any $ \mathcal{L} \in \mathrm{Pic}(A) $, define
$$
\begin{array}{rcrcl}
\lambda_\mathcal{L} & : & A(K) & \to & \mathrm{Pic}(A) \\
& & a & \mapsto & \tau_a^*\mathcal{L} \cdot \mathcal{L}^{-1}
\end{array}.
$$
}%
\only<4->{This is a homomorphism, by \textbf{theorem of the square}
$$ \tau_{a + b}^*\mathcal{L} \cdot \mathcal{L} \cong \tau_a^*\mathcal{L} \cdot \tau_b^*\mathcal{L}, \qquad a, b \in A(K). $$
}%
\only<5->{This follows from \textbf{theorem of the cube} \footnote{Theorem I.5.1} that
$$ (f + g + h)^*\mathcal{L} \cdot (f + g)^*\mathcal{L}^{-1} \cdot (f + h)^*\mathcal{L}^{-1} \cdot (g + h)^*\mathcal{L}^{-1} \cdot f^*\mathcal{L} \cdot g^*\mathcal{L} \cdot h^*\mathcal{L} $$
is trivial for any regular maps $ f, g, h : V \to A $ from a variety $ V / K $.}

\only<6>{\vspace{0.5cm} In fact, if $ \mathcal{L} \in \mathrm{Pic}(A) $ is ample, then $ \ker(\lambda_{\mathcal{L}}) \le A(K) $ is finite. \footnote{Proposition I.8.1}}

\end{frame}

\begin{frame}[t]{Invertible sheaves and Weil divisors}

\begin{remark}
Equivalently, $ \tau_a^* : \mathrm{Cl}(A) \to \mathrm{Cl}(A) $ translates a Weil divisor $ D $ by $ -a $,
\only<2->{so
$$
\begin{array}{rcrcl}
\lambda_{\mathcal{L}(D)} & : & A(K) & \to & \mathrm{Cl}(A) \\
& & a & \mapsto & D_{-a} - D
\end{array},
$$
where $ D_{-a} $ is translation of $ D $ by $ -a $.}
\only<3->{Theorem of the square becomes
$$ D_{-(a + b)} + D \sim D_{-a} + D_{-b}, \qquad a, b \in A(K). $$
}%
\only<4->{If $ A = E $, then
$$
\begin{array}{rcrcl}
\lambda_{\mathcal{L}((O))} & : & E(K) & \to & \mathrm{Cl}(E) \\
& & P & \mapsto & (-P) - (O)
\end{array}.
$$
}%
\only<5>{In fact, if $ D \in \mathrm{Cl}(E) $ is effective, then $ \deg D = 0 $ iff $ \lambda_{\mathcal{L}(D)} = 0 $. \footnote{Example I.8.3}}
\end{remark}

\end{frame}

\begin{frame}[t]{Translation-invariant invertible sheaves}

Let $ + : A \times A \to A $ be the addition map, and let $ \pi_i : A \times A \to A $ be the projection map to the $ i $-th component.
\only<2->{For any $ \mathcal{L} \in \mathrm{Pic}(A) $, define
$$ \mathrm{K}(\mathcal{L}) := \{a \in A : (+^*\mathcal{L} \cdot \pi_1^*\mathcal{L}^{-1})|_{A \times \{a\}} \cong \mathcal{O}_A\}. $$
}%
\only<3->{Then $ \mathrm{K}(\mathcal{L})(K) = \ker(\lambda_\mathcal{L}) $ as subgroups of $ A $, since
$$ (+^*\mathcal{L} \cdot \pi_1^*\mathcal{L}^{-1})|_{A \times \{a\}} = \tau_a^*\mathcal{L} \cdot \mathcal{L}^{-1}, \qquad a \in A(K). $$
}%
\only<4->{In fact, $ \mathrm{K}(\mathcal{L}) $ is closed as a subvariety of $ A $. \footnote{Proposition I.5.19}}

\only<5->{\vspace{0.5cm} Define the subgroup of \textbf{translation-invariant invertible sheaves}
$$ \mathrm{Pic}^0(A) := \{\mathcal{L} \in \mathrm{Pic}(A) : \mathrm{K}(\mathcal{L}) = A\}. $$
}%
\only<6->{Then $ \tau_a^*\mathcal{L} \cdot \mathcal{L}^{-1} \in \mathrm{Pic}^0(A) $ for any $ a \in A(K) $, so $ \mathrm{im}(\lambda_\mathcal{L}) \subseteq \mathrm{Pic}^0(A) $.}

\only<7>{\vspace{0.5cm} Need an abelian variety $ \widehat{A} $ such that $ \widehat{A}(K) \cong \mathrm{Pic}^0(A) $.}

\end{frame}

\begin{frame}[t]{Construction of dual abelian varieties}

Idea: $ \lambda_\mathcal{L} : A(K) \to \mathrm{Pic}^0(A) $ has kernel $ \mathrm{K}(\mathcal{L})(K) $, and in fact is surjective if $ \mathcal{L} \in \mathrm{Pic}(A) $ is ample, \footnote{Proposition I.8.14} \only<2->{so $ \widehat{A} $ should be the quotient variety $ A / \mathrm{K}(\mathcal{L}) $.}
\begin{itemize}
\item<3-> If $ \mathrm{char}(K) = 0 $, then $ \mathrm{K}(\mathcal{L}) $ is a reduced subgroup variety of $ A $, \only<4->{and $ A / \mathrm{K}(\mathcal{L}) $ is simply defined as the $ \mathrm{K}(\mathcal{L}) $-orbits of $ A $.}
\item<5-> If $ \mathrm{char}(K) \ne 0 $, then $ \mathrm{K}(\mathcal{L}) $ may not be reduced in general, so redefine $ \mathrm{K}(\mathcal{L}) $ as the maximal subscheme of $ A $ such that $ (+^*\mathcal{L} \cdot \pi_1^*\mathcal{L}^{-1})|_{A \times \mathrm{K}(\mathcal{L})} \cong \pi_2^*\mathcal{L}' $ for some $ \mathcal{L}' \in \mathrm{Pic}(\mathrm{K}(\mathcal{L})) $, \only<6->{and $ A / \mathrm{K}(\mathcal{L}) $ is naturally an algebraic space quotient of $ A $.}
\end{itemize}
\only<7->{The \textbf{dual abelian variety} of $ A $ is $ \widehat{A} := A / \mathrm{K}(\mathcal{L}) $.}

\only<8->{\vspace{0.5cm}
\begin{remark}
Since $ \mathcal{L} \in \mathrm{Pic}^0(A) $ iff $ +^*\mathcal{L} \cong \pi_1^*\mathcal{L} \cdot \pi_2^*\mathcal{L} $, addition on $ A $ lifts to multiplication on $ \mathcal{L} $ and makes $ \mathcal{G}(\mathcal{L}) := \mathcal{L} \setminus \{0\} $ an abelian group scheme over $ K $. \only<9>{In fact, $ \mathcal{G}(\mathcal{L}) $ is an extension of $ A $ by $ \mathbb{G}_\mathrm{m} $, and this defines an isomorphism $ \mathcal{G} : \mathrm{Pic}^0(A) \xrightarrow{\sim} \mathrm{Ext}_K^1(A, \mathbb{G}_\mathrm{m}) $ of abelian group schemes. \footnote{Proposition I.9.3}}
\end{remark}
}

\end{frame}

\begin{frame}[t]{Representability of dual abelian varieties}

Consider the functor $ \mathcal{F} : \textbf{Var}_K \to \textbf{Set} $ that associates a variety $ V / K $ to the set of isomorphism classes of $ \mathcal{L} \in \mathrm{Pic}(A \times V) $ such that
\begin{itemize}
\item $ \mathcal{L}|_{A \times \{x\}} \in \mathrm{Pic}^0(A_x) $ for any $ x \in V $, and
\item $ \mathcal{L}|_{\{0\} \times V} \cong \mathcal{O}_V $.
\end{itemize}

\only<2->{
\begin{theorem}
$ \widehat{A} $ represents $ \mathcal{F} $. In other words $ \mathcal{F}(V) = \mathrm{Hom}(V, \widehat{A}) $ for any variety $ V / K $.
\end{theorem}

\begin{proof}
Sketched in Section I.8.
\end{proof}
}

\only<3->{\vspace{0.5cm} By construction, $ \widehat{A}(L) = \mathrm{Pic}^0(A_L) $ for any field extension $ L / K $.}

\only<4->{\vspace{0.5cm} By universality, $ \widehat{A} $ is unique up to unique isomorphism.}
\only<5>{Its corresponding universal element is the \textbf{Poincar\'e sheaf} $ \mathcal{P}_A \in \mathcal{F}(\widehat{A}) $, which associates any $ \mathcal{L} \in \mathrm{Pic}^0(A) $ with a unique $ \mathcal{P}_A|_{A \times \{a\}} $ for some $ a \in \widehat{A}(K) $.}

\end{frame}

\begin{frame}[t]{Dualities on abelian varieties}

The functor $ A \mapsto \widehat{A} $ is a duality theory in the sense that $ \widehat{\widehat{A}} \cong A $.
\only<2->{This follows from $ \mathcal{P}_{\widehat{A}} \cong \mathcal{P}_A $, \footnote{Theorem I.8.9} since $ \mathcal{P}_A $ parameterises $ \widehat{A}(K) \cong \mathrm{Pic}^0(A) $.}

\only<3->{\vspace{0.5cm} Now let $ \phi : A \to B $ be a morphism.}
\only<4->{Then it has a dual morphism
$$
\begin{array}{rcrcl}
\widehat{\phi} & : & \widehat{B} & \mapsto & \widehat{A} \\
& & \mathcal{L} & \mapsto & \phi^*\mathcal{L}
\end{array}.
$$
}%
\only<5->{If $ \phi $ is an isogeny, then $ \ker(\widehat{\phi}) = \widehat{\ker(\phi)} $ is the \emph{Cartier dual} of $ \ker(\phi) $, \footnote{Theorem I.9.1} where $ \widehat{\widehat{\ker(\phi)}} \cong \ker(\phi) $.}
\only<6->{If $ K = K^{\text{sep}} $ with $ \mathrm{char}(K) \nmid n := \#\ker(\phi) $, then
$$ \widehat{\ker(\phi)} = \mathrm{Hom}(\ker(\phi), \mu_n). $$}
\only<7>{This defines a \emph{Weil pairing}
$$ \mathrm{e}_\phi : \ker(\phi) \times \ker(\widehat{\phi}) \to \mu_n. $$}

\end{frame}

\begin{frame}[t]{Polarisations on abelian varieties}

A \textbf{polarisation} on $ A $ is an isogeny $ \lambda : A \to \widehat{A} $ such that $ \lambda = \lambda_\mathcal{L} $ over $ \overline{K} $ for some ample $ \mathcal{L} \in \mathrm{Pic}(A_{\overline{K}}) $. \only<2->{It is \textbf{principal} if it has degree one.}

\only<3->{\vspace{0.5cm}
\begin{remark}
Zarhin proved that $ (A \times \widehat{A})^4 $ is always principally polarised. \footnote{Theorem I.13.12}
\end{remark}
}

\only<4->{\vspace{0.5cm} Let $ \lambda : A \to \widehat{A} $ be a polarisation.}
\only<5->{This defines an involution on $ \mathrm{End}^0(A) $ called the \textbf{Rosati involution} $ (\cdot)^\dagger : \mathrm{End}^0(A) \to \mathrm{End}^0(A) $, where
$$ A \xrightarrow{\phi} A \qquad \longmapsto \qquad A \xrightarrow{\lambda} \widehat{A} \xrightarrow{\widehat{\phi}} \widehat{A} \xrightarrow{\lambda^{-1}} A, $$
which is well-defined since $ \lambda^{-1} \in \mathrm{Hom}^0(\widehat{A}, A) $.}
\only<6>{It satisfies
$$ (\phi + \psi)^\dagger = \phi^\dagger + \psi^\dagger, \qquad (\phi \circ \psi)^\dagger = \psi^\dagger \circ \phi^\dagger, \qquad \phi, \psi \in \mathrm{End}^0(A), $$
and $ a^\dagger = a $ for any $ a \in \mathbb{Q} $.}

\end{frame}

\end{document}