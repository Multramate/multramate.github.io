\documentclass[10pt]{beamer}

\setbeamertemplate{footline}[page number]

\DeclareFontFamily{U}{wncyr}{}
\DeclareFontShape{U}{wncyr}{m}{n}{<->wncyr10}{}
\DeclareSymbolFont{cyr}{U}{wncyr}{m}{n}
\DeclareMathSymbol{\Sha}{\mathord}{cyr}{"58}

\usepackage{tikz-cd}
\usepackage{marvosym}

\theoremstyle{definition}
\newtheorem{assumptions}{Assumptions}
\newtheorem{proposition}{Proposition}
\newtheorem{consequences}{Consequences}
\newtheorem{remark}{Remark}

\begin{document}

\begin{frame}

\begin{center}

{\scriptsize London School of Geometry and Number Theory}

\vspace{0.5cm}

Mini Project

\vspace{1cm}

\textbf{\large Kolyvagin's work on the BSD conjecture} \footnote{\scriptsize Victor Kolyvagin, 1989. \textbf{Euler Systems}, in \emph{Grothendieck Festschrift}}

\vspace{1cm}

David Ang

\vspace{0.5cm}

{\scriptsize Thursday, 5 May 2022}

\end{center}

\end{frame}

\begin{frame}[t]{From Gross-Zagier to Kolyvagin}

\only<1-8>{
\begin{assumptions}
\begin{itemize}
\item<1-8> Elliptic curve $ E / \mathbb{Q} $ with modular parameterisation $ \phi : X_0(N) \twoheadrightarrow E $.
\item<2-8> Imaginary quadratic field $ K = \mathbb{Q}(\sqrt{-D}) $ with \textbf{Heegner condition}:
$$ p \mid N \qquad \implies \qquad p \ \text{is split in} \ K. $$
\end{itemize}
\end{assumptions}
}

\only<3-8>{\vspace{-1cm}
\begin{consequences}
\begin{itemize}
\item<3-8> An ideal $ \mathcal{N}_K \trianglelefteq \mathcal{O}_K $ such that $ \mathcal{O}_K / \mathcal{N}_K \cong \mathbb{Z} / N $.
\item<4-8> A cyclic $ N $-isogeny $ \mathbb{C} / \mathcal{O}_K \to \mathbb{C} / \mathcal{N}_K^{-1} $.
\item<5-8> A point $ x_1 \in X_0(N)\visible<6-8>{(K^1)} $ \visible<6-8>{by CM theory.}
\item<7-8> A \textbf{Heegner point} $ P_1 := \phi(x_1) \in E(K^1) $.
\item<8> A \textbf{basic Heegner point}
$$ P_K := \sum_{\sigma \in \mathrm{Gal}(K^1 / K)} \sigma(P_1) \in E(K). $$
\end{itemize}
\end{consequences}
}

\only<9-13>{Recall the Gross-Zagier formula. \vspace{0.5cm}
\begin{theorem}[Gross-Zagier, 1986]
$$ L'(E / K, 1) = c \cdot \widehat{h}(P_K). $$
\end{theorem}
}

\only<10-13>{If $ L'(E / K, 1) \ne 0 $, then $ \mathrm{rk}_{\mathbb{Z}} E(K) \ge 1 $. \vspace{0.5cm}}

\only<11->{
\begin{theorem}[Kolyvagin, 1989]
$$ \widehat{h}(P_K) \ne 0 \qquad \implies \qquad E(K)_{/ \mathrm{tors}} = \mathbb{Z} \cdot \tfrac{1}{n}P_K. $$
\end{theorem}
}

\only<12-13>{If $ L'(E / K, 1) \ne 0 $, then $ \mathrm{rk}_{\mathbb{Z}} E(K) = 1 $.}

\only<13>{\vspace{0.5cm} This \emph{almost} proves weak BSD for analytic rank $ \le 1 $!}

\only<14-18>{
\underline{Idea}: bound $ \mathrm{rk}_\mathbb{Z} E(K) $ with
$$ \delta : E(K) / \ell E(K) \hookrightarrow \mathrm{Sel}(K, E[\ell]), $$
for some prime $ \ell \in \mathbb{N} $.
\begin{itemize}
\item<15-18> Want $ \dim_{\mathbb{F}_\ell} E(K) / \ell E(K) = \mathrm{rk}_{\mathbb{Z}} E(K) $. \only<16-18>{Suffices to assume
$$ \mathrm{Gal}(\mathbb{Q}(E[\ell]) / \mathbb{Q}) \cong \mathrm{GL}_2(\mathbb{F}_\ell). $$
}%
\only<17-18>{\underline{Fact}: this implies $ E(K)[\ell] = 0 $.}
\item<18> Need
$$ P_K \notin \ell E(K). $$
\end{itemize}
}

\only<19->{
\begin{theorem}[main result \footnote{\scriptsize Benedict Gross, 1991. \textbf{Kolyvagin's work on modular elliptic curves}}]
Let $ \ell \in \mathbb{N} $ be an odd prime of good reduction such that
$$ \mathrm{Gal}(\mathbb{Q}(E[\ell]) / \mathbb{Q}) \cong \mathrm{GL}_2(\mathbb{F}_\ell), \qquad P_K \notin \ell E(K). $$
Then
$$ \mathrm{Sel}(K, E[\ell]) = \mathbb{F}_\ell \cdot \delta(P_K). $$
\end{theorem}
}

\only<20>{
\begin{remark}
There are infinitely many such $ \ell \in \mathbb{N} $.
\end{remark}
}

\end{frame}

\begin{frame}[t]{Generalised Selmer groups}

\only<1-5>{For each $ \ell \in \mathbb{N} $, there is a short exact sequence
$$ 0 \to E[\ell] \to E \xrightarrow{\cdot \ell} E \to 0. $$
}%

\only<2-5>{Applying $ \mathrm{Gal}(\overline{K} / K) $ cohomology,
$$
\begin{tikzcd}[ampersand replacement=\&, row sep=small]
0 \arrow{r} \& E(K)[\ell] \arrow{r} \& E(K) \arrow{r}{\cdot \ell} \& E(K) \arrow[in=165, out=-15, overlay]{dll}[swap]{\delta} \& \\
\& H^1(K, E[\ell]) \arrow{r} \& H^1(K, E) \arrow{r}[swap]{\cdot \ell} \& H^1(K, E) \arrow{r} \& \dots.
\end{tikzcd}
$$
}%

\only<3-9>{\only<3-5>{Truncating at $ H^1(K, E[\ell]) $,} \only<4-5>{and for each place $ v $ of $ K $,} \only<6-9>{There is an exact diagram}
$$
\begin{tikzcd}[ampersand replacement=\&, column sep=small, row sep=small]
0 \arrow{r} \& E(K) / \ell E(K) \arrow{r}{\delta} \only<5-9>{\arrow{d}} \& H^1(K, E[\ell]) \arrow{r} \only<5-9>{\arrow{d}[swap]{\only<9>{(\cdot)_v}}} \only<7-9>{\arrow[dashed]{dr}{(\cdot)^v}} \& H^1(K, E)[\ell] \arrow{r} \only<5-9>{\arrow{d}} \& 0 \\
\only<4-9>{0 \arrow{r}} \& \only<4-9>{E(K_v) / \ell E(K_v) \arrow{r}} \& \only<4-9>{H^1(K_v, E[\ell]) \arrow{r}} \& \only<4-9>{H^1(K_v, E)[\ell] \arrow{r}} \& \only<4-9>{0}
\end{tikzcd}.
$$
}

\only<7-9>{
\begin{itemize}
\item<7-9> The \textbf{classical} Selmer group is
$$ \mathrm{Sel}(K, E[\ell]) := \{c \in H^1(K, E[\ell]) \ \mid \ \forall v, \ c^v = 0\}. $$
\item<8-9> The \textbf{relaxed} Selmer group is
$$ \mathrm{Sel}^S(K, E[\ell]) := \{c \in H^1(K, E[\ell]) \ \mid \ \forall v \notin S, \ c^v = 0\}. $$
\item<9> The \textbf{restricted} Selmer group is
$$ \mathrm{Sel}_S(K, E[\ell]) := \{c \in \mathrm{Sel}^S(K, E[\ell]) \ \mid \ \forall v \in S, \ c_v = 0\}. $$
\end{itemize}
}

\only<10->{
\begin{proposition}
There is an exact sequence of $ \mathbb{F}_\ell $-vector spaces
$$ 0 \to \mathrm{Sel} \to \mathrm{Sel}^S \xrightarrow{\sigma_S} \displaystyle\prod_{v \in S} H^1(K_v, E)[\ell] \to \mathrm{Sel}^\vee \to \mathrm{Sel}_S^\vee \to 0. $$
\end{proposition}
}

\only<11->{\vspace{-0.5cm}
\begin{proof}
Local Tate duality and the Poitou-Tate exact sequence.
\end{proof}
}

\only<12->{
\only<12-16>{\begin{proposition}}
\only<17>{\begin{proposition}[sort of]}
There is a ``magical" set $ S $ of primes, inert in $ K / \mathbb{Q} $, such that
\begin{itemize}
\item<13-> $ H^1(K_p, E)[\ell] = \mathbb{F}_\ell \cdot c(p)^p $ for all $ p \in S $,
\item<14-> $ \mathrm{im}(\sigma_S) = \prod_{p \in S} \mathbb{F}_\ell \cdot c(p)^p $, and
\item<15-> $ \mathrm{Sel}_S = \mathbb{F}_\ell \cdot \delta(P_K) $.
\end{itemize}
\end{proposition}
}

\only<16->{
\begin{proof}
Chebotarev density and a lot of Galois cohomology.
\end{proof}
}

\end{frame}

\begin{frame}[t]{Derived Heegner points}

\only<1-8>{For any $ n \in \mathbb{N} $, there is a cohomology class $ c(n) \in H^1(K, E[\ell]) $ derived from a \textbf{Heegner point of conductor $ n $}.}

\only<2-8>{\renewcommand{\arraystretch}{2}
\begin{center}
\begin{tabular}{c|c}
conductor $ 1 $ & conductor $ n $ \\
\hline
ring of integers $ \mathcal{O}_K $ & \visible<3->{order $ \mathcal{O}_{K, n} := \mathbb{Z} + n\mathcal{O}_K $} \\
ideal $ \mathcal{N}_K \trianglelefteq \mathcal{O}_K $ & \visible<4->{ideal $ \mathcal{N}_{K, n} := \mathcal{N}_K \cap \mathcal{O}_{K, n} \trianglelefteq \mathcal{O}_{K, n} $} \\
$ N $-isogeny $ \mathbb{C} / \mathcal{O}_K \to \mathbb{C} / \mathcal{N}_K^{-1} $ & \visible<5->{$ N $-isogeny $ \mathbb{C} / \mathcal{O}_{K, n} \to \mathbb{C} / \mathcal{N}_{K, n}^{-1} $} \\
Hilbert class field $ K^1 $ & \visible<6->{ring class field $ K^n $} \\
point $ x_1 \in X_0(N)(K^1) $ & \visible<7->{point $ x_n \in X_0(N)(K^n) $} \\
Heegner point $ P_1 \in E(K^1) $ & \visible<8>{Heegner point $ P_n \in E(K^n) $}
\end{tabular}
\end{center}
}

\only<9-13>{The Heegner points $ P_n \in E(K^n) $ satisfy ``nice" relations over all $ n \in \mathbb{N} $.}

\only<10-13>{\vspace{0.5cm}
\underline{Fact}: If $ p $ is inert in $ K / \mathbb{Q} $, then
$$ \mathrm{Gal}(K^p / K^1) = \{1, \sigma_p, \sigma_p^2, \dots, \sigma_p^p\}. $$
}

\only<11-13>{\vspace{-0.5cm}
\only<11>{\begin{proposition}}
\only<12-13>{\begin{proposition}[don't worry about this]}
Let $ p \in S $. Then
$$ \sum_{i = 0}^p \sigma_p^iP_{pq} = a_pP_q \in E(K^q), \qquad \overline{P_{pq}} = \overline{\left(\dfrac{\mathfrak{p_q}}{K^q / K}\right)P_q} \in \overline{E}(\mathbb{F}_{\mathfrak{p_q}}). $$
\end{proposition}
}

\only<12-13>{
\begin{proof}
Consequence of the Eichler-Shimura congruence relation.
\end{proof}
}

\only<13>{\vspace{0.5cm} These are the axioms of an \textbf{AX3 Euler system}.}

\only<14->{Given $ P_p \in E(K^p) $, how to derive $ c(p) \in H^1(K, E[\ell]) $?}

\only<15->{\vspace{0.5cm} Define the \textbf{Kolyvagin derivative} operator by
$$ D_p := \sigma_p + 2\sigma_p^2 + \dots + p\sigma_p^p \in \mathbb{Z}[\mathrm{Gal}(K^p / K^1)]. $$
}%

\only<16->{Also define a ``trace" operator by
$$ T_p := \sum_{\tau \in T} \tau \in \mathbb{Z}[\mathrm{Gal}(K^p / K)], $$
where $ T $ is a set of coset representatives for $ \mathrm{Gal}(K^p / K^1) \le \mathrm{Gal}(K^p / K) $.}

\only<17>{\vspace{0.5cm} Define the \textbf{Kolyvagin class} $ c(p) \in H^1(K, E[\ell]) $ by
$$ c(p)(\sigma) := \sigma\left(\dfrac{1}{\ell}T_pD_pP_p\right) - \dfrac{1}{\ell}T_pD_pP_p - \dfrac{1}{\ell}(\sigma - 1)(T_pD_pP_p). $$
}

\end{frame}

\begin{frame}[t]{The Tate-Shafarevich group}

Kolyvagin proved something more.

\vspace{0.5cm}

\only<2-4>{
There is an exact diagram
$$
\begin{tikzcd}[ampersand replacement=\&, column sep=tiny]
0 \arrow{r} \& E(K) / \ell E(K) \arrow{r}{\delta} \arrow{d} \& H^1(K, E[\ell]) \arrow{r} \arrow{d} \only<3-4>{\arrow[dashed]{dr}[swap]{\sigma}} \& H^1(K, E)[\ell] \arrow{r} \arrow{d}{\only<4>{\tau[\ell]}} \& 0 \\
0 \arrow{r} \& \displaystyle\prod_v E(K_v) / \ell E(K_v) \arrow{r} \& \displaystyle\prod_v H^1(K_v, E[\ell]) \arrow{r} \& \displaystyle\prod_v H^1(K_v, E)[\ell] \arrow{r} \& 0
\end{tikzcd}.
$$
}%
\only<3-4>{The classical Selmer group is
$$ \mathrm{Sel}(K, E[\ell]) := \ker \sigma. $$
}%
\only<4>{The \textbf{Tate-Shafarevich group} is
$$ \Sha(K, E) := \ker \tau. $$
}

\only<5->{There is an exact sequence
$$ 0 \to E(K) / \ell E(K) \xrightarrow{\delta} \mathrm{Sel}(K, E[\ell]) \to \Sha(K, E)[\ell] \to 0. $$
}

\only<6->{
\begin{corollary}
Let $ \widehat{h}(P_K) \ne 0 $ and $ \ell \in \mathbb{N} $ be an odd prime of good reduction such that
$$ \mathrm{Gal}(\mathbb{Q}(E[\ell]) / \mathbb{Q}) \cong \mathrm{GL}_2(\mathbb{F}_\ell), \qquad P_K \notin \ell E(K). $$
Then $ \mathrm{rk}_{\mathbb{Z}} E(K) = 1 $ \only<7->{and $ \Sha(K, E)[\ell] = 0 $.}
\end{corollary}
}

\only<8>{\vspace{0.5cm} Kolyvagin also proved $ \Sha(K, E) $ is finite.}

\end{frame}

\begin{frame}{Thank you!}

\begin{center}
For more details:

\vspace{0.5cm}

\textbf{\large The Euler system of Heegner points}

\vspace{0.5cm}

{\small London Junior Number Theory Seminar}

{\footnotesize Tuesday, 10 May 2022, 17:15 -- 18:15}

{\scriptsize Room K6.63, King's Building, Strand Campus, King's College London}

\vspace{0.5cm}

Please come! \Smiley
\end{center}

\end{frame}

\end{document}