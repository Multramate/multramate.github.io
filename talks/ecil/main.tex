\documentclass[10pt]{beamer}

\setbeamertemplate{footline}[page number]

\usepackage{color}
\usepackage{listings}
\usepackage{tikz-cd}

\definecolor{keywordcolor}{rgb}{0.7, 0.1, 0.1}
\definecolor{tacticcolor}{rgb}{0.0, 0.1, 0.6}
\definecolor{commentcolor}{rgb}{0.4, 0.4, 0.4}
\definecolor{symbolcolor}{rgb}{0.0, 0.1, 0.6}
\definecolor{sortcolor}{rgb}{0.1, 0.5, 0.1}
\definecolor{attributecolor}{rgb}{0.7, 0.1, 0.1}
\def\lstlanguagefiles{lstlean.tex}
\lstset{language=lean}

\begin{document}

\begin{frame}

\begin{center}

\includegraphics[width=0.2\textwidth]{huawei.png}

{\large Mathematical Theorem Proving Workshop}

\vspace{0.5cm}

{\footnotesize Monday, 25 April 2022 --- Cambridge Research Centre}

\vspace{1cm}

\textbf{\Large Elliptic Curves in Lean}

\vspace{1cm}

David Kurniadi Angdinata

\vspace{0.5cm}

{\small London School of Geometry and Number Theory}

\end{center}

\end{frame}

\begin{frame}[t]{Informally}

What are elliptic curves?

\only<2->{
\begin{itemize}
\item Solutions to $ y^2 = x^3 + Ax + B $.
\end{itemize}
}

\only<3->{
\begin{center}
\includegraphics[width=0.4\textwidth]{ellipticcurves.png}
\end{center}
}

\only<4>{
\begin{itemize}
\item Points form a group!
\end{itemize}
}

\end{frame}

\begin{frame}[t]{Motivation}

Why do we care?

\vspace{0.5cm}

\only<2->{Public-key cryptography (over a large finite field)
\begin{itemize}
\item<3-> Integer factorisation (\emph{e.g.} Lenstra's method) \\
Breaks the RSA cryptosystem
\item<4-> Diffie-Hellman key exchange \\
Discrete logarithm (solve $ nQ = P $ given $ P $ and $ Q $)
\item<5-> Supersingular isogeny Diffie-Hellman key exchange
\end{itemize}
}

\vspace{0.5cm}

\only<6->{Number theory (over a field/ring/scheme)
\begin{itemize}
\item<7-> The simplest non-trivial objects in algebraic geometry
\item<8-> Rational elliptic curve associated to $ a^p + b^p = c^p $ cannot be modular \\
But rational elliptic curves are modular (modularity theorem)
\item<9> Distribution of ranks of rational elliptic curves \\
The BSD conjecture (analytic rank equals algebraic rank)
\end{itemize}
}

\end{frame}

\begin{frame}[t]{Globally}

An \textbf{elliptic curve $ E $ over a scheme $ S $} is a diagram
$$
\begin{tikzcd}[row sep=small]
E \arrow{d}[swap]{f} \\
S \only<2->{\arrow[bend right=90]{u}[swap]{0}}
\end{tikzcd}
$$
\only<3->{with a few technical conditions. \footnote{$ f $ is smooth, proper, and all its geometric fibres are integral curves of genus one}}

\vspace{0.5cm}

\only<4->{For a scheme $ T $ over $ S $,}
\only<5->{define the set of \textbf{$ T $-points} of $ E $ by $$ E(T) := \mathrm{Hom}_S(T, E), $$}
\only<6->{which is naturally identified with a \emph{Picard group} $ \mathrm{Pic}_{E / S}^0(T) $ of $ E $.}

\vspace{0.5cm}

\only<7->{This defines a contravariant functor $ \textbf{Sch}_S \to \textbf{Ab} $ given by $ T \mapsto E(T) $.}

\vspace{0.5cm}

\only<8>{Good for algebraic geometry, but not very friendly...}

\end{frame}

\begin{frame}[t]{Locally}

Let $ T / S $ be a field extension $ K / F $. \footnote{$ S = \mathrm{Spec} \ F $ and $ T = \mathrm{Spec} \ K $, or even rings whose class group has no $ 12 $-torsion}

\vspace{0.5cm}

\only<2->{An \textbf{elliptic curve $ E $ over a field $ F $} is a tuple $ (E, 0) $.}
\only<3->{
\begin{itemize}
\item<3-> $ E $ is a nice \footnote{smooth, proper, and geometrically integral} genus one curve over $ F $.
\item<4-> $ 0 $ is an $ F $-point.
\end{itemize}
}

\vspace{0.5cm}

\only<5->{The Picard group becomes $$ \mathrm{Pic}_{E / F}^0(K) = \dfrac{\{\text{degree zero divisors of} \ E \ \text{over} \ K\}}{\{\text{principal divisors of} \ E \ \text{over} \ K\}}. $$}

\only<6->{This defines a covariant functor $ \textbf{Alg}_F \to \textbf{Ab} $ given by $ K \mapsto E(K) $.}

\vspace{0.5cm}

\only<7>{Group law is free, but still need equations...}

\end{frame}

\begin{frame}[t]{Concretely}

The Riemann-Roch theorem gives \emph{Weierstrass equations}.

\vspace{0.5cm}

\only<2->{$ E(K) $ is basically the set of solutions $ (x, y) \in K^2 $ to $$ y^2 + a_1xy + a_3y = x^3 + a_2x^2 + a_4x + a_6, \qquad a_i \in F. $$}
\only<3->{If $ \mathrm{char} \ F \ne 2, 3 $, can reduce this to $$ y^2 = x^3 + Ax + B, \qquad A, B \in F. $$}
\only<4>{The \emph{group law} is reduced to drawing lines.
\begin{center}
\includegraphics[width=0.4\textwidth]{grouplaw.png}
\end{center}
}

\end{frame}

\begin{frame}[fragile, t]{Implementation}

Three definitions of elliptic curves:
\begin{enumerate}
\item Abstract definition over a scheme
\item Abstract definition over a field
\item Concrete definition over a field
\end{enumerate}
Generality: $ 1 \supset 2 \overset{\text{RR}}{=} 3 $
\begin{itemize}
\item<2-> $ 1. $ and $ 2. $ require much algebraic geometry (properness, genus, ...)
\item<3-> $ 2. = 3. $ also requires algebraic geometry (divisors, differentials, ...)
\item<4-> $ 3. $ requires just five coefficients (and a non-zero \emph{discriminant})!
\end{itemize}

\begin{onlyenv}<5->
\begin{lstlisting}[basicstyle=\scriptsize, frame=single]
def disc_aux {R : Type} [comm_ring R] (a₁ a₂ a₃ a₄ a₆ : R) : R :=
  -(a₁^2 + 4*a₂)^2*(a₁^2*a₆ + 4*a₂*a₆ - a₁*a₃*a₄ + a₂*a₃^2 - a₄^2)
  - 8*(2*a₄ + a₁*a₃)^3 - 27*(a₃^2 + 4*a₆)^2
  + 9*(a₁^2 + 4*a₂)*(2*a₄ + a₁*a₃)*(a₃^2 + 4*a₆)

structure EllipticCurve (R : Type) [comm_ring R] :=
  (a₁ a₂ a₃ a₄ a₆ : R) (disc : units R) (disc_eq : disc.val = disc_aux a₁ a₂ a₃ a₄ a₆)
\end{lstlisting}
\end{onlyenv}

\only<6>{This is the \emph{scheme} $ E $, but what about the \emph{abelian group} $ E(F) $?}

\end{frame}

\begin{frame}[fragile, t]{Points}

\begin{lstlisting}[basicstyle=\scriptsize, frame=single]
variables {F : Type} [field F] (E : EllipticCurve F) (K : Type) [field K] [algebra F K]

inductive point
  | zero
  | some (x y : K) (w : y^2 + E.a₁*x*y + E.a₃*y = x^3 + E.a₂*x^2 + E.a₄*x + E.a₆)

notation E(K) := point E K
\end{lstlisting}

\only<2-3>{\vspace{0.5cm} Identity}

\begin{onlyenv}<2-3>
\begin{lstlisting}[basicstyle=\scriptsize, frame=single]
instance : has_zero E(K) := ⟨zero⟩
\end{lstlisting}
\end{onlyenv}

\only<3>{\vspace{0.5cm} Negation}

\begin{onlyenv}<3>
\begin{lstlisting}[basicstyle=\scriptsize, frame=single]
def neg : E(K) → E(K)
  | zero := zero
  | (some x y w) := some x (-y - E.a₁*x - E.a₃) $ by { rw [← w], ring }

instance : has_neg E(K) := ⟨neg⟩
\end{lstlisting}
\end{onlyenv}

\only<4>{Addition}

\begin{onlyenv}<4>
\begin{lstlisting}[basicstyle=\scriptsize, frame=single]
def add : E(K) → E(K) → E(K)
  | zero P := P
  | P zero := P
  | (some x₁ y₁ w₁) (some x₂ y₂ w₂) :=
    if x_ne : x₁ ≠ x₂ then
      let L := (y₁ - y₂) / (x₁ - x₂),
          x₃ := L^2 + E.a₁*L - E.a₂ - x₁ - x₂,
          y₃ := -L*x₃ - E.a₁*x₃ - y₁ + L*x₁ - E.a₃
      in some x₃ y₃ $ by { ... }
    else if y_ne : y₁ + y₂ + E.a₁*x₂ + E.a₃ ≠ 0 then
      ...
    else
      zero

instance : has_add E(K) := ⟨add⟩
\end{lstlisting}
\end{onlyenv}

\only<5->{\vspace{0.5cm} Commutativity is doable}

\begin{onlyenv}<5->
\begin{lstlisting}[basicstyle=\scriptsize, frame=single]
lemma add_comm (P Q : E(K)) : P + Q = Q + P :=
  by { rcases ⟨P, Q⟩ with ⟨_ | _, _ | _⟩, ... }
\end{lstlisting}
\end{onlyenv}

\only<6>{\vspace{0.5cm} Associativity is difficult}

\begin{onlyenv}<6>
\begin{lstlisting}[basicstyle=\scriptsize, frame=single]
lemma add_assoc (P Q R : E(K)) : (P + Q) + R = P + (Q + R) :=
  by { rcases ⟨P, Q, R⟩ with ⟨_ | _, _ | _, _ | _⟩, ... }
\end{lstlisting}
\end{onlyenv}

\end{frame}

\begin{frame}[t]{Associativity}

Known to be very difficult with several proofs:
\only<2->{
\begin{itemize}
\item<2-> Just do it! \\
(times out with 130,000(!?) coefficients)
\item<3-> Uniformisation \\
(requires complex analysis and modular forms)
\item<4-> Cayley-Bacharach \\
(requires incidence geometry notions and B\'ezout's theorem)
\item<5-> $ E(K) \cong \mathrm{Pic}_{E / F}^0(K) $ \\
(requires divisors, differentials, and the Riemann-Roch theorem)
\end{itemize}
}

\vspace{0.5cm}

\only<6->{Current status:
\begin{itemize}
\item<6-> Left as a \texttt{sorry}
\item<7-> Attempt (by M Masdeu) to bash it out using \texttt{linear\_combination}
\item<8-> Proved (by E-I Bartzia and P-Y Strub) in Coq \only<8->{\footnote{A Formal Library for Elliptic Curves in the Coq Proof Assistant (2015)}} \\
that $ E(K) \cong \mathrm{Pic}_{E / F}^0(K) $ for $ \mathrm{char} \ F \ne 2, 3 $
\end{itemize}
}

\end{frame}

\begin{frame}[fragile, t]{Progress}

Modulo associativity, what has been done?

\only<2-3>{\vspace{0.5cm} Functoriality $ \textbf{Alg}_F \to \textbf{Ab} $}

\begin{onlyenv}<2-3>
\begin{lstlisting}[basicstyle=\tiny, frame=single]
def point_hom (φ : K →ₐ[F] L) : E(K) → E(L)
  | zero := zero
  | (some x y w) := some (φ x) (φ y) $ by { ... }

local notation K →[F] L := (algebra.of_id K L).restrict_scalars F

lemma point_hom.id (P : E(K)) : point_hom (K →[F] K) P = P := by cases P; refl

lemma point_hom.comp (P : E(K)) :
  point_hom (L →[F] M) (point_hom (K →[F] L) P) = point_hom ((L →[F] M).comp (K →[F] L)) P := by cases P; refl
\end{lstlisting}
\end{onlyenv}

\only<3>{\vspace{0.5cm} Galois module structure $ \mathrm{Gal}(L/K) \curvearrowright E(L) $}

\begin{onlyenv}<3>
\begin{lstlisting}[basicstyle=\tiny, frame=single]
def point_gal (σ : L ≃ₐ[K] L) : E(L) → E(L)
  | zero := zero
  | (some x y w) := some (σ • x) (σ • y) $ by { ... }

lemma point_gal.fixed : mul_action.fixed_points (L ≃ₐ[K] L) E(L) = (point_hom (K →[F] L)).range := by { ... }
\end{lstlisting}
\end{onlyenv}

\only<4>{\vspace{0.5cm} Isomorphisms $ (x, y) \mapsto (u^2x + r, u^3y + u^2sx + t) $}

\begin{onlyenv}<4>
\begin{lstlisting}[basicstyle=\tiny, frame=single]
variables (u : units F) (r s t : F)

def cov : EllipticCurve F :=
{ a₁ := u.inv*(E.a₁ + 2*s),
  a₂ := u.inv^2*(E.a₂ - s*E.a₁ + 3*r - s^2),
  a₃ := u.inv^3*(E.a₃ + r*E.a₁ + 2*t),
  a₄ := u.inv^4*(E.a₄ - s*E.a₃ + 2*r*E.a₂ - (t + r*s)*E.a₁ + 3*r^2 - 2*s*t),
  a₆ := u.inv^6*(E.a₆ + r*E.a₄ + r^2*E.a₂ + r^3 - t*E.a₃ - t^2 - r*t*E.a₁),
  disc := ⟨u.inv^12*E.disc.val, u.val^12*E.disc.inv, by { ... }, by { ... }⟩,
  disc_eq := by { simp only, rw [disc_eq, disc_aux, disc_aux], ring } }

def cov.to_fun : (E.cov u r s t)(K) → E(K)
  | zero := zero
  | (some x y w) := some (u.val^2*x + r) (u.val^3*y + u.val^2*s*x + t) $ by { ... }

def cov.inv_fun : E(K) → (E.cov u r s t)(K)
  | zero := zero
  | (some x y w) := some (u.inv^2*(x - r)) (u.inv^3*(y - s*x + r*s - t)) $ by { ... }

def cov.equiv_add : (E.cov u r s t)(K) ≃+ E(K) :=
  ⟨cov.to_fun u r s t, cov.inv_fun u r s t, by { ... }, by { ... }, by { ... }⟩
\end{lstlisting}
\end{onlyenv}

\only<5-6>{\vspace{0.5cm} $ 2 $-division polynomial $ \psi_2(x) $}

\begin{onlyenv}<5-6>
\begin{lstlisting}[basicstyle=\tiny, frame=single]
def ψ₂_x : cubic K := ⟨4, E.a₁^2 + 4*E.a₂, 4*E.a₄ + 2*E.a₁*E.a₃, E.a₃^2 + 4*E.a₆⟩

lemma ψ₂_x.disc_eq_disc : (ψ₂_x E K).disc = 16*E.disc := by { ... }
\end{lstlisting}
\end{onlyenv}

\only<6>{\vspace{0.5cm} Structure of $ E(K)[2] $}

\begin{onlyenv}<6>
\begin{lstlisting}[basicstyle=\tiny, frame=single]
notation E(K)[n] := ((•) n : E(K) →+ E(K)).ker

lemma E₂.x {x y w} : some x y w ∈ E(K)[2] ↔ x ∈ (ψ₂_x E K).roots := by { ... }

theorem E₂.card_le_four : fintype.card E(K)[2] ≤ 4 := by { ... }

variables [algebra ((ψ₂_x E F).splitting_field) K]

theorem E₂.card_eq_four : fintype.card E(K)[2] = 4 := by { ... }

lemma E₂.gal_fixed (σ : L ≃ₐ[K] L) (P : E(L)[2]) : σ • P = P := by { ... }
\end{lstlisting}
\end{onlyenv}

\end{frame}

\begin{frame}[t]{Mordell-Weil}

Let $ K $ be a number field. Then $ E(K) $ is finitely generated.

\vspace{0.5cm}

\only<2->{Show $ E(K)/2E(K) $ is finite:
\begin{enumerate}
\item<3-> Reduce to $ E[2] \subset E(K) $
\item<4-> Define $ 2 $-descent $ \delta : E(K)/2E(K) \hookrightarrow K^\times / (K^\times)^2 \times K^\times / (K^\times)^2 $
\item<5-> Show that $ \mathrm{im} \delta \le K(\emptyset, 2) \times K(\emptyset, 2) $
\item<6-> Prove exactness of $ 0 \to \mathcal{O}_K^\times / (\mathcal{O}_K^\times)^n \to K(\emptyset, n) \to \mathrm{Cl}_K[n] \to 0 $
\item<7-> Apply finiteness of $ \mathcal{O}_K^\times / (\mathcal{O}_K^\times)^n $ and $ \mathrm{Cl}_K[n] $
\end{enumerate}
}

\vspace{0.5cm}

\only<8->{Show this implies $ E(K) $ is finitely generated:
\begin{enumerate}
\item<9-> Define heights on elliptic curves
\item<10-> (J Zhang) Prove the descent theorem
\end{enumerate}
}

\vspace{0.5cm}

\only<11>{Soon: Mordell's theorem for $ E[2] \subset E(\mathbb{Q}) $.}

\end{frame}

\begin{frame}{Future}

Potential future projects:
\begin{itemize}
\item $ n $-division polynomials and the structure of $ E(K)[n] $
\item formal groups and local theory
\item ramification theory $ \implies $ Mordell-Weil theorem for number fields
\item Galois cohomology $ \implies $ Selmer and Tate-Shafarevich groups
\item modular forms $ \implies $ complex theory
\item algebraic geometry $ \implies $ proof of associativity
\end{itemize}

\visible<2>{Thank you!}

\end{frame}

\end{document}