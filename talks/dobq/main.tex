\documentclass[10pt]{beamer}

\setbeamertemplate{footline}[page number]

\DeclareFontFamily{U}{wncyr}{}
\DeclareFontShape{U}{wncyr}{m}{n}{<->wncyr10}{}
\DeclareSymbolFont{cyr}{U}{wncyr}{m}{n}
\DeclareMathSymbol{\Sha}{\mathord}{cyr}{"58}

\newtheorem{conjecture}{Conjecture}

\title{Denominators of BSD quotients}

\author{David Ang}

\institute{London School of Geometry and Number Theory}

\date{Wednesday, 31 July 2024}

\begin{document}

\frame{\titlepage}

\begin{frame}[t]{Mordell's theorem}

Let $ E $ be an elliptic curve over $ \mathbb{Q} $ given by a Weierstrass equation
$$ y^2 + a_1xy + a_3y = x^3 + a_2x^2 + a_4x + a_6, \qquad a_i \in \mathbb{Q}. $$

\pause

Its rational points forms a group $ E(\mathbb{Q}) $ under a geometric addition law.

\begin{theorem}[Mordell]
\vspace{-0.3cm} $$ E(\mathbb{Q}) \cong \mathrm{tor}(E) \oplus \mathbb{Z}^{\mathrm{rk}(E)}. $$
\end{theorem}

\pause

\vspace{0.3cm} The \textbf{torsion subgroup} $ \mathrm{tor}(E) $ is well understood.

\begin{theorem}[Mazur]
\vspace{-0.2cm} $$ \mathrm{tor}(E) \cong \begin{cases}
C_n & n = 1, 2, 3, 4, 5, 6, 7, 8, 9, 10, 12 \\
C_2 \oplus C_{2n} & n = 1, 2, 3, 4
\end{cases}. $$
\end{theorem}

\pause

\vspace{0.2cm} The \textbf{rank} $ \mathrm{rk}(E) $ is somewhat mysterious.

\end{frame}

\begin{frame}[t]{The Birch--Swinnerton-Dyer conjecture}

Assume $ E $ has conductor $ N $. The L-function of $ E $ is the infinite product
$$ L(E, s) := \prod_p \dfrac{1}{L_p(E, p^{-s})}. $$

\pause

Here,
$$ L_p(E, T) := \begin{cases}
1 \pm \epsilon T & \text{if} \ p \mid N \\
1 - a_p(E)T + pT^2 & \text{if} \ p \nmid N
\end{cases}, $$
where $ a_p(E) := 1 + p - \#E(\mathbb{F}_p) $ and $ \epsilon \in \{-1, 0, 1\} $.

\pause

\vspace{0.5cm}

\begin{conjecture}[weak Birch--Swinnerton-Dyer]
\vspace{-0.3cm} $$ \mathrm{ord}_{s = 1} L(E, s) = \mathrm{rk}(E). $$
\end{conjecture}

\pause

\vspace{0.3cm} This is known for $ \mathrm{ord}_{s = 1} L(E, s) \le 1 $. Assume that $ \mathrm{ord}_{s = 1} L(E, s) = 0 $.

\end{frame}

\begin{frame}[t]{The Birch--Swinnerton-Dyer quotient}

\begin{conjecture}[strong Birch--Swinnerton-Dyer]
\vspace{-0.2cm} $$ \dfrac{L(E, 1)}{\Omega(E)} = \dfrac{\mathrm{Tam}(E) \cdot \#\Sha(E)}{\#\mathrm{tor}(E)^2}. $$
\end{conjecture}

\only<2-7>{\vspace{0.2cm} The LHS is the \textbf{algebraic L-value} and the RHS is the \textbf{BSD quotient}.}
\only<3-7>{\begin{itemize}
\only<3-4>{\item The \textbf{Tamagawa product} is the finite product
$$ \mathrm{Tam}(E) := \prod_{p \mid N} [E(\mathbb{Q}_p) : E_0(\mathbb{Q}_p)], $$
where $ E_0(\mathbb{Q}_p) $ is the subgroup of points of $ E(\mathbb{Q}_p) $ whose reduction is \emph{nonsingular}. It can be computed by \emph{Tate's algorithm}.}
\only<4>{\item The \textbf{Tate--Shafarevich group} is the finite group
$$ \Sha(E) := \ker\left(H^2(\mathbb{Q}, E) \to H^2(\mathbb{R}, E) \times \prod_p H^2(\mathbb{Q}_p, E)\right). $$}
\only<5-7>{\item The \textbf{real period} is the integral
$$ \Omega(E) := \int_{E(\mathbb{R})} \omega_E, $$
where $ \omega_E $ is the \textbf{N\'eron differential}.} \only<6-7>{If $ E $ is given by a \emph{minimal} Weierstrass equation $ y^2 + a_1xy + a_3y = x^3 + a_2x^2 + a_4x + a_6 $,
$$ \omega_E = \dfrac{\mathrm{d}x}{2y + a_1x + a_3}. $$}
\only<7>{It is the least positive element of the \emph{real period lattice} of $ E $.}
\end{itemize}}

\end{frame}

{\usebackgroundtemplate{\includegraphics[width=\paperwidth]{lmfdb1.png}}
\begin{frame}[b]

\begin{minipage}{0.54\textwidth}\hfill\end{minipage}
\begin{minipage}{0.45\textwidth}\fbox{\tiny$ \dfrac{L(E, 1)}{\Omega(E)} = \dfrac{\mathrm{Tam}(E) \cdot \#\Sha(E)}{\#\mathrm{tor}(E)^2} $}\end{minipage}

\vspace{0.5cm}

\end{frame}
}

\begin{frame}[t]{Denominator bounds}

Observe that BSD quotients have bounded denominators.

\pause

\vspace{0.5cm}

\begin{theorem}[Mazur]
\vspace{-0.2cm} $$ \mathrm{tor}(E) \cong \begin{cases}
C_n & n = 1, 2, 3, 4, 5, 6, 7, 8, 9, 10, 12 \\
C_2 \oplus C_{2n} & n = 1, 2, 3, 4
\end{cases}. $$
\end{theorem}

\pause

\begin{corollary}
\vspace{-0.3cm} $$ \mathrm{ord}_p\left(\dfrac{\mathrm{Tam}(E) \cdot \#\Sha(E)}{\#\mathrm{tor}(E)^2}\right) \ge \begin{cases}
-8 & \text{if} \ p = 2 \\
-4 & \text{if} \ p = 3 \\
-2 & \text{if} \ p = 5, 7 \\
0 & \text{if} \ p \ge 11
\end{cases}. $$
\end{corollary}

\pause

There are typically cancellations between $ \mathrm{tor}(E) $ and $ \mathrm{Tam}(E) $.

\end{frame}

\begin{frame}[t]{Torsion cancellations}

\begin{theorem}[Lorenzini, 2010]
Assume that $ \mathrm{tor}(E) $ has a point of order $ n \ge 4 $.
\begin{itemize}
\item If $ n = 4 $, then $ 2 \mid \mathrm{Tam}(E) $, except for 15a7, 15a8, 17a4.
\item If $ n \ge 5 $, then $ n \mid \mathrm{Tam}(E) $, except for 11a3, 14a4, 14a6, 20a2.
\item If $ n = 9 $, then $ 27 \mid \mathrm{Tam}(E) $.
\end{itemize}
\end{theorem}

\pause

\begin{corollary}
With seven exceptions,
$$ \mathrm{ord}_p\left(\dfrac{\mathrm{Tam}(E) \cdot \#\Sha(E)}{\#\mathrm{tor}(E)^2}\right) \ge \begin{cases}
-5 & \text{if} \ p = 2 \ \text{and} \ \mathrm{tor}(E) \cong C_2 \oplus C_{2n} \\
-3 & \text{if} \ p = 2 \ \text{and} \ \mathrm{tor}(E) \not\cong C_2 \oplus C_{2n} \\
-2 & \text{if} \ p = 3 \ \text{and} \ \mathrm{tor}(E) \cong C_3 \\
-1 & \text{if} \ p = 3 \ \text{and} \ \mathrm{tor}(E) \not\cong C_3 \\
-1 & \text{if} \ p = 5, 7 \\
0 & \text{if} \ p \ge 11
\end{cases}. $$
\end{corollary}

\end{frame}

\begin{frame}[t]{The seven exceptions}

Let $ \mathrm{BSD}(E) $ denote the BSD quotient.
$$
\renewcommand{\arraystretch}{2}
\begin{array}{|c|c|c|c|c|c|c|c|}
\hline
E & 11a3 & 14a4 & 14a6 & 15a7 & 15a8 & 17a4 & 20a2 \\
\hline
\mathrm{tor}(E) & C_5 & C_6 & C_6 & C_4 & C_4 & C_4 & C_6 \\
\hline
\mathrm{Tam}(E) & 1 & 2 & 2 & 1 & 1 & 1 & 3 \\
\hline
\Sha(E) & 1 & 1 & 1 & 1 & 1 & 1 & 1 \\
\hline
\mathrm{BSD}(E) & \tfrac{1}{5^2} & \tfrac{1}{2 \cdot 3^2} & \tfrac{1}{2 \cdot 3^2} & \tfrac{1}{2^4} & \tfrac{1}{2^4} & \tfrac{1}{2^4} & \tfrac{1}{2^2 \cdot 3} \\
\hline\noalign{\pause}
c_0(E) & 5 & 3 & 3 & 2 & 4 & 4 & 2 \\
\hline
c_0(E)\mathrm{BSD}(E) & \tfrac{1}{5} & \tfrac{1}{2 \cdot 3} & \tfrac{1}{2 \cdot 3} & \tfrac{1}{2^3} & \tfrac{1}{2^2} & \tfrac{1}{2^2} & \tfrac{1}{2 \cdot 3} \\
\hline
\end{array}
$$

Here, $ c_0(E) $ is the \textbf{Manin constant} in the LMFDB.

\end{frame}

{\usebackgroundtemplate{\includegraphics[width=\paperwidth]{lmfdb2.png}}
\begin{frame}

\vspace{3cm}

\begin{minipage}{0.09\textwidth}\hfill\end{minipage}
\fbox{\begin{minipage}{0.13\textwidth}\vspace{0.5cm}\hfill\end{minipage}}

\end{frame}
}

\begin{frame}[t]{The Manin constant}

\begin{theorem}[Modularity, version $ L $]
There is an eigenform $ f_E \in S_2(\Gamma_0(N)) $ with eigenvalues $ a_p(E) $ such that
$$ L(f_E, s) = L(E, s). $$
\end{theorem}

In particular, this defines a differential $ f_E(q)\mathrm{d}q $ on $ X_0(N) $.

\pause

\begin{theorem}[Modularity, version $ X_\mathbb{Q} $]
There is a finite morphism $ \phi_E : X_0(N) \twoheadrightarrow E $ defined over $ \mathbb{Q} $ such that
$$ \phi_E^*\omega_E = c_0(E) \cdot f_E(q)\mathrm{d}q, $$
for some positive integer $ c_0(E) $.
\end{theorem}

\pause

\vspace{0.5cm} Conjecturally $ c_0(E) = 1 $ for all \emph{$ \Gamma_0(N) $-optimal} elliptic curves (known in the semistable case!), but the seven exceptions are not $ \Gamma_0(N) $-optimal.

\end{frame}

\begin{frame}[t]{A refined conjecture}

\begin{conjecture}
With no exceptions,
$$ \mathrm{ord}_p\left(\dfrac{c_0(E) \cdot \mathrm{Tam}(E) \cdot \#\Sha(E)}{\#\mathrm{tor}(E)^2}\right) \ge \begin{cases}
-3 & p = 2 \\
-1 & p = 3, 5, 7 \\
0 & p \ge 11
\end{cases}. $$
\end{conjecture}

\pause

This follows from Lorenzini's theorem, but the bound for $ p = 2 $ holds for $ \mathrm{tor}(E) \cong C_2 \oplus C_{2n} $, and the bound for $ p = 3 $ holds for $ \mathrm{tor}(E) \cong C_3 $.

\pause

\vspace{0.5cm}

\begin{conjecture}
Assume that $ \mathrm{tor}(E) \cong C_3 $. Then $ 3 \mid c_0(E) \cdot \mathrm{Tam}(E) \cdot \#\Sha(E) $.
\end{conjecture}

\pause

\vspace{0.5cm} I can prove this under the strong Birch--Swinnerton-Dyer conjecture.

\end{frame}

\begin{frame}[t]{Modular symbols}

If $ f \in S_2(\Gamma_0(N)) $ and $ p \nmid N $, the Hecke operator $ T_p $ acts on periods by
$$ (1 + p - T_p) \cdot \int_0^\infty f(q)\mathrm{d}q = \sum_{a = 1}^{p - 1} \int_0^{\tfrac{a}{p}} f(q)\mathrm{d}q. $$

\pause

If $ f = f_E $ and $ p $ is odd, this says that
$$ (1 + p - a_p(E)) \cdot (-L(E, 1)) = \dfrac{\Omega(E)}{c_0(E)} \cdot n, \quad n \in \mathbb{Z}. $$

\pause

If the strong Birch--Swinnerton-Dyer conjecture holds,
$$ (1 + p - a_p(E)) \cdot \dfrac{c_0(E) \cdot \mathrm{Tam}(E) \cdot \#\Sha(E)}{\#\mathrm{tor}(E)^2} \in \mathbb{Z}. $$

\pause

If $ \mathrm{tor}(E) \cong C_3 $, it suffices to find an odd prime $ p \nmid N $ such that
$$ 1 + p - a_p(E) \equiv 3 \mod 9. $$

\end{frame}

\begin{frame}[t]{3-adic Galois images}

In terms of $ \rho_{E, 3} : \mathrm{Gal}(\overline{\mathbb{Q}} / \mathbb{Q}) \to \mathrm{GL}_2(\mathbb{Q}_3) $,
$$ p = \det(\rho_{E, 3}(\mathrm{Fr}_p)), \qquad a_p(E) = \mathrm{tr}(\rho_{E, 3}(\mathrm{Fr}_p)). $$

\pause

Chebotarev's density theorem says that $ \mathrm{Fr}_p $ is uniformly distributed in $ \mathrm{im}(\rho_{E, 3}) $, so it suffices to find a matrix $ M \in \mathrm{im}(\rho_{E, 3}) $ such that
$$ 3 = 1 + \det(M) - \mathrm{tr}(M). $$

\pause

\begin{theorem}[Rouse--Sutherland--Zureick-Brown, 2022]
Assume that $ \mathrm{tor}(E) \cong C_3 $. Then $ \mathrm{im}(\rho_{E, 3}) $ is one of the explicit matrix subgroups 3.8.0.1, 3.24.0.1, 9.24.0.1/2, 9.72.0.1/2/3/4/6/7/8/9/10, 27.72.0.1, 27.648.13.25, 27.648.18.1, or 27.1944.55.31/37/43/44.
\end{theorem}

\pause

\vspace{0.5cm} Each $ \mathrm{im}(\rho_{E, 3}) $ contains a matrix $ M $ such that $ 3 = 1 + \det(M) - \mathrm{tr}(M) $, except for 9.72.0.1, but Tate's algorithm shows $ 3 \mid \mathrm{Tam}(E) $ in this case.

\end{frame}

\begin{frame}[t]{Concluding remarks}

\begin{theorem}[A., 2023]
Assume the $ 3 $-part of the strong Birch--Swinnerton-Dyer conjecture. Then
$$ \mathrm{ord}_p\left(\dfrac{c_0(E) \cdot \mathrm{Tam}(E) \cdot \#\Sha(E)}{\#\mathrm{tor}(E)^2}\right) \ge \begin{cases}
-3 & p = 2 \\
-1 & p = 3, 5, 7 \\
0 & p \ge 11
\end{cases}. $$
\end{theorem}

\pause

Note the similarity to a conjecture by Agashe--Stein (2005) that
$$ \dfrac{2 \cdot c_0(E) \cdot \mathrm{Tam}(E) \cdot \#\Sha(E)}{\#\mathrm{tor}(E)} \in \mathbb{Z}. $$

\pause

This is known for semistable optimal elliptic curves by Melistas (2023), building upon \v Cesnavi\v cius (2018) and Byeon--Kim--Yhee (2020).

\pause

\vspace{0.5cm} Does this generalise to $ \mathbb{F}_q(C) $ or $ \mathrm{ord}_{s = 1} L(E, s) \ge 1 $?

\end{frame}

\end{document}