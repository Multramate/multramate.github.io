\documentclass[10pt]{beamer}

\setbeamertemplate{footline}[page number]

\usepackage{tikz}
\usepackage{tikz-cd}

\DeclareFontFamily{U}{wncyr}{}
\DeclareFontShape{U}{wncyr}{m}{n}{<->wncyr10}{}
\DeclareSymbolFont{cyr}{U}{wncyr}{m}{n}
\DeclareMathSymbol{\Sha}{\mathord}{cyr}{"58}

\newtheorem*{conjecture}{Conjecture}

\begin{document}

\begin{frame}

\begin{center}

\vspace{1cm}

\textbf{\Large Rank heuristics for elliptic curves \footnote{\footnotesize partially based on the VaNTAGe seminar on 'Heuristics for the arithmetic of elliptic curves' by Bjorn Poonen on 1 September 2020}}

\vspace{1cm}

{\normalsize David Ang}

{\scriptsize Part III Seminar Series}

{\tiny Michaelmas 2020 - Friday, 4 December}

\end{center}

\end{frame}

\begin{frame}[t]{Elliptic curves}

\only<1-8>{
Let $ E $ be an elliptic curve over a number field $ K $.
}

\only<2-8>{
\vspace{0.5cm}
\begin{theorem}[Mordell-Weil]
$ E(K) $ is a finitely generated abelian group of the form
$$ E(K) \cong \operatorname{tors}(E / K) \oplus \mathbb{Z}^{\operatorname{rk}(E / K)}. $$
\end{theorem}
}

\only<3-5>{
The \textbf{torsion subgroup} $ \operatorname{tors}(E / K) $ is effectively computable.
}%
\only<4-5>{%
\begin{theorem}[Lutz-Nagell]
If $ (x, y) \in \operatorname{tors}(E / \mathbb{Q}) $, then $ y \in \mathbb{Z} $ and either $ y = 0 $ or $ y^2 \mid \Delta(E / \mathbb{Q}) $.
\end{theorem}
}%
\only<5>{%
\begin{theorem}[Mazur, Kamienny, Merel]
There are finitely many possibilities for $ \operatorname{tors}(E / K) $.
\end{theorem}
}

\only<6-8>{
The \textbf{rank} $ \operatorname{rk}(E / K) $ is computationally harder and more mysterious.
}%
\only<7-8>{%
\begin{conjecture}[Birch-Swinnerton-Dyer]
If $ K = \mathbb{Q} $, then
$$ \operatorname{ord}_{s = 1} L(E, s) = \operatorname{rk}(E / \mathbb{Q}). $$
\end{conjecture}
}%
\only<8>{%
\begin{theorem}[Kolyvagin]
BSD holds for modular elliptic curves with analytic rank zero and one.
\end{theorem}
}

\end{frame}

\begin{frame}[t]{Rank distribution conjecture}

\only<1-5>{
How is the rank distributed?
}

\only<2-5>{
\vspace{0.5cm}
Consider the set $ \mathcal{E}(\mathbb{Q}) $ of unique minimal representatives of isomorphism classes of elliptic curves over $ \mathbb{Q} $, ordered by the height function
$$ \mathfrak{h}(E : y^2 = x^3 + Ax + B) = \max(4|A|^3, 27|B|^2). $$
}%
\only<3-5>{%
\begin{conjecture}[Rank distribution]
The average rank of $ \mathcal{E}(\mathbb{Q}) $ is $ \tfrac{1}{2} $.
\end{conjecture}
}%
\only<4-5>{%
\begin{theorem}[Bhargava-Shankar 2015]
The average rank of $ \mathcal{E}(\mathbb{Q}) $ is at most $ \tfrac{7}{6} $.
\end{theorem}
}

\only<5>{
\vspace{0.5cm}
Combining these shows that BSD holds for a positive proportion of $ \mathcal{E}(\mathbb{Q}) $ (Kolyvagin 1989, Breuil-Conrad-Diamond-Taylor 2001, Nekov\'a\v r 2009, Dokchitser-Dokchitser 2010, Skinner-Urban 2015).
}

\end{frame}

\begin{frame}[t]{Rank boundedness conjecture}

\only<1-5>{
Is the rank bounded?
}%
\only<2-5>{%
Probably not...
}

\only<2-5>{
\vspace{0.5cm}
\begin{conjecture}[Rank boundedness]
There are $ E \in \mathcal{E}(\mathbb{Q}) $ of arbitrarily large rank.
\end{conjecture}
}%
\only<3-5>{%
\begin{theorem}[Shafarevich-Tate 1967, Ulmer 2002]
There are $ E \in \mathcal{E}(\mathbb{F}_p(T)) $ of arbitrarily large rank.
\end{theorem}
}%
\only<4-5>{%
\begin{theorem}[Elkies 2006]
There is $ E \in \mathcal{E}(\mathbb{Q}) $ with rank at least $ 28 $.
\end{theorem}
\begin{theorem}[Elkies-Klagsbrun 2020]
There is $ E \in \mathcal{E}(\mathbb{Q}) $ with rank exactly $ 20 $.
\end{theorem}
}%
\only<5>{
\vspace{0.5cm}
Many proponents of this (Cassels 1966, Tate 1974, Mestre 1982, Silverman 1986, Brumer 1992, Ulmer 2002, Farmer-Gonek-Hughes 2007).
}

\only<6-10>{
Is the rank bounded?
}%
\only<6-10>{%
Probably!
}

\only<6-10>{
\vspace{0.5cm}
\begin{conjecture}[Poonen et al \footnote{\footnotesize B. Poonen and E. Rains. 'Random maximal isotropic subspaces and Selmer groups'. In: J. Amer. Math. Soc 25 (2012)} \footnote{\footnotesize M. Bhargava, D. Kane, H. Lenstra, B. Poonen and E. Rains. 'Modelling the distribution of ranks, Selmer groups, and Shafarevich-Tate groups of elliptic curves'. In: Camb. J. Math. 3 (2015)} \footnote{\footnotesize J. Park, B. Poonen, J. Voight and M. Wood. 'A heuristic for boundedness of ranks of elliptic curves'. In: J. Eur. Math. Soc (2019)}]
There are finitely many $ E \in \mathcal{E}(\mathbb{Q}) $ with rank greater than $ 21 $.
\end{conjecture}
}%
\only<7-10>{%
\begin{itemize}
\item<7-10> Model $ p^e $-Selmer groups using intersection of quadratic submodules.
\item<8-10> Model Tate-Shafarevich groups using matrices with a fixed rank.
\item<9-10> Model the Mordell-Weil rank using matrices without fixing the rank.
\end{itemize}
}

\only<10>{
\vspace{0.5cm}
A few others also predict boundedness (N\'eron 1950, Honda 1960, Rubin-Silverberg 2000, Granville 2006, Watkins 2015).
}

\end{frame}

\begin{frame}[t]{The Selmer and Tate-Shafarevich groups}

\only<1-12>{
Let $ E $ be an elliptic curve over a number field $ K $.
}

\only<2-4>{
\vspace{0.5cm}
Multiplication by $ n \in \mathbb{N}^+ $ gives
$$ 0 \to E[n] \to E \xrightarrow{[n]} E \to 0. $$
}%
\only<3-4>{%
Applying $ \operatorname{Gal}(\overline{K} / K) $ cohomology gives
$$
\begin{tikzcd}[ampersand replacement=\&, column sep=tiny]
0 \arrow{r} \& E(K)[n] \arrow{r} \& E(K) \arrow{r} \& E(K) \arrow[in=180, out=0]{dll}[swap]{\delta} \& \\
\& H^1(K, E[n]) \arrow{r} \& H^1(K, E) \arrow{r} \& H^1(K, E) \arrow{r} \& \dots.
\end{tikzcd}
$$
}%
\only<4>{%
Truncating at $ H^1(K, E[n]) $ gives
$$
\begin{tikzcd}[ampersand replacement=\&]
0 \longrightarrow E(K) / n \arrow{r} \& H^1(K, E[n]) \arrow{r} \& H^1(K, E)[n] \longrightarrow 0
\end{tikzcd}.
$$
}

\only<5>{
\vspace{0.5cm}
There is a short exact sequence
}%
\only<6-7>{%
\vspace{0.5cm}
There are short exact sequences
}%
\only<5-7>{%
$$
\begin{tikzcd}[ampersand replacement=\&]
0 \longrightarrow E(K) / n \arrow{r} \& H^1(K, E[n]) \arrow{r} \& H^1(K, E)[n] \longrightarrow 0
\end{tikzcd}.
$$
}%
\only<6>{%
\vspace{0.05cm}
$$
\begin{tikzcd}[ampersand replacement=\&, column sep=small]
0 \longrightarrow E(K_v) / n \arrow{r} \& H^1(K_v, E[n]) \arrow{r} \& H^1(K_v, E)[n] \longrightarrow 0
\end{tikzcd}.
$$
}%
\only<7>{%
\vspace{0.05cm}
$$
\begin{tikzcd}[ampersand replacement=\&, column sep=tiny]
0 \arrow{r} \& \displaystyle\prod_v E(K_v) / n \arrow{r} \& \displaystyle\prod_v H^1(K_v, E[n]) \arrow{r} \& \displaystyle\prod_v H^1(K_v, E)[n] \arrow{r} \& 0
\end{tikzcd}.
$$
}%
\only<8-12>{%
\vspace{0.5cm}
There is a row-exact commutative diagram
$$
\begin{tikzcd}[ampersand replacement=\&, column sep=tiny]
0 \arrow{r} \& E(K) / n \arrow{r} \arrow{d} \& H^1(K, E[n]) \arrow{r} \arrow{d}[swap]{\only<10-12>{\lambda}} \only<9-12>{\arrow[dashed]{dr}{\sigma}} \& H^1(K, E)[n] \arrow{r} \arrow{d}{\only<11-12>{\tau[n]}} \& 0 \\
0 \arrow{r} \& \displaystyle\prod_v E(K_v) / n \arrow{r}[swap]{\only<10-12>{\kappa}} \& \displaystyle\prod_v H^1(K_v, E[n]) \arrow{r} \& \displaystyle\prod_v H^1(K_v, E)[n] \arrow{r} \& 0
\end{tikzcd}.
$$
}%
\only<9-10>{%
The \textbf{$ n $-Selmer group} is
$$ \mathcal{S}_n(E / K) = \ker (\sigma : H^1(K, E[n]) \to \textstyle\prod_v H^1(K_v, E)[n]). $$
}%
\only<10>{%
Exactness gives
$$ \mathcal{S}_n(E / K) / \ker \lambda \xrightarrow{\sim} \operatorname{im} \kappa \cap \operatorname{im} \lambda. $$
}%
\only<11-12>{%
The \textbf{Tate-Shafarevich group} is
$$ \Sha(E / K) = \ker (\tau : H^1(K, E) \to \textstyle\prod_v H^1(K_v, E)). $$
}%
\only<12>{%
There is an exact sequence
$$ 0 \to E(K) / n \to \mathcal{S}_n(E / K) \to \Sha(E / K)[n] \to 0. $$
}

\end{frame}

\begin{frame}[t]{Modelling $ p^e $-Selmer groups}

\only<1-22>{
\begin{theorem}
For almost all $ E \in \mathcal{E}(K) $, the $ p^e $-Selmer group $ \mathcal{S}_{p^e}(E / K) $ is the intersection of two maximal totally isotropic direct summands in a non-degenerate quadratic $ \mathbb{Z} / p^e $-module of infinite rank.
\end{theorem}
}

\only<2-3>{
\vspace{0.5cm}
Consider $ (\mathbb{Z} / p^e)^{2n} $, equipped with hyperbolic quadratic form
$$ (x_1, \dots, x_n, y_1, \dots, y_n) \mapsto \sum_{i = 1}^n x_iy_i, $$
with two MTIDS's $ (\mathbb{Z} / p^e)^n \oplus 0^n $ and $ 0^n \oplus (\mathbb{Z} / p^e)^n $.
}

\only<3>{
\vspace{0.5cm}
The result was known for a finite-dimensional vector space over $ \mathbb{F}_2 $ (Colliot-Th\'el\`ene-Skorobogatov-Swinnerton-Dyer 2002).
}

\only<4-18>{
\vspace{0.5cm}
\begin{proof}[Proof (Sketch)]
\renewcommand{\qedsymbol}{}
Recall that $ \mathcal{S}_n(E / K) / \ker \lambda \cong \operatorname{im} \kappa \cap \operatorname{im} \lambda $.
\begin{enumerate}
\item<5-18> Construct the local non-degenerate quadratic module.
\only<6-9>{
\begin{itemize}
\item<6-9> Construct $ \Theta $ such that $ 0 \to \overline{K_v}^\times \to \Theta \to E[n] \to 0 $.
\item<7-9> Construct $ \operatorname{Ob}_{K_v} : H^1(K_v, E[n]) \to \operatorname{Br} K_v \hookrightarrow \mathbb{Q} / \mathbb{Z} $.
\item<8-9> Prove $ \langle\cdot, \cdot\rangle_{\operatorname{Ob}_{K_v}} = [\cdot, \cdot] \circ \cup $, and deduce $ \operatorname{Ob}_{K_v} $ is a quadratic form.
\item<9> Deduce non-degeneracy with local arithmetic duality.
\end{itemize}
}
\item<10-18> Prove $ \operatorname{im} \kappa $ and $ \operatorname{im} \lambda $ are maximal totally isotropic.
\only<11-13>{
\begin{itemize}
\item<11-13> Use basic properties of Brauer-Severi diagrams to redefine $ \operatorname{Ob}_{K_v} $.
\item<12-13> Define $ M = \overline{\prod_v} H^1(K_v, E[n]) $ and $ \mathfrak{q} = \sum_v \operatorname{inv}_{K_v} \circ \operatorname{Ob}_{K_v} : M \to \mathbb{Q} / \mathbb{Z} $.
\item<13> Conclude by B-S diagrams, class field theory, and arithmetic duality.
\end{itemize}
}
\item<14-18> Prove $ \operatorname{im} \kappa $ and $ \operatorname{im} \lambda $ are direct summands.
\only<15>{
\begin{itemize}
\item<15> Use infinite group theory to characterise direct summands in terms of divisibility-preserving maps and apply global arithmetic duality.
\end{itemize}
}
\item<16-18> Attain good criterion for $ \ker \lambda = 0 $ when $ n = p^e $.
\only<17-18>{
\begin{itemize}
\item<17-18> Use Chebotarev's density theorem to reduce to $ H_c^1(\operatorname{im} \rho_{E[n]}, E[n]) $ and apply inflation-restriction repeatedly to reduce to $ \operatorname{SL}_2(\mathbb{Z} / n) $.
\item<18> Extract assumption $ \operatorname{SL}_2(\mathbb{Z} / n) \le \operatorname{im} \rho_{E[n]} $ and justify its ubiquity using Hilbert's irreducibility theorem and $ n $-division polynomials.
$ \square $
\end{itemize}
}
\end{enumerate}
\end{proof}
}

\only<19-22>{
\vspace{0.5cm}
\begin{conjecture}
The distribution of $ \mathcal{S}_{p^e}(E / \mathbb{Q}) $ coincides with the distribution of $ S_1 \cap S_2 $ for two randomly chosen MTIDS's $ S_1, S_2 \subseteq (\mathbb{Z} / p^e)^{2n} $ as $ n \to \infty $.
\end{conjecture}
}%
\only<20-22>{%
\begin{itemize}
\item<20-22> Variant for function fields is known (Feng-Landesman-Rains 2020).
\item<21-22> Variant for quadratic twist families over $ \mathbb{Q} $ is known for $ p^e = 2 $ (Heath-Brown 1994, Swinnerton-Dyer 2008, Kane 2013).
\item<22> Average of $ \#(S_1 \cap S_2) $ is $ \sigma_1(p^e) $, and average of $ \#\mathcal{S}_{p^e}(E / \mathbb{Q}) $ is $ \sigma_1(p^e) $ for $ p^e \le 5 $ (Bhargava-Shankar 2013-2015).
\end{itemize}
}

\end{frame}

\begin{frame}[t]{Modelling short exact sequences}

\only<1-7>{
Recall that
$$ 0 \to E(K) / n \to \mathcal{S}_n(E / K) \to \Sha(E / K)[n] \to 0. $$
}%
\only<2-7>{%
Setting $ n = p^e $ and taking direct limits gives
$$ 0 \to E(K) \otimes \mathbb{Q}_p / \mathbb{Z}_p \to \varinjlim_e \mathcal{S}_{p^e}(E / K) \to \Sha(E / K)[p^\infty] \to 0. $$
}%
\only<3-7>{%
Randomly choosing two MTIDS's $ S_1, S_2 \subseteq (\mathbb{Z}_p)^{2n} $ gives
$$ 0 \to \mathcal{R} \to \mathcal{S} \to \mathcal{T} \to 0, $$
where $ \mathcal{R} = (S_1 \cap S_2) \otimes \mathbb{Q}_p / \mathbb{Z}_p $ and $ \mathcal{S} = (S_1 \otimes \mathbb{Q}_p / \mathbb{Z}_p) \cap (S_2 \otimes \mathbb{Q}_p / \mathbb{Z}_p) $.
}%
\only<4-7>{%
\begin{itemize}
\item<4-7> Both $ \varinjlim_e \mathcal{S}_{p^e}(E / K) $ and $ \mathcal{S} $ are compatible with $ p^e $-parts.
\item<5-7> Both $ \Sha(E / K)[p^\infty] $ and $ \mathcal{T} $ are finite with an alternating pairing.
\item<6-7> Both $ E(K) \otimes \mathbb{Q}_p / \mathbb{Z}_p $ and $ \mathcal{R} $ satisfy the rank distribution conjecture.
\item<7> Variant for quadratic twist families is known for $ p = 2 $ (Smith 2020).
\end{itemize}
}

\end{frame}

\begin{frame}[t]{Modelling Tate-Shafarevich groups}

\only<1-6>{
The rank distribution conjecture gives
$$ \mathbb{P}(\operatorname{rk}_{\mathbb{Z}_p}(S_1 \cap S_2) = 0) = \mathbb{P}(\operatorname{rk}_{\mathbb{Z}_p}(S_1 \cap S_2) = 1) = \dfrac{1}{2}. $$
}%
\only<2-6>{%
If $ r \ge 2 $, then
$$ \{S_1, S_2 \subseteq \mathbb{Z}_p^{2n} \mid \operatorname{rk}_{\mathbb{Z}_p}(S_1 \cap S_2) = r\} $$
has measure zero as $ n \to \infty $.
}

\only<3-6>{
\vspace{0.5cm}
Instead choose $ M $ randomly from
$$ \{M \in \operatorname{Mat}_n \mathbb{Z}_p \mid M^\intercal = -M, \ \operatorname{rk}_{\mathbb{Z}_p}(\operatorname{ker} M) = r\}, \qquad n \equiv r \mod 2, $$
and let $ n \to \infty $.}
\only<4-6>{
Use distribution of $ \operatorname{tors}(\operatorname{coker} M) $ to model $ \mathcal{T} $.
}%
\only<5-6>{%
\begin{itemize}
\item<5-6> Coincides with original $ \mathbb{Z}_p^{2n} $ distribution for $ \mathcal{T} $ for rank zero and one.
\item<6> Coincides with Delaunay's distribution for $ \Sha(E / \mathbb{Q})[p^\infty] $ (Delaunay-Jouhet 2000-2014).
\end{itemize}
}

\end{frame}

\begin{frame}[t]{Modelling ranks}

\only<1-5>{
Instead of choosing $ M $ randomly from
$$ \{M \in \operatorname{Mat}_n \mathbb{Z}_p \mid M^\intercal = -M, \ \operatorname{rk}_{\mathbb{Z}_p}(\operatorname{ker} M) = r\}, \qquad n \equiv r \mod 2, $$
}%
\only<2-5>{%
choose $ M $ randomly from
$$ \{M \in \operatorname{Mat}_n \mathbb{Z}_p \mid M^\intercal = -M\}, \qquad n \equiv r \mod 2, $$
and use distribution of $ \operatorname{rk}_{\mathbb{Z}_p}(\operatorname{ker} M) $ to model $ r $.
}%
\only<3-5>{%
\begin{itemize}
\item<3-5> Measure zero locus.
\item<4-5> Alternating matrices have even rank.
\end{itemize}
}

\only<5>{
\vspace{0.5cm}
Need a more refined model.
}

\only<6-12>{
How to model an elliptic curve $ E $ over $ \mathbb{Q} $ of height $ h $?
}%
\only<7-12>{%
\begin{itemize}
\item<7-12> Choose functions $ X : \mathbb{N} \to \mathbb{R} $ and $ Y : \mathbb{N} \to \mathbb{R} $ such that
$$ X(x)^{Y(x)} = x^{\tfrac{1}{12} + \operatorname{o}(1)}, \qquad x \to \infty. $$
\item<8-12> Choose $ n $ randomly from $ \{\lceil Y(h) \rceil, \lceil Y(h) \rceil + 1\} $.
\item<9-12> Choose $ M $ randomly from
$$ \{M \in \operatorname{Mat}_n \mathbb{Z} \mid M^\intercal = -M, \ M_{ij} \le X(h)\}. $$
\item<10-12> Model $ \Sha(E / \mathbb{Q}) $ by $ \operatorname{tors}(\operatorname{coker} M) $ and $ \operatorname{rk}(E / \mathbb{Q}) $ by $ \operatorname{rk}_{\mathbb{Z}}(\operatorname{ker} M) $.
\end{itemize}
}%
\only<11-12>{%
Conditions are chosen such that the average size of
$$ \#\operatorname{coker}_0' M =
\begin{cases}
\#\operatorname{tors}(\operatorname{coker} M) & \operatorname{rk}_{\mathbb{Z}}(\operatorname{ker} M) = 0 \\
0 & \operatorname{rk}_{\mathbb{Z}}(\operatorname{ker} M) > 0
\end{cases}
$$
is $ h^{1 / 12 + \operatorname{o}(1)} $.
}
\only<12>{
The same is predicted for $ \Sha(E / \mathbb{Q}) $ by strong BSD.
}

\only<13-14>{
Denote the model for $ \operatorname{rk}(E / \mathbb{Q}) $ by $ \operatorname{rk}'(E / \mathbb{Q}) $.
}

\only<14>{
\vspace{0.5cm}
\begin{theorem}[Poonen et al]
The following hold with probability $ 1 $.
$$
\begin{align*}
\#\{E \in \mathcal{E}(\mathbb{Q}) \mid \mathfrak{h}(E) \le h, \ \operatorname{rk}'(E / \mathbb{Q}) = 0\} & = h^{20 / 24 + \operatorname{o}(1)} \\
\#\{E \in \mathcal{E}(\mathbb{Q}) \mid \mathfrak{h}(E) \le h, \ \operatorname{rk}'(E / \mathbb{Q}) = 1\} & = h^{20 / 24 + \operatorname{o}(1)} \\
\#\{E \in \mathcal{E}(\mathbb{Q}) \mid \mathfrak{h}(E) \le h, \ \operatorname{rk}'(E / \mathbb{Q}) \ge 2\} & = h^{19 / 24 + \operatorname{o}(1)} \\
& \vdots \\
\#\{E \in \mathcal{E}(\mathbb{Q}) \mid \mathfrak{h}(E) \le h, \ \operatorname{rk}'(E / \mathbb{Q}) \ge 20\} & = h^{1 / 24 + \operatorname{o}(1)} \\
\#\{E \in \mathcal{E}(\mathbb{Q}) \mid \mathfrak{h}(E) \le h, \ \operatorname{rk}'(E / \mathbb{Q}) \ge 21\} & \le h^{\operatorname{o}(1)} \\
\#\{E \in \mathcal{E}(\mathbb{Q}) \mid \operatorname{rk}'(E / \mathbb{Q}) > 21\} & \ \text{is finite}
\end{align*}.
$$
\end{theorem}
}

\end{frame}

\begin{frame}

\begin{center}

{\Huge THANK YOU}

\end{center}

\end{frame}

\end{document}