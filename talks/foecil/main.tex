\documentclass[10pt]{beamer}

\setbeamertemplate{footline}[page number]

\usepackage{color}
\usepackage{listings}

\definecolor{keywordcolor}{rgb}{0.7, 0.1, 0.1}
\definecolor{tacticcolor}{rgb}{0.0, 0.1, 0.6}
\definecolor{commentcolor}{rgb}{0.4, 0.4, 0.4}
\definecolor{symbolcolor}{rgb}{0.0, 0.1, 0.6}
\definecolor{sortcolor}{rgb}{0.1, 0.5, 0.1}
\definecolor{attributecolor}{rgb}{0.7, 0.1, 0.1}
\def\lstlanguagefiles{lstlean.tex}
\lstset{language=lean}

\begin{document}

\begin{frame}

\begin{center}

{\small Young Researchers in Algebraic Number Theory}

\vspace{0.5cm}

{\footnotesize Wednesday, 24 August 2022}

\vspace{1cm}

\textbf{\large Formalisation of elliptic curves in Lean}

\vspace{1cm}

David Kurniadi Angdinata

\vspace{0.5cm}

{\scriptsize London School of Geometry and Number Theory}

\end{center}

\end{frame}

\begin{frame}[c]{The Lean theorem prover}

\begin{center}
\includegraphics[width=\textwidth]{LEAN.png}
\end{center}

\visible<2->{
\begin{flushleft}
A functional programming language...
\end{flushleft}
}

\visible<3>{
\begin{flushright}
and an interactive theorem prover!
\end{flushright}
}

\end{frame}

\begin{frame}[fragile, t]{Programming in Lean}

\only<1-3>{
\begin{center}
Idea: \textit{set theory} is replaced by \texttt{Type Theory}.
$$ \textit{element} \ \in \ \textit{set} \quad \implies \quad \texttt{Term : Type} $$
\end{center}
}

\only<2-3>{\vspace{0.5cm} Can define inductive types.}

\begin{onlyenv}<2-3>
\begin{lstlisting}[basicstyle=\scriptsize, frame=single]
inductive Nat
  | zero : Nat
  | succ : Nat → Nat
\end{lstlisting}
\end{onlyenv}

\only<3>{\vspace{0.5cm} Can define functions recursively.}

\begin{onlyenv}<3>
\begin{lstlisting}[basicstyle=\scriptsize, frame=single]
def add : Nat → Nat → Nat
  | n zero := n
  | n (succ m) := succ (add n m)
\end{lstlisting}
\end{onlyenv}

\only<4->{How to prove $ \forall n \in \mathbb{N}, \ 0 + n = n $?}

\only<5->{\vspace{0.5cm} A theorem is a \texttt{Type} (of type \texttt{Prop}).}

\begin{onlyenv}<5->
\begin{lstlisting}[basicstyle=\scriptsize, frame=single]
theorem zero_add : ∀ (n : Nat), add zero n = n :=
\end{lstlisting}
\end{onlyenv}

\only<6->{\vspace{0.5cm} A proof of this theorem (if it exists) is the unique \texttt{Term} of this type.}

\begin{onlyenv}<7->
\begin{lstlisting}[basicstyle=\scriptsize, frame=single]
begin
  intro n,
  induction n with m hm,
  { refl },
  { rw [add, hm] }
end
\end{lstlisting}
\end{onlyenv}

\only<7->{The keywords \texttt{intro}, \texttt{induction}, \texttt{refl}, and \texttt{rw} are \textbf{tactics}.}

\only<8>{\vspace{0.5cm} Play \textbf{The Natural Number Game}!}

\end{frame}

\begin{frame}[fragile, t]{Lean's mathematical library \texttt{mathlib}}

\only<1-3>{Community-driven unified library of mathematics formalised in Lean.}

\only<2-3>{\vspace{0.5cm}
\begin{minipage}{0.49\textwidth}
\begin{itemize}
\item algebra
\item algebraic\_geometry
\item algebraic\_topology
\item analysis
\item category\_theory
\item combinatorics
\item computability
\item dynamics
\item field\_theory
\item geometry
\item group\_theory
\end{itemize}
\end{minipage}
\begin{minipage}{0.49\textwidth}
\begin{itemize}
\item information\_theory
\item linear\_algebra
\item measure\_theory
\item model\_theory
\item number\_theory
\item order
\item probability
\item representation\_theory
\item ring\_theory
\item set\_theory
\item topology
\end{itemize}
\end{minipage}
}

\only<3>{\vspace{0.5cm} 3k files, 1m lines, 40k definitions, 100k theorems, 270 contributors.}

\only<4->{Consider the following theorem in \texttt{group\_theory/quotient\_group}.}

\begin{onlyenv}<4->
\begin{lstlisting}[basicstyle=\scriptsize, frame=single]
variables {G H : Type} [group G] [group H]
variables (φ : G →* H) (ψ : H → G) (hφ : right_inverse ψ φ)

def quotient_ker_equiv_of_right_inverse : G / ker φ ≃* H :=
  { to_fun := ker_lift φ,
    inv_fun := mk ∘ ψ,
    left_inv := ...,
    right_inv := hφ,
    map_mul' := ... }
\end{lstlisting}
\end{onlyenv}

\only<5->{Why is this a definition?}

\only<6->{\vspace{0.5cm} Consider an immediate corollary.}

\begin{onlyenv}<6->
\begin{lstlisting}[basicstyle=\scriptsize, frame=single]
def quotient_bot : G / (⊥ : subgroup G) ≃* G :=
  quotient_ker_equiv_of_right_inverse (monoid_hom.id G) id (λ _, rfl)
\end{lstlisting}
\end{onlyenv}

\only<7->{Why is this not trivial?}

\only<8>{\vspace{0.5cm} Canonical isomorphisms are important data!}

\end{frame}

\begin{frame}[fragile, t]{Elliptic curves in Lean}

\only<1-6>{What generality?}
\only<2-6>{Ideally, defined abstractly over a scheme or a ring...}
\only<3-6>{However, \texttt{mathlib}'s algebraic geometry is still quite primitive.}

\only<4-6>{\vspace{0.5cm} Here is a working definition in \texttt{algebraic\_geometry/EllipticCurve}.}

\begin{onlyenv}<4-6>
\begin{lstlisting}[basicstyle=\scriptsize, frame=single]
def Δ_aux {R : Type} [comm_ring R] (a₁ a₂ a₃ a₄ a₆ : R) : R :=
  let
    b₂ := a₁^2 + 4*a₂,
    b₄ := 2*a₄ + a₁*a₃,
    b₆ := a₃^2 + 4*a₆,
    b₈ := a₁^2*a₆ + 4*a₂*a₆ - a₁*a₃*a₄ + a₂*a₃^2 - a₄^2
  in
    -b₂^2*b₈ - 8*b₄^3 - 27*b₆^2 + 9*b₂*b₄*b₆

structure EllipticCurve (R : Type) [comm_ring R] :=
  (a₁ a₂ a₃ a₄ a₆ : R) (Δ : units R) (Δ_eq : ↑Δ = Δ_aux a₁ a₂ a₃ a₄ a₆)
\end{lstlisting}
\end{onlyenv}

\only<5-6>{Accurate for rings $ R $ with $ \mathrm{Pic}(R)[12] = 0 $, such as PIDs!}

\only<6>{\vspace{0.5cm} Much can be done just with this definition.}

\only<7->{Can define $ K $-points.}

\begin{onlyenv}<7->
\begin{lstlisting}[basicstyle=\scriptsize, frame=single]
variables {F : Type} [field F] (E : EllipticCurve F) (K : Type) [field K] [algebra F K]

inductive point
  | zero
  | some (x y : K) (w : y^2 + E.a₁*x*y + E.a₃*y = x^3 + E.a₂*x^2 + E.a₄*x + E.a₆)

notation E(K) := point E K
\end{lstlisting}
\end{onlyenv}

\only<8-9>{Can define zero.}

\begin{onlyenv}<8-9>
\begin{lstlisting}[basicstyle=\scriptsize, frame=single]
instance : has_zero E(K) := ⟨zero⟩
\end{lstlisting}
\end{onlyenv}

\only<9>{Can define negation.}

\begin{onlyenv}<9>
\begin{lstlisting}[basicstyle=\scriptsize, frame=single]
def neg : E(K) → E(K)
  | zero := zero
  | (some x y w) := some x (-y - E.a₁*x - E.a₃)
    begin
      rw [← w],
      ring
    end

instance : has_neg E(K) := ⟨neg⟩
\end{lstlisting}
\end{onlyenv}

\only<10>{Can define addition.}

\begin{onlyenv}<10>
\begin{lstlisting}[basicstyle=\scriptsize, frame=single]
def add : E(K) → E(K) → E(K)
  | zero P := P
  | P zero := P
  | (some x₁ y₁ w₁) (some x₂ y₂ w₂) :=
    if x_ne : x₁ ≠ x₂ then
      let
        L := (y₁ - y₂) / (x₁ - x₂),
        x₃ := L^2 + E.a₁*L - E.a₂ - x₁ - x₂,
        y₃ := -L*x₃ - E.a₁*x₃ - y₁ + L*x₁ - E.a₃
      in
        some x₃ y₃ ... -- 100 lines
    else ... -- 100 lines

instance : has_add E(K) := ⟨add⟩
\end{lstlisting}
\end{onlyenv}

\only<11->{Can prove group axioms} \only<12->{(except associativity, which is left as a \texttt{sorry}).}

\begin{onlyenv}<11->
\begin{lstlisting}[basicstyle=\scriptsize, frame=single]
lemma zero_add (P : E(K)) : 0 + P = P := ...

lemma add_zero (P : E(K)) : P + 0 = P := ...

lemma add_left_neg (P : E(K)) : -P + P = 0 := ...

lemma add_comm (P Q : E(K)) : P + Q = Q + P := ... -- 100 lines

lemma add_assoc (P Q R : E(K)) : (P + Q) + R = P + (Q + R) := ... -- ?? lines
\end{lstlisting}
\end{onlyenv}

\only<13>{Can also prove Galois-theoretic properties and structure of torsion points.}

\end{frame}

\begin{frame}[t]{The Mordell-Weil theorem in Lean}

Can prove Mordell's theorem by complete $ 2 $-descent and na\"ive heights.

\begin{theorem}[Mordell]
$ E(\mathbb{Q}) $ is finitely generated.
\end{theorem}

\only<2-7>{
\begin{proof}[Proof ($ E(\mathbb{Q}) / 2E(\mathbb{Q}) $ finite)]
\renewcommand{\qedsymbol}{}
\begin{itemize}
\item<2-7> Reduce to $ K \supseteq E[2] $, so that $ y^2 = (x - e_1)(x - e_2)(x - e_3) $.
\item<3-7> Define the complete $ 2 $-descent homomorphism
$$
\begin{array}{rcc}
E(K) & \longrightarrow & K^\times / (K^\times)^2 \times K^\times / (K^\times)^2 \\
\mathcal{O} & \longmapsto & (1, 1) \\
(x, y) & \longmapsto & (x - e_1, x - e_2)
\end{array}.
$$
\item<4-7> Prove its kernel is $ 2E(K) $.
\item<5-7> Prove its image lies in a Selmer group $ K(S, 2) $.
\item<6-7> Prove $ 0 \to \mathcal{O}_K^\times / (\mathcal{O}_K^\times)^n \to K(\emptyset, n) \to \mathrm{Cl}_K[n] \to 0 $ is exact.
\item<7> Prove $ \mathrm{Cl}_K $ is finite (done) and $ \mathcal{O}_K^\times $ is finitely generated (soon). $ \square $
\end{itemize}
\end{proof}
}

\only<8->{
\begin{proof}[Proof ($ E(\mathbb{Q}) / 2E(\mathbb{Q}) $ finite $ \implies E(\mathbb{Q}) $ finitely generated)]
\renewcommand{\qedsymbol}{}
\begin{itemize}
\item<8-> Define the na\"ive height
$$
\begin{array}{rcrcl}
h & : & E(\mathbb{Q}) & \longrightarrow & \mathbb{R} \\
& & \mathcal{O} & \longmapsto & 0 \\
& & (\tfrac{n}{d}, y) & \longmapsto & \log\max(|n|, |d|)
\end{array}.
$$
\item<9-> Prove $ \forall Q \in E(\mathbb{Q}), \ \exists C \in \mathbb{R}, \ \forall P \in E(\mathbb{Q}), \ h(P + Q) \le 2h(P) + C $.
\item<10-> Prove $ \exists C \in \mathbb{R}, \ \forall P \in E(\mathbb{Q}), \ 4h(P) \le h(2P) + C $.
\item<11-> Prove $ \forall C \in \mathbb{R} $, the set $ \{P \in E(\mathbb{Q}) \mid h(P) \le C\} $ is finite.
\item<12-> Prove the descent theorem (done). $ \square $
\end{itemize}
\end{proof}
}

\only<13>{\vspace{-0.5cm} Can finally define the algebraic rank of $ E(\mathbb{Q}) $.}

\end{frame}

\begin{frame}[t]{Algebraic number theory in Lean}

Here are some recent developments.

\vspace{0.5cm}

\begin{minipage}{0.49\textwidth}
\only<2->{Completed:}
\begin{itemize}
\item<2-> Quadratic reciprocity
\item<2-> Hensel's lemma
\item<3-> UF in Dedekind domains
\item<3-> $ \#\mathrm{Cl}_K < \infty $ for global fields
\item<4-> Ad\`eles and id\`eles
\item<4-> Statement of global CFT
\item<5-> L-series of arithmetic functions
\item<5-> Bernoulli polynomials
\item<6-> Perfectoid spaces
\item<6-> Liquid tensor experiment
\end{itemize}
\end{minipage}
\begin{minipage}{0.49\textwidth}
\only<7->{Ongoing:}
\begin{itemize}
\item<7-> S-unit theorem (\textbf{HELP})
\item<7-> FLT for regular primes
\item<7-> p-adic L-functions
\item<7-> $ \mathrm{B}_{\mathrm{dR}} $, $ \mathrm{B}_{\mathrm{HT}} $, and $ \mathrm{B}_{\mathrm{cris}} $
\item<8-> Modular forms
\item<8-> \'Etale cohomology
\item<8-> Local CFT
\item<9> Statement of BSD
\item<9> Statement of GAGA
\item<9> Statement of R=T
\end{itemize}
\end{minipage}

\end{frame}

\begin{frame}

\begin{center}
\textbf{\Huge Thank you!}

\vspace{0.5cm}

\textbf{Check out the leanprover community!}
\end{center}

\end{frame}

\end{document}