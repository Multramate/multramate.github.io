\documentclass[10pt]{beamer}

\setbeamertemplate{footline}[page number]

\usepackage{soul}

\title{The group law on an elliptic curve \footnote{\tiny Angdinata, David Kurniadi and Xu, Junyan. \emph{An Elementary Formal Proof of the Group Law on Weierstrass Elliptic Curves in Any Characteristic}. Fourteenth International Conference on Interactive Theorem Proving (ITP 2023)}}

\author{David Ang}

\institute{London School of Geometry and Number Theory}

\date{Postgraduate Seminar \\ \vspace{0.5cm} \footnotesize Thursday, 5 October 2023}

\begin{document}

\frame{\titlepage}

\begin{frame}[t]{Introduction}

Pedagogical question:
\begin{center}
Is there an \emph{elementary} proof of the group law on \emph{any} elliptic curve?
\end{center}

\only<2-5>{Status quo:
\begin{center}
Yes. But it depends on what is considered \emph{elementary}.
\end{center}
}

\only<3-5>{Our answer:
\begin{center}
Yes. \only<4-5>{And we formalised the argument in the \emph{Lean theorem prover}.}
\end{center}
}

\only<5>{Talk overview:
\begin{itemize}
\item What is an elliptic curve?
\item Why is it a group?
\item Where is the problem then?
\item How did we do it?
\end{itemize}
}

\end{frame}

\begin{frame}[t]{Elliptic curves}

An \textbf{elliptic curve} over a field $ F $ is a pair $ (E, 0) $, where
\begin{itemize}
\item $ E $ is a \emph{smooth projective curve} of \emph{genus one} defined over $ F $, and
\item $ 0 $ is a distinguished point on $ E $ defined over $ F $.
\end{itemize}

\only<2-7>{\vspace{0.5cm} They are the simplest non-trivial objects in arithmetic geometry.
\begin{itemize}
\item<3-7> Wiles proved \emph{Fermat's last theorem} by drawing a correspondence between certain elliptic curves and certain \emph{modular forms}.
\item<4-7> The \emph{Birch and Swinnerton-Dyer conjecture} predicts the behaviour of the \emph{L-function} of an elliptic curve based on its arithmetic invariants.
\end{itemize}
}

\only<5-7>{\vspace{0.5cm} Outside pure mathematics, they see many computational applications.
\begin{itemize}
\item<6-7> Intractability of the \emph{discrete logarithm problem} for elliptic curves forms the basis behind many public key cryptographic protocols.
\item<7> The \emph{Atkin--Morain primality test} and \emph{Lenstra's factorisation method} use elliptic curves and are two of the fastest known algorithms.
\end{itemize}
}

\only<8-11>{
\begin{theorem}[long Weierstrass model]
Any elliptic curve $ E $ over $ F $ can be given by $ E(X, Y) = 0 $, where \vspace{-0.2cm} $$ \vspace{-0.2cm} E(X, Y) := Y^2 + a_1XY + a_3Y - (X^3 + a_2X^2 + a_4X + a_6), $$ for some $ a_i \in F $ such that $ \Delta \ne 0 $, \footnote{\tiny $ \Delta := -(a_1^2 + 4a_2)^2(a_1^2a_6 + 4a_2a_6 - a_1a_3a_4 + a_2a_3^2 - a_4^2) - 8(2a_4 + a_1a_3)^3 - 27(a_3^2 + 4a_6)^2 + 9(a_1^2 + 4a_2)(2a_4 + a_1a_3)(a_3^2 + 4a_6) $} \only<9-11>{with $ 0 $ being the ``point at infinity''.}
\end{theorem}
}

\only<10-11>{
\begin{proof}
Follows from the \emph{Riemann-Roch theorem} in algebraic geometry.
\end{proof}
}

\only<11>{\vspace{0.5cm} If $ \mathrm{char}(F) \ne 2, 3 $, then $ E $ has a \textbf{short Weierstrass model}, where \vspace{-0.2cm} $$ \vspace{-0.2cm} E(X, Y) := Y^2 - (X^3 + aX + b), $$ for some $ a, b \in F $ such that $ \Delta = -16(4a^3 + 27b^2) \ne 0 $.}

\end{frame}

\begin{frame}[t]{Group law}

\begin{theorem}[the group law]
The points of an elliptic curve form an abelian group, where the identity element is $ 0 $, and the addition law is characterised by \vspace{-0.2cm} $$ \vspace{-0.2cm} P + Q + R = 0 \qquad \iff \qquad P, Q, R \ \text{are collinear}. $$
\end{theorem}

\only<2-6>{If $ R = 0 $, then this translates to \vspace{-0.2cm} $$ \vspace{-0.2cm} P + Q = 0 \qquad \iff \qquad \text{line through} \ P \ \text{and} \ Q \ \text{is vertical}. $$}

\only<3-6>{Thus negation can be given by \vspace{-0.2cm} $$ \vspace{-0.2cm} -(x, y) := (x, -y - a_1x - a_3). $$}

\only<4-6>{Define an affine involution given by \vspace{-0.2cm} $$ \vspace{-0.2cm} \sigma(Y) := -Y - a_1X - a_3. $$}

\only<5-6>{Note that \visible<6>{in the \textbf{coordinate ring} $ F[E] := F[X, Y] / \langle E(X, Y) \rangle $,} \vspace{-0.2cm} $$ \vspace{-0.2cm} -(Y \cdot \sigma(Y)) = Y^2 + a_1XY + a_3Y \visible<6>{\equiv X^3 + a_2X^2 + a_4X + a_6,} $$\only<6>{which is a polynomial only in $ X $.}}

\only<7-8>{Addition can be given by $ (x_1, y_1) + (x_2, y_2) := -(x_3, y_3) $.} \only<8>{Here, \vspace{-0.2cm} \begin{align*} \lambda & := \begin{cases} \dfrac{y_1 - y_2}{x_1 - x_2} & x_1 \ne x_2 \vspace{0.2cm} \\ \dfrac{3x_1^2 + 2a_2x_1 + a_4 - a_1y_1}{y_1 - \sigma(y_1)} & y_1 \ne \sigma(y_1) \\ \infty & \text{otherwise} \end{cases}, \\ x_3 & := \lambda^2 + a_1\lambda - a_2 - x_1 - x_2, \\ y_3 & := \lambda(x_3 - x_1) + y_1. \end{align*}}

\end{frame}

\begin{frame}[t]{Hard problem}

\only<1-9>{One may attempt to prove the axioms directly.
\begin{itemize}
\item<2-9> Identity: $ 0 + P = P = P + 0 $ is trivial.
\item<3-9> Inverses: $ (-P) + P = 0 = P + (-P) $ is easy.
\item<4-9> Commutativity: $ P + Q = Q + P $ is easy.
\item<5-9> Associativity: $ (P + Q) + R = P + (Q + R) $ seems impossible?

\only<6-9>{\vspace{0.5cm} Recall that each addition operation has five cases!}

\only<7-9>{\vspace{0.5cm} In the generic case, \footnote{$ P $, $ Q $, $ R $, $ P + Q $, $ P + R $, and $ Q + R $ are affine and have distinct $ X $-coordinates} checking that their $ X $-coordinates are equal is an equality of polynomials with 26,082 terms.}

\only<8-9>{\vspace{0.5cm} In the short Weierstrass model, this reduces to 2,636 terms.}

\only<9>{\vspace{0.5cm} Automation in an interactive theorem prover enables manipulation of multivariate polynomials with at most 5,000 terms.}
\end{itemize}
}

\only<10-17>{Associativity is known to be mathematically difficult with many proofs.
\begin{itemize}
\item<11-17>[Pf 1.] Just do it. \\ Polynomial manipulation, but impossibly slow and many cases.
\item<12-17>[Pf 2.] Count dimensions. \\ Projective geometry (\emph{Cayley-Bacharach}), but only works generically.
\end{itemize}
}

\only<13-17>{One may instead identify the set of points $ E(F) $ with a known group $ G $.
\begin{itemize}
\item<14-17>[Pf 3.] $ G = \mathbb{C} / \Lambda_E $. \\ Riemann surfaces (\emph{uniformisation}), but only works for $ \mathrm{char}(F) = 0 $.
\item<15-17>[Pf 4.] $ G = \mathrm{Pic}_F^0(E) $. \\ Algebraic geometry (\emph{Riemann-Roch}) in general. \\ Ring theory (\emph{Fermat descent}), but only works for $ \mathrm{char}(F) \ne 2 $.
\end{itemize}
}

\only<16-17>{Undergraduate courses typically teach Pf 2 (assuming genericity), Pf 3 (assuming uniformisation), or Pf 4 (assuming Riemann-Roch).}

\only<17>{\vspace{0.5cm} Existing interactive theorem provers have used Pf 1 (Th\'ery 2007) or Pf 4 (Bartzia--Strub 2014), both assuming the short Weierstrass model.}

\end{frame}

\begin{frame}[t]{Algebraic analogue}

\only<1-5>{Let us examine the argument for Pf 3 and Pf 4 in more detail.}

\only<2-5>{\vspace{0.5cm} To identify $ E(F) $ with \only<3-10>{\emph{a subgroup of} }$ G $ is to
\begin{itemize}
\item define a function $ \phi : E(F) \to G $,
\item prove that $ \phi $ respects addition, and
\item prove that $ \phi $ is \only<2>{bijective}\only<3-10>{\st{bijective} injective}.
\end{itemize}
}

\only<4-5>{\vspace{0.5cm} Pf 4 sets $ G = \mathrm{Pic}_F^0(E) $, and \vspace{-0.2cm} $$ \vspace{-0.2cm} \phi \ \text{is injective} \qquad \iff \qquad \text{there is no isomorphism} \ E \xrightarrow{\sim} \mathbb{P}^1, $$ which follows from isomorphism invariance of the \emph{genus}.}

\only<5>{\vspace{0.5cm} Our proof sets $ G = \mathrm{Cl}(F[E]) $, and \vspace{-0.2cm} $$ \vspace{-0.2cm} \phi \ \text{is injective} \qquad \iff \qquad \text{an ideal of} \ F[E] \ \text{is not principal}, $$ which is just a statement in ring theory.}

\only<6-11>{The group $ \mathrm{Cl}(F[E]) $ is the \emph{ideal class group} of the coordinate ring \vspace{-0.2cm} $$ \vspace{-0.2cm} F[E] := F[X, Y] / \langle E(X, Y) \rangle. $$}

\only<7-11>{\underline{Exercise (easy)}: $ F[E] $ is an integral domain.}

\only<8-11>{\vspace{0.5cm} For any integral domain $ R $, the \textbf{ideal class group} $ \mathrm{Cl}(R) $ is the quotient group of \emph{invertible fractional ideals} by those that are \emph{principal}.}

\only<9-11>{
\begin{itemize}
\item A submodule $ I $ is a \textbf{fractional ideal} if $ \exists r \in R $ such that $ r \cdot I \subseteq R $.
\item $ I $ is \textbf{invertible} if there is a fractional ideal $ J $ such that $ I \cdot J = R $.
\item $ I $ is \textbf{principal} if $ \exists r, s \in R $ such that $ r \cdot I = \langle s \rangle $.
\end{itemize}
}

\only<10-11>{\underline{Exercise (hard)}: $ \mathrm{Cl}(R) $ is an abelian group.}

\only<11>{\vspace{0.5cm}
\begin{example}[of invertible fractional ideals]
Any nonzero ideal $ I $ such that $ I \cdot J $ is principal for some ideal $ J $.
\end{example}
}

\only<12-15>{
\begin{proof}[Pf 5 (A.--Xu)]
\begin{itemize}
\item Define a function $ \phi : E(F) \to \mathrm{Cl}(F[E]) $. \visible<13-15>{This will be \vspace{-0.2cm} $$ \vspace{-0.2cm} \begin{array}{rcrcl} \phi & : & E(F) & \longrightarrow & \mathrm{Cl}(F[E]) \\ & & 0 & \longmapsto & [\langle 1 \rangle] \\ & & (x, y) & \longmapsto & [\langle X - x, Y - y \rangle] \end{array}. $$} \visible<14-15>{Note that $ \phi $ is well-defined since \vspace{-0.2cm} $$ \vspace{-0.2cm} \langle X - x, Y - y \rangle \cdot \langle X - x, Y - \sigma(y) \rangle = \langle X - x \rangle. $$}
\item Prove that $ \phi $ respects addition. \visible<15>{This holds since \vspace{-0.2cm} $$ \vspace{-0.2cm} \langle X - x_1, Y - y_1 \rangle \cdot \langle X - x_2, Y - y_2 \rangle \cdot \langle X - x_3, Y - \sigma(y_3) \rangle \qquad $$ $$ \qquad = \langle (Y - y_3) - \lambda(X - x_3) \rangle. $$}
\item Prove that $ \phi $ is injective.
\end{itemize}
\vspace{-0.5cm}
\end{proof}
}

\end{frame}

\begin{frame}[t]{Injectivity proof}

\only<1-14>{Note that $ F[E] $ is free over $ F[X] $ with basis $ \{1, Y\} $.} \only<2-14>{Thus it has a \textbf{norm} \vspace{-0.2cm} $$ \vspace{-0.2cm} \begin{array}{rcrcl} \mathrm{Nm} & : & F[E] & \longmapsto & F[X] \\ & & f & \longmapsto & \det([\cdot f]) \end{array}. $$}

\only<3-8>{\vspace{-0.2cm}
\begin{example}[of norms]
Recall that $ Y \cdot Y \equiv -(a_1X + a_3) \cdot Y + (X^3 + a_2X^2 + a_4X + a_6) $. \only<4-8>{Then \vspace{-0.2cm} \begin{align*} \mathrm{Nm}(Y) & \equiv \det\begin{pmatrix} 0 & 1 \\ X^3 + a_2X^2 + a_4X + a_6 & -(a_1X + a_3) \end{pmatrix} \only<5-8>{\\ & = X^3 + a_2X^2 + a_4X + a_6.} \end{align*}}
\end{example}
}

\only<6-8>{In general, if $ f = p + qY \in F[E] $ for some $ p, q \in F[X] $, \vspace{-0.2cm} $$ \vspace{-0.2cm} \mathrm{Nm}(f) = p^2 - pq(a_1X + a_3) - q^2(X^3 + a_2X^2 + a_4X + a_6). $$}

\only<7-8>{This has degree $ \max(2\deg(p), 2\deg(q) + 3) $.} \only<8>{In particular, \vspace{-0.2cm} \begin{equation} \vspace{-0.2cm} \label{eq:1} \tag{$ \star $} \deg(\mathrm{Nm}(f)) \ne 1. \end{equation}}

\only<9-14>{Now $ [\cdot f] $ has a Smith normal form \vspace{-0.2cm} $$ \vspace{-0.2cm} [\cdot f] \sim \begin{pmatrix} p & 0 \\ 0 & q \end{pmatrix}, \qquad p, q \in F[X]. $$}

\only<10-14>{On one hand, $ F[E] / \langle f \rangle \cong F[X] / \langle p \rangle \oplus F[X] / \langle q \rangle $.} \only<11-14>{Then \vspace{-0.2cm} $$ \vspace{-0.2cm} \dim(F[E] / \langle f \rangle) = \deg(p) + \deg(q). $$}

\only<12-14>{On the other hand, $ \mathrm{Nm}(f) = pq $.} \only<13-14>{Then \vspace{-0.2cm} $$ \vspace{-0.2cm} \deg(\mathrm{Nm}(f)) = \deg(p) + \deg(q). $$}

\only<14>{Combining both equalities and $ (\star) $ yields \vspace{-0.2cm} \begin{equation} \vspace{-0.2cm} \label{eq:1} \tag{$ \dagger $} \dim(F[E] / \langle f \rangle) \ne 1. \end{equation}}

\only<15-22>{
\begin{proof}[Pf 5 (A.--Xu)]
\begin{itemize}
\item Define a function $ \phi : E(F) \to \mathrm{Cl}(F[E]) $.
\item Prove that $ \phi $ respects addition.
\item Prove that $ \phi $ is injective. \visible<16-22>{It suffices to show that $ \langle X - x, Y - y \rangle $ is not principal for any $ (x, y) \in E(F) $.} \visible<17-22>{Suppose otherwise, that \vspace{-0.2cm} $$ \vspace{-0.2cm} \langle X - x, Y - y \rangle = \langle f \rangle, \qquad f \in F[E]. $$}\visible<18-22>{Then \vspace{-0.2cm} \begin{align*} F[E] / \langle f \rangle & = F[E] / \langle X - x, Y - y \rangle \visible<19-22>{\\ & \cong F[X, Y] / \langle E(X, Y), X - x, Y - y \rangle & \text{3\textsuperscript{rd} iso thm}} \visible<20-22>{\\ & = F[X, Y] / \langle X - x, Y - y \rangle & (x, y) \in E(F)} \visible<21-22>{\\ & \cong F & \text{1\textsuperscript{st} iso thm}.} \end{align*}}\visible<22>{Since $ \dim(F) = 1 $, this contradicts ($ \dagger $)!}
\end{itemize}
\end{proof}
}

\end{frame}

\begin{frame}{Conclusions}

Some retrospectives:
\begin{itemize}
\item formalisation encouraged proof accessible to undergraduates
\item novel injectivity proof and novel formalisation
\item proof works for \emph{nonsingular} points of \emph{Weierstrass} curves
\item heavy use of linear algebra and ring theory in Lean's \texttt{mathlib}
\item generality of ideal class groups of integral domains
\item plans for many more formalisation projects!
\end{itemize}
Thank you!

\end{frame}

\end{document}