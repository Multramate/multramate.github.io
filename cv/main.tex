\documentclass[10pt]{moderncv}

\moderncvstyle{classic}
\moderncvcolor{blue}

\usepackage[margin=1in]{geometry}

\nopagenumbers

\firstname{\Huge David Kurniadi}
\familyname{\Huge Angdinata}
\mobile{+44 7551 733331}
\email{ucahdka@ucl.ac.uk}
\homepage{multramate.github.io}
\extrainfo{\href{https://github.com/Multramate}{github.com/Multramate}}

\begin{document}

\makecvtitle

\section{Education}

\cventry{09/21 -- 09/25}{PhD}{London School of Geometry and Number Theory}{London}{}{Research interests: algebraic number theory and arithmetic geometry.}
\cventry{10/20 -- 06/21}{MASt Pure Mathematics}{University of Cambridge}{Cambridge}{}{Pass with Merit with courses in: local fields, elliptic curves, class field theory, profinite groups and group cohomology, schemes and sheaf cohomology, homology and cohomology of manifolds.}
\cventry{10/16 -- 06/20}{MEng Pure Mathematics and Computational Logic}{Imperial College}{London}{}{Best overall performance in degree cohort with courses in: number theory (elementary number theory, algebraic number theory, elliptic curves, modular forms), algebra (Galois theory, commutative algebra, homological algebra), geometry (curves, varieties, topology, manifolds), theoretical computer science (logic, program reasoning, computability, type systems, complexity).}
\cventry{01/14 -- 12/15}{Singapore-Cambridge GCE A-level}{Temasek Junior College}{Singapore}{}{A in four subjects and merit in NUS Modern Physics.}
\cventry{01/12 -- 12/13}{Singapore-Cambridge GCE O-level}{Anderson Secondary School}{Singapore}{}{A in seven subjects and a top student award.}

\section{Employment}

\cventry{07/22 -- 09/22}{Research assistant}{Huawei Technologies R\&D UK Ltd}{London}{}{Summer internship on formalisation of modern mathematics in automated theorem proving.}
\cventry{06/19 -- 09/19}{Cryptography engineer}{Adjoint UK Ltd}{London}{}{Developed three highly polymorphic libraries for zero-knowledge proof protocols in Haskell:
\begin{itemize}
\item \href{http://hackage.haskell.org/package/galois-field}{\texttt{galois-field}} --- an efficient implementation of finite field arithmetic,
\item \href{http://hackage.haskell.org/package/elliptic-curve}{\texttt{elliptic-curve}} --- an extensible database of elliptic curve operations, and
\item \href{http://hackage.haskell.org/package/pairing}{\texttt{pairing}} --- a polymorphic library for bilinear pairing algorithms.
\end{itemize}
Published on Hackage as: \href{http://hackage.haskell.org}{\texttt{hackage.haskell.org/package/\textit{<name>}}}
}

\section{Projects}

\subsection{Number theory}

\cventry{12/21 -- 06/22}{The Mordell-Weil theorem in Lean}{}{}{}{\emph{Mini project, LSGNT (supervised by Prof Kevin Buzzard)} \\ Aim to formalise $ 2 $-descent and heights assuming associativity of the group law.}
\cventry{12/21 -- 04/22}{The Birch and Swinnerton-Dyer conjecture}{}{}{}{\emph{Group mini project, LSGNT (supervised by Prof Vladimir Dokchitser)} \\ Traced the work of Kolyvagin on the Euler system of Heegner points and computed examples.}
\cventry{10/19 -- 06/20}{Arithmetic statistics for elliptic curves}{}{}{}{\emph{Masters thesis (88\%), Imperial College (supervised by Prof Toby Gee)} \\ Detailed a recent heuristic on the boundedness of Mordell-Weil ranks and average size of Selmer groups of elliptic curves over number fields (by modelling the Selmer group as an intersection of two Lagrangian direct summands in an ambient metabolic quadratic module of infinite rank).}
\cventry{07/19 -- 09/19}{Class field theory and applications}{}{}{}{\emph{UROP project, Imperial College (supervised by Dr David Helm)} \\ Reproduced the proof of global Artin reciprocity via Galois cohomology of idele class groups.}
\cventry{08/18 -- 09/18}{The arithmetic of elliptic curves}{}{}{}{\emph{UROP project, Imperial College (supervised by Prof Johannes Nicaise)} \\ Explored the arithmetic of elliptic curves over finite fields (the Hasse-Weil bound, Schoof's algorithm) and the rationals (the Mordell-Weil theorem, torsion computation in Haskell), and discussed their contemporary applications (primality testing, integer factorisation, cryptography).}

\pagebreak

\subsection{Miscellaneous}

\cventry{06/18}{An introduction to finite projective planes}{}{}{}{\emph{Group project (77\%), Imperial College (supervised by Dr Ambrus P\'al)} \\ Characterised existence and uniqueness of Desarguesian and non-Desarguesian projective planes.}
\cventry{01/18 -- 03/18}{Pintos}{}{}{}{\emph{C group project (81\%), Imperial College} \\ Implemented algorithms for thread scheduling and system calls for user programs.}
\cventry{10/17 -- 12/17}{WACC}{}{}{}{\emph{Haskell group project (94\%), Imperial College} \\ Wrote a syntax/semantics parser with Alex/Happy and a code generator with monad transformers.}
\cventry{05/17 -- 06/17}{MelodyPi}{}{}{}{\emph{C group project (4th amongst first year groups) \\ Parsed MIDI files and directed signals through a Raspberry Pi to play a mechanical glockenspiel.}}

\section{Talks}

\cventry{24/08/22}{Formalisation of elliptic curves in Lean}{}{}{}{\emph{Young Researchers in Algebraic Number Theory, University of Glasgow}}
\cventry{05/08/22}{\'Etale cohomology}{}{}{}{\emph{Study group on \'etale cohomology, LSGNT}}
\cventry{05/07/22}{The Tate-Shafarevich and Brauer groups}{}{}{}{\emph{Study group on curves over function fields, University College}}
\cventry{26/05/22}{Elliptic curves and the Mordell-Weil theorem}{}{}{}{\emph{London Learning Lean, Imperial College}}
\cventry{10/05/22}{The Euler system of Heegner points}{}{}{}{\emph{London Junior Number Theory, King's College}}
\cventry{05/05/22}{Kolyagin's work on the BSD conjecture}{}{}{}{\emph{Mini project presentation, LSGNT}}
\cventry{25/04/22}{Elliptic curves in Lean}{}{}{}{\emph{Mathematical Theorem Proving Workshop, Huawei Technologies R\&D UK Ltd}}
\cventry{06/10/21}{Ideal class groups}{}{}{}{\emph{Short introductory talk, LSGNT}}
\cventry{04/12/20}{Rank heuristics for elliptic curves}{}{}{}{\emph{Part III Seminar Series, University of Cambridge}}
\cventry{22/06/20}{Arithmetic statistics for elliptic curves}{}{}{}{\emph{Masters thesis presentation, Imperial College}}
\cventry{11/03/20}{The ideal class group is a Tate-Shafarevich group}{}{}{}{\emph{Presentation, Essen Seminar for Algebraic Geometry and Arithmetic}}
\cventry{04/10/19}{Cryptography engineering at Adjoint UK Ltd}{}{}{}{\emph{Industrial placement presentation, Imperial College}}
\cventry{13/09/19}{Pairing-based elliptic curve cryptography}{}{}{}{\emph{Lunch and Learn, Adjoint UK Ltd}}
\cventry{16/01/19}{An unusual cubic representation problem ($ \frac{a}{b + c} + \frac{b}{a + c} + \frac{c}{a + b} = 4 $)}{}{}{}{\emph{Undergraduate Mathematics Colloquium, Imperial College}}

\section{Teaching}

\cventry{05/22}{Department of Mathematics project supervisor}{University College}{London}{}{Supervised year 1 term 3 projects on continued fractions, cryptography, and Lean.}
\cventry{01/22 -- $ \quad \ \quad $}{Private university tutor}{TutorChase/ElitePrep}{London}{}{Tutored a range of topics in introductory mathematics (proof writing, differential equations, linear algebra, group theory, real analysis, complex analysis, metric spaces, topological spaces), advanced mathematics (topological manifolds, Galois theory, algebraic number theory, local fields, algebraic geometry), and basic computer science (\LaTeX \ writing, Java programming, discrete mathematics, programme reasoning, information structures, graph algorithms)}

\pagebreak

\cventry{10/21 -- $ \quad \ \quad $}{Department of Mathematics teaching assistant}{University College}{London}{}{Ran weekly group tutoring and drop-in revision sessions for mathematics courses in year 1 (algebra 2, analysis 2) and year 2 (further linear algebra, groups and rings, number theory).}
\cventry{10/18 -- 03/20}{Department of Computing teaching assistant}{Imperial College}{London}{}{Held weekly group tutoring sessions for computing courses in year 1 (logic, discrete mathematics, reasoning about programmes, graphs and algorithms).}

\section{Conferences}

\cvline{23 -- 25/08/22}{Young Researchers in Algebraic Number Theory, University of Glasgow}
\cvline{15 -- 19/08/22}{Mordell 2022, Cambridge}
\cvline{08 -- 12/08/22}{Elliptic Curves 2022, Baskerville Hall}
\cvline{06 -- 09/06/22}{73rd British Mathematical Colloquium, King's College}

\section{Awards}

\subsection{Scholarships}

\cvline{2021 -- 2025}{Full funding for 4-year PhD research (EPSRC CDT LSGNT, University College)}
\cvline{2018}{UROP research studentship (Department of Mathematics, Imperial College)}
\cvline{2012 -- 2015}{Full 4-year school-based scholarship (Ministry of Education, Singapore)}

\subsection{Academic}

\cvline{2020}{\emph{Governors' MSci JMC Prize} for best overall performance in final year, £500}
\cvline{2020}{\emph{Donald Davies Prize} for best final year individual project, £500}
\cvline{2017, 18, 20}{Imperial College Faculty of Engineering Dean's List}
\cvline{2017}{\emph{G Research Ltd Prize} for academic excellence, £100}

\subsection{Competitions}

\cvline{2018}{Champion Award in \emph{Imperial College Mathematics Competition Group Round}}
\cvline{2013}{Gold and Team Awards in \emph{Singapore and ASEAN Schools Mathematics Olympiad}}
\cvline{2012, 13}{Certificates of Distinction in \emph{Australian Mathematics Competition}}
\cvline{2012, 13}{Bronze Awards in \emph{Singapore Mathematics Olympiad}}

\section{Skills}

\cvline{Languages}{English, Mandarin/Hokkien, Indonesian/Malay, Japanese}
\cvline{Programming}{Haskell, Lean, Java, C/C++, Python, Prolog}
\cvline{Markup}{LaTeX, XHTML/CSS}
\cvline{Tools}{Git, Stack, Vim}

\section{Miscellaneous}

\cvline{2020 -- 2021}{Owner and moderator of the \emph{Cambridge Part III Mathematics} Discord server}
\cvline{2018 -- 2021}{Live-TeXed lecture notes for geometry, algebra, and number theory available on GitHub}
\cvline{2019 -- 2020}{Problems curator for the \emph{Imperial College Mathematics Competition}}
\cvline{2018 -- 2020}{Organiser for the Imperial College \emph{Undergraduate Mathematics Colloquium}}
\cvline{2012 -- 2018}{Solved 180 \emph{Project Euler} problems primarily in Java and Haskell}

\end{document}